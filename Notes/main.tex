\documentclass[oneside, a4paper,11pt]{book}
\usepackage[margin=1in]{geometry}

\usepackage{graphicx}
\usepackage[utf8]{inputenc} % allow utf-8 input
\usepackage[T1]{fontenc}    % use 8-bit T1 fonts
\usepackage{hyperref}       % hyperlinks
\usepackage{url}            % simple URL typesetting
\usepackage{booktabs}       % professional-quality tables
\usepackage{nicefrac}       % compact symbols for 1/2, etc.
\usepackage{microtype}      % microtypography

\usepackage{listings}
\usepackage{xcolor}
\definecolor{codegreen}{rgb}{0,0.6,0}
\definecolor{codegray}{rgb}{0.5,0.5,0.5}
\definecolor{codepurple}{rgb}{0.58,0,0.82}
\definecolor{backcolour}{rgb}{0.95,0.95,0.92}

\lstdefinestyle{mystyle}{
    backgroundcolor=\color{backcolour},
    commentstyle=\color{codegreen},
    keywordstyle=\color{magenta},
    numberstyle=\tiny\color{codegray},
    stringstyle=\color{codepurple},
    basicstyle=\ttfamily\footnotesize,
    breakatwhitespace=false,
    breaklines=true,
    captionpos=b,
    keepspaces=true,
    numbers=left,
    numbersep=5pt,
    showspaces=false,
    showstringspaces=false,
    showtabs=false,
    tabsize=2
}
\lstset{style=mystyle}


\usepackage[autostyle]{csquotes}
\usepackage{dsfont}

% \usepackage{algorithm,algpseudocode}
\usepackage[ruled,vlined]{algorithm2e}
\usepackage{multicol}

\usepackage{hyperref}
\usepackage{amssymb}

\setlength{\parindent}{0pt}
\setlength{\parskip}{1em}

\newtheorem{theorem}{Theorem}
\newtheorem{lemma}{lemma}
\newtheorem{proof}{proof}
\newtheorem{definition}{Definition}
\newtheorem{proposition}{Proposition}
\newtheorem{corollary}{Corollary}

\newcommand\scalemath[2]{\scalebox{#1}{\mbox{\ensuremath{\displaystyle #2}}}}

\newcommand{\cyan}[1]{\textcolor{cyan}{#1}}

\input{math-com.tex}

\begin{document}

\begin{titlepage}
	\begin{center}
		\vspace*{5.5cm}
		\textbf{\Huge Deep Statistical Learning}\\
        \vspace{2.5cm}
		\includegraphics[width=0.4\textwidth]{./logo/new_logo.pdf}\\
        \vspace{1.5cm}
        % \Large My Note \\
        \vspace{1.5cm}
		Han Cheol Moon\\
		School of Computer Science and Engineering\\
		Nanyang Technological University\\
		Singapore\\
		\texttt{hancheol001@e.ntu.edu.sg}
		\date{\today}
	\end{center}
\end{titlepage}

% \frontmatter
% \maketitle
\tableofcontents
\newpage

\mainmatter
\part{Introduction}
% \section{Probability}
\label{sec:intro_prob}

\begin{definition}{Independence}
	\begin{align*}
		X\perp Y \leftrightarrow p(X,Y)=p(X)p(Y)
	\end{align*}
\end{definition}
\begin{definition}{Conditional independence}
	\begin{align*}
		X\perp Y|Z \leftrightarrow p(X,Y|Z)=p(X|Z)p(Y|Z)
	\end{align*}
\end{definition}
All the dependencies between $X$ and $Y$ are mediated via $Z$. If $X$ and $Y$ are conditionally independent, then 
\begin{align*}
	p(X|Y,Z)&=\frac{p(X,Y|Z)}{p(Y|Z)}\\
	&=\frac{p(X|Z)p(Y|Z)}{p(Y|Z)}\\
	&=p(X|Z).
\end{align*}


\subsection{The law of total expectation}
$$\mathbb{E}[\mathbb{E}[Y|X]] = \mathbb{E}[Y].$$

It is worth trying to parse this formula more carefully and adding some subscripts:
$$\mathbb{E}_X[\mathbb{E}_{Y|X}[Y|X]] = \mathbb{E}[Y].$$
Intuitively, this expression has a divide and conquer flavour, \ie what it says is that to compute the average of a random variable $Y$, you can first compute its average over a bunch of partitions of the sample space (where some other random variable $X$ is fixed to different values), and then average the resulting averages. 

Example: Suppose I had a population of people, 47\% of whom weer men and the remaining 53\% were women. Suppose that the average height of the men was 70 inches, and the women was 71 inches. What is the average height of the entire population?

By the law of total expectation:

\begin{align*}
	\mathbb{E}[H] &= \mathbb{E}[\mathbb{E}[H|S]]\\
				  &= \mathbb{E}[H|S]\mathbb{P}(S=m) + \mathbb{E}[H|S]\mathbb{P}(S=f)\\
			   &= 70\times 0.47 + 71\times 0.53 = 70.53
\end{align*}

\section{Transformations of Random Variable}
\subsection{Formal Definition}
Suppose $X$ is a continuous random variable with pdf $f(x)$. If we define $Y=g(X)$, where $g(\cdot)$ is a monotonically increasing function, then the pdf of $Y$ can be obtained as follows:
\begin{align*}
	p(Y\leq y) &= p(g(X)\leq y)\\
	& = p(X\leq g^{-1}(y))
\end{align*}
This can be re-written as by definition
\begin{align*}
F_Y(y) = F_X(g^{-1}(y))
\end{align*}
By differentiating the CDFs on both sides w.r.t. $y$, we can get the pdf of $Y$. If the function $g(\cdot)$ is monotonically increasing, then the pdf of $Y$ is given by
$$f_Y(y) = f_X(g^{-1}(y))\frac{d}{dy}g^{-1}(y)$$
On the other hand, if it is monotonically decreasing, then the pdf of $Y$ is given by
$$f_Y(y) = - f_X(g^{-1}(y))\frac{d}{dy}g^{-1}(y)$$
Compactly, the above two equations can be combined into a following equation:
$$f_Y(y) = f_X(g^{-1}(y))\Bigg|\frac{d}{dy}g^{-1}(y)\Bigg|$$

\subsection{Intuition}
Given a random variable $z$ and its known probability density function $z\sim \pi(z)$, we would like to construct a new random variable using a one-to-one mapping function $x=f(z)$. The function $f$ is invertible, so $z = f^{-1}(x)$. Now the question is how to infer the unknown probability density function of the new variable, $p(x)$?

$$
\begin{aligned}
& \int p(x)dx = \int \pi(z)dz = 1 \scriptstyle{\text{   ; Definition of probability distribution.}}\\
& p(x) = \pi(z) \left\vert\frac{dz}{dx}\right\vert = \pi(f^{-1}(x)) \left\vert\frac{d f^{-1}(x)}{dx}\right\vert = \pi(f^{-1}(x)) \vert (f^{-1})'(x) \vert
\end{aligned}$$

In multivariate case, 

\begin{align}
\mathbf{z} &\sim \pi(\mathbf{z}), \mathbf{x} = f(\mathbf{z}), \mathbf{z} = f^{-1}(\mathbf{x}) \\
p(\mathbf{x}) 
&= \pi(\mathbf{z}) \left\vert \det \dfrac{d \mathbf{z}}{d \mathbf{x}} \right\vert  
= \pi(f^{-1}(\mathbf{x})) \left\vert \det \dfrac{d f^{-1}}{d \mathbf{x}} \right\vert
\end{align}

\section{Bernoulli distribution}
\label{sec:bernoulli}

The probability of $x=1$ will be denoted by the parameter $\mu$ so that 
$$p(x=1|\mu) = \mu,$$
where $0 \leq \mu \leq 1$, from which it follows that $p(x=0|\mu) = 1-\mu$. The probability distribution over $x$ can be written in the form 
\begin{align}
	Bern(x|\mu) = \mu^x(1-\mu)^{1-x},
	\label{eq:bernoulli}
\end{align}
which is known as the \textit{Bernoulli distribution}.  

Now suppose we have a dataset $\mathcal{D} = \{x_1, \dots, \x_{N}\}$. We can construct the likelihood function, which is a function of $\mu$, on the assumption that the observations are drawn independently from $p(x|\mu)$, so that

$$p(\mathcal{D}|\mu) = \prod_{n=1}^Np(x_n|\mu)= \prod_{n=1}^N\mu^{x_n}(1-\mu)^{1-x_n}$$

In a frequentist, setting, we can estimate a value for $\mu$ by maximizing the likelihood function, or equivalently maximizing the logarithm of the likelihood. 
$$L(w) = p(\rvt|\rvw, \sigma^2) = $$

\section{Gaussian Distribution}
For a $D$-dimensional vector $\rvx$, the multivariate Gaussian distribution takes the form
\begin{align}
	\mathcal{N}(\mathbf{x}|\boldsymbol{\mu},\boldsymbol{\Sigma}) &= \frac{1}{(2\pi)^{D/2}|\boldsymbol{\Sigma}|^{1/2}}\exp\bigg(-\frac{1}{2}(\mathbf{x}-\boldsymbol{\mu})^T\boldsymbol{\Sigma}^{-1}(\mathbf{x}-\boldsymbol{\mu})\bigg)\\
	\label{eq:normal_distribution}
\end{align}

\subsection{Conditional Gaussian Distribution}
Consider first the case of conditional distributions. Suppose $\rvx$ is a $D$-dimensional vector with Gaussian distribution $\mathcal{N}(\mathbf{x}|\boldsymbol{\mu},\boldsymbol{\Sigma})$ and that we partition $\rvx$ into two disjoint subsets $\rvx_a$ and $\rvx_b$. Thus, $\rvx_a$ has $M$ components and $\rvx_b$ has $D-M$ components.
\begin{align*}
	\rvx = \begin{bmatrix}
		\rvx_a\\
		\rvx_b
	\end{bmatrix}.
\end{align*}
Similarly, 
\begin{align*}
	\boldsymbol{\mu} = \begin{bmatrix}
		\boldsymbol{\mu}_a\\
		\boldsymbol{\mu}_b
	\end{bmatrix} 
\end{align*}
and the covariance matrix is given by
\begin{align*}
	\boldsymbol{\Sigma} = \begin{bmatrix}
		\boldsymbol{\Sigma}_{aa} & \boldsymbol{\Sigma}_{ab}\\
		\boldsymbol{\Sigma}_{ba} & \boldsymbol{\Sigma}_{bb} 
	\end{bmatrix} 
\end{align*}
Note that the symmetry $\mSigma^T = \mSigma$ implies that $\mSigma_{ab}^T=\mSigma_{ba}$. We can also define a \textit{precision matrix} as follows: 
$$\mLambda \equiv \mSigma^{-1}$$
We also introduce a partitioned form of the precision matrix:
\begin{align}
	\boldsymbol{\Lambda} = \begin{bmatrix}
		\mLambda_{aa} & \mLambda_{ab}\\
		\mLambda_{ba} & \mLambda_{bb} 
	\end{bmatrix} 
	\label{eq:precision_matrix}
\end{align}
Because the inverse of a symmetric matrix is also symmetric, we see that $\mLambda_{aa}$ and $\mLambda_{bb}$ are symmetric and $\mLambda_{ab}^T =\mLambda_{ba}$. Note that, for instance, $\mLambda_{aa}$ is not simply given by the inverse of $\boldsymbol{\Sigma}_{aa}$. 

Now let's compute the conditional probability:
\begin{align*}
	-\frac{1}{2}(\mathbf{x}-\boldsymbol{\mu})^T\boldsymbol{\Sigma}^{-1}(\mathbf{x}-\boldsymbol{\mu}) &= \frac{1}{2}\Bigg(\bigg(\begin{bmatrix}
		\rvx_a\\
		\rvx_b
	\end{bmatrix}-
	\begin{bmatrix}
		\boldsymbol{\mu}_a\\
		\boldsymbol{\mu}_b
	\end{bmatrix} 
\bigg)^T\begin{bmatrix}
		\mLambda_{aa} & \mLambda_{ab}\\
		\mLambda_{ba} & \mLambda_{bb} 
	\end{bmatrix}\bigg(\begin{bmatrix}
		\rvx_a\\
		\rvx_b
	\end{bmatrix}-
	\begin{bmatrix}
		\boldsymbol{\mu}_a\\
		\boldsymbol{\mu}_b
	\end{bmatrix}\bigg)\Bigg)\\
	&= -\frac{1}{2}\big((\rvx_a-\vmu_a)^T\mLambda_{aa}(\rvx_a-\vmu_a)+(\rvx_a-\vmu_a)^T\mLambda_{ab}(\rvx_a-\vmu_a)\\
	&\quad+(\rvx_a-\vmu_a)^T\mLambda_{ba}(\rvx_a-\vmu_a)+(\rvx_a-\vmu_a)^T\mLambda_{bb}(\rvx_a-\vmu_a)\big)\\
\end{align*}





\begin{figure}[h]
	\centering
	\includegraphics[scale=0.53]{./images/generative/flows/generative_models.png}
	\caption{An overview of generative models.}
\end{figure}

\begin{itemize}
	\item Flow-models get an exact estimate of the likelihood of your sample, as well as in the reverse direction. 
	\item VAEs optimize a lower bound on the (log) likelihood 
	\item GANs minimize a discrepancy between your input and transformed noise distributions. 
\end{itemize}



\section{The Method of Transformations of Random Variables}

If we are interested in finding the PDF of $Y=g(X)$, where $g(\cdot)$ is some deterministic transformation of $X$, and the function $g$ satisfies following properties, we can utilize a method called the method of transformations.
\begin{itemize}
	\item $g(x)$ is differentiable;
	\item $g(x)$ is a strictly (or monotonically) increasing function, that is, if $x_1<x_2$, then $g(x_1)<g(x_2)$.
\end{itemize}

% Now, let $X$ be a continuous random variable and $Y=g(X)$. We will show that you can directly find the PDF of $Y$ using the following formula.
% \begin{equation*}
% 	f_Y(y) = 
% 	\begin{cases}
% 	\frac{f_X(x_1)}{g'(x_1)}=f_X(x_1). \frac{dx_1}{dy} & \quad \textrm{where } g(x_1)=y\\
% 	0 & \quad \textrm{if }g(x)=y \textrm{ does not have a solution}
% 	\end{cases} 
% \end{equation*}

% Note that the derivative $\frac{dx}{dy}$ or $\frac{d}{dy}(g^{-1}(y))$ \textbf{measures how $X$ changes with respect to $Y$}.
% % Note that start with the function $y=f^{-1}(x)$. Write this as $x=f(y)$ and differentiate both sides implicitly with respect to$x$ using the Chain Rule:
% % \begin{align*}
% % 	1 &= f'(y)\frac{dy}{dx}\\
% % 	\frac{dy}{dx} &= \frac{1}{f'(y)}\\
% % 	y &= f^{-1}(x)\\
% % 	[f^{-1}]'(x) &= \frac{1}{f'(f^{-1}(x))}
% % \end{align*}
% Since $g$ is strictly increasing, its inverse function $g^{-1}$ is well defined. You can imagine a simple function like a linear function, \eg $Y=3X+1$. Then, for each $y\in R_Y$, there exists a \textbf{unique} $x_1$ such that $g(x_1)=y$. We can write $x_1=g^{-1}(y)$. Then,
% \begin{align*}
% 	\{Y\leq y\} = \{g(X)\leq y\} = \{X \leq g^{-1}(x)\}.
% \end{align*}
% Thus, 
% \begin{align*}
% 	F_Y(y) &= P(Y\leq y)\\
% 		   &= P(g(X)\leq y)\\
% 		   &= P(X\leq g^{-1}(y))\quad \text{, since }g \text{ is strictly increasing.}\\
% 		   &= F_X(g^{-1}(y)).
% \end{align*}
% To find the PDF of $Y$, we differentiate $F_Y(y)$ as follows:
% \begin{align*}
% 	f_Y(y) &= \frac{d}{dy}F_X(x_1)\quad \text{by } g(x_1)=y\\
% 		   &=\frac{dx_1}{dy}\cdot \underbrace{\frac{d}{dx_1}F_X(x_1)}_{=F'_X(x_1)}\\
% 		   &=\frac{dx_1}{dy}f_X(x_1)\\
% 		   &=f_X(g^{-1}(y))\left|\frac{d}{dy}(g^{-1}(y))\right|
% 		   % &= \frac{f_X(x_1)}{g'(x_1)} \quad \text{, since } \frac{dx}{dy}=\frac{1}{\frac{dy}{dx}}.
% \end{align*}
% We can repeat the same argument for the case where $g$ is \textbf{strictly decreasing}. In that case, $g'(x_1)$ will be \textbf{negative}, so we need to use $|g'(x_1)|$ . Thus, we can state the following theorem for a \textit{strictly monotonic function}. (A function $g:R\to R$ is called strictly monotonic if it is strictly increasing or strictly decreasing.)

% Actually, we assumed that $g$ was one-to-one out of convenience: the condition that $g$ is one-to-one is not necessary for change of variables to work: Consider a continuous random variable $X$ with domain $R_X$, and let $Y=g(X)$. Suppose that we can partition $R_X$ into a finite number of intervals such that $g(x)$ is strictly monotone and differentiable on each partition. Then the PDF of $Y$ is given by 
% \begin{align*}
% 	f_Y(y)= \sum_{i=1}^{n} \frac{f_X(x_i)}{|g'(x_i)|}= \sum_{i=1}^{n} f_X(x_i).
% 	\left|\frac{dx_i}{dy}\right|,
% \end{align*}
% where $x_1,\dots,x_n$ are real solutions to $g(x)=y$.

% \subsection{Intuitive Explanation}
How to derive the PDF of the random variable $Y=g(X)$ when one knows the PDF of the random variable $X$? If $X$ is discrete, we can derive the pmf for $Y$ by simply summing up the probability mass for all the $x$'s such that $f(x)=y$. For a general function $g$, there is no direct formula to get the PDF of the random variable $Y=g(X)$ knowing $p(X)$. There is a formula in case when $h$ is a differentiable one-to-one mapping from the range (\ie the support) of $X$ to the range of $Y$.

Take for example a random variable $X\sim \mathcal{N}(\mu, \sigma)$ and set $Y=\exp(X)$. The figure below shows some simulations of $X$ and the corresponding values of $Y$. The density of $X$ is shown in blue and the one of $Y$ is shown in orange in the vertical direction.

\begin{figure}[t]
	\centering
	\includegraphics[scale=0.23]{./images/generative/flows/change_of_vars.pdf}
\end{figure}

\begin{figure}[t]
    \centering
    \includegraphics[scale=0.7]{./images/generative/flows/change_intuition.pdf}
    \caption{Area would be approximately $p(z)dz = q(z)dx$. Thus, $q(x) = p(z)\Big|\frac{dz}{dx}\Big|$}
    \label{fig:change_intuition}
\end{figure}


% \begin{figure}[ht]
%     \centering
%     \begin{minipage}[b]{0.45\textwidth}
%         \centering
%         \includegraphics[width=\textwidth]{./images/generative/flows/sample.png}
%     \end{minipage}
%     \hfill
%     \begin{minipage}[b]{0.45\textwidth}
%         \centering
%         \includegraphics[width=\textwidth]{./images/generative/flows/sample2.png}
%     \end{minipage}
% \end{figure}
Now the question is: knowing the density of $X$, what is the density of $Y$?
Taking a point $y$ in the range of $Y$, the PDF $f_Y$ provides the probability of $Y$, belong to a small area $dy$ around $y$ by the formula below
$$P(Y\in dy)\approx f_Y(y)|dy|,$$
where $P(Y\in dy)$ is the area below the curve. Similarly, we can define
$$P(X\in dx)\approx f_X(x)|dx|$$
The above two areas are approximately the same in case of very small region. Note that if $dy$ and $dx$ are very small, we can approximate the derivative of $g'(x)=\frac{|dy|}{|dx|}$. Compactly, this can be expressed as follows:
$$P(X\in dx) = f_X(x)\frac{|dy|}{g'(x)}$$
With $y=g(x)$ we can get 
\begin{align*}
	P(X\in dx)\approx P(Y\in dy) &= f_X(x)\frac{|dy|}{g'(x)}\\
	& = f_X(g^{-1}(y))\frac{|dy|}{g'(g^{-1}(y))}\\
	& = f_X(g^{-1}(y))|dy|(g^{-1})'(y)
\end{align*}
The last line is by the derivative of inverse function which is 
\begin{align*}
	\frac{d}{dx}f^{-1}(x) = \frac{1}{f'(f^{-1}(x))}
\end{align*}
Then, 
\begin{align*}
	P(Y\in dy)\approx f_Y(y)|dy| = f_X(g^{-1}(y))|(g^{-1})'(y)|\cdot |dy|
\end{align*}
Finally, we can get 
$$f_Y(y) = f_X(g^{-1}(y))|(g^{-1})'(y)|$$
Note that the absolute is determined by the function $h$. This is the so-called \textit{change of variables formula}.



\subsection{Vector to Vector}

$Z$ and $X$ be random variables which are related by a mapping $f:\mathbb{R}^n\to \mathbb{R}^n$ such that $X=f(Z)$ and $Z=f^{-1}(X)$. Then
\begin{align*}
	p_X(\mathbf{x}) = p_Z(f^{-1}(\mathbf{x})) \left\vert \text{det}\left(\frac{\partial f^{-1}(\mathbf{x})}{\partial \mathbf{x}}\right) \right\vert
\end{align*}

Note that for any invertible square matrix $A$ over a field (\eg real or complex numbers),
\begin{align*}
	\det\bigl(A^{-1}\bigr)=\frac{1}{\det(A)}.
\end{align*}

Let
$$
A=\begin{bmatrix}
2 & 1\\[2pt]
3 & 4
\end{bmatrix}.
$$

Then, determinant of $A$ is given by
$$
\det(A)=2\cdot4-3\cdot1 = 8-3 = 5.
$$

The inverse of $A$ is 
$$
A^{-1}= \frac{1}{5}\begin{bmatrix}
4 & -1\\[2pt] -3 & 2
\end{bmatrix}.
$$

Then, the determinant of $A^{-1}$ is
$$
\det(A^{-1}) = \frac{1}{5^2}\bigl(4\cdot2-(-3)(-1)\bigr)
=\frac{1}{25}(8-3)=\frac{5}{25}=\frac{1}{5}.
$$

You can confirm the property:

$$
\det(A^{-1})=\frac{1}{5}=\frac{1}{\det(A)}.
$$


For example, we can transform $(x_1,x_2)$ to $(r,\theta)$ via $x_1=r\cos\theta$ and $x_2=r\sin\theta$.

Then
\[
J_{y\to x}=
\begin{pmatrix}
\dfrac{\partial x_1}{\partial r} & \dfrac{\partial x_1}{\partial\theta}\\[6pt]
\dfrac{\partial x_2}{\partial r} & \dfrac{\partial x_2}{\partial\theta}
\end{pmatrix}
=
\begin{pmatrix}
\cos\theta & -r\sin\theta\\
\sin\theta & \phantom{-}r\cos\theta
\end{pmatrix},
\tag{3}
\]
so
\[
\bigl|\det J_{y\to x}\bigr|
 = \bigl|\,r\cos^{2}\theta + r\sin^{2}\theta\,\bigr|
 = |r|.
\tag{4}
\]

Hence
\[
p_{\mathbf{y}}(\mathbf{y}) = p_{\mathbf{x}}(\mathbf{x})\,|J_{y\to x}|
\quad\Longrightarrow\quad
p_{R,\Theta}(r,\theta) = p_{X_1,X_2}(x_1,x_2)\,r.
\tag{5--6}
\]

For a two dimensional random vector $(X,Y)$ with density $p_{X,Y}$, 
\begin{align*}
	Pr((X,Y)\in A) = \int\int_A p_{X,Y}(x,y)dxdy.
\end{align*}
For a infinitesimally small region, we can approximate as follows:
\begin{align*}
	\int_x^{x+dx} p_{X}(x)dx \approx  p_{X}(x)dx.
\end{align*}

Similarly, we can get the probability as follows:
\[
P\!\bigl(r\le R\le r+dr,\;
        \theta\le\Theta\le\theta+d\theta\bigr)
  = p_{R,\Theta}(r,\theta)\,dr\,d\theta.
\tag{7}
\]
Note that the length of arc is $r\times d\theta$. Thus, the area $r\,dr\,d\theta$ or probability is given by
\[
P\!\bigl(r\le R\le r+dr,\;
        \theta\le\Theta\le\theta+d\theta\bigr)
  = p_{X,Y}(r\cos\theta, r\sin\theta)\,r\,dr\,d\theta,
\tag{8--9}
\]
so that finally
\[
p_{R,\Theta}(r,\theta)
  = p_{X,Y}(r\cos\theta, r\sin\theta)\,r.
\tag{10}
\]

% \begin{figure}[h]
%   \centering
%   % Replace the filename with the actual graphic if you have it.
%   \includegraphics[width=.55\linewidth]{polar_patch}
%   \caption{Change of variables from polar to Cartesian.
%            The area of the shaded patch is $r\,dr\,d\theta$.}
%   \label{fig:polarPatch}
% \end{figure}

\chapter{Fourier Analysis}
Reference: Y.D. Chong 2021, NTU Lecture Note

The Fourier transform is one of the most important mathematical tools used for analyzing functions. Given an arbitrary function $f(x)$, with a real domain $(x \in \mathbb{R})$, we can express it as a linear combination of complex waves

\section{Fourier Series}
\label{sec:fourier_series}

We begin by discussing the Fourier series, which is used to analyze functions that are periodic in their inputs. A periodic function $f(x)$ is a function of a real variable $x$ that repeats itself every time $x$ changes by $a$. The constant $a$ is called the period. We can write the periodicity condition as
$$f(x+a) = f(x), \forall x\in \mathbb{R}.$$
The value of $f(x)$ can be real or complex, but $x$ should be real.

Let's consider what it means to specify a periodic function $f(x)$. One way to specify the function is to give an explicit mathematical formula for it. Another approach might be to specify the function values in $−a/2 \leq x < a/2$. Since there's an uncountably infinite number of points in this domain, we can generally only achieve an approximate specification of $f$ this way, by giving the values of $f$ at a large but finite set $x$ points. 

There is another interesting approach to specifying f. We can express it as a linear combination of simpler periodic functions, consisting of sines and cosines:
$$f(x) = \sum_{n=1}^{\infty} \alpha_n \sin\bigg(\frac{2\pi n x}{a}\bigg)+\sum_{m=0}^{\infty}\beta_m \cos\bigg(\frac{2\pi m x}{a}\bigg).$$
Note that the index $n$ does not include 0; since the sine term with $n = 0$ vanishes for all $x$, it's redundant. The above formula is called a \textit{Fourier series}. Given the numbers $\{\alpha_n, \beta_m\}$, which are called the \textit{Fourier coefficients}, $f(x)$ can be calculated for any $x$. The Fourier coefficients are real if $f(x)$ is a real function, or complex if $f(x)$ is complex.

\subsection{Square-integrable functions}
Can arbitrary periodic functions always be expressed as a Fourier series? It turns out that a certain class of periodic functions, commonly encountered in physical contexts, are guaranteed to always be expressible as Fourier series. These are called square-integrable functions such that the integral of the square of the absolute value is finite:
\begin{align*}
\int_{-a/2}^{a/2}|f(x)|^2 dx < \infty.
\end{align*}
Unless otherwise stated, we will always assume that the functions we're dealing with are square-integrable.

\subsection{Complex Fourier series and inverse relations}

We have written the Fourier series as a sum of sine and cosine functions. However, sines and cosines can be expressed by exponential functions by using \textit{Euler's formula}. 
\begin{align}
	e^{ix}=\cos x+i\sin x
	\label{eq:euler_formula}
\end{align}
\begin{itemize}
	\item $\cos x=\frac{e^{ix}+e^{-ix}}{2}$
	\item $\sin x=\frac{e^{ix}-e^{-ix}}{2i}$
\end{itemize}
Thus, Fourier series can be expressed as follows:
$$f(x) = \sum_{n=-\infty}^{\infty}e^{2\pi i n x/a}f_n$$
\begin{itemize}
	\item $i$: complex number
	\item $n$: integer
	\item $a$: period
	\item $f_n$: Fourier coefficient
\end{itemize}



\chapter{Training, Testing, and Regularization}
\section{Sources of Error in ML}
$$E_{out} \leq E_{ml}+\Omega$$
\begin{itemize}
	\item $E_{out}$: estimation of error. 
	\item $E_{ml}$: error from a learning algorithm
	\item $\Omega$: error caused by the variance from observations. 
\end{itemize}
We also define 
\begin{itemize}
	\item $f$: target function
	\item $g$: learning function
	\item $g^{(D)}$: learned function based on $D$, or simply \textit{hypothesis}.
	\item $D$: dataset drawn from the real world.
	\item $\bar{g}$: the average hypothesis of a given infinite number of $D$s. 
		$$\bar{g}(x) = \mathbb{E}_D[g^{(D)}(x)].$$
\end{itemize}

Error of a single instance $x$ from $g$ learnt from $D$ is given by
\begin{align*}
	Err_{\textrm{out}}(g^{(D)}(x)) = \mathbb{E}_{X}[(g^{(D)}(x)-f(x))^2],
\end{align*}
where $X$ can be considered as test sets. Then, the expected error over the infinite number of datasets $D$ sampled from a true data distribution is
\begin{align*}
	\mathbb{E}_D[Err_{\textrm{out}}(g^{(D)}(x))] &= \mathbb{E}_D[\mathbb{E}_{X}[(g^{(D)}(x)-f(x))^2]]\\
												 &= \mathbb{E}_X[\mathbb{E}_D[(g^{(D)}(x)-f(x))^2]]
\end{align*}
Let's simplify the term inside with an average of hypothesis $\bar{g}(x)$:
\begin{align*}
	\mathbb{E}_D[(g^{(D)}(x)-f(x))^2]&= \mathbb{E}_D[(g^{(D)}(x)-\bar{g}(x)+\bar{g}(x)-f(x))^2]\\
	&= \mathbb{E}_D\big[(g^{(D)}(x)-\bar{g}(x))^2+(\bar{g}(x)-f(x))^2\\ &\quad + 2 (g^{(D)}(x)-\bar{g}(x))(\bar{g}(x)-f(x))\big]\\
	&= \mathbb{E}_D\big[(g^{(D)}(x)-\bar{g}(x))^2\big]+(\bar{g}(x)-f(x))^2\\ &\quad + \mathbb{E}_D\big[2 (g^{(D)}(x)-\bar{g}(x))(\bar{g}(x)-f(x))\big]
\end{align*}
Since, $\mathbb{E}_D\big[2 (g^{(D)}(x)-\bar{g}(x))(\bar{g}(x)-f(x))\big]$ is 0, the expectation of the error becomes
\begin{align*}
	\mathbb{E}_D[Err_{\textrm{out}}(g^{(D)}(x))] = \mathbb{E}_X\big[\mathbb{E}_D\big[(g^{(D)}(x)-\bar{g}(x))^2\big]+(\bar{g}(x)-f(x))^2\big].
\end{align*}
Let's closely look at this formula. The errors are from two sources:
\begin{itemize}
	\item \textbf{Variance}: $\mathbb{E}_D\big[(g^{(D)}(x)-\bar{g}(x))^2\big]$. Variance captures how much your classifier changes if you train on a different training set. We need to collect more data to reduce the variance. 
	\item \textbf{Bias}: $(\bar{g}(x)-f(x))^2$. Bias is the inherent error that you obtain from your classifier even with infinite training data. We need to build a more complex model to reduce the bias. 
\end{itemize}
However, if we reduce the bias, then the variance tends to increase. 

\subsection{Alternative Derivation}

The derivation of the bias–variance decomposition for squared error proceeds as follows.[6][7] For notational convenience, abbreviate $f = f(x)$ and $\hat{f} = \hat{f}(x)$. First, recall that, by definition, for any random variable $\mathbf{X}$, we have
$$\displaystyle \operatorname {Var} {\big [}{\hat {f}}(x){\big ]}=\operatorname {E} [X^{2}]-\operatorname {E} [X]^{2}.$$
By rearranging, we get 
$$\operatorname {E} [X^{2}] = \displaystyle \operatorname {Var} {\big [}{\hat {f}}(x){\big ]}+\operatorname {E} [X]^{2}.$$
Since $f$ is deterministic
$$\operatorname {E} [f] = f$$

Thus, given $y = f+\varepsilon$ and $\operatorname {E} [\varepsilon] = 0$, implies $\operatorname {E} [y] = \operatorname {E} [f+\varepsilon] = \operatorname {E} [f] = f$

Also, since $\operatorname{Var} [\varepsilon ]=\sigma ^{2}$
$$\displaystyle \operatorname {Var} [y]=\operatorname {E} [(y-\operatorname {E} [y])^{2}]=\operatorname {E} [(y-f)^{2}]=\operatorname {E} [(f+\varepsilon -f)^{2}]=\operatorname {E} [\varepsilon ^{2}]=\operatorname {Var} [\varepsilon ]+{\Big (}\operatorname {E} [\varepsilon ]{\Big )}^{2}=\sigma ^{2}$$

Thus, since $\varepsilon$ and $\hat {f}$ are independent, we can write:
\begin{align*}
	\operatorname {E} {\big [}(y-{\hat {f}})^{2}{\big ]}&=\operatorname {E} {\big [}(f+\varepsilon -{\hat {f}})^{2}{\big ]}\\
	&=\operatorname {E} {\big [}(f+\varepsilon -{\hat {f}}+\operatorname {E} [{\hat {f}}]-\operatorname {E} [{\hat {f}}])^{2}{\big ]}\\
	&=\operatorname {E} {\big [}(f-\operatorname {E} [{\hat {f}}])^{2}{\big ]}+\operatorname {E} [\varepsilon ^{2}]+\operatorname {E} {\big [}(\operatorname {E} [{\hat {f}}]-{\hat {f}})^{2}{\big ]}+2\operatorname {E} {\big [}(f-\operatorname {E} [{\hat {f}}])\varepsilon {\big ]}+\\
	&\quad 2\operatorname {E} {\big [}\varepsilon (\operatorname {E} [{\hat {f}}]-{\hat {f}}){\big ]}+2\operatorname {E} {\big [}(\operatorname {E} [{\hat {f}}]-{\hat {f}})(f-\operatorname {E} [{\hat {f}}]){\big ]}\\
	&=(f-\operatorname {E} [{\hat {f}}])^{2}+\operatorname {E} [\varepsilon ^{2}]+\operatorname {E} {\big [}(\operatorname {E} [{\hat {f}}]-{\hat {f}})^{2}{\big ]}+\\
	&\quad 2(f-\operatorname {E} [{\hat {f}}])\operatorname {E} [\varepsilon ]+2\operatorname {E} [\varepsilon ]\operatorname {E} {\big [}\operatorname {E} [{\hat {f}}]-{\hat {f}}{\big ]}+2\operatorname {E} {\big [}\operatorname {E} [{\hat {f}}]-{\hat {f}}{\big ]}(f-\operatorname {E} [{\hat {f}}])\\
	&=(f-\operatorname {E} [{\hat {f}}])^{2}+\operatorname {E} [\varepsilon ^{2}]+\operatorname {E} {\big [}(\operatorname {E} [{\hat {f}}]-{\hat {f}})^{2}{\big ]}\\
	&=(f-\operatorname {E} [{\hat {f}}])^{2}+\operatorname {Var} [y]+\operatorname {Var} {\big [}{\hat {f}}{\big ]}\\
	&=\operatorname {Bias} [{\hat {f}}]^{2}+\operatorname {Var} [y]+\operatorname {Var} {\big [}{\hat {f}}{\big ]}\\
	&=\operatorname {Bias} [{\hat {f}}]^{2}+\sigma ^{2}+\operatorname {Var} {\big [}{\hat {f}}{\big ]}
\end{align*}

\chapter{Optimization}
\label{ch:optimization}

\section{Intuition of Gradient}

Gradient is a vector function that represents a direction of steepest increase of a function to be differentiated at a certain point. For example, a convex function $z = ax^2+by^2$ has a gradient $[2ax, 2by]$. Its steepest descent direction is $[-2ax, -2by]$. At the point $x_0,y_0$, the gradient direction for the function is $2ax_0(y-y_0) = 2by_0(x-x_0)$. But then the lowest point $(x,y) = (0,0)$ does not lie on the line. Thus, we cannot find that minimum point in one step of ``gradient descent''. The steepeset direction does not lead to the bottom of the bowl. 

\subsection{Direction of Gradient Descent}

% Most deep learning algorithms involve \textbf{optimization}.

\begin{itemize}
	\item The derivative of the objective function $f(\mathbf{x})$ provides the slope of $f(\mathbf{x})$ at the point $f(\mathbf{x})$.
	\item It tells us how to change $\mathbf{x}$ in order to make a small improvement in our goal.
\end{itemize}

	A function $f(\mathbf{x})$ can be approximated by its first-order Taylor expansion at $\bar{\mathbf{x}}$:
	$$f(\mathbf{x})\approx f(\bar{\mathbf{x}})+\nabla f(\bar{\mathbf{x}})^T(x-\bar{\mathbf{x}})$$
	Now let $\mathbf{d}\neq0, \|\mathbf{d}\|=1$ be a direction, and in consideration of a new point $\mathbf{x}:=\bar{\mathbf{x}}+\mathbf{d}$, we define:
	$$f(\bar{\mathbf{x}}+\mathbf{d})\approx f(\bar{\mathbf{x}})+\nabla f(\bar{\mathbf{x}})^T\mathbf{d}$$

We would like to choose $\mathbf{d}$ that minimizes the function $f$. From the Cauchy-Schwarz inequality \footnote{Cauchy-Schwarz Inequaility: $|\mathbf{a}\cdot \mathbf{b}|\leq \|\mathbf{a}\|\textrm{ } \|\mathbf{b}\|$. Equality holds if and only if either $\mathbf{a}$ or $\mathbf{b}$ is a multiple of the other.}, we know that
$$|\nabla f(\bar{\mathbf{x}})^T\mathbf{d}|\leq \|\nabla f(\bar{\mathbf{x}})\|\textrm{ }\|\mathbf{d}\|.$$
The equality holds if and only if $\mathbf{d}=\lambda \nabla f(\bar{\mathbf{x}})$, where $\lambda\in \mathbb{R}$. Since we want to minimize the function $f$, we negate the steepest direction $\mathbf{d}^{*}$ so the iterations of steepest descent becomes
	$$\mathbf{x}^{(k+1)} = \mathbf{x}^{(k)} - \eta \nabla f(\mathbf{x}^{(k)})$$
	, where $k$ is the index of iteration step and $\eta$ is the learning rate parameter.



%	$\nabla f(\bar{\mathbf{x}})^T\mathbf{d}/\|\mathbf{d}\|\geq \|\nabla f(\bar{\mathbf{x}})\|=\nabla f(\bar{\mathbf{x}})^T\Bigg(\frac{\nabla f(\bar{\mathbf{x}})}{\|\nabla f(\bar{\mathbf{x}})\|}\Bigg)$

\section{Normalized Gradient Descent}

The underlying issue of the vanila gradient descent is the presence of saddle points in nonconvex functions; the gradient $\nabla f(x)$ vanishes near saddle points, which causes GD to ``stall'' in neighboring regions. This both slows the overall convergence rate and makes detection of local minima difficult. The detrimental effects of this issue become particularly severe in high-dimensional problems where the number of saddle points may proliferate.


However, in the normalized gradient descent
$$ \frac{\nabla f(x)}{\|\nabla f(x)\|}$$
The normalized gradient preserves the direction of the gradient but ignores magnitude, because the normalization does not vanish near saddle points, the intuitive expectation is that NGD should not slow down in the neighborhood of saddle points and should therefore escape quickly. 

\section{Projected Gradient Descent}
Gradient Descent (GD) is a standard way to solve unconstrained optimization problem. Starting from an initial point $x\in \mathbb{R}^n$, GD itereates until a stopping criterion is met. Projected Gradient Descent (PGD) is a way to solve constrained optimization problem. Consider a constraint set $\mathcal{Q}$, starting from a initial point $x_0 \in \mathcal{Q}$, PGD iterates the following equation until a stopping condition is met:
$$x_{k+1} = P_\mathcal{Q} \Big(x_k - t_k \nabla f(x_k)\Big),$$
where $P_\mathcal{Q}$ is the projection operator
$$P_\mathcal{Q}(x_0) = \argmin_{x\in \mathcal{Q}} \frac{1}{2}\|x-x_0\|^2_2$$
In other words, given a point $x_0$, $P_\mathcal{Q}$ tries to to find a point $x\in \mathcal{Q}$ which is ``closest''to $x_0$.


Note that a vector projection can be expressed as follows:
$$a_1 = \|\mathbf{a}\|\cos \theta = \mathbf{a}\cdot \hat{\mathbf{b}} = \mathbf{a}\cdot \frac{\mathbf{b}}{\|\mathbf{b}\|}$$
Thus, a projection for unit $L_2$ ball is given by the solution of the equation as follows:
$$\mathbf{x} = \mathcal{P}_{\|x\|_2\leq 1}(\mathbf{y})$$
The solution is
$$\mathbf{x} = \frac{\mathbf{y}}{\max \{1,\|\mathbf{y}\|_2\}}$$
The ``geometric'' proof is given as follows: Let $\mathcal{S} = \{\mathbf{x}\in \mathbb{R}^n: \|\mathbf{x}\|_2 \leq 1\}.$
\begin{itemize}
		\item If $\mathbf{y}\in \mathcal{S}$, then $\|y\|_2\leq 1$ and $\mathbf{y}$ itself is the closest point to $\mathbf{y}$.
		\item If $\mathbf{y}\notin \mathcal{S}$, then $\|y\|_2> 1$ and the closest point $\mathbf{x}\in \mathcal{S}$ to $\mathbf{y}$ will be simply $\frac{\mathbf{y}}{\|\mathbf{y}\|_2}$ as the norm of $\frac{\mathbf{y}}{\|\mathbf{y}\|_2}=1$.
\end{itemize}
By combining the bost cases, we have
$$\mathbf{x} = \frac{\mathbf{y}}{\max \{1,\|\mathbf{y}\|_2\}}$$


\section{Exponentially Weighted Average}
\begin{align*}
    v_t = \beta v_{t-1} + (1-\beta) \theta_t
\end{align*}
Larger $\beta$ value covers more longer history. EMA is exponentially weighted average the previous result. 

\section{Bias Correction}
The initial values of $v_t$ will be very low which need to be compensated, since the curve starts from 0, there are not many values to average on in the initial points. Thus, the curve is lower than the correct value initially and then moves in line with expected values. The $\beta$ is the same as the averaging coefficient. As $t$ becomes large, the impact of the bias correction will be decreased. 
\begin{align*}
    v_t = \frac{v_t}{1-\beta^t}
\end{align*}


\section{Momentum}
Momentum can reduce the oscillation in the gradients. Let's say $w$ has a small value and $b$ is in charge of oscillation. Then momentum can cancel out $db$ by averaging them. 

\begin{align*}
    &v_{dw} = \beta v_{dw} + (1-\beta)dw \\
    &v_{db} = \beta v_{db} + (1-\beta)db \\
    & w = w-\alpha v_{dw}\\
    & b = b-\alpha v_{db}\\
\end{align*}

\section{Adagrad: Adaptive Gradient}
\begin{align*}
    &v_{dw} = v_{dw} + dw \cdot dw \\
	& w = w-\frac{\alpha}{\sqrt{v_{dw}}+\epsilon} v_{dw}\\
\end{align*}
A con of Adagrad is learning rate will become very small


\section{RMS Prop}

\begin{align*}
    &s_{dw} = \beta s_{dw} + (1-\beta)dw^2 \\
    &s_{db} = \beta s_{db} + (1-\beta)db^2 \\
	& w = w-\alpha  \frac{dw}{\sqrt{s_{dw}}}\\
	& b = b-\alpha \frac{db}{\sqrt{s_{db}}} \\
\end{align*}

\section{ADAM}
Its name is derived from adaptive moment estimation, and the reason it's called that is because Adam uses estimations of first and second moments of gradient to adapt the learning rate for each weight of the neural network. $N$-th moment of a random variable is defined as the expected value of that variable to the power of $n$. More formally:
$$m_n = \mathbb{E}[X^n]$$

To estimates the moments, Adam utilizes exponentially moving averages, computed on the gradient evaluated on a current mini-batch:

Since $m$ and $v$ are estimates of first and second moments, we want to have the following property:
\begin{align}
	\mathbb{E}[m_t] &= \mathbb{E}[g_t]\\
	\mathbb{E}[v_t] &= \mathbb{E}[g_t^2]
\end{align}
Unbiased estimators

ADAM uses both momentum style and RMS prop style averaging. 
\begin{itemize}
	\item $v_{dw} = \beta v_{dw} + (1-\beta)dw$
	\item $v_{db} = \beta v_{db} + (1-\beta)db$
	\item $s_{dw} = \beta s_{dw} + (1-\beta)dw^2 $
	\item $s_{db} = \beta s_{db} + (1-\beta)db^2 $
\end{itemize}

Using them,
\begin{itemize}
	\item $ v^{\textrm{corr}}_{dw} = \frac{v_{dw}}{1-\beta_1^t}$
	\item $ v^{\textrm{corr}}_{db} = \frac{v_{db}}{1-\beta_1^t}$
	\item $ s^{\textrm{corr}}_{dw} = \frac{s_{dw}}{1-\beta_2^t}$
	\item $ s^{\textrm{corr}}_{db} = \frac{s_{db}}{1-\beta_2^t}$
\end{itemize}

Finally, 
\begin{align*}
	& w = w-\alpha  \frac{v^{\textrm{corr}}_{dw}}{\sqrt{s^{\textrm{corr}}_{dw}}+\varepsilon}\\
	& b = b-\alpha  \frac{v^{\textrm{corr}}_{db}}{\sqrt{s^{\textrm{corr}}_{db}}+\varepsilon}\\
\end{align*}





\section{Principal Component Analysis}
\subsection{Covariance and the weight vector}

When deriving PCA, we seek a vector \( \mathbf{w} \) (the weight vector or loading vector) such that the projection of the data onto this vector maximizes the variance. For a given data matrix \( \mathbf{X} \) with mean zero (mean-centered data), the projection of the data onto \( \mathbf{w} \) is given by \( \mathbf{X}\mathbf{w} \).

The variance of the projected data can be expressed as:

\[
\text{Var}(\mathbf{X}\mathbf{w}) = \frac{1}{n} (\mathbf{X}\mathbf{w})^T (\mathbf{X}\mathbf{w}) = \frac{1}{n} \mathbf{w}^T \mathbf{X}^T \mathbf{X} \mathbf{w}
\]

Where \( n \) is the number of data points. The matrix \( \mathbf{X}^T \mathbf{X} \) is the covariance matrix of the data (up to a scaling factor).

The goal is to maximize the variance \( \mathbf{w}^T \mathbf{X}^T \mathbf{X} \mathbf{w} \) with respect to the weight vector \( \mathbf{w} \), subject to the constraint that \( \mathbf{w}^T \mathbf{w} = 1 \) (to prevent the trivial solution where the variance could be made arbitrarily large just by scaling \( \mathbf{w} \)).
\begin{align*}
	\mathcal{L} = \frac{1}{n} \mathbf{w}^\mathsf{T} \mathbf{X}^\mathsf{T} \mathbf{X} \mathbf{w} - \lambda \left( \mathbf{w}^\mathsf{T} \mathbf{w}  - 1 \right)
\end{align*}

\begin{align*}
	\frac{\partial \mathcal{L}}{\partial \mathbf{w}} = \frac{2}{n} \mathbf{X}^\mathsf{T} \mathbf{X} \mathbf{w} - 2 \lambda \mathbf{w} = \mathbf{0}
\end{align*}

\begin{align*}
	\underbrace{ \frac{1}{n} \mathbf{X}^\mathsf{T} \mathbf{X} }_{:= \mathbf{S} } \mathbf{w} = \lambda \mathbf{w}  \quad \Rightarrow \quad \mathbf{S} \mathbf{w} = \lambda \mathbf{w}
\end{align*}
This is exactly the eigenvalue equation. The eigenvectors $\rvw$ are the directions that maximize the variance, and the eigenvalues $\lambda$ represent the magnitude of the variance along those directions.

\begin{itemize}
	\item Eigenvectors: Each eigenvector of the covariance matrix represents a direction in the feature space. These directions are the principal components.
	\item Eigenvalues: The corresponding eigenvalue tells us how much variance is captured along that direction. The larger the eigenvalue, the more variance is captured by the corresponding eigenvector.
\end{itemize}

\[
\text{Var}(\mathbf{X}\mathbf{w}) = \frac{1}{n} (\mathbf{X}\mathbf{w})^T (\mathbf{X}\mathbf{w}) = \frac{1}{n} \mathbf{w}^T \mathbf{X}^T \mathbf{X} \mathbf{w} = \frac{1}{n} \mathbf{w}^T \mathbf{S} \mathbf{w} = \frac{1}{n} \mathbf{w}^T \lambda \mathbf{w} = \frac{1}{n} \lambda
\]

Capturing the Most Information:

\begin{itemize}
	\item Variance is a measure of how spread out the data is along a particular direction. By maximizing the variance, we ensure that the principal components capture the most significant patterns in the data.
	\item If we reduce the dimensionality by selecting components with the highest variance, we retain the most information about the data, effectively compressing the data without losing critical details.
	\item High variance indicates that the data points are spread out and less likely to be redundant. Conversely, low variance implies that data points are clustered close together, often making the information less significant.
\end{itemize}



\part{Regression}
\chapter{Introduction to Regression Methods}
\label{chapter:regression_intro}
\section{Regression}
\label{sec:basic_regression}
Suppose a noisy measurement $\rvy = [y_1, \dots, y_m]^T$ with noise $\boldsymbol{\eta} = [\eta_1, \dots, \eta_d]^T$ and we want to estimate parameter $\boldsymbol{\beta} = [\beta_1,\dots,\beta_d]^T$ by using our input $\rvx = [x_1, \dots, x_d]$. 

The measurement $\rvy$ can be modeled as follows:
$$\rvy = \mathbf{X}\boldsymbol{\beta}+\boldsymbol{\eta},$$
where $\mathbf{X}$ is a $m\times d$ input matrix (or our observations). Given a parameter $\boldsymbol{\beta}$, we consider the difference between the noisy measurements and estimated value as follows:
$$\boldsymbol{\epsilon} = \rvy - \mathbf{X}\boldsymbol{\beta}$$
Then, we can establish an objective function as follows:
$$J(\boldsymbol{\beta}) = \boldsymbol{\epsilon}^T\boldsymbol{\epsilon}$$
Note that this is equivalent to minimizing the mean squared error:
$$MSE = \frac{1}{n}\sum_{i=1}^n (y_i-\rvx_i\boldsymbol{\beta})^2.$$
% The OLEs' solution can be optimized by a closed form as follows:
% $$f(\rvx) = \mathbf{X}\boldsymbol{\beta},$$
% where $\rvx = [x_1, \dots, x_d]$ and $\boldsymbol{\beta} = [\beta_1,\dots,\beta_d]^T$. The ridge regression for $\mathbf{X}\in \mathbb{R}^{n\times d}$ matrix can be modeled as follows:
We can optimize this in a closed-form as follows:
\begin{align*}
	J(\boldsymbol{\beta}) &= \|\mathbf{y}-\mathbf{X}\boldsymbol{\beta}\|^2_2 \\
			&= (\mathbf{y}-\mathbf{X}\boldsymbol{\beta})^T(\mathbf{y}-\mathbf{X}\boldsymbol{\beta})\\
			&= (\mathbf{y}^T-\boldsymbol{\beta}^T\mathbf{X}^T)(\mathbf{y}-\mathbf{X}\boldsymbol{\beta})\\
			&= \rvy^T\rvy-\boldsymbol{\beta}^T\mathbf{X}^T\rvy-\rvy^T\mathbf{X}\boldsymbol{\beta}+\boldsymbol{\beta}^T\mathbf{X}^T\mathbf{X}\boldsymbol{\beta}
\end{align*}
To find the $\boldsymbol{\beta}$ that minimizes the objective function, we will compute a derivative of the function while setting it equal to zero:
\begin{align*}
	\frac{\partial J}{\partial \boldsymbol{\beta}}&= -\mathbf{X}^T\rvy-\mathbf{X}^T\rvy+\mathbf{X}^T\mathbf{X}\boldsymbol{\beta}+\mathbf{X}^T\mathbf{X}\boldsymbol{\beta} = 0\\
	\boldsymbol{\beta}	&= (\mathbf{X}^T\mathbf{X})^{-1}\mathbf{X}^T\rvy
\end{align*}
\begin{lstlisting}[language=Python]
import numpy as np
import matplotlib.pyplot as plt

N = 50
x = np.random.randn(N)
w_1 = 3.4 # True Parameter
w_0 = 0.9 # True Parameter
y = w_1*x + w_0 + 0.3*np.random.randn(N) # Synthesize training data

X = np.column_stack((x, np.ones(N)))
W = np.array([w_1, w_0])

# From Scratch
XtX    = np.dot(X.T, X)
XtXinvX = np.dot(np.linalg.inv(XtX), X.T) # d x m
W_best = np.dot(XtXinvX, y.T)
print(f"W_best: {W_best}") 

# Pythonic Approach
theta = np.linalg.lstsq(X, y, rcond=None)[0]
print(f"Theta: {theta}") 

t = np.linspace(0, 1, 200)
y_pred = W_best[0]*t+W_best[1]
yhat = theta[0]*t+theta[1]
plt.plot(x, y, 'o')
plt.plot(t, y_pred, 'r', linewidth=4)
plt.show()
\end{lstlisting}

\subsection{Ridge Regression}
With the ridge regression principle, we can optimize it as follows:
\begin{align}
	J(\boldsymbol{\beta}) &= \|\mathbf{y}-\mathbf{X}\boldsymbol{\beta}\|^2_2 + \lambda \|\boldsymbol{\beta}\|^2_2 \\
			&= (\mathbf{y}-\mathbf{X}\boldsymbol{\beta})^T(\mathbf{y}-\mathbf{X}\boldsymbol{\beta})+\lambda\boldsymbol{\beta}^T\boldsymbol{\beta}\\
			&= (\mathbf{y}^T-\boldsymbol{\beta}^T\mathbf{X}^T)(\mathbf{y}-\mathbf{X}\boldsymbol{\beta})+\lambda\boldsymbol{\beta}^T\boldsymbol{\beta}\\
			&= \rvy^T\rvy-\boldsymbol{\beta}^T\mathbf{X}^T\rvy-\rvy^T\mathbf{X}\boldsymbol{\beta}+\boldsymbol{\beta}^T\mathbf{X}^T\mathbf{X}\boldsymbol{\beta}+\boldsymbol{\beta}^T\lambda\mathbf{I}\boldsymbol{\beta}\\
	\frac{\partial J}{\partial \boldsymbol{\beta}}&= -\mathbf{X}^T\rvy-\mathbf{X}^T\rvy+\mathbf{X}^T\mathbf{X}\boldsymbol{\beta}+\mathbf{X}^T\mathbf{X}\boldsymbol{\beta}+2\lambda\mathbf{I}\boldsymbol{\beta} = 0\\
	\boldsymbol{\beta}	&= (\mathbf{X}^T\mathbf{X}+\lambda\mathbf{I})^{-1}\mathbf{X}^T\rvy
	\label{eq:ridge_regression}
\end{align}

\subsection{Weighted LSE}
The OLEs assume an equal confidence on all the measurements. Now we look at varying confidence in the measurements. We assume that the noise for each measurement has zero mean and is independent, then the covariance matrix for all measurement noise is given by
\begin{align*}
	R &= \mathbb{E}(\eta\eta^T)\\
	  &= \begin{bmatrix}
		  \sigma_1^2 & \dots & 0\\
		  \vdots & \ddots & \vdots\\
		  0 & \ddots & \sigma_l^2\\
	  \end{bmatrix}
\end{align*}
By denoting the error vector $\rvy-\mathbf{X}\boldsymbol{\beta}$ as $\boldsymbol{\epsilon} = (\epsilon_1, \dots, \epsilon_l)^T$, we will minimize the sum of squared differences weighted over the variations of the measurements:
$$J(\tilda{\rvx}) = \boldsymbol{\epsilon}^TR^{-1}\boldsymbol{\epsilon}=\frac{\boldsymbol{\epsilon}_1^2}{\sigma_1^2}+\dots+\frac{\boldsymbol{\epsilon}_l^2}{\sigma_l^2}$$
% This is equivalent to
% $$\frac{1}{n}\sum_{i=1}^{n}\sum_{i=1}^n \alpha_i (y_i-\rvx_i\boldsymbol{\beta}).$$

The best estimate of the parameter is given by
$$\boldsymbol{\beta} = (\mathbf{X}^TR^{-1}\mathbf{X})^{-1}\mathbf{X}^TR^{-1}\rvy.$$
Note that the measurement noise matrix $R$ must be non-singular for a solution to exist.

\section{Recursive Least Squares}
\label{sec:recursive_least_square}

\begin{itemize}
	\item $H==X$
	\item $x==\beta$
\end{itemize}

The ordinary least-squares assumes that all measurements are available at a certain time. However, this often might not be the case in practice. More often, we obtain measurements sequentially and want to update our estimate with each new measurement. In this case, the matrix $H$ needs to be augmented. This update can be very expensive. When then number of measurements is extremely large, the solutions of the least squares problem are difficult to compute. 

These motivate the RLS. Suppose we have an estimate $\tilda{\rvx}_{k-1}$ after $k-1$ measurements, and obtain a new measurement $\rvy_k$. To be general, every measurements is now an $m$-vector with values yielded by, say, several measuring instruments. 

A linear recursive estimator can be written in the following form:
\begin{align*}
	\rvy_k &= H_k\rvx+\eta_k\\
	\tilda{\rvx}_k &= \tilda{\rvx}_{k-1}+K_k (\rvy_k-H_k\tilda{\rvx}_{k-1})
\end{align*}
Here $H_k$ is an $m\times n$ matrix, and $K_k$ is $n\times m$ and referred to as the \textit{estimator gain matrix}. We refer to $\rvy_k-H_k\tilda{\rvx}_{k-1}$ as the \textit{correction term}. Namely, the new estimate is modified from the previous estimate $\tilda{\rvx}_{k-1}$ with a correction via the gain vector. 

The current estimation error is

\begin{align*}
	\boldsymbol{\epsilon}_k	&= \rvx-\tilda{\rvx}_k \\
							&= \rvx-\tilda{\rvx}_{k-1} - K_k (\rvy_k-H_k\tilda{\rvx}_{k-1})\\
							&= \boldsymbol{\epsilon}_{k-1}-K_k (H_k\rvx+\eta_k-H_k\tilda{\rvx}_{k-1})\\
							&= \boldsymbol{\epsilon}_{k-1}-K_k H_k(\rvx-\tilda{\rvx}_{k-1})-K_k\eta_k\\
							&= (I-K_k H_k)\boldsymbol{\epsilon}_{k-1}-K_k\eta_k,
\end{align*}
where $I$ is the $n\times n$ identity matrix. The mean of this error is then
$$\mathbb{E}[\boldsymbol{\epsilon}_{k}] = (I-K_k H_k)\mathbb{E}[\boldsymbol{\epsilon}_{k-1}]-K_k\mathbb{E}[\boldsymbol{\eta}_{k}]$$
If $\mathbb{E}[\boldsymbol{\eta}_{k}]=0$ and $\mathbb{E}[\boldsymbol{\epsilon}_{k-1}]=0$, then $\mathbb{E}[\boldsymbol{\epsilon}_{k}]=0$. So if the measurement noise has zero mean for all $k$, and the initial estimate of $\rvx$ is set equal to its expected value, then $\tilda{\rvx}_k=\rvx_k, \forall k$. With this property, the estimator $\tilda{\rvx}_k &= \tilda{\rvx}_{k-1}+K_k (\rvy_k-H_k\tilda{\rvx}_{k-1})$ is \textit{unbiased}. The property holds regardless of the value of the gain vector $K_k$. This means the estimate will be equal to the true value $\rvx$ on average. 

The key is to determine the optimal value of the gain vector $K_k$. The optimality criterion used by us is to minimize the aggregated variance of the estimation errors at time $k$: 
\begin{align*}
	J_k &= \mathbb{E}[||\rvx-\tilda{\rvx}_k||^2]\\
		&= \mathbb{E}[\boldsymbol{\epsilon}_{k}^T\boldsymbol{\epsilon}_{k}]\\
		&= \mathbb{E}[tr(\boldsymbol{\epsilon}_{k}\boldsymbol{\epsilon}_{k}^T)]\\
		&= tr(P_k),
\end{align*}
where $P_k=\mathbb{E}[\boldsymbol{\epsilon}_{k}\boldsymbol{\epsilon}_{k}^T]$ is the estimation-error covariance and the third line is done by the trace of a product (or cyclic property). The expectation in the third line can go into the trace operator by its linearity. Next, we can obtain $P_k$ by
\begin{align*}
	P_k &= \mathbb{E}\bigg[\big((I-K_k H_k)\boldsymbol{\epsilon}_{k-1}-K_k\eta_k\big)\big((I-K_k H_k)\boldsymbol{\epsilon}_{k-1}-K_k\eta_k\big)^T\bigg]
\end{align*}
By rearranging the above equation with the property that the mean of noise is zero, we can get
$P_k = (I-K_k H_k)P_{k-1}(I-K_k H_k)^T+K_kR_kK_k^T.$

This equation is the recurrence for the covariance of the least squares estimation error. It is consistent with the intuition that as the measurement noise $R_k$ increases, the uncertainty $P_k$ increases. 

Next, we have to compute $K_k$ that minimizes the cost function given by error equation. 








\chapter{Recursive Least Squares}
\section{Recursive Least Squares}
\label{sec:recursive_least_square}

The ordinary least-squares assumes that all measurements are available at a certain time. However, this often might not be the case in practice. \textbf{More often, we obtain measurements sequentially and want to update our estimate with each new measurement.} In this case, the matrix $\mathbf{X}$ needs to be augmented. This update can be very expensive especially if the number of measurements is extremely large, the solutions of the least squares problem become prohibitive to compute. 

This motivates the Recursive Least Squares (RLS). Suppose we have an estimate $\boldsymbol{\theta}_{k-1}$ after $(k-1)$ measurements and obtain a new measurement $\rvy_k$. How can we update our estimate without completely reworking on the pseudo-inverse problem?

A linear recursive estimator can be expressed in the following form:
\begin{align*}
	\rvy_k &= \mathbf{X}_k\boldsymbol{\theta}+\boldsymbol{\eta}_k\\
	\boldsymbol{\theta}_k &= \boldsymbol{\theta}_{k-1}+K_k (\rvy_k-\mathbf{X}_k\boldsymbol{\theta}_{k-1})
\end{align*}
Here, $\mathbf{X}_k$ is an $m\times d$ matrix (observations) and $K_k$ is $d\times m$ and referred to as the \textit{estimator gain matrix}. We refer to $(\rvy_k-\mathbf{X}_k\boldsymbol{\theta}_{k-1})$ as the \textit{correction term}. Also, $\boldsymbol{\eta}_k$ is the measurement error. The new estimate is modified from the previous estimate $\boldsymbol{\theta}_{k-1}$ with a correction via the gain matrix. 

Intuitively, we can notice that we have to compute the optimal gain matrix to update our estimate. To this end, we have to set an \textit{estimation error}, which is our learning objective. The error can be expressed as follows: 
\begin{align*}
	\boldsymbol{\epsilon}_k	&= \boldsymbol{\theta}-\boldsymbol{\theta}_k \\
							&= \boldsymbol{\theta}-\boldsymbol{\theta}_{k-1} - K_k (\rvy_k-\mathbf{X}_k\boldsymbol{\theta}_{k-1})\\
							&= \boldsymbol{\epsilon}_{k-1}-K_k (\mathbf{X}_k\boldsymbol{\theta}+\boldsymbol{\eta}_k-\mathbf{X}_k\boldsymbol{\theta}_{k-1})\\
							&= \boldsymbol{\epsilon}_{k-1}-K_k \mathbf{X}_k(\boldsymbol{\theta}-\boldsymbol{\theta}_{k-1})-K_k\boldsymbol{\eta}_k\\
							&= (I-K_k \mathbf{X}_k)\boldsymbol{\epsilon}_{k-1}-K_k\boldsymbol{\eta}_k,
\end{align*}
where $I$ is the $d\times d$ identity matrix. The mean of this error is then
$$\mathbb{E}[\boldsymbol{\epsilon}_{k}] = (I-K_k \mathbf{X}_k)\mathbb{E}[\boldsymbol{\epsilon}_{k-1}]-K_k\mathbb{E}[\boldsymbol{\eta}_{k}]$$
If $\mathbb{E}[\boldsymbol{\eta}_{k}]=0$ and $\mathbb{E}[\boldsymbol{\epsilon}_{k-1}]=0$, then $\mathbb{E}[\boldsymbol{\epsilon}_{k}]=0$. So if the measurement noise has zero mean for all $k$, and the initial estimate of $\boldsymbol{\theta}$ is set equal to its expected value, then $\boldsymbol{\theta}_k=\boldsymbol{\theta}_k, \forall k$. This property tells us that the estimator $\boldsymbol{\theta}_k = \boldsymbol{\theta}_{k-1}+K_k (\rvy_k-\mathbf{X}_k\boldsymbol{\theta}_{k-1})$ is \textit{unbiased}. This property holds regardless of the value of the gain vector $K_k$. This means the estimate will be equal to the true value $\boldsymbol{\theta}$ on average. 

The key is to \textbf{determine the optimal value of the gain vector} $K_k$. The optimality criterion is to \textbf{minimize the aggregated variance of the estimation errors at time} $k$: 
\begin{align*}
	J_k &= \mathbb{E}[\|\boldsymbol{\theta}-\boldsymbol{\theta}_k\|^2]\\
		&= \mathbb{E}[\boldsymbol{\epsilon}_{k}^T\boldsymbol{\epsilon}_{k}]\\
		&= \mathbb{E}[tr(\boldsymbol{\epsilon}_{k}\boldsymbol{\epsilon}_{k}^T)]\\
		&= tr(P_k),
\end{align*}
where $P_k=\mathbb{E}[\boldsymbol{\epsilon}_{k}\boldsymbol{\epsilon}_{k}^T]$ is \textit{the estimation-error covariance} (\ie \textbf{covariance matrix}). Note that the third line holds by the trace of a product (\ie \textit{cyclic property}) and the expectation in the third line can go into the trace operator by its linearity. Next, we can obtain $P_k$ by
\begin{align*}
	P_k &= \mathbb{E}\bigg[\big((I-K_k \mathbf{X}_k)\boldsymbol{\epsilon}_{k-1}-K_k\boldsymbol{\eta}_k\big)\big((I-K_k \mathbf{X}_k)\boldsymbol{\epsilon}_{k-1}-K_k\boldsymbol{\eta}_k\big)^T\bigg]
\end{align*}
By rearranging the above equation with the property that the mean of noise is zero, we can get
\begin{align}
	P_k = (I-K_k \mathbf{X}_k)P_{k-1}(I-K_k \mathbf{X}_k)^T+K_kR_kK_k^T,
	\label{eq:rls_estimation_cov}
\end{align}
where $R_k = \mathbb{E}[\boldsymbol{\eta}_k\boldsymbol{\eta}_k^T]$ as covariance of $\boldsymbol{\eta}_k$. This equation is the recurrence for the covariance of the least squares estimation error. It is consistent with the intuition that as the measurement noise $R_k$ increases, the uncertainty in our estimate also increases (\ie $P_k$ increases).  Note that $P_k$ should be positive definite since it is a covariance matrix.

Next, let's compute $K_k$ that minimizes the cost function given by the error equation. We are going to utilize the following property:
\begin{align*}
	\frac{\partial tr(CA^T)}{\partial A} &= C\\
	\frac{\partial tr(ACA^T)}{\partial A} &= AC + AC^T
\end{align*}
Next, we are going to take a derivative to the objective function:
\begin{align*}
	\frac{\partial J_k}{\partial K_k} &= \frac{\partial tr(P_k)}{\partial K_k} = \frac{\partial tr}{\partial K_k}\underbrace{(I-K_k \mathbf{X}_k)P_{k-1}(I-K_k \mathbf{X}_k)^T}_{=ACA^T}+\frac{\partial}{\partial K_k}tr\left(K_k R_k K_k^T\right)\\ 
									  &= \frac{\partial tr(ACA^T)}{\partial (I-K_k \mathbf{X}_k)}\frac{\partial (I-K_k \mathbf{X}_k)}{\partial K_k} +\frac{\partial}{\partial K_k}tr\left(K_k R_k K_k^T\right) \quad \text{by Chain Rule}\\
	&= \left((I-K_k \mathbf{X}_k)P_{k-1}+ (I-K_k \mathbf{X}_k)P_{k-1}^T\right)(-\mathbf{X}_k^T) + \frac{\partial}{\partial K_k}tr\left(K_k R_k K_k^T\right)\\
	&= 2(I-K_k \mathbf{X}_k)P_{k-1}(-\mathbf{X}_k^T) + \frac{\partial}{\partial K_k}tr\left(K_k R_k K_k^T\right)\quad \text{, since } P_{k-1} \text{ is symmetric.}\\
									  &= -2(I-K_k \mathbf{X}_k)P_{k-1}\mathbf{X}_k^T+2K_kR_k
\end{align*}
By setting the partial derivative to zero, we get
$$K_k = P_{k-1}\mathbf{X}_k^T(\mathbf{X}_kP_{k-1}\mathbf{X}_k^T+R_k)^{-1}.$$

\subsection{Alternative Form}
Sometimes it is useful to write the equations for $P_k$ and $K_k$ in alternate forms. Although these alternate forms are mathematically identical, they can be beneficial from a computational point of view. Let's first set $\mathbf{X}_kP_{k-1}\mathbf{X}_k^T+R_k = S_k$, then we get 
$$K_k = P_{k-1}\mathbf{X}_k^TS_k^{-1}.$$
By putting this into \Cref{eq:rls_estimation_cov},
\begin{align*}
	P_k &= (I-P_{k-1}\mathbf{X}_k^TS_k^{-1} \mathbf{X}_k)P_{k-1}(I-P_{k-1}\mathbf{X}_k^TS_k^{-1} \mathbf{X}_k)^T+P_{k-1}\mathbf{X}_k^TS_k^{-1} R_k S_k^{-1}\mathbf{X}_kP_{k-1}\\ 
		&\quad \vdots\\
		&= P_{k-1}-P_{k-1}\mathbf{X}_k^TS_k^{-1}\mathbf{X}_k^TP_{k-1}\\
		&= (I-K_k\mathbf{X}_k)P_{k-1}.
\end{align*}
Note that $P_k$ is symmetric (\cf $P_k=\boldsymbol{\epsilon}_{k}\boldsymbol{\epsilon}_{k}^T$), since it is a covariance matrix, and so is $S_k$.

We take the inverse of both sides of 
$$P_{k-1}^{-1} = \bigg(\underbrace{P_{k-1}}_{A}-\underbrace{P_{k-1}\mathbf{X}_k^T}_{B}\big(\underbrace{\mathbf{X}_kP_{k-1}\mathbf{X}_k^T}_{D}\big)^{-1}\underbrace{\mathbf{X}_kP_{k-1}}_{C}\bigg)^{-1}.$$
Next, we apply the matrix inversion lemma which is known as \textit{Sherman-Morrison-Woodbury matrix identity} (or \textit{matrix inversion lemma}) identity: 
$$(A-BD^{-1}C)^{-1} = A^{-1}+A^{-1}B(D-CA^{-1}B)^{-1}CA^{-1}.$$
Then, rewrite $P_k^{-1}$ as follows:
\begin{align*}
	P_k^{-1} &= P_{k-1}^{-1}+P_{k-1}^{-1}P_{k-1}\mathbf{X}_k^T\big((\mathbf{X}_kP_{k-1}\mathbf{X}_k^T+R_k)-\mathbf{X}_kP_{k-1}P_{k-1}^{-1}(P_{k-1}\mathbf{X}_k^T)\big)^{-1}\mathbf{X}_kP_{k-1}P_{k-1}^{-1}\\ 
			 &= P_{k-1}^{-1}+\mathbf{X}_k^TR_{k}^{-1}\mathbf{X}_k
\end{align*}
This yields an alternative expression for the covariance matrix:
\begin{align*}
	P_k = \big(P_{k-1}^{-1}+\mathbf{X}_k^TR_{k}^{-1}\mathbf{X}_k\big)^{-1}
\end{align*}
We can also obtain
\begin{align*}
	K_k = P_{k}\mathbf{X}_k^TR_{k}^{-1}
\end{align*}
By
\begin{align*}
	P_k &= (I-K_k\mathbf{X}_k)P_{k-1}\\
	P_kP_{k-1}^{-1} &= (I-K_k\mathbf{X}_k)\\
	P_kP_k^{-1} &= P_kP_{k-1}^{-1}+P_k\mathbf{X}_k^TR_{k}^{-1}\mathbf{X}_k=I\\
	I &= (I-K_k\mathbf{X}_k)+P_k\mathbf{X}_k^TR_{k}^{-1}\mathbf{X}_k\\
	K_k &= P_{k}\mathbf{X}_k^TR_{k}^{-1}.
\end{align*}

\subsection{Summary of RLS}
In sum, RLS can be updated as follows: 
\begin{itemize}
	\item Update the gain matrix: 
		\begin{itemize}
			\item $K_k = P_{k-1}\mathbf{X}_k^T(\mathbf{X}_kP_{k-1}\mathbf{X}_k^T+R_k)^{-1}$ or
			\item $K_k = P_{k}\mathbf{X}_k^TR_{k}^{-1}$
		\end{itemize}
	\item Update estimate: $\boldsymbol{\theta}_k = \boldsymbol{\theta}_{k-1}+K_k (\rvy_k-\mathbf{X}_k\boldsymbol{\theta}_{k-1})$
	\item Update error covariance matrix by either: 
		\begin{itemize}
			\item $P_k = (I-K_k\mathbf{X}_k)P_{k-1}$.
			\item $P_k = (I-K_k \mathbf{X}_k)P_{k-1}(I-K_k \mathbf{X}_k)^T+K_kR_kK_k^T,$
		\end{itemize}
\end{itemize}

\paragraph{Example: }
At sample time $k$, our measurement is
\begin{itemize}
	\item $y_k = X_k\theta+\eta_k$
	\item $X_k = 1$
	\item $R_k = \mathbb{E}[\eta_k^2]$
\end{itemize}
For this scalar problem, the measurement matrix $X_k$ is a scalar too, and the measurement noise covariance $R_k$ is also a scalar. We will suppose that each measurement has the same covariance so the measurement covariance $R_k$ is not a function of $k$, and can be written as $R$. Initially, before we have any measurements, we have some idea about the value of the $\theta$, and this forms our initial estimate. We also have some uncertainty about our initial estimate, and this forms our initial covariance:
\begin{align*}
	\hat{\theta}_0 &= \mathbb{E}[\theta]\\
	P_0 &= \mathbb{E}[(\theta-\hat{\theta}_0)(\theta-\hat{\theta}_0)^T]\\
		&= \mathbb{E}[(\theta-\hat{\theta}_0)^2]
\end{align*}
If we have absolutely no idea about $\theta$, then $P(0)=\infty I$. If we are 100\% certain about the $\theta$ before taking any measurements, then $P(0)=0$. Let's compute the gain matrix at $k=1$ by using the following equation:
$$K_k = P_{k-1}\mathbf{X}_k^T(\mathbf{X}_kP_{k-1}\mathbf{X}_k^T+R_k)^{-1}.$$
Then, we get
$$K_1 = P_{0}(P_{0}+R)^{-1}.$$
Similarly, by
$$\boldsymbol{\theta}_k = \boldsymbol{\theta}_{k-1}+K_k (\rvy_k-\mathbf{X}_k\boldsymbol{\theta}_{k-1}),$$
we obtain
$$\hat{\theta}_1 = \hat{\theta}_{0}+\frac{P_{0}}{P_{0}+R} (y_1-\hat{\theta}_{0}).$$
Finally, let's update our covariance matrix $P_k$ by 
$$P_k = (I-K_k \mathbf{X}_k)P_{k-1}(I-K_k \mathbf{X}_k)^T+K_kR_kK_k^T.$$
Then, 
\begin{align*}
	P_1 &= \left(I-\frac{P_{0}}{P_{0}+R}\right)P_{0}I-\frac{P_{0}}{P_{0}+R}+\frac{P_{0}}{P_{0}+R}R\frac{P_{0}}{P_{0}+R}\\
		&= \left(\frac{P_0R^2}{(P_{0}+R)^2}\right)+\frac{P_{0}^2R}{(P_{0}+R)^2}\\
		&= \frac{P_{0}R(P_0+R)}{(P_{0}+R)^2}\\
		&= \frac{P_{0}R}{P_{0}+R}
\end{align*}
By repeating these calculations, we can update the above parameters and find general expressions: 
\begin{align*}
	P_{k-1}&= \frac{P_0R}{(k-1)P_0+R}\\
	K_{k}&= \frac{P_0}{kP_0+R}\\
	\hat{\theta}_{k}&= \frac{(k-1)P_0+R}{kP_0+R}\hat{\theta}_{k-1}+\frac{P_0}{kP_0+R}y_k
\end{align*}
Note that if $\theta$ is known perfectly \textit{a priori} (\ie $\theta$ is known perfectly before any measurements are obtained) then $P_0 =0$ and the above equation show that $K_k=0$ and $\hat{\theta} = \hat{\theta}_0$. That is, the optimal estimate of $\theta$ is independent of any measurements that are obtained. In sum, this indicates that no update from measurements is needed, as the estimate is already perfect.

On the other hand, if $x$ is completely unknown a priori, then $P_0\to \infty$, and the above equations simplify to
\begin{align*}
\hat{\theta}_k &= \frac{(k-1)P_0}{kP_0} \hat{\theta}_{k-1} + \frac{P_0}{kP_0} y_k \\
         &= \frac{k-1}{k} \hat{\theta}_{k-1} + \frac{1}{k} y_k \\
         &= \frac{1}{k} [(k-1)\hat{\theta}_{k-1} + y_k]
\end{align*}
In other words, the optimal estimate of \(\theta\) is equal to the running average of the measurements \(y_k\), which can be written as
\begin{align*}
\bar{y}_k &= \frac{1}{k} \sum_{j=1}^k y_j \\
          &= \frac{1}{k} \left(\sum_{j=1}^{k-1} y_j + y_k\right) \\
          &= \frac{1}{k} \left[(k-1) \frac{1}{k-1} \sum_{j=1}^{k-1} y_j + y_k\right] \\
          &= \frac{1}{k} [(k-1) \bar{y}_{k-1} + y_k]
\end{align*}


\subsection{Curve Fitting} 
In the recursive curve fitting problem, we measure data one sample at a time $(y_1, y_2 \dots, )$ and want to find the best fit of a curve to the data. The curve that we want to fit to the data could be constrained to be linear or quadratic and so on. 

\paragraph{Example:} Suppose that we want to fit a straight line to a set of data points. The linear data fitting problem can be written as 
\begin{align*}
	y_k &= \theta_1+\theta_2t_k+\eta_k\\
	\mathbb{E}[\eta_k] &= R_k
\end{align*}
$t_k$ is the independent variable, $y_k$ is the noisy data, and we want to find the linear relationship between $y_k$ and $t_k$. In sum, we want to estimate the constants $\theta_1$ and $\theta_2$. The measurement matrix can be written as 
\begin{align*}
	\mathbf{X}_k = \begin{bmatrix}
		1 & t_k
	\end{bmatrix}.
\end{align*}
Then, 
$$\rvy_k = \mathbf{X}_k\boldsymbol{\theta}+\boldsymbol{\eta}_k.$$


\subsection{Python Implementation}

\begin{lstlisting}[language=Python]
class RecursiveLeastSquares(object):
    
    # theta0 - initial estimate used to initialize the estimator
    # P0 - initial estimation error covariance matrix
    # R  - covariance matrix of the measurement noise
    def __init__(self,theta0,P0,R)
        
        # initialize the values
        self.theta0=theta0
        self.P0=P0
        self.R=R
        
        # this variable is used to track the current time step k of the estimator 
        # after every time step arrives, this variables increases for one 
        # in this way, we can track the number of variblaes
        self.curr_step=0
                  
        # this list is used to store the estimates xk starting from the initial estimate 
        self.estimates=[]
        self.estimates.append(theta0)
         
        # this list is used to store the estimation error covariance matrices Pk
        self.est_error_cov=[]
        self.est_error_cov.append(P0)
        
        # this list is used to store the gain matrices Kk
        self.gainMatrices=[]
         
        # this list is used to store estimation error vectors
        self.errors=[]
    
     
    # this function takes the current measurement and the current measurement matrix X
    # and computes the estimation error covariance matrix, updates the estimate, 
    # computes the gain matrix, and the estimation error
    # it fills the lists self.estimates, self.est_error_cov, self.gainMatrices, and self.errors
    # it also increments the variable curr_step for 1
    
    # measurementValue (theta) - measurement obtained at the time instant k
    # X - measurement matrix at the time instant k
    
    def predict(self,measurementValue,X):
        import numpy as np
        
        # Compute the L matrix and its inverse 
        # K_k = P_{k-1}X_k^T(R_k+X_kP_{k-1}X_k^T)^{-1}
        Lmatrix=self.R+np.matmul(X,np.matmul(self.est_error_cov[self.curr_step],X.T))
        LmatrixInv=np.linalg.inv(Lmatrix)
        # Compute the gain matrix
        gainMatrix=np.matmul(self.est_error_cov[self.curr_step],np.matmul(X.T,LmatrixInv))

        # Compute the estimation error                    
        # \theta_k = \theta_{k-1}+K_k (y_k-X_k\theta_{k-1})
        error=measurementValue-np.matmul(X,self.estimates[self.curr_step])
        # Compute the estimate
        estimate=self.estimates[self.curr_step]+np.matmul(gainMatrix,error)
        
        # Propagate the estimation error covariance matrix
        # P_k = (I-K_k X_k)P_{k-1}(I-K_k X_k)^T+K_kR_kK_k^T
        ImKc=np.eye(np.size(self.theta0),np.size(self.theta0))-np.matmul(gainMatrix,X)
        error_cov=np.matmul(ImKc,self.est_error_cov[self.curr_step])
        
        # add computed elements to the list
        self.estimates.append(estimate)
        self.est_error_cov.append(error_cov)
        self.gainMatrices.append(gainMatrix)
        self.errors.append(error)
        
        # increment the current time step
        self.curr_step=self.curr_step+1
\end{lstlisting}


\section{Alternate Derivation of RLS}
Suppose the training samples arrive one by one in the following sequence $\rvx_1, \dots, \rvx_m, \rvx_{m+1}$, where $\rvx_{m+1}$ denotes the newly arrived sample vector. These samples can be projected onto the feature space by linear projection and expressed into a matrix $\mathbf{P}^T \in \mathbb{R}^{(d+1)\times (m+1)}$ as follows:
\begin{align*}
	\mathbf{P}^T = [\rvp(\rvx_1), \dots, \rvp(\rvx_{m+1})],
\end{align*}
where $\rvp(\cdot)\in \mathbb{R}^{d+1}$. Subsequently, let 
\begin{align*}
	\mathbf{R}_{m+1} &= \mathbf{P}^T \mathbf{P} + b\mathbf{I}\\
	\mathbf{Q}_{m+1} &= \mathbf{P}^T \mathbf{y}.
\end{align*}
By separating the covariance of the newly arrived sample \( p(\rvx_{m+1}) \) from the remaining stack, we can write:
\begin{align*}
	\mathbf{P}^T \mathbf{P} &= \sum_{i=1}^{m+1} \rvp(\rvx_i) \rvp(\rvx_i)^T\\
							&= \sum_{i=1}^m \rvp(\rvx_i) \rvp(\rvx_i)^T + \rvp(\rvx_{m+1}) \rvp(\rvx_{m+1})^T\\
							&= \mathbf{P}_m^T \mathbf{P}_m + \rvp(\rvx_{m+1}) \rvp(\rvx_{m+1})^T.
\end{align*}
Hence, 
\begin{align*}
	\mathbf{R}_{m+1} &= \mathbf{P}^T \mathbf{P} + b\mathbf{I}\\
					 &= \left(\mathbf{P}_m^T \mathbf{P}_m + \rvp(\rvx_{m+1}) \rvp(\rvx_{m+1})^T\right)+ b\mathbf{I}\\
					 &= \underbrace{\mathbf{P}_m^T \mathbf{P}_m + b\mathbf{I}}_{=\mathbf{R}_m} + \rvp(\rvx_{m+1})\rvp(\rvx_{m+1})^T\\
					 &= \mathbf{R}_m + \rvp(\rvx_{m+1}) \rvp(\rvx_{m+1})^T
\end{align*}
Similarly, 
\begin{align*}
	\mathbf{Q}_{m+1} &= \mathbf{Q}_m + \rvp(\rvx_{m+1}) y_{m+1}
\end{align*}
If the system is designed to forget the old training samples (\ie weighted averaging), 
\begin{align*}
	\mathbf{R}_{m+1} &= (1 - \lambda) \mathbf{R}_m + \lambda \rvp(\rvx_{m+1}) \rvp(\rvx_{m+1})^T,\\
	\mathbf{Q}_{m+1} &= (1 - \lambda) \mathbf{Q}_m + \lambda \rvp(\rvx_{m+1}) y_{m+1},
\end{align*}
where \( \lambda \in (0, 1) \) is often called a \textit{forgetting factor}.

Let \( \mathbf{A} = \mathbf{R}_m, \mathbf{B} = p(\rvx_{m+1}), \mathbf{C} = 1 \) (scalar), \( \mathbf{D} = p(\rvx_{m+1})^T = \mathbf{p}^T \), then based on the matrix inversion lemma (Woodbury, 1950; Sherman and Morrison, 1950),
\begin{align*}
	(\mathbf{A} + \mathbf{BCD})^{-1} = \mathbf{A}^{-1} - \mathbf{A}^{-1} \mathbf{B} (\mathbf{C}^{-1} + \mathbf{DA}^{-1} \mathbf{B})^{-1} \mathbf{DA}^{-1},
\end{align*}
we have
\begin{align*}
	\mathbf{R}_{m+1}^{-1} &= \left[ (1 - \lambda) \mathbf{R}_m + \lambda \mathbf{p}\rvp^T \right]^{-1}\\
						  &= \frac{1}{1 - \lambda} \mathbf{R}_m^{-1} - \frac{1}{1 - \lambda} \mathbf{R}_m^{-1}\lambda \mathbf{p} \left( \mathbf{I} + \mathbf{p}^T \frac{\lambda}{1-\lambda} \mathbf{R}_m^{-1} \mathbf{p} \right)^{-1} \mathbf{p}^T \frac{1}{1-\lambda} \mathbf{R}_m^{-1}\\
						  &= \frac{1}{1 - \lambda} \mathbf{R}_m^{-1} - \frac{1}{(1 - \lambda)^2} \mathbf{R}_m^{-1} \mathbf{pp}^T \mathbf{R}_m^{-1} \left( \frac{1}{\lambda} + \frac{1}{1-\lambda} \mathbf{p}^T \mathbf{R}_m^{-1} \mathbf{p} \right)^{-1}.
\end{align*}
We can obtain
\begin{align*}
	\rvw_{m+1} = \left( \mathbf{P}^T \mathbf{P} + b\mathbf{I} \right)^{-1} \mathbf{P}^T \mathbf{y} = \mathbf{R}_{m+1}^{-1} \mathbf{Q}_{m+1},
\end{align*}
Substitute \(\mathbf{R}_{m+1}^{-1}\) and \(\mathbf{Q}_{m+1} = (1 - \lambda) \mathbf{Q}_m +  \lambda\mathbf{p}(\rvx_{m+1}) y_{m+1}\):
\begin{align*}
	\mathbf{w}_{m+1} &= \left[ \frac{1}{1 - \lambda} \mathbf{R}_m^{-1} - \frac{1}{(1 - \lambda)^2} \mathbf{R}_m^{-1} \mathbf{p}\mathbf{p}^T \mathbf{R}_m^{-1} \left( \frac{1}{\lambda} + \frac{1}{1 - \lambda} \mathbf{p}^T \mathbf{R}_m^{-1} \mathbf{p} \right)^{-1} \right] \left[ (1 - \lambda) \mathbf{Q}_m + \lambda y_{m+1} \mathbf{p} \right]\\
					 &= \underbrace{\mathbf{R}_m^{-1} \mathbf{Q}_m}_{=w_m} + \frac{1}{1 - \lambda}\mathbf{R}_m^{-1} \mathbf{p}\mathbf{p}^T \mathbf{R}_m^{-1} \left( \frac{1}{\lambda} + \frac{1}{1 - \lambda} \mathbf{p}^T \mathbf{R}_m^{-1} \mathbf{p} \right)^{-1} \mathbf{Q}_m + \frac{\lambda}{1 - \lambda} \mathbf{R}_m^{-1} \mathbf{p} y_{m+1} \\
					 &\quad - \frac{\lambda}{(1 - \lambda)^2}\mathbf{R}_m^{-1} \mathbf{p}\mathbf{p}^T \mathbf{R}_m^{-1} \left( \frac{1}{\lambda} + \frac{1}{1 - \lambda} \mathbf{p}^T \mathbf{R}_m^{-1} \mathbf{p} \right)^{-1}\mathbf{p} y_{m+1}\\
\end{align*}
Let 
\begin{align*}
	A &= \left( \frac{1}{\lambda} + \frac{1}{1 - \lambda} \mathbf{p}^T \mathbf{R}_m^{-1} \mathbf{p} \right)^{-1} \\
	  &= \frac{\lambda(1-\lambda)}{\lambda \mathbf{p}^T \mathbf{R}_m^{-1} \mathbf{p}+(1-\lambda)},
\end{align*}
which is a constant. Then 
\begin{align*}
	w_{m+1} &= w_m - \frac{\mathbf{R}_m^{-1} \mathbf{p}}{(1-\lambda)^2} \cdot A \cdot \left((1-\lambda)\mathbf{p}^T w_m + \lambda \mathbf{p}^T \mathbf{R}_m^{-1}\mathbf{p} y_{m+1}\right)+ \frac{\lambda}{1 - \lambda} \mathbf{R}_m^{-1} \mathbf{p} y_{m+1}\\
			&= w_m - \frac{\lambda \mathbf{R}_m^{-1} \mathbf{p}}{(1-\lambda)} \cdot \frac{1}{\lambda \mathbf{p}^T \mathbf{R}_m^{-1} \mathbf{p}+(1-\lambda)}\left((1-\lambda)\mathbf{p}^T w_m + \lambda \mathbf{p}^T \mathbf{R}_m^{-1}\mathbf{p} y_{m+1}\right) + \frac{\lambda}{1 - \lambda} \mathbf{R}_m^{-1} \mathbf{p} y_{m+1}\\
			&= w_m - \frac{\lambda \mathbf{R}_m^{-1} \mathbf{p}}{(1-\lambda)} \cdot \frac{1}{\lambda \mathbf{p}^T \mathbf{R}_m^{-1} \mathbf{p}+(1-\lambda)}\left((1-\lambda)\mathbf{p}^T w_m + \lambda \mathbf{p}^T \mathbf{R}_m^{-1}\mathbf{p} y_{m+1}\right) \\ 
			&\quad +\frac{\lambda}{(1 - \lambda)}\cdot \frac{\lambda \mathbf{p}^T \mathbf{R}_m^{-1} \mathbf{p}+(1-\lambda)}{\lambda \mathbf{p}^T \mathbf{R}_m^{-1} \mathbf{p}+(1-\lambda)} \mathbf{R}_m^{-1} \mathbf{p} y_{m+1} \,\, \text{, since $\mathbf{p}^T \mathbf{R}_m^{-1} \mathbf{p}$ is a scalar, we can put $\mathbf{R}_m^{-1} \mathbf{p}$ to the right}\\
			&= w_m + \frac{\lambda}{\lambda \mathbf{p}^T \mathbf{R}_m^{-1} \mathbf{p}+(1-\lambda)}\cdot\frac{1}{(1-\lambda)}\big(-(1-\lambda)\mathbf{R}_m^{-1}\mathbf{p}\mathbf{p}^T w_m - \lambda \mathbf{R}_m^{-1}\mathbf{p}\mathbf{p}^T\mathbf{R}_m^{-1}\mathbf{p}y_{m+1}\\
			&\quad + \lambda \mathbf{R}_m^{-1}\mathbf{p}\mathbf{p}^T\mathbf{R}_m^{-1}\mathbf{p}y_{m+1} + (1-\lambda)\mathbf{R}_m^{-1}\mathbf{p}y_{m+1}\big)\\
			&= w_m + \frac{\lambda}{\lambda \mathbf{p}^T \mathbf{R}_m^{-1} \mathbf{p}+(1-\lambda)}\cdot\frac{1}{(1-\lambda)}\big(-(1-\lambda)\mathbf{R}_m^{-1}\mathbf{p}\mathbf{p}^T w_m + (1-\lambda)\mathbf{R}_m^{-1}\mathbf{p}y_{m+1}\big)
\end{align*}
Thus, the final recursive solution for the weight vector \(\mathbf{w}_{m+1}\) is given by
\[
\mathbf{w}_{m+1} = \mathbf{w}_m + \frac{\lambda \mathbf{R}_m^{-1} \mathbf{p} (y_{m+1} - \mathbf{p}^T \mathbf{w}_m)}{\lambda \mathbf{p}^T \mathbf{R}_m^{-1} \mathbf{p} + (1 - \lambda)}
\]
Here, we note that $\lambda$ controls the strength of update with respect to the accumulated solution with $\lambda \to 1$ having the strongest weight for newly arrived sample. When $\lambda = 0.5$ in , we have the following regularized recursive least squares solution:
\begin{align*}
	w_{m+1} = w_m + \frac{\mathbf{R}_m^{-1} \rvp (y_{m+1} - \rvp^T w_m)}{\rvp^T \mathbf{R}_m^{-1} \rvp + 1}. 
\end{align*}




\chapter{Logistic Regression}
\section{Logistic Regression}
\label{sec:logistic_regression}

Logistic regression corresponds to the following binary classification model parameterized by $\rvw$:
$$p(y|\mathbf{x},\mathbf{w})=\textrm{Ber}(y|\sigma(\mathbf{w}^T\mathbf{x}))$$

Logistic regression models \textit{logit}s (log odds) through a linear model. For binary data, the goal is to model the probability $p$ that one of two outcomes occurs. Recall that an ordinary linear regression model is not bounded. Thus, we will pass a linear model through a sigmoid function, which is also known as logistic function. 

$$\sigma(z) = \frac{1}{1+\exp^{z}},$$
where $z=wx+b.$
The sigmoid function has the property
$$\sigma(-x) = 1-\sigma(x).$$
The $z$ is often called the logit. Note that the inverse of the sigmoid is the log of the odds ratio $\frac{p}{1-p}.$

The logit function is $\textrm{log}\frac{p}{1-p}$, which varies between $-\infty$ and $+\infty$ as $p$ varies between $0$ and $1$.
$$\textrm{log}\frac{p}{1-p} = w_0x_0 +  w_1x_1 + ... + w_nx_n$$
Note that \textbf{the logistic regression model assumes that the log-odds (\textit{logit}) of an observation $y$ can be expressed as a linear function}. In this context, the logit function is called the \textbf{\textit{link function}} because it ``links'' the probability to the linear function of the predictor variables.

% Simplest solution to model a dependant variable $y$ is a linear regression. However, $y$ should be in a range of $[0,1]$. So we need to introduce the logit function. 

% The linear regression can be generalized to the classification setting with two changes:
% \begin{itemize}
% 	\item Replacing the Gaussian distribution for $y$ with a Bernoulli distribution: $p(y|\mathbf{x},\mathbf{w})=Ber(y|\mu(\mathbf{x}))$
% 	\item Squashing input data into sigmoid function $\sigma(\eta)$ that range from 0 to 1: $\sigma(\eta)\triangleq \frac{1}{1+exp(-\eta)}$.
% \end{itemize}
% $$p(y|\mathbf{x},\mathbf{w})=Ber(y|\sigma(\mathbf{w}^T\mathbf{x})),\$$

The negative log-likelihood for logistic regression is given by
\begin{align*}
	\textrm{NLL}(\mathbf{w}) &= -\ln \prod_{i=1}^N p(\rvx)^{\mathds{I}(y_i=1)}(1-p(\rvx))^{\mathds{I}(y_i=0)}\,\footnotemark\\
							 &= -\ln \prod_{i=1}^N \sigma(\mathbf{w}^T\rvx)^{\mathds{I}(y_i=1)}(1-\sigma(\mathbf{w}^T\rvx))^{\mathds{I}(y_i=0)}\\
							 &= -\sum_{i=1}^N y_i\ln\sigma(\mathbf{w}^T\rvx)+\ln(1-y_i)(1-\sigma(\mathbf{w}^T\rvx)).
							 % &= -\sum_{i=1}^{N}\textrm{log}[\mu_i^{\mathds{I}(y_i=1)}\times (1-\mu_i)^{\mathds{I}(y_i=0)}]\\
	% &=-\sum_{i=1}^{N}[y_i\textrm{log}\mu_i + (1-y_i) \textrm{log}(1-\mu_i)], \textrm{ }
\end{align*}
This is also called \textbf{cross-entropy} error function. 
\footnotetext{$\mathds{I}(y_i=1) = y_i$, because $y_i\in \{0, 1\}$ is a binary variable} 

To compute the derivative of NLL, we first need to know the following tricks:
\begin{itemize}
	\item The derivative of $\ln (x)$:
$$\frac{\partial }{\partial x}\ln (x) = \frac{1}{x}.$$
\item The derivative of the sigmoid is given by:
$$\frac{\partial \sigma(z)}{\partial x} = \sigma(x)(1-\sigma(x)).$$
\item Finally, the chain rule of derivative. Suppose we are computing the derivative of a composite function $f(x) = u(v(x))$. The derivative of $f(x)$ is the derivative of $u(x)$ with respect to $v(x)$ times the derivative of $v(x)$ with respect to $x$.
$$\frac{\partial f}{\partial x} = \frac{\partial u}{\partial v} \frac{\partial v}{\partial x}$$
\end{itemize}
The derivative of the loss function w.r.t., a single weight $w_j$ is given by
\begin{align*}
	\frac{\partial \mathcal{L}}{\partial w_j} &= \frac{\partial }{\partial w_j} -[y\ln \sigma(wx + b)+(1-y) \ln (1-\sigma(wx+b))]\\
											  &=  -[\frac{\partial }{\partial w_j}y\ln \sigma(wx + b)+\frac{\partial }{\partial w_j}(1-y) \ln (1-\sigma(wx+b))]\\
											  &= -\frac{y}{\sigma(wx + b)}\frac{\partial }{\partial w_j}\sigma(wx + b) - \frac{1-y}{1-\sigma(wx+b)} \frac{\partial }{\partial w_j}1-\sigma(wx+b)\\
											  &= -\bigg[\frac{y}{\sigma(wx + b)}-\frac{1-y}{1-\sigma(wx + b)}\bigg]\frac{\partial }{\partial w_j}\sigma(wx + b)\\
											  &= -\bigg[\frac{y-\sigma(wx + b)}{\sigma(wx + b)[1-\sigma(wx + b)]}\bigg]\sigma(wx + b)[1-\sigma(wx + b)]\frac{\partial \sigma(wx + b)}{\partial w_j}\\
											  &= -\bigg[\frac{y-\sigma(wx + b)}{\sigma(wx + b)[1-\sigma(wx + b)]}\bigg]\sigma(wx + b)[1-\sigma(wx + b)]x_j\\ 
											  &= -( y-\sigma(wx + b) )x_j\\
											  &= ( \sigma(wx + b)-y )x_j.
\end{align*}





Another way to express \textrm{NLL} is as follows. Suppose $\hat{y}_i\in\{-1,+1\}$ instead of $y_i\in\{0,1\}$. We have $p(y=1)=\frac{1}{1+\mathrm{exp}(-\mathbf{w}^T\mathbf{x})}$ and $p(y=-1)=\frac{1}{1+\mathrm{exp}(+\mathbf{w}^T\mathbf{x})}$. Hence
\begin{align*}
	\textrm{NLL}(\mathbf{w}) &= -\frac{1}{N}\sum_{n=1}^N [\mathbb{I}(\hat{y}_n=1)\log(\sigma(a_n))+\mathbb{I}(\hat{y}_n=-1)\log(\sigma(-a_n))]\\
							 &= -\frac{1}{N}\sum_{n=1}^N \log(\sigma(\hat{y}_na_n))\\
							 &=  \frac{1}{N}\sum_{i=1}^{N}\textrm{log}(1+\mathrm{exp}(-\hat{y}_i\mathbf{w}^T\mathbf{x}_i).
\end{align*}
Note that the sigmoid is used for compressing the output into $[0,1]$ and $\sigma(-a_n) = 1-\sigma(a_n)$. Unlike the linear regression, there is no closed from solution for logistic regression, thus we need optimization algorithms for it. Typically, optimization process involves the gradient and Hessian. 
\begin{align*}
	\mathbf{g}&=\frac{d}{d\mathbf{w}}\mathrm{NLL}(\mathbf{w})=\frac{d}{d\mu_i}\mathrm{NLL}(\mathbf{w})\frac{d\mu_i}{d\mathbf{h}}\frac{d\mathbf{h}}{d\mathbf{w}}\\
	& = \sum_{i=1}\Bigg[-\frac{y_i}{\mu_i} + \frac{(1-y_i) }{(1-\mu_i)}\Bigg]\frac{d\mu_i}{d\mathbf{h}}\frac{d\mathbf{h}}{d\mathbf{w}}=\sum_{i=1}\Bigg[\frac{\mu_i-y_i }{\mu_i(1-\mu_i)}\Bigg]\frac{d\mu_i}{d\mathbf{h}}\frac{d\mathbf{h}}{d\mathbf{w}}\\
	&=\sum_{i}(\mu_i-y_i)\mathbf{x}_i=\mathbf{X}^T(\boldsymbol{\mu}-\mathbf{y})\\
	\frac{d\mu_i}{d\mathbf{h}}& = \mu_i(1-\mu_i)\\
	\frac{d\mathbf{h}}{d\mathbf{w}}& = \mathbf{x}_i
\end{align*}
where $\mathbf{h}=\mathbf{w}^T\mathbf{x}$. 

We can also use the second-order method. 
\begin{align*}
\mathbf{H}&=\frac{d}{d\mathbf{w}}g(\mathbf{w})^T=\sum_{i}(\nabla_{\mathbf{w}}\mu_i)\mathbf{x}_i^T=\sum_{i}\mu_i(1-\mu_i)\mathbf{x}_i\mathbf{x}_i^T\\
&=\mathbf{X}^T\mathbf{S}\mathbf{X},
\end{align*}
where $\mathbf{S}\triangleq \mathrm{diag}(\mu_i(1-\mu_i))$. Note that $\mathbf{H}$ is positive definite, because the \textrm{NLL} is convex and has a global minimum. 

\chapter{Bayesian Regression}
\chapter{Bayesian Regression}

Integrate over all $\theta$

\begin{align}
P(heads \mid D) =& \int_{\theta} P(heads, \theta \mid D) d\theta\\
 =& \int_{\theta} P(heads \mid \theta, D) P(\theta \mid D) d\theta \ \ \ \ \ \  \textrm{(Chain rule: $P(A,B|C)=P(A|B,C)P(B|C)$.)}\\ 
  =& \int_{\theta} \theta P(\theta \mid D) d\theta\\ 
  =&E\left[\theta|D\right]\\
 =&\frac{n_H + \alpha}{n_H + \alpha + n_T + \beta}
\end{align}


% \chapter{Introduction}
\section{Curve Fitting}
We can assume that a target variable has a Gaussian distribution with a mean equal to the value $y(x,\mathbf{w})$ of the polynomial curven given by
\begin{equation}
	p(t|x, \mathbf{w}, \beta) = \mathcal{N}(t|y(x,\mathbf{w}), \beta^{-1}),
	\label{eq:curve}
\end{equation}
\begin{itemize}
	\item $t$: target variable
		$$t = y(\rvx, \rvw) +\epsilon,$$
	where $\epsilon$ is a zero mean Gaussian noise with precision (inverse variance) $\beta$. 
	% \item $x$: input
	% \item $\beta$: an inverse variance of the distribution.
\end{itemize}

We not use the training data $\{\mathbf{x,y}\}$ to determine the values of the unknown parameters $\mathbf{w}$ and $\beta$ by maximum likelihood. If the data are assumed to be drawn independently from the distribution, then the likelihood function is given by 
\begin{equation}
	p(\mathbf{t}|\mathbf{x,w},\beta) = \prod_{n=1}^{N}\mathcal{N}(t_n|y(x_n,\mathbf{w}), \beta^{-1}).
	\label{eq:curve_ml}
\end{equation}

We can take a step towards a more Bayesian approach and introduce a prior distribution over the polynomial coefficients $\mathbf{w}$. For simplicity, we can use a Gaussian distribution from
\begin{equation}
	p(\mathbf{w}|\alpha) = \mathcal{N}(\mathbf{w|0},\alpha^{-1}\mathbf{I}),
	\label{eq:prior_hyper}
\end{equation}
where $\alpha$ is the precision of the distribution. Using Bayes' theorem, the posterior distribution for $\mathbf{w}$ is proportional to the product of the prior distribution and the likelihood function
\begin{equation}
	p(\mathbf{w|x,t},\alpha,\beta)\propto p(\mathbf{w|x,t},\beta)p(\mathbf{w},\alpha).
	\label{eq:bayes_reg}
\end{equation}
We can now determined $\mathbf{w}$ by finding the most probable value of $\mathbf{w}$ given the data, in other words by maximizing the posterior distribution, MAP (maximum posterior). Taking a negative logarithm, then we can find that the maximum of the posterior is given by the minimum of 
\begin{equation}
	\frac{\beta}{2}\sum_{n=1}^N \{y(x_n,\mathbf{w})-t_n\}^2+\frac{\alpha}{2}\mathbf{w}^T\mathbf{w}.
	\label{eq:bayes_}
\end{equation}
Thus we see that maximizing the posteiror distribution si equivalent to minimizing the regularized sum of squares error function encountered earler with a regularization parameter given by $\lambda = \alpha/\beta$.



\section{Bayesian Curve Fitting}


% \section{Adding Noise to Regression Predictors is Ridge Regression}
% In linear regression, we seek a vector $\hat{\beta}$ which solves the following optimization problem:
% \begin{equation}
% 	\hat{\beta} = \arg\min_{\beta}|y-X\beta|^2
% 	\label{eq:linear_regression}
% \end{equation}

% Ridge regression is

% \begin{equation}
% 	\hat{\beta} = \arg\min_{\beta}|y-X\beta|^2+\lambda|\beta|^2
% 	\label{eq:ridge_linear_regression}
% \end{equation}

% $$\varepsilon_1,\varepsilon_2,\cdots\sim \mathcal{N}(1,\sigma),$$
% We can add a multiplicitive random noise to $\mathbf{X}$.
% $$x_{ij}\to \varepsilon_{ij}x_{ij}$$

% $$\hat{\beta} \sim \arg\min_{\beta} E_{G}[|y-G\cdot X\beta|^2]$$
% where $G$ is a matrix of random Gaussian noise. 

% \begin{align*}
% 	\mathbb{E} \left[ \left| y - (G * X) \beta  \right|^2 \right] &= E \left[ y^t y - 2 y^t (G * X) \beta + \beta^t (G * X)^t (G * X) \right] \\
% &= y^t y - 2 y^t (E[G] * X) \beta + \beta^t E \left[ M \right] \beta \\
% &= y^t y - 2 y^t X \beta + \beta^t X^t X \beta + \beta^t diag(\sigma^2) X^t X \beta \\
% &= \left| y - X \beta \right|^2 + \beta^t diag(\sigma^2) X^t X \beta \\
% &= \left| y - X \beta \right|^2 + \sigma^2 \left| \Gamma \beta \right|^2
% \end{align*}
% where $M = (G\cdot X)^T(G\cdot X)$
% $$m_{ij} = \sum_k e_{ki}e_{kj}x_{ki}x_{kj}$$

	




\part{Kernel Methods}
\chapter{Kernel Methods}
\begin{itemize}
	\item The main idea is to use large set of fixed non-linear basis functions.
	\item The complexity depends on number of basis functions, but dual trick changes it to a size of dataset. 
\end{itemize}

Kernel function: Let $\phi(\rvx)$ be a set of basis functions that map inputs $\rvx$ to a feature space. In many algorithms, this feature space only appears in a dot product $\phi(\rvx)^T\phi(\rvx')$ of input pairs $\rvx$ and $\rvx'$. Then, kernel function can be defined as $k(\rvx, \rvx') = \phi(\rvx)^T\phi(\rvx')$. Note that we only need to know $k(\rvx, \rvx')$, not $\phi(\rvx)$.

Recall that the objective of linear regression is:
$$E =\sum_n (\rvw^T\phi(\rvx_n)-y_n)^2 + \lambda \rvw^T\rvw.$$
The solution (gradient) is given by
$$\rvw =-\frac{1}{\lambda}\sum_n (\rvw^T\phi(\rvx_n)-y_n)\phi(\rvx_n).$$
Thus, $\rvw$ is a linear combination of inputs in a feature space. 

We can rewrite it by $\rvw = \Phi \rva$, where $\Phi = [\phi(\rvx_1),\dots,\phi(\rvx_N)]$, $\rva = [a_1,\dots,a_N]^T$, and $a_n =-\frac{1}{\lambda} (\rvw^T\phi(\rvx_n)-y_n).$. Then, we get a dual objective, which aims to minimize $E$ with respect to $\rva$.

Let $K = \Phi^T\Phi$ be the Gram matrix. 


\chapter{Gaussian Process}
\section{Introduction}
\label{sec:gaussian_process}
Gaussian processes are distributions over functions $f(x)$ of which the distribution is defined by a mean function $m(x)$ and positive definite covariance function $k(x,x')$:
$$f(x) \sim \mathcal{GP}(m(x),k(x,x')).$$

Let $f(x) = \phi(x)^Tw$, with $w\sim \mathcal{N}(0, \alpha^{-1}\mathbf{I})$. Then, $m(x) = \mathbb{E}[f(x)] = \mathbb{E}[w]^T\phi(x) = 0$

$k(x, x') = \mathbb{E}[(f(x)-m(x))(f(x')-m(x'))] =\mathbb{E}[f(x)(f(x')] $



\begin{figure}[h]
	\centering
	\includegraphics[scale=0.53]{./images/generative/flows/generative_models.png}
	\caption{An overview of generative models.}
\end{figure}

\begin{itemize}
	\item Flow-models get an exact estimate of the likelihood of your sample, as well as in the reverse direction. 
	\item VAEs optimize a lower bound on the (log) likelihood 
	\item GANs minimize a discrepancy between your input and transformed noise distributions. 
\end{itemize}



\section{The Method of Transformations of Random Variables}

If we are interested in finding the PDF of $Y=g(X)$, where $g(\cdot)$ is some deterministic transformation of $X$, and the function $g$ satisfies following properties, we can utilize a method called the method of transformations.
\begin{itemize}
	\item $g(x)$ is differentiable;
	\item $g(x)$ is a strictly (or monotonically) increasing function, that is, if $x_1<x_2$, then $g(x_1)<g(x_2)$.
\end{itemize}

% Now, let $X$ be a continuous random variable and $Y=g(X)$. We will show that you can directly find the PDF of $Y$ using the following formula.
% \begin{equation*}
% 	f_Y(y) = 
% 	\begin{cases}
% 	\frac{f_X(x_1)}{g'(x_1)}=f_X(x_1). \frac{dx_1}{dy} & \quad \textrm{where } g(x_1)=y\\
% 	0 & \quad \textrm{if }g(x)=y \textrm{ does not have a solution}
% 	\end{cases} 
% \end{equation*}

% Note that the derivative $\frac{dx}{dy}$ or $\frac{d}{dy}(g^{-1}(y))$ \textbf{measures how $X$ changes with respect to $Y$}.
% % Note that start with the function $y=f^{-1}(x)$. Write this as $x=f(y)$ and differentiate both sides implicitly with respect to$x$ using the Chain Rule:
% % \begin{align*}
% % 	1 &= f'(y)\frac{dy}{dx}\\
% % 	\frac{dy}{dx} &= \frac{1}{f'(y)}\\
% % 	y &= f^{-1}(x)\\
% % 	[f^{-1}]'(x) &= \frac{1}{f'(f^{-1}(x))}
% % \end{align*}
% Since $g$ is strictly increasing, its inverse function $g^{-1}$ is well defined. You can imagine a simple function like a linear function, \eg $Y=3X+1$. Then, for each $y\in R_Y$, there exists a \textbf{unique} $x_1$ such that $g(x_1)=y$. We can write $x_1=g^{-1}(y)$. Then,
% \begin{align*}
% 	\{Y\leq y\} = \{g(X)\leq y\} = \{X \leq g^{-1}(x)\}.
% \end{align*}
% Thus, 
% \begin{align*}
% 	F_Y(y) &= P(Y\leq y)\\
% 		   &= P(g(X)\leq y)\\
% 		   &= P(X\leq g^{-1}(y))\quad \text{, since }g \text{ is strictly increasing.}\\
% 		   &= F_X(g^{-1}(y)).
% \end{align*}
% To find the PDF of $Y$, we differentiate $F_Y(y)$ as follows:
% \begin{align*}
% 	f_Y(y) &= \frac{d}{dy}F_X(x_1)\quad \text{by } g(x_1)=y\\
% 		   &=\frac{dx_1}{dy}\cdot \underbrace{\frac{d}{dx_1}F_X(x_1)}_{=F'_X(x_1)}\\
% 		   &=\frac{dx_1}{dy}f_X(x_1)\\
% 		   &=f_X(g^{-1}(y))\left|\frac{d}{dy}(g^{-1}(y))\right|
% 		   % &= \frac{f_X(x_1)}{g'(x_1)} \quad \text{, since } \frac{dx}{dy}=\frac{1}{\frac{dy}{dx}}.
% \end{align*}
% We can repeat the same argument for the case where $g$ is \textbf{strictly decreasing}. In that case, $g'(x_1)$ will be \textbf{negative}, so we need to use $|g'(x_1)|$ . Thus, we can state the following theorem for a \textit{strictly monotonic function}. (A function $g:R\to R$ is called strictly monotonic if it is strictly increasing or strictly decreasing.)

% Actually, we assumed that $g$ was one-to-one out of convenience: the condition that $g$ is one-to-one is not necessary for change of variables to work: Consider a continuous random variable $X$ with domain $R_X$, and let $Y=g(X)$. Suppose that we can partition $R_X$ into a finite number of intervals such that $g(x)$ is strictly monotone and differentiable on each partition. Then the PDF of $Y$ is given by 
% \begin{align*}
% 	f_Y(y)= \sum_{i=1}^{n} \frac{f_X(x_i)}{|g'(x_i)|}= \sum_{i=1}^{n} f_X(x_i).
% 	\left|\frac{dx_i}{dy}\right|,
% \end{align*}
% where $x_1,\dots,x_n$ are real solutions to $g(x)=y$.

% \subsection{Intuitive Explanation}
How to derive the PDF of the random variable $Y=g(X)$ when one knows the PDF of the random variable $X$? If $X$ is discrete, we can derive the pmf for $Y$ by simply summing up the probability mass for all the $x$'s such that $f(x)=y$. For a general function $g$, there is no direct formula to get the PDF of the random variable $Y=g(X)$ knowing $p(X)$. There is a formula in case when $h$ is a differentiable one-to-one mapping from the range (\ie the support) of $X$ to the range of $Y$.

Take for example a random variable $X\sim \mathcal{N}(\mu, \sigma)$ and set $Y=\exp(X)$. The figure below shows some simulations of $X$ and the corresponding values of $Y$. The density of $X$ is shown in blue and the one of $Y$ is shown in orange in the vertical direction.

\begin{figure}[t]
	\centering
	\includegraphics[scale=0.23]{./images/generative/flows/change_of_vars.pdf}
\end{figure}

\begin{figure}[t]
    \centering
    \includegraphics[scale=0.7]{./images/generative/flows/change_intuition.pdf}
    \caption{Area would be approximately $p(z)dz = q(z)dx$. Thus, $q(x) = p(z)\Big|\frac{dz}{dx}\Big|$}
    \label{fig:change_intuition}
\end{figure}


% \begin{figure}[ht]
%     \centering
%     \begin{minipage}[b]{0.45\textwidth}
%         \centering
%         \includegraphics[width=\textwidth]{./images/generative/flows/sample.png}
%     \end{minipage}
%     \hfill
%     \begin{minipage}[b]{0.45\textwidth}
%         \centering
%         \includegraphics[width=\textwidth]{./images/generative/flows/sample2.png}
%     \end{minipage}
% \end{figure}
Now the question is: knowing the density of $X$, what is the density of $Y$?
Taking a point $y$ in the range of $Y$, the PDF $f_Y$ provides the probability of $Y$, belong to a small area $dy$ around $y$ by the formula below
$$P(Y\in dy)\approx f_Y(y)|dy|,$$
where $P(Y\in dy)$ is the area below the curve. Similarly, we can define
$$P(X\in dx)\approx f_X(x)|dx|$$
The above two areas are approximately the same in case of very small region. Note that if $dy$ and $dx$ are very small, we can approximate the derivative of $g'(x)=\frac{|dy|}{|dx|}$. Compactly, this can be expressed as follows:
$$P(X\in dx) = f_X(x)\frac{|dy|}{g'(x)}$$
With $y=g(x)$ we can get 
\begin{align*}
	P(X\in dx)\approx P(Y\in dy) &= f_X(x)\frac{|dy|}{g'(x)}\\
	& = f_X(g^{-1}(y))\frac{|dy|}{g'(g^{-1}(y))}\\
	& = f_X(g^{-1}(y))|dy|(g^{-1})'(y)
\end{align*}
The last line is by the derivative of inverse function which is 
\begin{align*}
	\frac{d}{dx}f^{-1}(x) = \frac{1}{f'(f^{-1}(x))}
\end{align*}
Then, 
\begin{align*}
	P(Y\in dy)\approx f_Y(y)|dy| = f_X(g^{-1}(y))|(g^{-1})'(y)|\cdot |dy|
\end{align*}
Finally, we can get 
$$f_Y(y) = f_X(g^{-1}(y))|(g^{-1})'(y)|$$
Note that the absolute is determined by the function $h$. This is the so-called \textit{change of variables formula}.



\subsection{Vector to Vector}

$Z$ and $X$ be random variables which are related by a mapping $f:\mathbb{R}^n\to \mathbb{R}^n$ such that $X=f(Z)$ and $Z=f^{-1}(X)$. Then
\begin{align*}
	p_X(\mathbf{x}) = p_Z(f^{-1}(\mathbf{x})) \left\vert \text{det}\left(\frac{\partial f^{-1}(\mathbf{x})}{\partial \mathbf{x}}\right) \right\vert
\end{align*}

Note that for any invertible square matrix $A$ over a field (\eg real or complex numbers),
\begin{align*}
	\det\bigl(A^{-1}\bigr)=\frac{1}{\det(A)}.
\end{align*}

Let
$$
A=\begin{bmatrix}
2 & 1\\[2pt]
3 & 4
\end{bmatrix}.
$$

Then, determinant of $A$ is given by
$$
\det(A)=2\cdot4-3\cdot1 = 8-3 = 5.
$$

The inverse of $A$ is 
$$
A^{-1}= \frac{1}{5}\begin{bmatrix}
4 & -1\\[2pt] -3 & 2
\end{bmatrix}.
$$

Then, the determinant of $A^{-1}$ is
$$
\det(A^{-1}) = \frac{1}{5^2}\bigl(4\cdot2-(-3)(-1)\bigr)
=\frac{1}{25}(8-3)=\frac{5}{25}=\frac{1}{5}.
$$

You can confirm the property:

$$
\det(A^{-1})=\frac{1}{5}=\frac{1}{\det(A)}.
$$


For example, we can transform $(x_1,x_2)$ to $(r,\theta)$ via $x_1=r\cos\theta$ and $x_2=r\sin\theta$.

Then
\[
J_{y\to x}=
\begin{pmatrix}
\dfrac{\partial x_1}{\partial r} & \dfrac{\partial x_1}{\partial\theta}\\[6pt]
\dfrac{\partial x_2}{\partial r} & \dfrac{\partial x_2}{\partial\theta}
\end{pmatrix}
=
\begin{pmatrix}
\cos\theta & -r\sin\theta\\
\sin\theta & \phantom{-}r\cos\theta
\end{pmatrix},
\tag{3}
\]
so
\[
\bigl|\det J_{y\to x}\bigr|
 = \bigl|\,r\cos^{2}\theta + r\sin^{2}\theta\,\bigr|
 = |r|.
\tag{4}
\]

Hence
\[
p_{\mathbf{y}}(\mathbf{y}) = p_{\mathbf{x}}(\mathbf{x})\,|J_{y\to x}|
\quad\Longrightarrow\quad
p_{R,\Theta}(r,\theta) = p_{X_1,X_2}(x_1,x_2)\,r.
\tag{5--6}
\]

For a two dimensional random vector $(X,Y)$ with density $p_{X,Y}$, 
\begin{align*}
	Pr((X,Y)\in A) = \int\int_A p_{X,Y}(x,y)dxdy.
\end{align*}
For a infinitesimally small region, we can approximate as follows:
\begin{align*}
	\int_x^{x+dx} p_{X}(x)dx \approx  p_{X}(x)dx.
\end{align*}

Similarly, we can get the probability as follows:
\[
P\!\bigl(r\le R\le r+dr,\;
        \theta\le\Theta\le\theta+d\theta\bigr)
  = p_{R,\Theta}(r,\theta)\,dr\,d\theta.
\tag{7}
\]
Note that the length of arc is $r\times d\theta$. Thus, the area $r\,dr\,d\theta$ or probability is given by
\[
P\!\bigl(r\le R\le r+dr,\;
        \theta\le\Theta\le\theta+d\theta\bigr)
  = p_{X,Y}(r\cos\theta, r\sin\theta)\,r\,dr\,d\theta,
\tag{8--9}
\]
so that finally
\[
p_{R,\Theta}(r,\theta)
  = p_{X,Y}(r\cos\theta, r\sin\theta)\,r.
\tag{10}
\]

% \begin{figure}[h]
%   \centering
%   % Replace the filename with the actual graphic if you have it.
%   \includegraphics[width=.55\linewidth]{polar_patch}
%   \caption{Change of variables from polar to Cartesian.
%            The area of the shaded patch is $r\,dr\,d\theta$.}
%   \label{fig:polarPatch}
% \end{figure}

\section{SVM Optimization: Lagrange Multipliers}

\begin{align*}
&\min_{\rvx} f(\rvx) \\
&\textrm{subject to}\quad g(\rvx)=0.\\
\end{align*}
The minimum of $f$ is found when its gradient point in the same direction as the gradient of $g$. In other words, when:  
$$\nabla f(\rvx) = \alpha\nabla g(\rvx)$$
So if we want to find the minimum of $f$ under the constraint $g$, we just need to solve the following function: 
$$\mathcal{L}(\rvx, \alpha) = f(\rvx) - \alpha g(\rvx)$$
Note that the $\alpha$ is called a \textit{Lagrange multiplier}. 

Recall that we want to solve the following convex quadratic optimization problem:
\begin{align*}
	&\min \frac{1}{2}||w||^2,\quad \textrm{subject to } \\
	&y_i(\mathbf{w}\cdot \mathbf{x}_i+b)\geq 1 \quad\forall i.
\end{align*}
We can reformulate the above problem as follows:
\begin{align}
	\mathcal{L}(\rvw, b, \alpha) = \frac{1}{2}||w||^2 - \sum_i \alpha_i \Big[y_i(\mathbf{w}\cdot \mathbf{x}_i+b)-1\Big]
	\label{eq:objective_function}
\end{align}
We could try to solve the optimization problem, but this problem can only be solved analytically when the number of examples is small. Thus, we will reformulate the problem in the duality principle. 

To get the solution of the primal problem, we need to solve the following \textbf{Lagrangian problem}: 
\begin{align*}
	&\max_{\mathbf{w}, b}\min_\alpha \mathcal{L}(\rvw, b, \alpha)\\
	&\textrm{subject to}\quad \alpha_i\geq 0, \forall i.\\
\end{align*}

You may have noticed that the method of Lagrange multipliers is used for solving problems with equality constraints, and here we are using them with inequality constraints. This is because the method still works for inequality constraints, provided some additional conditions (the \textbf{KKT conditions}) are met. We will discuss about this soon.

\section{The Wolfe Dual Problem}
The Lagrangian problem has $m$ inequality constraints (where $m$ is the number of training examples) and is typically solved using its \textit{dual form}. The duality principle tells us that \textbf{an optimization problem can be viewed from two perspectives}: \Ni The first one is the \textit{primal problem}, a minimization problem in our case. \Nii The other one is the \textit{dual problem}, which will be a maximization problem. What is interesting is that the maximum of the dual problem will always be less than or equal to the minimum of the primal problem (we say it \textbf{provides a lower bound to the solution of the primal problem}). 

In our case, we are trying to solve a convex optimization problem, and \textbf{Slater's condition} holds for affine constraints (Gretton, 2016), so Slater’s theorem tells us that strong duality holds. Note that the strong duality denotes that the solutions from the dual and the primal are identical (the maximum of the dual problem is equal to the minimum of the primal problem). 

Solving the minimization problem involves taking the partial derivatives of $\mathcal{L}$ with respect to $\rvw$ and $b$.  
\begin{align*}
	&\nabla_\rvw\mathcal{L}(\rvw, b, \alpha) = \rvw - \sum_i \alpha_i y_i \mathbf{x}_i\\
	& \nabla_b\mathcal{L}(\rvw, b, \alpha) = -\sum_i \alpha_i y_i
\end{align*}
The first term gives
\begin{align*}
	&\rvw = \sum_{i=1}^m \alpha_i y_i \mathbf{x}_i\\
\end{align*}
Let's substitute the objective function \Cref{eq:objective_function} with $\rvw$:
\begin{align*}
	\mathbf{W}(\alpha, b) &= \frac{1}{2}\Big(\sum_i \alpha_i y_i \mathbf{x}_i\Big)\cdot \Big(\sum_j \alpha_j y_j \mathbf{x}_j\Big) - \sum_i \alpha_i \Bigg[y_i\Bigg(\Big(\sum_j \alpha_j y_j \mathbf{x}_j\Big)\cdot \mathbf{x}_i+b\Bigg)-1\Bigg]\\
						  &= \frac{1}{2}\Big(\sum_i\sum_j \alpha_i\alpha_j y_iy_j \mathbf{x}_i\cdot \mathbf{x}_j\Big) - \sum_i \alpha_i \Bigg[y_i\Bigg(\Big(\sum_j \alpha_j y_j \mathbf{x}_j\Big)\cdot \mathbf{x}_i+b\Bigg)\Bigg]+\sum_i \alpha_i \\
						  &= \frac{1}{2}\sum_i\sum_j \alpha_i\alpha_j y_iy_j \mathbf{x}_i\cdot \mathbf{x}_j - \sum_i\sum_j \alpha_i\alpha_j y_iy_j \mathbf{x}_i \cdot \mathbf{x}_j-\sum_i \alpha_i y_i b+\sum_i \alpha_i \\
						  &= \sum_i \alpha_i -\frac{1}{2}\sum_i\sum_j \alpha_i\alpha_j y_iy_j \mathbf{x}_i\cdot \mathbf{x}_j-\sum_i \alpha_i y_i b
\end{align*}
There is still $b$, but $b=0$, so we can just remove it. Finally, we get
\begin{align}
	 \mathbf{W}(\alpha, b) = \sum_i \alpha_i -\frac{1}{2}\sum_i\sum_j \alpha_i\alpha_j y_iy_j \mathbf{x}_i\cdot \mathbf{x}_j
	 \label{eq:dual_form}
\end{align}
This is the \textbf{Wolfe dual Lagrangian function}. Note that we transform the problem as a problem with regard to $\alpha$. Also, this is again a quadratic programming problem. The optimization problem is also called the \textbf{Wolfe dual problem}: 
\begin{align*}
	 &\max_\alpha \mathbf{W}(\alpha, b) = \sum_i \alpha_i -\frac{1}{2}\sum_i\sum_j \alpha_i\alpha_j y_iy_j \mathbf{x}_i\cdot \mathbf{x}_j\\
	 &\textrm{subject to } \alpha_i\geq 0 \textrm{ for any } i=1,\dots,m\\
	 & \sum_{i=1}^m \alpha_iy_i=0
\end{align*}
Once we get the value of $\alpha$, we can get the optimal $\rvw$ and $b$ can be obtained by using $\alpha_i(y_i(\mathbf{w}\cdot \mathbf{x}_i+b)-1=0$. Details will be covered in the following section.


\section{Karush-Kuhn-Tucker conditions }
Because we are dealing with inequality constraints, there is an additional requirement. The solution must also satisfy the \textbf{Karush-Kuhn-Tucker (KKT) conditions}. 

The KKT conditions are first-order necessary conditions for a solution of an optimization problem to be optimal. Moreover, the problem should satisfy some regularity conditions. Luckily for us, one of the regularity conditions is Slater’s condition, and we just saw that it holds for SVMs. Because the primal problem we are trying to solve is a convex problem, the KKT conditions are also sufficient for the point to be primal and dual optimal, and there is zero duality gap. 

To sum up, if a solution satisfies the KKT conditions, we are guaranteed that \textbf{it is the optimal solution}. Note that solving the SVM problem is equivalent to finding a solution to the KKT conditions. 





% \chapter{Explicit Generative Models}
Two important factors of explicit models to be determined:
\begin{itemize}
	\item Distributions
	\item Parameters of the distributions.
\end{itemize}

\part{Generative Modeling}
\chapter{Introduction to Regression Methods}
\label{chapter:regression_intro}
\section{Regression}
\label{sec:basic_regression}
Suppose a noisy measurement $\rvy = [y_1, \dots, y_m]^T$ with noise $\boldsymbol{\eta} = [\eta_1, \dots, \eta_d]^T$ and we want to estimate parameter $\boldsymbol{\beta} = [\beta_1,\dots,\beta_d]^T$ by using our input $\rvx = [x_1, \dots, x_d]$. 

The measurement $\rvy$ can be modeled as follows:
$$\rvy = \mathbf{X}\boldsymbol{\beta}+\boldsymbol{\eta},$$
where $\mathbf{X}$ is a $m\times d$ input matrix (or our observations). Given a parameter $\boldsymbol{\beta}$, we consider the difference between the noisy measurements and estimated value as follows:
$$\boldsymbol{\epsilon} = \rvy - \mathbf{X}\boldsymbol{\beta}$$
Then, we can establish an objective function as follows:
$$J(\boldsymbol{\beta}) = \boldsymbol{\epsilon}^T\boldsymbol{\epsilon}$$
Note that this is equivalent to minimizing the mean squared error:
$$MSE = \frac{1}{n}\sum_{i=1}^n (y_i-\rvx_i\boldsymbol{\beta})^2.$$
% The OLEs' solution can be optimized by a closed form as follows:
% $$f(\rvx) = \mathbf{X}\boldsymbol{\beta},$$
% where $\rvx = [x_1, \dots, x_d]$ and $\boldsymbol{\beta} = [\beta_1,\dots,\beta_d]^T$. The ridge regression for $\mathbf{X}\in \mathbb{R}^{n\times d}$ matrix can be modeled as follows:
We can optimize this in a closed-form as follows:
\begin{align*}
	J(\boldsymbol{\beta}) &= \|\mathbf{y}-\mathbf{X}\boldsymbol{\beta}\|^2_2 \\
			&= (\mathbf{y}-\mathbf{X}\boldsymbol{\beta})^T(\mathbf{y}-\mathbf{X}\boldsymbol{\beta})\\
			&= (\mathbf{y}^T-\boldsymbol{\beta}^T\mathbf{X}^T)(\mathbf{y}-\mathbf{X}\boldsymbol{\beta})\\
			&= \rvy^T\rvy-\boldsymbol{\beta}^T\mathbf{X}^T\rvy-\rvy^T\mathbf{X}\boldsymbol{\beta}+\boldsymbol{\beta}^T\mathbf{X}^T\mathbf{X}\boldsymbol{\beta}
\end{align*}
To find the $\boldsymbol{\beta}$ that minimizes the objective function, we will compute a derivative of the function while setting it equal to zero:
\begin{align*}
	\frac{\partial J}{\partial \boldsymbol{\beta}}&= -\mathbf{X}^T\rvy-\mathbf{X}^T\rvy+\mathbf{X}^T\mathbf{X}\boldsymbol{\beta}+\mathbf{X}^T\mathbf{X}\boldsymbol{\beta} = 0\\
	\boldsymbol{\beta}	&= (\mathbf{X}^T\mathbf{X})^{-1}\mathbf{X}^T\rvy
\end{align*}
\begin{lstlisting}[language=Python]
import numpy as np
import matplotlib.pyplot as plt

N = 50
x = np.random.randn(N)
w_1 = 3.4 # True Parameter
w_0 = 0.9 # True Parameter
y = w_1*x + w_0 + 0.3*np.random.randn(N) # Synthesize training data

X = np.column_stack((x, np.ones(N)))
W = np.array([w_1, w_0])

# From Scratch
XtX    = np.dot(X.T, X)
XtXinvX = np.dot(np.linalg.inv(XtX), X.T) # d x m
W_best = np.dot(XtXinvX, y.T)
print(f"W_best: {W_best}") 

# Pythonic Approach
theta = np.linalg.lstsq(X, y, rcond=None)[0]
print(f"Theta: {theta}") 

t = np.linspace(0, 1, 200)
y_pred = W_best[0]*t+W_best[1]
yhat = theta[0]*t+theta[1]
plt.plot(x, y, 'o')
plt.plot(t, y_pred, 'r', linewidth=4)
plt.show()
\end{lstlisting}

\subsection{Ridge Regression}
With the ridge regression principle, we can optimize it as follows:
\begin{align}
	J(\boldsymbol{\beta}) &= \|\mathbf{y}-\mathbf{X}\boldsymbol{\beta}\|^2_2 + \lambda \|\boldsymbol{\beta}\|^2_2 \\
			&= (\mathbf{y}-\mathbf{X}\boldsymbol{\beta})^T(\mathbf{y}-\mathbf{X}\boldsymbol{\beta})+\lambda\boldsymbol{\beta}^T\boldsymbol{\beta}\\
			&= (\mathbf{y}^T-\boldsymbol{\beta}^T\mathbf{X}^T)(\mathbf{y}-\mathbf{X}\boldsymbol{\beta})+\lambda\boldsymbol{\beta}^T\boldsymbol{\beta}\\
			&= \rvy^T\rvy-\boldsymbol{\beta}^T\mathbf{X}^T\rvy-\rvy^T\mathbf{X}\boldsymbol{\beta}+\boldsymbol{\beta}^T\mathbf{X}^T\mathbf{X}\boldsymbol{\beta}+\boldsymbol{\beta}^T\lambda\mathbf{I}\boldsymbol{\beta}\\
	\frac{\partial J}{\partial \boldsymbol{\beta}}&= -\mathbf{X}^T\rvy-\mathbf{X}^T\rvy+\mathbf{X}^T\mathbf{X}\boldsymbol{\beta}+\mathbf{X}^T\mathbf{X}\boldsymbol{\beta}+2\lambda\mathbf{I}\boldsymbol{\beta} = 0\\
	\boldsymbol{\beta}	&= (\mathbf{X}^T\mathbf{X}+\lambda\mathbf{I})^{-1}\mathbf{X}^T\rvy
	\label{eq:ridge_regression}
\end{align}

\subsection{Weighted LSE}
The OLEs assume an equal confidence on all the measurements. Now we look at varying confidence in the measurements. We assume that the noise for each measurement has zero mean and is independent, then the covariance matrix for all measurement noise is given by
\begin{align*}
	R &= \mathbb{E}(\eta\eta^T)\\
	  &= \begin{bmatrix}
		  \sigma_1^2 & \dots & 0\\
		  \vdots & \ddots & \vdots\\
		  0 & \ddots & \sigma_l^2\\
	  \end{bmatrix}
\end{align*}
By denoting the error vector $\rvy-\mathbf{X}\boldsymbol{\beta}$ as $\boldsymbol{\epsilon} = (\epsilon_1, \dots, \epsilon_l)^T$, we will minimize the sum of squared differences weighted over the variations of the measurements:
$$J(\tilda{\rvx}) = \boldsymbol{\epsilon}^TR^{-1}\boldsymbol{\epsilon}=\frac{\boldsymbol{\epsilon}_1^2}{\sigma_1^2}+\dots+\frac{\boldsymbol{\epsilon}_l^2}{\sigma_l^2}$$
% This is equivalent to
% $$\frac{1}{n}\sum_{i=1}^{n}\sum_{i=1}^n \alpha_i (y_i-\rvx_i\boldsymbol{\beta}).$$

The best estimate of the parameter is given by
$$\boldsymbol{\beta} = (\mathbf{X}^TR^{-1}\mathbf{X})^{-1}\mathbf{X}^TR^{-1}\rvy.$$
Note that the measurement noise matrix $R$ must be non-singular for a solution to exist.

\section{Recursive Least Squares}
\label{sec:recursive_least_square}

\begin{itemize}
	\item $H==X$
	\item $x==\beta$
\end{itemize}

The ordinary least-squares assumes that all measurements are available at a certain time. However, this often might not be the case in practice. More often, we obtain measurements sequentially and want to update our estimate with each new measurement. In this case, the matrix $H$ needs to be augmented. This update can be very expensive. When then number of measurements is extremely large, the solutions of the least squares problem are difficult to compute. 

These motivate the RLS. Suppose we have an estimate $\tilda{\rvx}_{k-1}$ after $k-1$ measurements, and obtain a new measurement $\rvy_k$. To be general, every measurements is now an $m$-vector with values yielded by, say, several measuring instruments. 

A linear recursive estimator can be written in the following form:
\begin{align*}
	\rvy_k &= H_k\rvx+\eta_k\\
	\tilda{\rvx}_k &= \tilda{\rvx}_{k-1}+K_k (\rvy_k-H_k\tilda{\rvx}_{k-1})
\end{align*}
Here $H_k$ is an $m\times n$ matrix, and $K_k$ is $n\times m$ and referred to as the \textit{estimator gain matrix}. We refer to $\rvy_k-H_k\tilda{\rvx}_{k-1}$ as the \textit{correction term}. Namely, the new estimate is modified from the previous estimate $\tilda{\rvx}_{k-1}$ with a correction via the gain vector. 

The current estimation error is

\begin{align*}
	\boldsymbol{\epsilon}_k	&= \rvx-\tilda{\rvx}_k \\
							&= \rvx-\tilda{\rvx}_{k-1} - K_k (\rvy_k-H_k\tilda{\rvx}_{k-1})\\
							&= \boldsymbol{\epsilon}_{k-1}-K_k (H_k\rvx+\eta_k-H_k\tilda{\rvx}_{k-1})\\
							&= \boldsymbol{\epsilon}_{k-1}-K_k H_k(\rvx-\tilda{\rvx}_{k-1})-K_k\eta_k\\
							&= (I-K_k H_k)\boldsymbol{\epsilon}_{k-1}-K_k\eta_k,
\end{align*}
where $I$ is the $n\times n$ identity matrix. The mean of this error is then
$$\mathbb{E}[\boldsymbol{\epsilon}_{k}] = (I-K_k H_k)\mathbb{E}[\boldsymbol{\epsilon}_{k-1}]-K_k\mathbb{E}[\boldsymbol{\eta}_{k}]$$
If $\mathbb{E}[\boldsymbol{\eta}_{k}]=0$ and $\mathbb{E}[\boldsymbol{\epsilon}_{k-1}]=0$, then $\mathbb{E}[\boldsymbol{\epsilon}_{k}]=0$. So if the measurement noise has zero mean for all $k$, and the initial estimate of $\rvx$ is set equal to its expected value, then $\tilda{\rvx}_k=\rvx_k, \forall k$. With this property, the estimator $\tilda{\rvx}_k &= \tilda{\rvx}_{k-1}+K_k (\rvy_k-H_k\tilda{\rvx}_{k-1})$ is \textit{unbiased}. The property holds regardless of the value of the gain vector $K_k$. This means the estimate will be equal to the true value $\rvx$ on average. 

The key is to determine the optimal value of the gain vector $K_k$. The optimality criterion used by us is to minimize the aggregated variance of the estimation errors at time $k$: 
\begin{align*}
	J_k &= \mathbb{E}[||\rvx-\tilda{\rvx}_k||^2]\\
		&= \mathbb{E}[\boldsymbol{\epsilon}_{k}^T\boldsymbol{\epsilon}_{k}]\\
		&= \mathbb{E}[tr(\boldsymbol{\epsilon}_{k}\boldsymbol{\epsilon}_{k}^T)]\\
		&= tr(P_k),
\end{align*}
where $P_k=\mathbb{E}[\boldsymbol{\epsilon}_{k}\boldsymbol{\epsilon}_{k}^T]$ is the estimation-error covariance and the third line is done by the trace of a product (or cyclic property). The expectation in the third line can go into the trace operator by its linearity. Next, we can obtain $P_k$ by
\begin{align*}
	P_k &= \mathbb{E}\bigg[\big((I-K_k H_k)\boldsymbol{\epsilon}_{k-1}-K_k\eta_k\big)\big((I-K_k H_k)\boldsymbol{\epsilon}_{k-1}-K_k\eta_k\big)^T\bigg]
\end{align*}
By rearranging the above equation with the property that the mean of noise is zero, we can get
$P_k = (I-K_k H_k)P_{k-1}(I-K_k H_k)^T+K_kR_kK_k^T.$

This equation is the recurrence for the covariance of the least squares estimation error. It is consistent with the intuition that as the measurement noise $R_k$ increases, the uncertainty $P_k$ increases. 

Next, we have to compute $K_k$ that minimizes the cost function given by error equation. 








\chapter{Recursive Least Squares}
\section{Recursive Least Squares}
\label{sec:recursive_least_square}

The ordinary least-squares assumes that all measurements are available at a certain time. However, this often might not be the case in practice. \textbf{More often, we obtain measurements sequentially and want to update our estimate with each new measurement.} In this case, the matrix $\mathbf{X}$ needs to be augmented. This update can be very expensive especially if the number of measurements is extremely large, the solutions of the least squares problem become prohibitive to compute. 

This motivates the Recursive Least Squares (RLS). Suppose we have an estimate $\boldsymbol{\theta}_{k-1}$ after $(k-1)$ measurements and obtain a new measurement $\rvy_k$. How can we update our estimate without completely reworking on the pseudo-inverse problem?

A linear recursive estimator can be expressed in the following form:
\begin{align*}
	\rvy_k &= \mathbf{X}_k\boldsymbol{\theta}+\boldsymbol{\eta}_k\\
	\boldsymbol{\theta}_k &= \boldsymbol{\theta}_{k-1}+K_k (\rvy_k-\mathbf{X}_k\boldsymbol{\theta}_{k-1})
\end{align*}
Here, $\mathbf{X}_k$ is an $m\times d$ matrix (observations) and $K_k$ is $d\times m$ and referred to as the \textit{estimator gain matrix}. We refer to $(\rvy_k-\mathbf{X}_k\boldsymbol{\theta}_{k-1})$ as the \textit{correction term}. Also, $\boldsymbol{\eta}_k$ is the measurement error. The new estimate is modified from the previous estimate $\boldsymbol{\theta}_{k-1}$ with a correction via the gain matrix. 

Intuitively, we can notice that we have to compute the optimal gain matrix to update our estimate. To this end, we have to set an \textit{estimation error}, which is our learning objective. The error can be expressed as follows: 
\begin{align*}
	\boldsymbol{\epsilon}_k	&= \boldsymbol{\theta}-\boldsymbol{\theta}_k \\
							&= \boldsymbol{\theta}-\boldsymbol{\theta}_{k-1} - K_k (\rvy_k-\mathbf{X}_k\boldsymbol{\theta}_{k-1})\\
							&= \boldsymbol{\epsilon}_{k-1}-K_k (\mathbf{X}_k\boldsymbol{\theta}+\boldsymbol{\eta}_k-\mathbf{X}_k\boldsymbol{\theta}_{k-1})\\
							&= \boldsymbol{\epsilon}_{k-1}-K_k \mathbf{X}_k(\boldsymbol{\theta}-\boldsymbol{\theta}_{k-1})-K_k\boldsymbol{\eta}_k\\
							&= (I-K_k \mathbf{X}_k)\boldsymbol{\epsilon}_{k-1}-K_k\boldsymbol{\eta}_k,
\end{align*}
where $I$ is the $d\times d$ identity matrix. The mean of this error is then
$$\mathbb{E}[\boldsymbol{\epsilon}_{k}] = (I-K_k \mathbf{X}_k)\mathbb{E}[\boldsymbol{\epsilon}_{k-1}]-K_k\mathbb{E}[\boldsymbol{\eta}_{k}]$$
If $\mathbb{E}[\boldsymbol{\eta}_{k}]=0$ and $\mathbb{E}[\boldsymbol{\epsilon}_{k-1}]=0$, then $\mathbb{E}[\boldsymbol{\epsilon}_{k}]=0$. So if the measurement noise has zero mean for all $k$, and the initial estimate of $\boldsymbol{\theta}$ is set equal to its expected value, then $\boldsymbol{\theta}_k=\boldsymbol{\theta}_k, \forall k$. This property tells us that the estimator $\boldsymbol{\theta}_k = \boldsymbol{\theta}_{k-1}+K_k (\rvy_k-\mathbf{X}_k\boldsymbol{\theta}_{k-1})$ is \textit{unbiased}. This property holds regardless of the value of the gain vector $K_k$. This means the estimate will be equal to the true value $\boldsymbol{\theta}$ on average. 

The key is to \textbf{determine the optimal value of the gain vector} $K_k$. The optimality criterion is to \textbf{minimize the aggregated variance of the estimation errors at time} $k$: 
\begin{align*}
	J_k &= \mathbb{E}[\|\boldsymbol{\theta}-\boldsymbol{\theta}_k\|^2]\\
		&= \mathbb{E}[\boldsymbol{\epsilon}_{k}^T\boldsymbol{\epsilon}_{k}]\\
		&= \mathbb{E}[tr(\boldsymbol{\epsilon}_{k}\boldsymbol{\epsilon}_{k}^T)]\\
		&= tr(P_k),
\end{align*}
where $P_k=\mathbb{E}[\boldsymbol{\epsilon}_{k}\boldsymbol{\epsilon}_{k}^T]$ is \textit{the estimation-error covariance} (\ie \textbf{covariance matrix}). Note that the third line holds by the trace of a product (\ie \textit{cyclic property}) and the expectation in the third line can go into the trace operator by its linearity. Next, we can obtain $P_k$ by
\begin{align*}
	P_k &= \mathbb{E}\bigg[\big((I-K_k \mathbf{X}_k)\boldsymbol{\epsilon}_{k-1}-K_k\boldsymbol{\eta}_k\big)\big((I-K_k \mathbf{X}_k)\boldsymbol{\epsilon}_{k-1}-K_k\boldsymbol{\eta}_k\big)^T\bigg]
\end{align*}
By rearranging the above equation with the property that the mean of noise is zero, we can get
\begin{align}
	P_k = (I-K_k \mathbf{X}_k)P_{k-1}(I-K_k \mathbf{X}_k)^T+K_kR_kK_k^T,
	\label{eq:rls_estimation_cov}
\end{align}
where $R_k = \mathbb{E}[\boldsymbol{\eta}_k\boldsymbol{\eta}_k^T]$ as covariance of $\boldsymbol{\eta}_k$. This equation is the recurrence for the covariance of the least squares estimation error. It is consistent with the intuition that as the measurement noise $R_k$ increases, the uncertainty in our estimate also increases (\ie $P_k$ increases).  Note that $P_k$ should be positive definite since it is a covariance matrix.

Next, let's compute $K_k$ that minimizes the cost function given by the error equation. We are going to utilize the following property:
\begin{align*}
	\frac{\partial tr(CA^T)}{\partial A} &= C\\
	\frac{\partial tr(ACA^T)}{\partial A} &= AC + AC^T
\end{align*}
Next, we are going to take a derivative to the objective function:
\begin{align*}
	\frac{\partial J_k}{\partial K_k} &= \frac{\partial tr(P_k)}{\partial K_k} = \frac{\partial tr}{\partial K_k}\underbrace{(I-K_k \mathbf{X}_k)P_{k-1}(I-K_k \mathbf{X}_k)^T}_{=ACA^T}+\frac{\partial}{\partial K_k}tr\left(K_k R_k K_k^T\right)\\ 
									  &= \frac{\partial tr(ACA^T)}{\partial (I-K_k \mathbf{X}_k)}\frac{\partial (I-K_k \mathbf{X}_k)}{\partial K_k} +\frac{\partial}{\partial K_k}tr\left(K_k R_k K_k^T\right) \quad \text{by Chain Rule}\\
	&= \left((I-K_k \mathbf{X}_k)P_{k-1}+ (I-K_k \mathbf{X}_k)P_{k-1}^T\right)(-\mathbf{X}_k^T) + \frac{\partial}{\partial K_k}tr\left(K_k R_k K_k^T\right)\\
	&= 2(I-K_k \mathbf{X}_k)P_{k-1}(-\mathbf{X}_k^T) + \frac{\partial}{\partial K_k}tr\left(K_k R_k K_k^T\right)\quad \text{, since } P_{k-1} \text{ is symmetric.}\\
									  &= -2(I-K_k \mathbf{X}_k)P_{k-1}\mathbf{X}_k^T+2K_kR_k
\end{align*}
By setting the partial derivative to zero, we get
$$K_k = P_{k-1}\mathbf{X}_k^T(\mathbf{X}_kP_{k-1}\mathbf{X}_k^T+R_k)^{-1}.$$

\subsection{Alternative Form}
Sometimes it is useful to write the equations for $P_k$ and $K_k$ in alternate forms. Although these alternate forms are mathematically identical, they can be beneficial from a computational point of view. Let's first set $\mathbf{X}_kP_{k-1}\mathbf{X}_k^T+R_k = S_k$, then we get 
$$K_k = P_{k-1}\mathbf{X}_k^TS_k^{-1}.$$
By putting this into \Cref{eq:rls_estimation_cov},
\begin{align*}
	P_k &= (I-P_{k-1}\mathbf{X}_k^TS_k^{-1} \mathbf{X}_k)P_{k-1}(I-P_{k-1}\mathbf{X}_k^TS_k^{-1} \mathbf{X}_k)^T+P_{k-1}\mathbf{X}_k^TS_k^{-1} R_k S_k^{-1}\mathbf{X}_kP_{k-1}\\ 
		&\quad \vdots\\
		&= P_{k-1}-P_{k-1}\mathbf{X}_k^TS_k^{-1}\mathbf{X}_k^TP_{k-1}\\
		&= (I-K_k\mathbf{X}_k)P_{k-1}.
\end{align*}
Note that $P_k$ is symmetric (\cf $P_k=\boldsymbol{\epsilon}_{k}\boldsymbol{\epsilon}_{k}^T$), since it is a covariance matrix, and so is $S_k$.

We take the inverse of both sides of 
$$P_{k-1}^{-1} = \bigg(\underbrace{P_{k-1}}_{A}-\underbrace{P_{k-1}\mathbf{X}_k^T}_{B}\big(\underbrace{\mathbf{X}_kP_{k-1}\mathbf{X}_k^T}_{D}\big)^{-1}\underbrace{\mathbf{X}_kP_{k-1}}_{C}\bigg)^{-1}.$$
Next, we apply the matrix inversion lemma which is known as \textit{Sherman-Morrison-Woodbury matrix identity} (or \textit{matrix inversion lemma}) identity: 
$$(A-BD^{-1}C)^{-1} = A^{-1}+A^{-1}B(D-CA^{-1}B)^{-1}CA^{-1}.$$
Then, rewrite $P_k^{-1}$ as follows:
\begin{align*}
	P_k^{-1} &= P_{k-1}^{-1}+P_{k-1}^{-1}P_{k-1}\mathbf{X}_k^T\big((\mathbf{X}_kP_{k-1}\mathbf{X}_k^T+R_k)-\mathbf{X}_kP_{k-1}P_{k-1}^{-1}(P_{k-1}\mathbf{X}_k^T)\big)^{-1}\mathbf{X}_kP_{k-1}P_{k-1}^{-1}\\ 
			 &= P_{k-1}^{-1}+\mathbf{X}_k^TR_{k}^{-1}\mathbf{X}_k
\end{align*}
This yields an alternative expression for the covariance matrix:
\begin{align*}
	P_k = \big(P_{k-1}^{-1}+\mathbf{X}_k^TR_{k}^{-1}\mathbf{X}_k\big)^{-1}
\end{align*}
We can also obtain
\begin{align*}
	K_k = P_{k}\mathbf{X}_k^TR_{k}^{-1}
\end{align*}
By
\begin{align*}
	P_k &= (I-K_k\mathbf{X}_k)P_{k-1}\\
	P_kP_{k-1}^{-1} &= (I-K_k\mathbf{X}_k)\\
	P_kP_k^{-1} &= P_kP_{k-1}^{-1}+P_k\mathbf{X}_k^TR_{k}^{-1}\mathbf{X}_k=I\\
	I &= (I-K_k\mathbf{X}_k)+P_k\mathbf{X}_k^TR_{k}^{-1}\mathbf{X}_k\\
	K_k &= P_{k}\mathbf{X}_k^TR_{k}^{-1}.
\end{align*}

\subsection{Summary of RLS}
In sum, RLS can be updated as follows: 
\begin{itemize}
	\item Update the gain matrix: 
		\begin{itemize}
			\item $K_k = P_{k-1}\mathbf{X}_k^T(\mathbf{X}_kP_{k-1}\mathbf{X}_k^T+R_k)^{-1}$ or
			\item $K_k = P_{k}\mathbf{X}_k^TR_{k}^{-1}$
		\end{itemize}
	\item Update estimate: $\boldsymbol{\theta}_k = \boldsymbol{\theta}_{k-1}+K_k (\rvy_k-\mathbf{X}_k\boldsymbol{\theta}_{k-1})$
	\item Update error covariance matrix by either: 
		\begin{itemize}
			\item $P_k = (I-K_k\mathbf{X}_k)P_{k-1}$.
			\item $P_k = (I-K_k \mathbf{X}_k)P_{k-1}(I-K_k \mathbf{X}_k)^T+K_kR_kK_k^T,$
		\end{itemize}
\end{itemize}

\paragraph{Example: }
At sample time $k$, our measurement is
\begin{itemize}
	\item $y_k = X_k\theta+\eta_k$
	\item $X_k = 1$
	\item $R_k = \mathbb{E}[\eta_k^2]$
\end{itemize}
For this scalar problem, the measurement matrix $X_k$ is a scalar too, and the measurement noise covariance $R_k$ is also a scalar. We will suppose that each measurement has the same covariance so the measurement covariance $R_k$ is not a function of $k$, and can be written as $R$. Initially, before we have any measurements, we have some idea about the value of the $\theta$, and this forms our initial estimate. We also have some uncertainty about our initial estimate, and this forms our initial covariance:
\begin{align*}
	\hat{\theta}_0 &= \mathbb{E}[\theta]\\
	P_0 &= \mathbb{E}[(\theta-\hat{\theta}_0)(\theta-\hat{\theta}_0)^T]\\
		&= \mathbb{E}[(\theta-\hat{\theta}_0)^2]
\end{align*}
If we have absolutely no idea about $\theta$, then $P(0)=\infty I$. If we are 100\% certain about the $\theta$ before taking any measurements, then $P(0)=0$. Let's compute the gain matrix at $k=1$ by using the following equation:
$$K_k = P_{k-1}\mathbf{X}_k^T(\mathbf{X}_kP_{k-1}\mathbf{X}_k^T+R_k)^{-1}.$$
Then, we get
$$K_1 = P_{0}(P_{0}+R)^{-1}.$$
Similarly, by
$$\boldsymbol{\theta}_k = \boldsymbol{\theta}_{k-1}+K_k (\rvy_k-\mathbf{X}_k\boldsymbol{\theta}_{k-1}),$$
we obtain
$$\hat{\theta}_1 = \hat{\theta}_{0}+\frac{P_{0}}{P_{0}+R} (y_1-\hat{\theta}_{0}).$$
Finally, let's update our covariance matrix $P_k$ by 
$$P_k = (I-K_k \mathbf{X}_k)P_{k-1}(I-K_k \mathbf{X}_k)^T+K_kR_kK_k^T.$$
Then, 
\begin{align*}
	P_1 &= \left(I-\frac{P_{0}}{P_{0}+R}\right)P_{0}I-\frac{P_{0}}{P_{0}+R}+\frac{P_{0}}{P_{0}+R}R\frac{P_{0}}{P_{0}+R}\\
		&= \left(\frac{P_0R^2}{(P_{0}+R)^2}\right)+\frac{P_{0}^2R}{(P_{0}+R)^2}\\
		&= \frac{P_{0}R(P_0+R)}{(P_{0}+R)^2}\\
		&= \frac{P_{0}R}{P_{0}+R}
\end{align*}
By repeating these calculations, we can update the above parameters and find general expressions: 
\begin{align*}
	P_{k-1}&= \frac{P_0R}{(k-1)P_0+R}\\
	K_{k}&= \frac{P_0}{kP_0+R}\\
	\hat{\theta}_{k}&= \frac{(k-1)P_0+R}{kP_0+R}\hat{\theta}_{k-1}+\frac{P_0}{kP_0+R}y_k
\end{align*}
Note that if $\theta$ is known perfectly \textit{a priori} (\ie $\theta$ is known perfectly before any measurements are obtained) then $P_0 =0$ and the above equation show that $K_k=0$ and $\hat{\theta} = \hat{\theta}_0$. That is, the optimal estimate of $\theta$ is independent of any measurements that are obtained. In sum, this indicates that no update from measurements is needed, as the estimate is already perfect.

On the other hand, if $x$ is completely unknown a priori, then $P_0\to \infty$, and the above equations simplify to
\begin{align*}
\hat{\theta}_k &= \frac{(k-1)P_0}{kP_0} \hat{\theta}_{k-1} + \frac{P_0}{kP_0} y_k \\
         &= \frac{k-1}{k} \hat{\theta}_{k-1} + \frac{1}{k} y_k \\
         &= \frac{1}{k} [(k-1)\hat{\theta}_{k-1} + y_k]
\end{align*}
In other words, the optimal estimate of \(\theta\) is equal to the running average of the measurements \(y_k\), which can be written as
\begin{align*}
\bar{y}_k &= \frac{1}{k} \sum_{j=1}^k y_j \\
          &= \frac{1}{k} \left(\sum_{j=1}^{k-1} y_j + y_k\right) \\
          &= \frac{1}{k} \left[(k-1) \frac{1}{k-1} \sum_{j=1}^{k-1} y_j + y_k\right] \\
          &= \frac{1}{k} [(k-1) \bar{y}_{k-1} + y_k]
\end{align*}


\subsection{Curve Fitting} 
In the recursive curve fitting problem, we measure data one sample at a time $(y_1, y_2 \dots, )$ and want to find the best fit of a curve to the data. The curve that we want to fit to the data could be constrained to be linear or quadratic and so on. 

\paragraph{Example:} Suppose that we want to fit a straight line to a set of data points. The linear data fitting problem can be written as 
\begin{align*}
	y_k &= \theta_1+\theta_2t_k+\eta_k\\
	\mathbb{E}[\eta_k] &= R_k
\end{align*}
$t_k$ is the independent variable, $y_k$ is the noisy data, and we want to find the linear relationship between $y_k$ and $t_k$. In sum, we want to estimate the constants $\theta_1$ and $\theta_2$. The measurement matrix can be written as 
\begin{align*}
	\mathbf{X}_k = \begin{bmatrix}
		1 & t_k
	\end{bmatrix}.
\end{align*}
Then, 
$$\rvy_k = \mathbf{X}_k\boldsymbol{\theta}+\boldsymbol{\eta}_k.$$


\subsection{Python Implementation}

\begin{lstlisting}[language=Python]
class RecursiveLeastSquares(object):
    
    # theta0 - initial estimate used to initialize the estimator
    # P0 - initial estimation error covariance matrix
    # R  - covariance matrix of the measurement noise
    def __init__(self,theta0,P0,R)
        
        # initialize the values
        self.theta0=theta0
        self.P0=P0
        self.R=R
        
        # this variable is used to track the current time step k of the estimator 
        # after every time step arrives, this variables increases for one 
        # in this way, we can track the number of variblaes
        self.curr_step=0
                  
        # this list is used to store the estimates xk starting from the initial estimate 
        self.estimates=[]
        self.estimates.append(theta0)
         
        # this list is used to store the estimation error covariance matrices Pk
        self.est_error_cov=[]
        self.est_error_cov.append(P0)
        
        # this list is used to store the gain matrices Kk
        self.gainMatrices=[]
         
        # this list is used to store estimation error vectors
        self.errors=[]
    
     
    # this function takes the current measurement and the current measurement matrix X
    # and computes the estimation error covariance matrix, updates the estimate, 
    # computes the gain matrix, and the estimation error
    # it fills the lists self.estimates, self.est_error_cov, self.gainMatrices, and self.errors
    # it also increments the variable curr_step for 1
    
    # measurementValue (theta) - measurement obtained at the time instant k
    # X - measurement matrix at the time instant k
    
    def predict(self,measurementValue,X):
        import numpy as np
        
        # Compute the L matrix and its inverse 
        # K_k = P_{k-1}X_k^T(R_k+X_kP_{k-1}X_k^T)^{-1}
        Lmatrix=self.R+np.matmul(X,np.matmul(self.est_error_cov[self.curr_step],X.T))
        LmatrixInv=np.linalg.inv(Lmatrix)
        # Compute the gain matrix
        gainMatrix=np.matmul(self.est_error_cov[self.curr_step],np.matmul(X.T,LmatrixInv))

        # Compute the estimation error                    
        # \theta_k = \theta_{k-1}+K_k (y_k-X_k\theta_{k-1})
        error=measurementValue-np.matmul(X,self.estimates[self.curr_step])
        # Compute the estimate
        estimate=self.estimates[self.curr_step]+np.matmul(gainMatrix,error)
        
        # Propagate the estimation error covariance matrix
        # P_k = (I-K_k X_k)P_{k-1}(I-K_k X_k)^T+K_kR_kK_k^T
        ImKc=np.eye(np.size(self.theta0),np.size(self.theta0))-np.matmul(gainMatrix,X)
        error_cov=np.matmul(ImKc,self.est_error_cov[self.curr_step])
        
        # add computed elements to the list
        self.estimates.append(estimate)
        self.est_error_cov.append(error_cov)
        self.gainMatrices.append(gainMatrix)
        self.errors.append(error)
        
        # increment the current time step
        self.curr_step=self.curr_step+1
\end{lstlisting}


\section{Alternate Derivation of RLS}
Suppose the training samples arrive one by one in the following sequence $\rvx_1, \dots, \rvx_m, \rvx_{m+1}$, where $\rvx_{m+1}$ denotes the newly arrived sample vector. These samples can be projected onto the feature space by linear projection and expressed into a matrix $\mathbf{P}^T \in \mathbb{R}^{(d+1)\times (m+1)}$ as follows:
\begin{align*}
	\mathbf{P}^T = [\rvp(\rvx_1), \dots, \rvp(\rvx_{m+1})],
\end{align*}
where $\rvp(\cdot)\in \mathbb{R}^{d+1}$. Subsequently, let 
\begin{align*}
	\mathbf{R}_{m+1} &= \mathbf{P}^T \mathbf{P} + b\mathbf{I}\\
	\mathbf{Q}_{m+1} &= \mathbf{P}^T \mathbf{y}.
\end{align*}
By separating the covariance of the newly arrived sample \( p(\rvx_{m+1}) \) from the remaining stack, we can write:
\begin{align*}
	\mathbf{P}^T \mathbf{P} &= \sum_{i=1}^{m+1} \rvp(\rvx_i) \rvp(\rvx_i)^T\\
							&= \sum_{i=1}^m \rvp(\rvx_i) \rvp(\rvx_i)^T + \rvp(\rvx_{m+1}) \rvp(\rvx_{m+1})^T\\
							&= \mathbf{P}_m^T \mathbf{P}_m + \rvp(\rvx_{m+1}) \rvp(\rvx_{m+1})^T.
\end{align*}
Hence, 
\begin{align*}
	\mathbf{R}_{m+1} &= \mathbf{P}^T \mathbf{P} + b\mathbf{I}\\
					 &= \left(\mathbf{P}_m^T \mathbf{P}_m + \rvp(\rvx_{m+1}) \rvp(\rvx_{m+1})^T\right)+ b\mathbf{I}\\
					 &= \underbrace{\mathbf{P}_m^T \mathbf{P}_m + b\mathbf{I}}_{=\mathbf{R}_m} + \rvp(\rvx_{m+1})\rvp(\rvx_{m+1})^T\\
					 &= \mathbf{R}_m + \rvp(\rvx_{m+1}) \rvp(\rvx_{m+1})^T
\end{align*}
Similarly, 
\begin{align*}
	\mathbf{Q}_{m+1} &= \mathbf{Q}_m + \rvp(\rvx_{m+1}) y_{m+1}
\end{align*}
If the system is designed to forget the old training samples (\ie weighted averaging), 
\begin{align*}
	\mathbf{R}_{m+1} &= (1 - \lambda) \mathbf{R}_m + \lambda \rvp(\rvx_{m+1}) \rvp(\rvx_{m+1})^T,\\
	\mathbf{Q}_{m+1} &= (1 - \lambda) \mathbf{Q}_m + \lambda \rvp(\rvx_{m+1}) y_{m+1},
\end{align*}
where \( \lambda \in (0, 1) \) is often called a \textit{forgetting factor}.

Let \( \mathbf{A} = \mathbf{R}_m, \mathbf{B} = p(\rvx_{m+1}), \mathbf{C} = 1 \) (scalar), \( \mathbf{D} = p(\rvx_{m+1})^T = \mathbf{p}^T \), then based on the matrix inversion lemma (Woodbury, 1950; Sherman and Morrison, 1950),
\begin{align*}
	(\mathbf{A} + \mathbf{BCD})^{-1} = \mathbf{A}^{-1} - \mathbf{A}^{-1} \mathbf{B} (\mathbf{C}^{-1} + \mathbf{DA}^{-1} \mathbf{B})^{-1} \mathbf{DA}^{-1},
\end{align*}
we have
\begin{align*}
	\mathbf{R}_{m+1}^{-1} &= \left[ (1 - \lambda) \mathbf{R}_m + \lambda \mathbf{p}\rvp^T \right]^{-1}\\
						  &= \frac{1}{1 - \lambda} \mathbf{R}_m^{-1} - \frac{1}{1 - \lambda} \mathbf{R}_m^{-1}\lambda \mathbf{p} \left( \mathbf{I} + \mathbf{p}^T \frac{\lambda}{1-\lambda} \mathbf{R}_m^{-1} \mathbf{p} \right)^{-1} \mathbf{p}^T \frac{1}{1-\lambda} \mathbf{R}_m^{-1}\\
						  &= \frac{1}{1 - \lambda} \mathbf{R}_m^{-1} - \frac{1}{(1 - \lambda)^2} \mathbf{R}_m^{-1} \mathbf{pp}^T \mathbf{R}_m^{-1} \left( \frac{1}{\lambda} + \frac{1}{1-\lambda} \mathbf{p}^T \mathbf{R}_m^{-1} \mathbf{p} \right)^{-1}.
\end{align*}
We can obtain
\begin{align*}
	\rvw_{m+1} = \left( \mathbf{P}^T \mathbf{P} + b\mathbf{I} \right)^{-1} \mathbf{P}^T \mathbf{y} = \mathbf{R}_{m+1}^{-1} \mathbf{Q}_{m+1},
\end{align*}
Substitute \(\mathbf{R}_{m+1}^{-1}\) and \(\mathbf{Q}_{m+1} = (1 - \lambda) \mathbf{Q}_m +  \lambda\mathbf{p}(\rvx_{m+1}) y_{m+1}\):
\begin{align*}
	\mathbf{w}_{m+1} &= \left[ \frac{1}{1 - \lambda} \mathbf{R}_m^{-1} - \frac{1}{(1 - \lambda)^2} \mathbf{R}_m^{-1} \mathbf{p}\mathbf{p}^T \mathbf{R}_m^{-1} \left( \frac{1}{\lambda} + \frac{1}{1 - \lambda} \mathbf{p}^T \mathbf{R}_m^{-1} \mathbf{p} \right)^{-1} \right] \left[ (1 - \lambda) \mathbf{Q}_m + \lambda y_{m+1} \mathbf{p} \right]\\
					 &= \underbrace{\mathbf{R}_m^{-1} \mathbf{Q}_m}_{=w_m} + \frac{1}{1 - \lambda}\mathbf{R}_m^{-1} \mathbf{p}\mathbf{p}^T \mathbf{R}_m^{-1} \left( \frac{1}{\lambda} + \frac{1}{1 - \lambda} \mathbf{p}^T \mathbf{R}_m^{-1} \mathbf{p} \right)^{-1} \mathbf{Q}_m + \frac{\lambda}{1 - \lambda} \mathbf{R}_m^{-1} \mathbf{p} y_{m+1} \\
					 &\quad - \frac{\lambda}{(1 - \lambda)^2}\mathbf{R}_m^{-1} \mathbf{p}\mathbf{p}^T \mathbf{R}_m^{-1} \left( \frac{1}{\lambda} + \frac{1}{1 - \lambda} \mathbf{p}^T \mathbf{R}_m^{-1} \mathbf{p} \right)^{-1}\mathbf{p} y_{m+1}\\
\end{align*}
Let 
\begin{align*}
	A &= \left( \frac{1}{\lambda} + \frac{1}{1 - \lambda} \mathbf{p}^T \mathbf{R}_m^{-1} \mathbf{p} \right)^{-1} \\
	  &= \frac{\lambda(1-\lambda)}{\lambda \mathbf{p}^T \mathbf{R}_m^{-1} \mathbf{p}+(1-\lambda)},
\end{align*}
which is a constant. Then 
\begin{align*}
	w_{m+1} &= w_m - \frac{\mathbf{R}_m^{-1} \mathbf{p}}{(1-\lambda)^2} \cdot A \cdot \left((1-\lambda)\mathbf{p}^T w_m + \lambda \mathbf{p}^T \mathbf{R}_m^{-1}\mathbf{p} y_{m+1}\right)+ \frac{\lambda}{1 - \lambda} \mathbf{R}_m^{-1} \mathbf{p} y_{m+1}\\
			&= w_m - \frac{\lambda \mathbf{R}_m^{-1} \mathbf{p}}{(1-\lambda)} \cdot \frac{1}{\lambda \mathbf{p}^T \mathbf{R}_m^{-1} \mathbf{p}+(1-\lambda)}\left((1-\lambda)\mathbf{p}^T w_m + \lambda \mathbf{p}^T \mathbf{R}_m^{-1}\mathbf{p} y_{m+1}\right) + \frac{\lambda}{1 - \lambda} \mathbf{R}_m^{-1} \mathbf{p} y_{m+1}\\
			&= w_m - \frac{\lambda \mathbf{R}_m^{-1} \mathbf{p}}{(1-\lambda)} \cdot \frac{1}{\lambda \mathbf{p}^T \mathbf{R}_m^{-1} \mathbf{p}+(1-\lambda)}\left((1-\lambda)\mathbf{p}^T w_m + \lambda \mathbf{p}^T \mathbf{R}_m^{-1}\mathbf{p} y_{m+1}\right) \\ 
			&\quad +\frac{\lambda}{(1 - \lambda)}\cdot \frac{\lambda \mathbf{p}^T \mathbf{R}_m^{-1} \mathbf{p}+(1-\lambda)}{\lambda \mathbf{p}^T \mathbf{R}_m^{-1} \mathbf{p}+(1-\lambda)} \mathbf{R}_m^{-1} \mathbf{p} y_{m+1} \,\, \text{, since $\mathbf{p}^T \mathbf{R}_m^{-1} \mathbf{p}$ is a scalar, we can put $\mathbf{R}_m^{-1} \mathbf{p}$ to the right}\\
			&= w_m + \frac{\lambda}{\lambda \mathbf{p}^T \mathbf{R}_m^{-1} \mathbf{p}+(1-\lambda)}\cdot\frac{1}{(1-\lambda)}\big(-(1-\lambda)\mathbf{R}_m^{-1}\mathbf{p}\mathbf{p}^T w_m - \lambda \mathbf{R}_m^{-1}\mathbf{p}\mathbf{p}^T\mathbf{R}_m^{-1}\mathbf{p}y_{m+1}\\
			&\quad + \lambda \mathbf{R}_m^{-1}\mathbf{p}\mathbf{p}^T\mathbf{R}_m^{-1}\mathbf{p}y_{m+1} + (1-\lambda)\mathbf{R}_m^{-1}\mathbf{p}y_{m+1}\big)\\
			&= w_m + \frac{\lambda}{\lambda \mathbf{p}^T \mathbf{R}_m^{-1} \mathbf{p}+(1-\lambda)}\cdot\frac{1}{(1-\lambda)}\big(-(1-\lambda)\mathbf{R}_m^{-1}\mathbf{p}\mathbf{p}^T w_m + (1-\lambda)\mathbf{R}_m^{-1}\mathbf{p}y_{m+1}\big)
\end{align*}
Thus, the final recursive solution for the weight vector \(\mathbf{w}_{m+1}\) is given by
\[
\mathbf{w}_{m+1} = \mathbf{w}_m + \frac{\lambda \mathbf{R}_m^{-1} \mathbf{p} (y_{m+1} - \mathbf{p}^T \mathbf{w}_m)}{\lambda \mathbf{p}^T \mathbf{R}_m^{-1} \mathbf{p} + (1 - \lambda)}
\]
Here, we note that $\lambda$ controls the strength of update with respect to the accumulated solution with $\lambda \to 1$ having the strongest weight for newly arrived sample. When $\lambda = 0.5$ in , we have the following regularized recursive least squares solution:
\begin{align*}
	w_{m+1} = w_m + \frac{\mathbf{R}_m^{-1} \rvp (y_{m+1} - \rvp^T w_m)}{\rvp^T \mathbf{R}_m^{-1} \rvp + 1}. 
\end{align*}




\chapter{Logistic Regression}
\section{Logistic Regression}
\label{sec:logistic_regression}

Logistic regression corresponds to the following binary classification model parameterized by $\rvw$:
$$p(y|\mathbf{x},\mathbf{w})=\textrm{Ber}(y|\sigma(\mathbf{w}^T\mathbf{x}))$$

Logistic regression models \textit{logit}s (log odds) through a linear model. For binary data, the goal is to model the probability $p$ that one of two outcomes occurs. Recall that an ordinary linear regression model is not bounded. Thus, we will pass a linear model through a sigmoid function, which is also known as logistic function. 

$$\sigma(z) = \frac{1}{1+\exp^{z}},$$
where $z=wx+b.$
The sigmoid function has the property
$$\sigma(-x) = 1-\sigma(x).$$
The $z$ is often called the logit. Note that the inverse of the sigmoid is the log of the odds ratio $\frac{p}{1-p}.$

The logit function is $\textrm{log}\frac{p}{1-p}$, which varies between $-\infty$ and $+\infty$ as $p$ varies between $0$ and $1$.
$$\textrm{log}\frac{p}{1-p} = w_0x_0 +  w_1x_1 + ... + w_nx_n$$
Note that \textbf{the logistic regression model assumes that the log-odds (\textit{logit}) of an observation $y$ can be expressed as a linear function}. In this context, the logit function is called the \textbf{\textit{link function}} because it ``links'' the probability to the linear function of the predictor variables.

% Simplest solution to model a dependant variable $y$ is a linear regression. However, $y$ should be in a range of $[0,1]$. So we need to introduce the logit function. 

% The linear regression can be generalized to the classification setting with two changes:
% \begin{itemize}
% 	\item Replacing the Gaussian distribution for $y$ with a Bernoulli distribution: $p(y|\mathbf{x},\mathbf{w})=Ber(y|\mu(\mathbf{x}))$
% 	\item Squashing input data into sigmoid function $\sigma(\eta)$ that range from 0 to 1: $\sigma(\eta)\triangleq \frac{1}{1+exp(-\eta)}$.
% \end{itemize}
% $$p(y|\mathbf{x},\mathbf{w})=Ber(y|\sigma(\mathbf{w}^T\mathbf{x})),\$$

The negative log-likelihood for logistic regression is given by
\begin{align*}
	\textrm{NLL}(\mathbf{w}) &= -\ln \prod_{i=1}^N p(\rvx)^{\mathds{I}(y_i=1)}(1-p(\rvx))^{\mathds{I}(y_i=0)}\,\footnotemark\\
							 &= -\ln \prod_{i=1}^N \sigma(\mathbf{w}^T\rvx)^{\mathds{I}(y_i=1)}(1-\sigma(\mathbf{w}^T\rvx))^{\mathds{I}(y_i=0)}\\
							 &= -\sum_{i=1}^N y_i\ln\sigma(\mathbf{w}^T\rvx)+\ln(1-y_i)(1-\sigma(\mathbf{w}^T\rvx)).
							 % &= -\sum_{i=1}^{N}\textrm{log}[\mu_i^{\mathds{I}(y_i=1)}\times (1-\mu_i)^{\mathds{I}(y_i=0)}]\\
	% &=-\sum_{i=1}^{N}[y_i\textrm{log}\mu_i + (1-y_i) \textrm{log}(1-\mu_i)], \textrm{ }
\end{align*}
This is also called \textbf{cross-entropy} error function. 
\footnotetext{$\mathds{I}(y_i=1) = y_i$, because $y_i\in \{0, 1\}$ is a binary variable} 

To compute the derivative of NLL, we first need to know the following tricks:
\begin{itemize}
	\item The derivative of $\ln (x)$:
$$\frac{\partial }{\partial x}\ln (x) = \frac{1}{x}.$$
\item The derivative of the sigmoid is given by:
$$\frac{\partial \sigma(z)}{\partial x} = \sigma(x)(1-\sigma(x)).$$
\item Finally, the chain rule of derivative. Suppose we are computing the derivative of a composite function $f(x) = u(v(x))$. The derivative of $f(x)$ is the derivative of $u(x)$ with respect to $v(x)$ times the derivative of $v(x)$ with respect to $x$.
$$\frac{\partial f}{\partial x} = \frac{\partial u}{\partial v} \frac{\partial v}{\partial x}$$
\end{itemize}
The derivative of the loss function w.r.t., a single weight $w_j$ is given by
\begin{align*}
	\frac{\partial \mathcal{L}}{\partial w_j} &= \frac{\partial }{\partial w_j} -[y\ln \sigma(wx + b)+(1-y) \ln (1-\sigma(wx+b))]\\
											  &=  -[\frac{\partial }{\partial w_j}y\ln \sigma(wx + b)+\frac{\partial }{\partial w_j}(1-y) \ln (1-\sigma(wx+b))]\\
											  &= -\frac{y}{\sigma(wx + b)}\frac{\partial }{\partial w_j}\sigma(wx + b) - \frac{1-y}{1-\sigma(wx+b)} \frac{\partial }{\partial w_j}1-\sigma(wx+b)\\
											  &= -\bigg[\frac{y}{\sigma(wx + b)}-\frac{1-y}{1-\sigma(wx + b)}\bigg]\frac{\partial }{\partial w_j}\sigma(wx + b)\\
											  &= -\bigg[\frac{y-\sigma(wx + b)}{\sigma(wx + b)[1-\sigma(wx + b)]}\bigg]\sigma(wx + b)[1-\sigma(wx + b)]\frac{\partial \sigma(wx + b)}{\partial w_j}\\
											  &= -\bigg[\frac{y-\sigma(wx + b)}{\sigma(wx + b)[1-\sigma(wx + b)]}\bigg]\sigma(wx + b)[1-\sigma(wx + b)]x_j\\ 
											  &= -( y-\sigma(wx + b) )x_j\\
											  &= ( \sigma(wx + b)-y )x_j.
\end{align*}





Another way to express \textrm{NLL} is as follows. Suppose $\hat{y}_i\in\{-1,+1\}$ instead of $y_i\in\{0,1\}$. We have $p(y=1)=\frac{1}{1+\mathrm{exp}(-\mathbf{w}^T\mathbf{x})}$ and $p(y=-1)=\frac{1}{1+\mathrm{exp}(+\mathbf{w}^T\mathbf{x})}$. Hence
\begin{align*}
	\textrm{NLL}(\mathbf{w}) &= -\frac{1}{N}\sum_{n=1}^N [\mathbb{I}(\hat{y}_n=1)\log(\sigma(a_n))+\mathbb{I}(\hat{y}_n=-1)\log(\sigma(-a_n))]\\
							 &= -\frac{1}{N}\sum_{n=1}^N \log(\sigma(\hat{y}_na_n))\\
							 &=  \frac{1}{N}\sum_{i=1}^{N}\textrm{log}(1+\mathrm{exp}(-\hat{y}_i\mathbf{w}^T\mathbf{x}_i).
\end{align*}
Note that the sigmoid is used for compressing the output into $[0,1]$ and $\sigma(-a_n) = 1-\sigma(a_n)$. Unlike the linear regression, there is no closed from solution for logistic regression, thus we need optimization algorithms for it. Typically, optimization process involves the gradient and Hessian. 
\begin{align*}
	\mathbf{g}&=\frac{d}{d\mathbf{w}}\mathrm{NLL}(\mathbf{w})=\frac{d}{d\mu_i}\mathrm{NLL}(\mathbf{w})\frac{d\mu_i}{d\mathbf{h}}\frac{d\mathbf{h}}{d\mathbf{w}}\\
	& = \sum_{i=1}\Bigg[-\frac{y_i}{\mu_i} + \frac{(1-y_i) }{(1-\mu_i)}\Bigg]\frac{d\mu_i}{d\mathbf{h}}\frac{d\mathbf{h}}{d\mathbf{w}}=\sum_{i=1}\Bigg[\frac{\mu_i-y_i }{\mu_i(1-\mu_i)}\Bigg]\frac{d\mu_i}{d\mathbf{h}}\frac{d\mathbf{h}}{d\mathbf{w}}\\
	&=\sum_{i}(\mu_i-y_i)\mathbf{x}_i=\mathbf{X}^T(\boldsymbol{\mu}-\mathbf{y})\\
	\frac{d\mu_i}{d\mathbf{h}}& = \mu_i(1-\mu_i)\\
	\frac{d\mathbf{h}}{d\mathbf{w}}& = \mathbf{x}_i
\end{align*}
where $\mathbf{h}=\mathbf{w}^T\mathbf{x}$. 

We can also use the second-order method. 
\begin{align*}
\mathbf{H}&=\frac{d}{d\mathbf{w}}g(\mathbf{w})^T=\sum_{i}(\nabla_{\mathbf{w}}\mu_i)\mathbf{x}_i^T=\sum_{i}\mu_i(1-\mu_i)\mathbf{x}_i\mathbf{x}_i^T\\
&=\mathbf{X}^T\mathbf{S}\mathbf{X},
\end{align*}
where $\mathbf{S}\triangleq \mathrm{diag}(\mu_i(1-\mu_i))$. Note that $\mathbf{H}$ is positive definite, because the \textrm{NLL} is convex and has a global minimum. 

\chapter{Bayesian Regression}
\chapter{Bayesian Regression}

Integrate over all $\theta$

\begin{align}
P(heads \mid D) =& \int_{\theta} P(heads, \theta \mid D) d\theta\\
 =& \int_{\theta} P(heads \mid \theta, D) P(\theta \mid D) d\theta \ \ \ \ \ \  \textrm{(Chain rule: $P(A,B|C)=P(A|B,C)P(B|C)$.)}\\ 
  =& \int_{\theta} \theta P(\theta \mid D) d\theta\\ 
  =&E\left[\theta|D\right]\\
 =&\frac{n_H + \alpha}{n_H + \alpha + n_T + \beta}
\end{align}


% \chapter{Introduction}
\section{Curve Fitting}
We can assume that a target variable has a Gaussian distribution with a mean equal to the value $y(x,\mathbf{w})$ of the polynomial curven given by
\begin{equation}
	p(t|x, \mathbf{w}, \beta) = \mathcal{N}(t|y(x,\mathbf{w}), \beta^{-1}),
	\label{eq:curve}
\end{equation}
\begin{itemize}
	\item $t$: target variable
		$$t = y(\rvx, \rvw) +\epsilon,$$
	where $\epsilon$ is a zero mean Gaussian noise with precision (inverse variance) $\beta$. 
	% \item $x$: input
	% \item $\beta$: an inverse variance of the distribution.
\end{itemize}

We not use the training data $\{\mathbf{x,y}\}$ to determine the values of the unknown parameters $\mathbf{w}$ and $\beta$ by maximum likelihood. If the data are assumed to be drawn independently from the distribution, then the likelihood function is given by 
\begin{equation}
	p(\mathbf{t}|\mathbf{x,w},\beta) = \prod_{n=1}^{N}\mathcal{N}(t_n|y(x_n,\mathbf{w}), \beta^{-1}).
	\label{eq:curve_ml}
\end{equation}

We can take a step towards a more Bayesian approach and introduce a prior distribution over the polynomial coefficients $\mathbf{w}$. For simplicity, we can use a Gaussian distribution from
\begin{equation}
	p(\mathbf{w}|\alpha) = \mathcal{N}(\mathbf{w|0},\alpha^{-1}\mathbf{I}),
	\label{eq:prior_hyper}
\end{equation}
where $\alpha$ is the precision of the distribution. Using Bayes' theorem, the posterior distribution for $\mathbf{w}$ is proportional to the product of the prior distribution and the likelihood function
\begin{equation}
	p(\mathbf{w|x,t},\alpha,\beta)\propto p(\mathbf{w|x,t},\beta)p(\mathbf{w},\alpha).
	\label{eq:bayes_reg}
\end{equation}
We can now determined $\mathbf{w}$ by finding the most probable value of $\mathbf{w}$ given the data, in other words by maximizing the posterior distribution, MAP (maximum posterior). Taking a negative logarithm, then we can find that the maximum of the posterior is given by the minimum of 
\begin{equation}
	\frac{\beta}{2}\sum_{n=1}^N \{y(x_n,\mathbf{w})-t_n\}^2+\frac{\alpha}{2}\mathbf{w}^T\mathbf{w}.
	\label{eq:bayes_}
\end{equation}
Thus we see that maximizing the posteiror distribution si equivalent to minimizing the regularized sum of squares error function encountered earler with a regularization parameter given by $\lambda = \alpha/\beta$.



\section{Bayesian Curve Fitting}


% \section{Adding Noise to Regression Predictors is Ridge Regression}
% In linear regression, we seek a vector $\hat{\beta}$ which solves the following optimization problem:
% \begin{equation}
% 	\hat{\beta} = \arg\min_{\beta}|y-X\beta|^2
% 	\label{eq:linear_regression}
% \end{equation}

% Ridge regression is

% \begin{equation}
% 	\hat{\beta} = \arg\min_{\beta}|y-X\beta|^2+\lambda|\beta|^2
% 	\label{eq:ridge_linear_regression}
% \end{equation}

% $$\varepsilon_1,\varepsilon_2,\cdots\sim \mathcal{N}(1,\sigma),$$
% We can add a multiplicitive random noise to $\mathbf{X}$.
% $$x_{ij}\to \varepsilon_{ij}x_{ij}$$

% $$\hat{\beta} \sim \arg\min_{\beta} E_{G}[|y-G\cdot X\beta|^2]$$
% where $G$ is a matrix of random Gaussian noise. 

% \begin{align*}
% 	\mathbb{E} \left[ \left| y - (G * X) \beta  \right|^2 \right] &= E \left[ y^t y - 2 y^t (G * X) \beta + \beta^t (G * X)^t (G * X) \right] \\
% &= y^t y - 2 y^t (E[G] * X) \beta + \beta^t E \left[ M \right] \beta \\
% &= y^t y - 2 y^t X \beta + \beta^t X^t X \beta + \beta^t diag(\sigma^2) X^t X \beta \\
% &= \left| y - X \beta \right|^2 + \beta^t diag(\sigma^2) X^t X \beta \\
% &= \left| y - X \beta \right|^2 + \sigma^2 \left| \Gamma \beta \right|^2
% \end{align*}
% where $M = (G\cdot X)^T(G\cdot X)$
% $$m_{ij} = \sum_k e_{ki}e_{kj}x_{ki}x_{kj}$$

	




\chapter{Sampling Based Inference}
\section{Basic Sampling Methods}
% \subsection{Forward Sampling}

\subsection{Inverse Transform Sampling}
Inverse transform sampling is a basic method for pseudo-random number sampling, \ie for generating sample numbers at random from any probability distribution given its cumulative distribution function (CDF).

\textbf{Assume that we already have a uniformly distributed random number generator}, \eg \textrm{np.random.randn()}
\begin{enumerate}
	\item Generate a random number $u \sim Unif[0,1]$
	\item Find the inverse of the desired CDF, $F_{X}^{-1}(x)$.
	\item Compute $X=F_{X}^{-1}(u)$. The computed random variable $X$ has distribution $F_X(x)$
\end{enumerate}
However, it is \textbf{hard to compute the inverse of CDF} ($F_X(x)$)

	\begin{itemize}
		\item $F_X(x):\mathbb{R}\mapsto [0,1]$ is any CDF. 
		\item CDF is a non-negative and non-decreasing (monotone) function that is continuous. 
		\item Our objective is to simulate a random variable $X$ distributed as $F$; that is, we want to simularte a $X$ such that $P(X\leq x)=F(x)$.
		\item $F$ is invertible since it is continuous and strictly increasing.
%		\item $y$-axis: Uniform distribution
%		\item $x$-axis: sample value
	\end{itemize}

\begin{figure}[t]
	\begin{center}
		\includegraphics[scale=0.6]{./images/sampling/inverse.jpg}
	\end{center}
	\caption{$y$-axis: Uniform distribution, $x$-axis: sample value}
\end{figure}

\begin{figure}[t]
	\begin{center}
		\includegraphics[scale=1.2]{./images/sampling/cdf.pdf}
	\end{center}
	\caption{How can this sampling method recover the original distribution?}
\end{figure}

\subsection{Ancestral Sampling}
$$p(\mathbf{x}) = p(\mathbf{x}_1)p(\mathbf{x}_2|\mathbf{x}_1)p(\mathbf{x}_3|\mathbf{x}_2)\cdots$$
Sampling steps:
\begin{enumerate}
	\item sample $\mathbf{x}_1$
	\item sample $\mathbf{x}_2$ conditioned by $\mathbf{x}_1$
	\item sample $\mathbf{x}_3$ conditioned by $\mathbf{x}_2$
\end{enumerate}



\subsection{Rejection Sampling}
Rejection sampling is a simple method. It rejects samples violating a given condition (\eg conditions of conditional probability.). Let's see its theory. 

Rejection sampling is a method for sampling from a distribution $p(x)=\frac{1}{Z}p'(x)$ that is difficulut to sample directly, but its unnormalized pdf $p'(x)$ is east to evaluate ($Z$ is hard to compute). In rejection sampling, we need some simpler distribution $q(x)$, called a \textbf{proposal distribution}. 

The intuition of rejection sampling is actually similar to Monte-Carlo estimation. By setting a large area (proposal distribution), we can sample points and take them that are inside the our target distribution

% The $k$ must be sufficiently large to envelope the true distribution.
To run the rejection sampling, introduce a constant $k$ whose value is chosen such that $kq(x)\geq p'(x)$ for all values of $x$. The function $kq(x)$ is called a comparison function. Each step of the rejection sampler involves generating two random variables:
	\begin{enumerate}
		\item Sample $x_0\sim q$
		\item Sample $u_0\sim U[0,kq(x_0)]$.
	\end{enumerate}
	Finally, If $u_0>p'(x_0)$, then the sample $x_0$ will be rejected, otherwise we add the sample $x_0$ to our set of samples $\{x^{r}\}$.

	\begin{figure}[h]
		\begin{center}
			\includegraphics[scale=0.5]{./images/sampling/rejection.pdf}
		\end{center}
	\end{figure}

	The original values of $x$ are generated from the distribution $q(x)$ and these samples are then accepted with probability $p'(x)/kq(x)$ (see the figure above. The acceptance probability (\ie length) is the $p'$ divided by $kq$). Then, the probability that a sample will be accepted is given by   
	\begin{align*}
	p(accept) &= p\Bigg(u\leq \frac{p'(x)}{kq(x)}\Bigg)\\
	&= \int p\Bigg(u\leq \frac{p'(x)}{kq(x)}\bigg|x\Bigg)q(x)dx\\
	% & = \mathbb{E}[p'(x)/kq(x)] \tiny \textrm{ by Lemma 1} \\
	& = \int \frac{p'(x)}{kq(x)}q(x)dx\\
	& = \frac{1}{k}\int p'(x)dx
	\end{align*}
	Thus, the sampling will be more efficient if we choose small $k$ to increase the change of acceptance. 
%	\begin{align*}
%	p(accpet) &= \int p(accpet, x)dx\\
%	& = \int \frac{p'(x)}{kq(x)}q(x)dx\\
%	& = \frac{1}{k}\int p'(x)dx\\
%	& = \frac{1}{k}
%	\end{align*}
%	\begin{align*}
%		p(x^*) &= \frac{[p'(x^*)/kq(x^*)]q(x^*)}{\int [p'(x)/kq(x)]q(x)dx}\\
%		& = \frac{p'(x^*)}{\int p'(x^*)dx}\\
%		& = p(x^*)
%	\end{align*}

\subsection{Importance Sampling}

We want to estimate an expectation of function $f(x),$ where $x\sim p(x)$, but it is hard to estimate the distribution $p(x)$. Again, the importance sampling is not a method for generating samples from $p(\mathbf{x})$. In this case, we can use a simple distribution $q(x)$ by
% $$\mathbb{E}(f) = \int f(x)p(x)dx = \int f(x)\frac{p(x)}{q(x)}q(x)dx \approx \frac{1}{N}\sum_{n=1}^N \frac{p(x_i)}{q(x_i)}f(x_i) .$$

\begin{align*}
	\mathds{E}_p[f(\bf x)] &=  \int p(\mathbf{x}) f(\mathbf{x}) d\rvx\\
	&= \int p(\mathbf{x}) f(\mathbf{x}) \frac{q(\rvx)}{q(\rvx)} d\rvx\\
	&= \int q(\rvx) \Bigg[ f(\mathbf{x}) \frac{p(\mathbf{x})}{q(\rvx)}\Bigg] d\rvx\\
	&= \mathds{E}_q\Bigg[ f(\mathbf{x}) \frac{p(\mathbf{x})}{q(\rvx)}\Bigg]
\end{align*}

\begin{itemize}
	\item Assume that $p(\mathbf{x})$ is known and too complicated to be sampled directly. 
	\item Samples are independently drawn from a \textbf{proposal density} $Q(\mathbf{x})$, which is designed to be close to the true density $p(\mathbf{x})$ and \textbf{simpler}
	\item Generate $R$ samples from $Q(\mathbf{x})$
\end{itemize}

By applying the Monte-Carlo method, we can get
% By the Monte-Carlo, we can estimate the following equation with samples from $q$.
\begin{align}
	\mathds{E}_q\Bigg[ f(\mathbf{x}) \frac{p(\mathbf{x})}{q(\rvx)}\Bigg]&
	\approx \frac{1}{N}\sum_{i=1}^N f(\mathbf{x}) \frac{p(\mathbf{x})}{q(\rvx)}, \quad \rvx_i\sim p(\rvx)
	\label{eq:imporatnce_sampling}
\end{align}

\begin{itemize}
	\item Unbiased estimation
	\item (Potentially) Smaller variance compared to the vanilla Monte-Carlo method above.
		\begin{itemize}
			\item $Var_q\Bigg[f(\mathbf{x}) \frac{p(\mathbf{x})}{q(\rvx)}\Bigg]<Var_p[f(\rvx)]$
			\item When $q(\rvx)$ is high where $|p(\rvx)f(\rvx)|$ is high.
		\end{itemize}
\end{itemize}


\section{Gibbs Sampling}
% \label{sec:}

The phrase ``Markov chain Monte Carlo'' encompasses a broad array of techniques that have in common a few key ideas. The setup for all the techniques that we will discuss in this book is as follows:

\begin{enumerate}
	\item We want to sample from a some complicated density or probability mass function $\pi$. Often, this density is the result of a Bayesian computation so it can be interpreted as a posterior density. The presumption here is that we can evaluate $\pi$ but we cannot sample from it.
	\item We know that certain stochastic processes called Markov chains will converge to a stationary distribution (if it exists and if specific conditions are satisfied). Simulating from such a Markov chain for a long enough time will eventually give us a sample from the chain’s stationary distribution.
	\item Given the functional form of the density $\pi$, we want to construct a Markov chain that has $\pi$ as its stationary distribution.
	\item We want to sample values from the Markov chain such that the sequence of values $\{x_n\}$ generated by the chain converges in distribution to the density $\pi$.
\end{enumerate}

In order for all these ideas to make sense, we need to first go through some background on Markov chains. The rest of this chapter will be spent defining all these terms, the conditions under which they make sense, and giving examples of how they can be implemented in practice.

\section{Markov Chain}
Reference: \href{https://gregorygundersen.com/blog/2019/10/28/ergodic-markov-chains/}{Link}

A Markov chain is a stochastic process that evolves over time by transitioning into different states. The sequence of states is denoted by the collection $\{X_i\}$ and the transition between states is random, following the rule 

\begin{definition}
	Let $D$ be a finite set. A random process $X_1, X2,\dots$  with values in $D$ is called a Markov chain if
$$P(X_t=x_{t+1}|X_{t}=x_t,\dots,X_0=x_0)=P(X_{t+1}=x_{t+1}|X_{t}=x_t)$$
\label{def:markov_chain}
\end{definition}
We can think of $X_t$ as a random state at time $t$, and the Markovian assumption is that the probability of transitioning from $x_t$ to $x_{t+1}$ only depends on $x_t$. In words, the future depends only on the present. Let $p_{ij}$ be the probability of transitioning from state $i$ to state $j$. A Markov chain can be defined by a transition probability matrix:
\begin{definition}
	The matrix $\mathbf{P}=(p_{ij})_{i,j}\in D$ is called the transition probability matrix.
\label{def:markov_chain_transition_matrix}
\end{definition}
Thus, $P$ is a $∣D∣\times∣D∣$ matrix, where $∣D∣$ denotes the cardinality of $D$, and the cell value $p_{ij}$ is the probability of transitioning from state $i$ to state j$,$ and the rows of $P$ must sum to one. In this post, we will restrict ourselves to time \textit{homogeneous Markov chains}:

\begin{definition}
A Markov chain is called time homogeneous if 
$$\mathbb{P}{X_{t+1}=j∣X_n=i}=p_{ij},\, \forall n.$$
\end{definition}
In words, the transition probabilities are not changing as a function of time. Finally, let's introduce some useful notation for the initial state of the Markov chain. Let

\begin{itemize}
	\item Each node has a probability distribution of states.
	\item Each link represents a probability state transition.  
		% \begin{align*}
		% 	P(X_1=i) = \sum_{i=1}^N P(X_i=j|X_0=i)P(X_0=i)\\
		% \end{align*}
	\item $i \to j$: Accessible if a state $j$ is accessible from $i$. 
		\begin{itemize}
			\item $i \leftrightarrow j$: Communicate between the two states.
		\end{itemize}
	\item Reducibility: A Markov chain is \textbf{\textit{irreducible}} if $i\leftrightarrow j, \forall i,j\in S$. Simply, if all states are able to visit other states, it is irreducible. 
	\item Periodic: State $i$ has a period $d$ (\ie periodically visit the state $i$) $\leftrightarrow$ aperiodic.
	\item Transience: A state is \textit{transient} if, when we leave this state, there is a non-zero probability that we will never return to it. Conversely, a state is \textit{recurrent} if we know that we will return to that state, in the future, with probability 1 after leaving it (if it is not transient). 
		\begin{itemize}
			\item Stationary Distribution: A probability of being in a state s at time-step t is equal to a probability of being in the state s at the next time-step. Then, it is a stationary probability distribution.
		\end{itemize}
	\item Ergodicity: A state is ergodic if the state is recurrent, aperiodic. Markov chain is ergodic if all states are ergodic. 
\end{itemize}

\subsection{Stationary Distribution}
% The return time $RT_i = \min\{n>0: X_n=i|X_0=i\}$ is the minimum time when we observe the state $X_n$ is at $i$ after the first visit ($X_0$) at $n$. 

% Limit theorem of Markov chain:
% \begin{itemize}
% 	\item If a MC is irreducible and ergodic.
% \end{itemize}

\paragraph{Limit theorem of Markov chain}

For a Markov chain with a discrete state space and transition matrix $P$, let $\pi_*$ be such that $\pi_*P=\pi_*$. Then $\pi_*$ is a stationary distribution of the Markov chain and the chain is said to be stationary if it reaches this distribution.

The basic limit theorem for Markov chains says that, under a specific set of assumptions that we will detail below, we have 
$$||\pi_*-\pi_n|| \to 0$$
as $n\to\infty$, where $||\cdot||$ is the total variation distance between the two densities. Therefore, no matter where we start the Markov chain ($\pi_0$), $\pi_n$ will eventually approach the stationary distribution. Another way to think of this is that 
$$\lim_{n\to\infty}\pi_n(i)=\pi_*(i).$$
for all states $i$ in the state space. Note that $\pi_0$ is the probability distribution of the Markov chain at time 0. Also, $\pi_n$ denote the distribution of the chain at time $n$.

\url{https://bookdown.org/rdpeng/advstatcomp/background.html}


\paragraph{Reversible MC}
Consider a stationary ergodic Markov chain with transition probability$p(i, j)$ and stationary distribution $\pi(i)$, if we reverse the process, we will get a reversed Markov chain with transition probability$q(i, j)$: 
\begin{align*}
	q(j,i) &= P(X_m=i|X_{m+1}=j)\\
		   &= \frac{P(X_m=i,X_{m+1}=j)}{P(X_{m+1}=j)}\\
		   &= \frac{P(X_m=i|X_{m+1}=j)P(X_{m+1}=j)}{P(X_{m+1}=j)}\\
		   &= \frac{\pi(i)p(i,j)}{\pi(j)}\\
	\pi(i)p(i,j) &= \pi(j)q(j,i)
\end{align*}
If $p(i,j) = q(j,i)$, it is called time-reversible Markov chain. 

% \section{Markov Chain for Sampling}


\section{Markov Chain Monte Carlo}
The basic sampling methods we have learnt so far do not leverage past information, which assumes all samples are independent. In Markov chain based sampling, we will treat random variables as a sequence of sampling process. 

In Markov Chain Monte Carlo(MCMC), we assume that a stationary distribution is already known. We are more interested in estimating a transition rule that describing the stationary distribution. 

Ground rules for MCMCs:
\begin{itemize}
	\item MCMCs stochastically explore the parameter space in such a way that the histogram of their samples produces the target distribution.
	\item Markovian: Evolution of the chain (\ie collections of samples from one iteration to the other) only depends on the current position and some transition probability distribution (\ie how we move from one point in parameter space to another). This means that the chain has no memory and past samples cannot be used to determine new positions in parameter space.
	\item The chain will converge to the target distribution if the transition probability is:
		\begin{itemize}
			\item Irreducible: From any point in parameter space, we must be able to reach any other point in the space in a finite number of steps.
			\item Positive recurrent: For any point in parameter space, the expected number of steps for the chain to return to that point is finite. This means that the chain must be able to re-visit previously explored areas of parameter space.
			\item Aperiodic: The number of steps to return to a given point must not be a multiple of some value $k$. This means that the chain cannot get caught in cycles.
		\end{itemize}
\end{itemize}



\subsection{Metropolis-Hasting Algorithm}
% \begin{itemize}
% 	\item Current value $z^t$
% 	\item Propose a candidate $z^*\sim q(z^*|z^t)$ where $q_t$ is a proposal distribution (\ie Normal distribution). 
% 	\item Same as importance and rejections samplings, yet the difference is the Markov property idea in the proposal distribution. 
% 	\item With an acceptance probability (or moving criteria), $\alpha$.
% \end{itemize}

% \href{https://youtu.be/oX2wIGSn4jY}{Ref: YouTube Lecture}

% % The product of MCMC is a posterior distribution.

Suppose we have a target posterior distribution $\pi(x)$, where $x$ here can be any collection of parameters (not a single parameter). In order to move around this parameter space we must formulate some proposal distribution:

$$q(x_{i+1}\mid x_i),$$

that specifies the probability of moving to a point in parameter space, $x_{i+1},$ given that we are currently at $x_i$. The Metropolis Hastings algorithm accepts a ``jump'' to $x_{i+1}$ with the following probability

$$\kappa(x_{i+1}\mid x_i) = \mathrm{min}\left(1, \frac{\pi(x_{i+1})q(x_i\mid x_{i+1})}{\pi(x_{i})q(x_{i+1}\mid x_{i})}\right) = \mathrm{min}(1, H),$$

where the fraction above is called the Hastings ratio, $H$. The above expression represents that the probability of transitioning from point $x_{i+1}$ given the current position $x_{i}$ is a function of the ratio of the value of the posterior at the new point to the old point (\ie $\pi(x_{i+1})/\pi(x_i)$) and the ratio of the transition probabilities at the new point to the old point (\ie $q(x_i\mid x_{i+1})/q(x_{i+1}\mid x_i)$). Firstly, it is clear that if the ratio is bigger than 1 then the jump will be accepted. Secondly, the ratio of the target posteriors ensures that the chain will gradually move to high probability regions. Lastly, the ratio of the transition probabilities ensures that the chain is not ``favored'' toward certain locations by the proposal distribution function. Note that many proposal distributions are symmetric (\ie $q(x_{i+1}\mid x_i) = q(x_i\mid x_{i+1})$).

The Metropolis-Hasting algorithm is then:

\begin{lstlisting}[language=Python]
def mh_sampler(x0, lnprob_fn, prop_fn, prop_fn_kwargs={}, iterations=100000):
    """Simple metropolis hastings sampler.

    :param x0: Initial array of parameters.
    :param lnprob_fn: Function to compute log-posterior.
    :param prop_fn: Function to perform jumps.
    :param prop_fn_kwargs: Keyword arguments for proposal function
    :param iterations: Number of iterations to run sampler. Default=100000

    :returns:
        (chain, acceptance, lnprob) tuple of parameter chain , acceptance rate
        and log-posterior chain.
    """

    # number of dimensions
    ndim = len(x0)

    # initialize chain, acceptance rate and lnprob
    chain = np.zeros((iterations, ndim))
    lnprob = np.zeros(iterations)
    accept_rate = np.zeros(iterations)

    # first samples
    chain[0] = x0
    lnprob0 = lnprob_fn(x0)
    lnprob[0] = lnprob0

    # start loop
    naccept = 0
    for ii in range(1, iterations):

        # propose
        x_star, factor = prop_fn(x0, **prop_fn_kwargs)

        # draw random uniform number
        u = np.random.uniform(0, 1)

        # compute hastings ratio
        lnprob_star = lnprob_fn(x_star)
        H = np.exp(lnprob_star - lnprob0) * factor

        # accept/reject step (update acceptance counter)
        if u < H:
            x0 = x_star
            lnprob0 = lnprob_star
            naccept += 1

        # update chain
        chain[ii] = x0
        lnprob[ii] = lnprob0
        accept_rate[ii] = naccept / ii

    return chain, accept_rate, lnprob

\end{lstlisting}


% \begin{algorithm}
% 	Initialization: $x_0$\\
% 	\For {$i=1,\dots,N$}{
		
% 	}
% 	Return $p(X) = \sum_i^M \alpha_T^i$
% 	\caption{Metropolis-Hasting algorithm}
% 	\label{algo:metro-polis}
% \end{algorithm}




\chapter{Topic Modeling}
\section{Latent Dirichlet Allocation}
\label{sec:topic_modeling_lda}


The assumptions of LDA:
\begin{itemize}
	\item Each topic is a distribution over words.
	\item Each document is a mixture of corpus-wide topics.
	\item Each word is sampled from one of topics. 
\end{itemize}
The LDA attempts to model the document generation process stochastically. However, we have to infer the latent structure (the distributions) of documents. 

\begin{figure}[h]
	\centering
	\includegraphics[scale=1.0]{./images/lda/lda.pdf}
\end{figure}
\begin{itemize}
	\item $\theta_d\sim Dir(\alpha)$: For each document, draw topic distribution. 
		\begin{itemize}
			\item $\alpha$: Dirichlet parameter
		\end{itemize}
	\item $z_{d,n}\sim Mult(\theta_d)$: per-word topic assignment. The $n$-th word of document $d$ is from which topic?
	\item $w_{d,n}\sim Mult(\phi_{z_{d,n}},n)$: observed word. The $n$-th word in a document $d$ is from a certain topic ($z_{d,n}$) distribution $\phi_{z_{d,n}}$.
	\item $\phi_k\sim Dir(\beta), i=\{1,\dots,K\}$: topics.
		\begin{itemize}
			\item $\beta$: topic hyperparameter (Dirichlet parameter).
		\end{itemize}
\end{itemize}
The document generation process can be modelled as follows:
\begin{align*}
	p(\phi_{1:K}, \theta_{1:D}, z_{1:D}, w_{1:D}) = \prod_{i=1}^K p(\phi_i|\beta)\prod_{d=1}^D p(\theta_d|\alpha)\bigg(\prod_{n=1}^N p(w_{d,n}|\phi_{1:K},z_{d,n})p(z_{d,n}|\theta_d)\bigg).
\end{align*}

\subsection{LDA Inference}
The posterior of the latent variables given the document is
\begin{align*}
	p(\phi, \theta, \mathbf{z}|\mathbf{w}) = \frac{p(\phi, \theta, \mathbf{z},\mathbf{w})}{\int_{\phi}\int_{\theta}\sum_{\mathbf{z}}p(\phi, \theta, \mathbf{z},\mathbf{w})}
\end{align*}
\begin{itemize}
	\item The denominator is intractable
\end{itemize}
We want to estimate the topic distribution $\mathbf{z}$. 

\subsection{Dirichlet Distribution}
The Dirichlet Distribution can be considered as a extension of the beta distribution. 
\begin{align}
	p(P=\{p_i\}|\alpha_i) = \frac{\Gamma(\sum_i\alpha_i)}{\prod_i\Gamma(\alpha_i)}\prod_ip_i^{\alpha_i-1}
	\label{eq:dirichlet_dist}
\end{align}
\begin{itemize}
	\item $\sum_ip_i = 1$
	\item The posterior distribution of Dirichlet distribution is also Dirichlet distribution. 
\end{itemize}


% \part{Introduction to Machine Learning}
\chapter{Latent Variable Models}
\section{Introduction}
\subsection{Motivation of Latent Variable Models}
\label{sec:intro_motivation}
If we knew a corresponding latent variable for each observation, then modelling might be easier. Imagine, how can we find $z^* = \argmax_{z} p(\mathbf{x|z})$ for the data $\mathbf{x}$ as shown in Fig. \ref{fig:clusters}(b)

\begin{figure}[h]
	\begin{center}			
		\includegraphics[scale=0.25]{./images/generative/latent.png}
	\end{center}
	\caption{(a) Complete data set $p(\mathbf{x|z})$. (b) Incomplete data set $p(\mathbf{x})$. (c) Inference result}
	\label{fig:clusters}
\end{figure}
For example, we want to model the complete data set $p(\mathbf{x|z})$ under the i.i.d. assumption 
\begin{equation*}
p(\mathbf{x}_i, \mathbf{z}_i|\boldsymbol{\theta}) = 
\begin{cases}
p(\mathcal{C}_1)p(\mathbf{x}_i|\mathcal{C}_1) \textrm{ if } z_i=0\\
p(\mathcal{C}_2)p(\mathbf{x}_i|\mathcal{C}_2) \textrm{ if } z_i=1\\
p(\mathcal{C}_3)p(\mathbf{x}_i|\mathcal{C}_3) \textrm{ if } z_i=2\\
\end{cases}
\end{equation*}
\begin{align*}
p(\mathbf{x}_1, \mathbf{x}_2,...,\mathbf{x}_N, \mathbf{z}_1, \mathbf{z}_2, ..., \mathbf{z}_N|\boldsymbol{\theta}) = \prod_{n=1}^{N}\prod_{k=1}^{K}\pi_k^{z_{nk}}\mathcal{N}(\mathbf{x}_n|\boldsymbol{\mu}_k, \boldsymbol{\Sigma}_k)^{z_{nk}}
\end{align*}
, where $\pi_k=p(\mathcal{C}_k)$ and $p(\mathbf{x}_i|\mathcal{C}_k)=\mathcal{N}(\mathbf{x}_n|\boldsymbol{\mu}_k, \boldsymbol{\Sigma}_k)$. However, in many cases, it is not observable. 

%In this case, what we can only do is find a posterior $p(\mathbf{x}|\mathbf{z},\boldsymbol{\theta})$.

\chapter{Clustering}


\section{K-Means Clustering}
Suppose we have a data set $\mathbf{X} = \{\mathbf{x}_1,...,\mathbf{x}_n\}$ consisting of $N$ observations of a random $D$-dimensional variable $\mathbf{x}\in \mathbb{R}^{D}$. Our goal is to partition the data into some number $K$ of clusters.  Intuitively, we may think of a cluster as comprising a group of data points whose inter-point distances are small compared with the distances to points outside of the cluster.

This notion can be formalized by introducing a set of $D$-dimensional vectors $\boldsymbol{\mu}_k$, which represents the centers of the clusters. Our goal is to find an assignment of data points to clusters, as well as a set of vectors $\{\boldsymbol{\mu}_k\}$. Objective function of $K$-means clustering (\textit{distortion measure}) can be defined as follows:
$$J =  \sum_{n=1}^{N}\sum_{k=1}^{K}r_{nk}||\boldsymbol{x}_n-\boldsymbol{\mu}_k||^2$$
, where $r_{nk}\in\{0,1\}$ is a binary indicator variable which represents the \textbf{membership of data} $\mathbf{x}_n$. Our goal is to find values for the $\boldsymbol{\mu}_k$ and the $r_{nk}$ so as to minimize $J$. 

We can minimize $J$ through an iterative procedure in which each iteration involves two successive steps corresponding to successive optimizations with respect to the $\boldsymbol{\mu}_k$ and the $r_{nk}$ First we choose some initial values for the $\boldsymbol{\mu}_k$. Then in the first phase we minimize $J$ with respect to the $r_{nk}$, keeping the $\boldsymbol{\mu}_k$ fixed. In the second phase we minimize $J$ with respect to the $\boldsymbol{\mu}_k$, keeping $r_{nk}$ fixed. This two-stage optimization is then repeated until convergence.

The $r_{nk}$ can be optimized in a closed-form solution as follows:
$$r_{nk}=\begin{cases}
1 & \textrm{if } k=\argmin_{j} ||\boldsymbol{x}_n-\boldsymbol{\mu}_j||^2\\
0 & \textrm{otherwise}
\end{cases}$$

Now consider the optimization of the $\boldsymbol{\mu}_k$ with the $r_{nk}$ held fixed. The objective function $J$ is a quadratic function of $\boldsymbol{\mu}_k$, and it can be minimized by setting its derivative with respect to $\boldsymbol{\mu}_k$ to zero giving
\begin{align*}
2\sum_{n=1}^{N}r_{nk}(\boldsymbol{x}_n-\boldsymbol{\mu}_k) = 0.
\end{align*}
We can arrange as
\begin{align*}
\boldsymbol{\mu}_k = \frac{\sum_n r_{nk}\boldsymbol{x}_n}{\sum_n r_{nk}}.
\end{align*}

% \begin{enumerate}
% 	\item Expectation (expectation of a log-likelihood give parameters): $$2\sum_{n=1}^{N}r_{nk}(\boldsymbol{x}_n-\boldsymbol{\mu}_k) = 0.$$
% 	\item Maximization (maximize parameters give a log-likelihood function): $$\boldsymbol{\mu}_k = \frac{\sum_n r_{nk}\boldsymbol{x}_n}{\sum_n r_{nk}}.$$ This updates the centroids. 
% \end{enumerate}

The denominator of $\boldsymbol{\mu}_k$ is equal to the number of points assigned to cluster $k$. The mean of cluster $k$ is essentially the same as the mean of data points $\mathbf{x}_n$ assigned to cluster $k$. For this reason, the procedure is known as the $K$-means clustering algorithm. 

The two phases of re-assigning data points to clusters and re-computing the cluster means are repeated in turn until there is no further change in the assignments. These two phases reduce the value of the objective function $J$, so the convergence of the algorithm is assured. However, it may converge to a local rather than global minimum of $J$. 

Some properties:
\begin{itemize}
	\item Hard clustering ($\leftrightarrow$ Soft clusstering)
	\item Centroid initialization issue.
	\item The number of clusters is uncertain. 
	\item Distance metric issue (\eg Euclidean?)
\end{itemize}



\newpage
\section{Gaussian Mixture Models}

\subsection{Multinomial Distribution}
$$P(X|\boldsymbol{\mu}) = \prod_{n} \prod_k \mu_k^{x_{nk}} = \prod_k\mu_k^{\sum_n x_{nk}}.$$
How to determine the MLE solution of $\boldsymbol{\mu}?$ \ie maximize$P(X|\boldsymbol{\mu})$ subject to $\mu_k\geq 0$ and $ \sum_k \mu_k = 1$. We can use the Lagrange method. 

\begin{align*}
	&\mathcal{L} = \sum_k\sum_nx_{nk}\ln \mu_k +\lambda(\sum_k\mu_k-1)\\ 
	&\frac{\partial \mathcal{L}}{\partial \mu_k} = \frac{\sum_nx_{nk}}{\mu_k}+\lambda. \\
	& \mu_k^{\textrm{ML}} = \frac{m_k}{N},
\end{align*}
where $m_k = \sum_n x_{nk}.$ We can consider the joint distribution of the quantities $m_1, \dots, m_K$, conditioned on the parameters $\boldsymbol{\mu}$ and on the total number $N$ observations: 
$$\textrm{Mult}(m_1, \dots, m_K|\boldsymbol{\mu}, N) = \binom{N}{m_1, \dots, m_K} = \frac{N!}{m_1!,\dots, m_K!}.$$
Note that the variables $m_k$ are subject to the constraint
$$\sum_k m_k = N.$$


\subsection{Multivariate Gaussian Distribution}
\begin{align*}
	\mathcal{N}(x|\mu,\sigma^2) = \frac{1}{\sqrt{2\pi\sigma^2}}\exp\bigg(-\frac{1}{2\sigma^2}(x-\mu)^2\bigg)
\end{align*}
For a $D$-dimensional vector $\mathbf{x}$, 
\begin{align*}
	\mathcal{N}(\mathbf{x}|\boldsymbol{\mu},\boldsymbol{\Sigma}) &= \frac{1}{(2\pi)^{D/2}|\boldsymbol{\Sigma}|^{1/2}}\exp\bigg(-\frac{1}{2}(\mathbf{x}-\boldsymbol{\mu})^T\boldsymbol{\Sigma}^{-1}(\mathbf{x}-\boldsymbol{\mu})\bigg)\\
	\ln\mathcal{N}(\mathbf{x}|\boldsymbol{\mu},\boldsymbol{\Sigma}) &= -\frac{1}{2}\ln|\boldsymbol{\Sigma}|-\frac{1}{2}(\mathbf{x}-\boldsymbol{\mu})^T\boldsymbol{\Sigma}^{-1}(\mathbf{x}-\boldsymbol{\mu})+C.
\end{align*}
Note that the functional dependence of the Gaussian on $\mathbf{x}$ is through the quadratic form: 
$$\Delta^2 = (\mathbf{x}-\boldsymbol{\mu})^T\boldsymbol{\Sigma}^{-1}(\mathbf{x}-\boldsymbol{\mu}).$$
The quantity $\Delta$ is called the \textit{Mahalanobis distance} from $\boldsymbol{\mu}$ to $\mathbf{x}$ and reduces to the Euclidean distance when $\boldsymbol{\Sigma}$ is the identity matrix.

Also,by using i.i.d. condition of a dataset, we can also express as follows:
\begin{align*}
	\ln\mathcal{N}(\mathbf{X}|\boldsymbol{\mu},\boldsymbol{\Sigma}) &= -\frac{N}{2}\ln|\boldsymbol{\Sigma}|-\frac{1}{2}\sum_n(\mathbf{x}_n-\boldsymbol{\mu})^T\boldsymbol{\Sigma}^{-1}(\mathbf{x}_n-\boldsymbol{\mu})+C
\end{align*}








\subsection{Gaussian Mixture Models}


K-means clustering is a hard-clustering, but in some cases soft-clustering provides a better model in practice. Gaussian mixture model assumes a simple \textbf{linear superposition} of Gaussian components, aimed at providing a richer class of density models than the single Gaussian. Let's consider a single sample case and it can be expressed as follows:
$$p(\mathbf{x})= \sum_{k=1}^{K}\pi_k\mathcal{N}(\mathbf{x}|\boldsymbol{\mu_k}, \boldsymbol{\Sigma}_k)$$
Let us introduce a $K$-dimensional binary random variable $\mathbf{z}$ having a 1-of-$K$ representation in which a particular element $z_k$ is equal to 1 and all other elements are 0. I will explain more about $\mathbf{z}$ later. It satisfied the following properties:
\begin{itemize}
	\item $z_k\in\{0,1\}$
	\item $\sum_kz_k=1$
\end{itemize}
The marginal distribution over $\mathbf{z}$ is specified in terms of the mixing coefficients $\pi_k$, such that 
$$p(z_k=1) = \pi_k$$
, where the mixing coefficients must satisfy
$$0\leq\pi_k\leq1$$
and 
$$\sum_{k=1}^{K}\pi_k = 1 $$
in order to be valid probabilities. We can also write pdf of $\mathbf{z}$ in a product of mixing coefficient because it is a 1-of-$K$ representaion.
$$p(\mathbf{z}) = \prod_{k=1}^{K}\pi_k^{z_k} = \pi_k \because z_k\in\{
0,1\}$$
Similarly, the conditional distribution of $\mathbf{x}$ given a particular $\mathbf{z}$ can be modeled to be a Gaussian distribution.
\begin{equation*}
p(\mathbf{x}|z_k=1) = \mathcal{N}(\mathbf{x}|\boldsymbol{\mu}_k, \boldsymbol{\Sigma}_k) 
\end{equation*}
Also can be represented in the form 
\begin{align*}
p(\mathbf{x}|\mathbf{z}) &= \prod_{k=1}^{K}\mathcal{N}(\mathbf{x}|\boldsymbol{\mu}_k, \boldsymbol{\Sigma}_k)^{z_k}\\
& = \mathcal{N}(\mathbf{x}|\boldsymbol{\mu}_k, \boldsymbol{\Sigma}_k) \because z_k\in\{
0,1\}
\end{align*}
Finally, marginal data distribution can be obtrained by summing the joint distribution over all possible states of $\mathbf{z}$ to give
\begin{align*}
	p(\mathbf{x}) &  = \sum_{\mathbf{z}} p(\mathbf{x},\mathbf{z})\\
	& = \sum_{\mathbf{z}} p(\mathbf{z})p(\mathbf{x}|\mathbf{z})= \sum_{z_1,...,z_K} p(z_1,...,z_K)p(\mathbf{x}|z_1,...,z_K)\\
	& = \sum_{k=1}^{K}\pi_k \mathcal{N}(\mathbf{x}|\boldsymbol{\mu}_k, \boldsymbol{\Sigma}_k) 
\end{align*}
Note that for every observed data point $\mathbf{x}_n$ there is a corresponding latent variable $\mathbf{z}_n$, which \textbf{indicates the membership of} $\mathbf{x}_n$. This can be represented as in Fig. \ref{fig:gmm}.

\begin{figure}[h]
	\begin{center}			
		\includegraphics[scale=0.4]{./images/generative/gmm.png}
	\end{center}
	\caption{Graphical representation of GMM model. The GMM models a joint distribution $p(\mathbf{x}, \mathbf{z})$ in terms of a marginal distribution $p(\mathbf{z})$ and conditional distribution $p(\mathbf{x}|\mathbf{z})$ to model $p(\mathbf{x})$. Each $\mathbf{x}_n$ is coupled with $\mathbf{z}_n$}
	\label{fig:gmm}
\end{figure}
Now we can work with the joint distribution $p(\mathbf{x,z})$ instead of the marginal distribution $p(\mathbf{x})$, which is hard to estimate directly as explained in \Cref{sec:intro_motivation}. 

Another quantity which plays a central role is the conditional proability of $\mathbf{z}$ given $\mathbf{x}$, $p(z_k=1|\mathbf{x})$. 
\begin{itemize}
\item $p(z_k=1) = \pi_k$ can be viewed as a prior of $z_k=1$
\item $\gamma(z_k)$: assignment probability or responsibility. This represents the probability of assignment of a sample.  This quantity will be updated through the Bayes Theorem.
\item[] $\rightarrow$  A simple explanation is that \textbf{this is the classification result} of $\mathbf{x}_n$.
\end{itemize}
\begin{align*}
\gamma(z_k) \equiv p(z_k=1|\mathbf{x}) & \equiv \frac{p(z_k=1)p(\mathbf{x}|z_k=1)}{\sum_{j=1}^{K}p(z_j=1)p(\mathbf{x}|z_j=1)} \\
& = \frac{\pi_k\mathcal{N}(\mathbf{x}|\boldsymbol{\mu}_k, \boldsymbol{\Sigma}_k)}{\sum_{j=1}^{K} \pi_j\mathcal{N}(\mathbf{x}|\boldsymbol{\mu}_j, \boldsymbol{\Sigma}_j)}
\end{align*}

\subsection{Maximum Likelihood}
Suppose we have a data set of observations $\mathbf{X}=\{\mathbf{x}_1,...,\mathbf{x}_n\}^{T}\in\mathbbm{R}^{N\times D}$ and we want to model the data distribution $p(\mathbf{X})$ using GMM. If we assume an \textrm{i.i.d.} data set, it can be expressed as follows: 
\begin{align*}
p(\mathbf{X}|\boldsymbol{\pi},\boldsymbol{\mu},\boldsymbol{\Sigma}) &=\prod_{n=1}^{N}\Bigg(\sum_{k=1}^{K}\pi_k\mathcal{N}(\mathbf{x}_n|\boldsymbol{\mu}_k, \boldsymbol{\Sigma}_k)\Bigg)\\
\end{align*}
then its \textbf{loglikelihood function for GMM} is given by:
\begin{align*}
\ln p(\mathbf{X}|\boldsymbol{\pi},\boldsymbol{\mu},\boldsymbol{\Sigma}) &= \sum_{n=1}^{N}\ln \Bigg(\sum_{k=1}^{K}\pi_k\mathcal{N}(\mathbf{x}_n|\boldsymbol{\mu}_k, \boldsymbol{\Sigma}_k)\Bigg)
\end{align*}

%In a single dimension case, 
%\begin{align*}
%	\ln p(x, \pi, \mu, \sigma) & =\sum_{n=1}^{N}\ln \sum_{k=1}^{K}\pi_k \frac{1}{\sigma_k \sqrt{2\pi_k}}\exp\Big(-\frac{1}{2}\Big(\frac{x_n-\mu_k}{\sigma_k}\Big)^2\Big)\\
%	\frac{\partial }{\partial \mu_k}\ln p(x, \pi, \mu, \sigma) & =\sum_{n=1}^{N} \frac{\pi_k \frac{1}{\sigma_k \sqrt{2\pi}}\exp\Big(-\frac{1}{2}\Big(\frac{x_n-\mu_k}{\sigma_k}\Big)^2\Big) \frac{x_n-\mu_k}{\sigma_k^2}}{\sum_{k=1}^{K}\pi_k \frac{1}{\sigma_k \sqrt{2\pi}}\exp\Big(-\frac{1}{2}\Big(\frac{x_n-\mu_k}{\sigma_k}\Big)^2\Big)}\\
%	& =\sum_{n=1}^{N} \underbrace{\frac{\pi_k \mathcal{N}(x_n|\mu_k, \sigma_k) }{\sum_{k=1}^{K}\pi_k \mathcal{N}(x_n|\mu_k, \sigma_k)}}_{=\gamma(z_{nk})}\frac{x_n-\mu_k}{\sigma_k^2}\\
%	\mu_k &=\frac{1}{N_k}\sum_{n=1}^{N} \gamma(z_{nk}) x_n,
%\end{align*}
%where 
%$N_k = \sum_{n=1}^{N} \gamma(z_{nk})$. $N_k$ can be interpreted as the effective number of points assigned to cluster $k$. 

How to solve this MLE? While a gradient-based optimization is possible, we consider the iterative \textit{Expectation Maximization} algorithm.

Before, maximizing the likelihood, it is worth to emphasize two issues in GMM: (i) \textit{singularities} and (ii) \textit{identifiability}.

%\subsection{Singularity and Identifiability}
\paragraph{Singularity}
% Suppose that one of the components of the mixture model, let us say the $j$-th component, has its mean $\mathbf{\mu}_j$ exactly equal to one of the data points so that $\mathbf{\mu}_j=\mathbf{x}_n$ for some value of $n$. This data point will then contributes a term in the likelihood function of the form 
% $$\mathcal{N}(\mathbf{x}_n, \sigma_j^2\mathbf{I}) = \frac{1}{(2\pi)^{1/2}} \frac{1}{\sigma_j}$$
% If we consider the limit $\sigma_j \to 0$, then we see that this term goes to infinity and so the log likelihood function will also go to infinity. Thus the maximization of the log likelihood function will also go to infinity. Thus the maximization of the log likelihood function is not a well posed problem because such sigularities will walways be present and will occur whenever one of the Gaussian components collapses onto a specific data point. 

Before discussing how to maximize this function, it is worth emphasizing that there is a significant problem associated with the maximum likelihood framework applied to Gaussian mixture models, due to the presence of singularities. For simplicity, consider a Gaussian mixture whose components have covariance matrices given by $\Sigma_k = \sigma^2_kI$, where $I$ is the unit matrix, although the conclusions will hold for general covariance matrices. Suppose that one of the components of the mixture model, let us say the $j$-th component, has its mean $\boldsymbol{\mu}_j$ exactly equal to one of the data points so that $\boldsymbol{\mu}_j = \mathbf{x}_n$ for some value of $n$. This data point will then contribute a term in the likelihood function of the form
\begin{align*}
	\mathcal{N}(\mathbf{x}_n|\mathbf{x}_n, \sigma^2_jI) = \frac{1}{\sqrt{2\pi}\sigma_j}
\end{align*}
If we consider the limit $\sigma_j \to 0$, then we see that this term goes to infinity and so the log likelihood function will also go to infinity. Thus the maximization of the log likelihood function is not a well posed problem because such singularities will always be present and will occur whenever one of the Gaussian components `collapses' onto a specific data point. Recall that this problem did not arise in the case of a single Gaussian distribution as the variance can not be zero (recall the definition of variance). 


\paragraph{Identifiability}
A further issue in finding MLE based solutions arises from the fact that for any given maximum likelihood solution, a $K$-component mixture will have a total ok $K!$ equivalent solutions corrsponding to the $K!$ ways of assigning $K$ sets of parameters to $K$ components. In other words, for any given point in the space of parameter values there will be a further $K!-1$ additional points all of which give rise to exactly the same distribution. 

\subsection{Expectation Maximization for GMM}

The goal of Expectation Maximization (EM) is to find maximum likelihood solutions for models having latent variables 
\begin{itemize}
	\item Suppose that it is hard to optimize $p(\mathbf{X}|\boldsymbol{\theta})$ directly.
	\item However, it is easier to optimize the complete-data likelihood function $p(\mathbf{X}, \mathbf{Z}|\boldsymbol{\theta})$ 
	\item In this case, we can use \textbf{EM algorithm}. EM algorithm is a general technique for finding maximum likelihood solutions for latent variable models. 
\end{itemize}
Let us begin by writing down the conditions that must be satisfied at a maximum of the likelihood function. Setting the derivatives of $\ln p(\mathbf{X}|\boldsymbol{\pi},\boldsymbol{\mu},\boldsymbol{\Sigma})$  with respect to the means $\boldsymbol{\mu}_k$ of the Gaussian components to zero, we obtain
\begin{align*}
	0 = -\sum_{n=1}^N\frac{\pi_k\mathcal{N}(\mathbf{x}|\boldsymbol{\mu}_k, \boldsymbol{\Sigma}_k)}{\sum_{j=1}^{K} \pi_j\mathcal{N}(\mathbf{x}|\boldsymbol{\mu}_j, \boldsymbol{\Sigma}_j)}\boldsymbol{\Sigma}_k(\mathbf{x}_n-\boldsymbol{\mu}_k)
\end{align*}
Multiplying by $\boldsymbol{\Sigma}_k^{-1}$ (which we assume to be non-singular) and rearranging we obtain
\begin{align*}
	\boldsymbol{\mu}_k = \frac{1}{N_k}\sum_{n=1}^{N}\gamma(z_{nk})\mathbf{x}_n, 
\end{align*}
where we have defined
\begin{align*}
	N_k = \sum_{n=1}^{N}\gamma(z_{nk}).
\end{align*}
We can interpret $N_k$ as the effective number of points assigned to cluster $k$. We can obtain the MLE solutions for other variables similarly.
\begin{algorithm}
	Initialize the means $\boldsymbol{\mu}_k$, covariances $\boldsymbol{\Sigma}_k$ and mixing coefficients $\pi_k$ and evaluate the initial value of the log likelihood.\\
	\For{n}{
		E step: evaluate the responsibilities of $\mathbf{x}_n$ based on the current parameter values with the given parameters
		$$ \gamma(z_{nk})= p(z_k=1|\mathbf{x}_n) =  \frac{\pi_k\mathcal{N}(\mathbf{x}_n|\boldsymbol{\mu}_k, \boldsymbol{\Sigma}_k)}{\sum_{j=1}^{K} \pi_j\mathcal{N}(\mathbf{x}_n|\boldsymbol{\mu}_j, \boldsymbol{\Sigma}_j)}$$\\
		where $z_{nk}$ denote the $k$-th component of $\mathbf{z}_n$\\
		M step: maximize expectation
		\begin{itemize}
			\item $\boldsymbol{\mu}_k^{\textrm{new}} = \frac{1}{N_k}\sum_{n=1}^{N}\gamma(z_{nk})\mathbf{x}_n$
			\item $\boldsymbol{\Sigma}_k^{\textrm{new}} = \frac{1}{N_k}\sum_{n=1}^{N}\gamma(z_{nk})(\mathbf{x}_n-\boldsymbol{\mu}_k^{\textrm{new}})(\mathbf{x}_n-\boldsymbol{\mu}_k^{\textrm{new}})^T$
			\item $\pi_k^{\textrm{new}} = p(z_k=1) = \frac{N_k}{N}$
		\end{itemize}
	Evaluate the log likelihood to check for convergence of parameters
	$$\textrm{ln}p(\mathbf{X}|\boldsymbol{\pi},\boldsymbol{\mu},\boldsymbol{\Sigma}) = \sum_{n=1}^{N}\textrm{ln}\Bigg(\sum_{k=1}^{K}\pi_k\mathcal{N}(\mathbf{x}_n|\boldsymbol{\mu}_k, \boldsymbol{\Sigma}_k)\Bigg)$$
	}
	\caption{EM algorithm for GMM}
\end{algorithm}
	\begin{figure}[h]
	\begin{center}
		\includegraphics[scale=0.3]{./images/generative/em_update.png}
	\end{center}
	\caption{M-step of EM algorithm}
	\label{fig:em2}
\end{figure}

\section{Alternative View of EM}
The goal of the EM algorithm is to find maximum likelihood (loglikelihood) solutions for models having latent variables.
$$\ln p(X|\theta) = ln\sum_Z p(X,Z|\theta).$$
We are not given the complete data set ${X, Z}$, but only the incomplete data $X$. Our state of knowledge of the values of the latent variables
in $Z$ is given only by the posterior distribution $p(Z|X, \theta)$. Because we cannot use the complete-data log likelihood, we consider instead its expected value under the posterior distribution of the latent variable, which corresponds (as we shall see) to the E step of the EM algorithm.

In the subsequent M step, we maximize this expectation. If the current estimate for the parameters is denoted $\theta_{old}$, then a pair of successive E and M steps gives rise to a revised estimate $\theta^{new}$.

The algorithm is initialized by choosing some starting value for the parameters $\theta_0$. The use of the expectation may seem somewhat arbitrary.

In the E step, we use the current parameter values $\theta^{old}$ to find the posterior distribution of the latent variables given by $p(Z|X, \theta^{old})$. We then use this posterior distribution to find the expectation of the complete-data log likelihood evaluated for some general parameter value $\theta$. This expectation, denoted $Q(\theta, \theta^{old})$, is given by 
$$Q(\theta, \theta^{old}) = \sum_Z p(Z|X, \theta^{old})\ln p(X,Z|\theta).$$
In the M step, we determine the revised parameter estimate $\theta^{new}$ by maximizing this function
$$\theta^{new}=\argmax Q(\theta, \theta^{old}).$$


\begin{algorithm}
The goal is to maximize the likelihood function $p(X|\theta)$ with respect to $\theta$ given a joint distribution $p(X, Z|\theta)$.\\
1. Init $\theta^{old}$\\
2. E-Step: evaluate $p(Z|X, \theta^{old})$ \\
3. M-Step: evaluate $\theta^{new}$ given by 
$$\theta^{new} = \argmax Q(\theta, \theta^{old}),$$
where
$$Q(\theta, \theta^{old}) = \sum_Z p(Z|X, \theta^{old})\ln p(X,Z|\theta).$$
4. Check for convergence of either the log likelihood or the parameter values. If the convergence criterion is not satisfied, then let
$$\theta^{old}\leftarrow \theta^{new}.$$
Return to the step 2. 
\caption{General EM algorithm}
\end{algorithm}



\section{Latent Variable Modeling}

For each object $x_i$, we establish additional latent variable $z_i$ which denotes the index of gaussian from which $i$-th object was generated. Then our model is
$$p(X,Z|\theta) = \prod_{i=1}^{n}p(x_i,z_i|\theta) = \prod_{i=1}^{n}p(x_i|z_i,\theta)p(z_i|\theta) = \prod_{i=1}^{n}\mathcal{N}(x_i|\mu_{z_i},\sigma_{z_i}^2)\pi_{z_i},$$
where $\pi_{j} = p(z_i=j)$ are prior probability of $j$-th gaussian and $\theta = \{\mu_j, \sigma_j, \pi_j\}_{j=1}^K$. If we know both $X$ and $Z$ then we can obtain explicit ML-solution:
$$\theta_{ML} = \argmax_{\theta}p(X,Z|\theta) = \argmax_{\theta}\log p(X,Z|\theta).$$
However, in practice, we don't know $Z$, but only know $X$. Thus, we need to maximize w.r.t. $\theta$ the log of incomplete likelihood
\begin{align}
	\log p(X|\theta) & = \ln \int  p(X, Z|\theta)dZ\\
					 & = \ln\int q(Z|X) \frac{p(X, Z|\theta)}{q(Z|X)}dZ\\
					 & \geq \underbrace{\int q(Z|X) \ln\frac{p(X, Z|\theta)}{q(Z|X)}dZ}_{\textrm{ELBO, } \mathcal{L}(q,\theta)} \quad\textrm{by Jensen's Inequality.}\\
					 &= \int q(Z|X) \ln p(X, Z|\theta) - q(Z|X)\ln q(Z|X)dZ\\
					 &= \int q(Z|X)[\ln p(X|Z,\theta) + \ln p(Z|\theta)]  - q(Z|X)\ln q(Z|X)dZ\\
					 &= \int q(Z|X)\ln p(X|Z,\theta)  - q(Z|X)\ln\frac{q(Z|X)}{p(Z|\theta)}dZ\\
					 &= \mathbb{E}_{q(Z|X)} \ln p(X|Z,\theta)  - KL(q(Z|X)||p(Z|\theta)) 
	% & = \int q(Z)\log \frac{p(X,Z|\theta)}{p(Z|X,\theta)}dZ\\
	% & = \int q(Z)\log \frac{q(Z)p(X,Z|\theta)}{q(Z)p(Z|X,\theta)}dZ\\
	% & = \int q(Z)\log \frac{p(X,Z|\theta)}{q(Z)}dZ+ \int q(Z)\log \frac{q(Z)}{p(Z|X,\theta)}dZ\\
	% & = \underbrace{\int q(Z)\log \frac{p(X,Z|\theta)}{q(Z)}dZ}_{\textrm{ELBO, } \mathcal{L}(q,\theta)}+ \textrm{KL}(q(Z)||\log p(Z|X,\theta))
	\label{eq:elbo}
\end{align}
% Note that $\textrm{KL}( \cdot|| \cdot)\geq 0$, thus $\mathcal{L}(q,\theta) \leq \log p(X|\theta)$. In other words, $\mathcal{L}(q,\theta)$ is a \textbf{lower bound} on $\log p(X|\theta)$.

% Let's see ELBO in a different perspective
% \begin{align*}
% 	\mathcal{L}(q,\theta) &=  \int q(Z)\log \frac{p(X,Z|\theta)}{q(Z)}dZ\\
% 	&= \int q(Z)\Big[\log p(X,Z|\theta)- \log q(Z)\Big]dZ\\
% 	&= \int q(Z)\Big[\log p(Z|X,\theta)+\log p(X|\theta)-\log q(Z)\Big]dZ\\
% 	&= \int q(Z)\log p(X|\theta)dZ+\int q(Z) \log\frac{ p(Z|X,\theta)}{ q(Z)}dZ\\
% 	&= \log p(X|\theta)-\textrm{KL}(q(Z)||p(Z|X,\theta))
% \end{align*}
To maximize the above equation, we need to minimize KL divergence. 

% Also note that $p$ does not depend of $q$, \textbf{so maximizing ELBO is equal to minimizing the KL divergence}. 

% By using ELBO, we are able to maximize the incomplete likelihood. If you see the KL term, it is trying to minimize the divergence between $q(Z)$ and $p(Z)$ through maximizing ELBO.


\subsection{Evidence Lower Bound (ELBO)}
For any choice of inference model $q_{\phi}(z|x)$, we can represent the marginal probability of data (or model evidence) distribution, since the $z$ is not related to $x$, so the integration does not affect $x$. Thus, we can also derive ELBO as follows:
\begin{align*}
	\log p_{\theta}(x) &= \mathbb{E}_{q_{\phi}(z|x)}[\log p_{\theta}(x)]\\
	& = \mathbb{E}_{q_{\phi}(z|x)}\Bigg[\log \frac{p_{\theta}(x,z)}{p_{\theta}(z|x)}\Bigg]\\
	& = \mathbb{E}_{q_{\phi}(z|x)}\Bigg[\log \frac{p_{\theta}(x,z)q_{\phi}(z|x)}{q_{\phi}(z|x) p_{\theta}(z|x)}\Bigg]\\
	& = \underbrace{\mathbb{E}_{q_{\phi}(z|x)}\Bigg[\log \frac{p_{\theta}(x,z)}{q_{\phi}(z|x) }\Bigg]}_{=\mathcal{L}(\phi,\theta)(x)}+\underbrace{ \mathbb{E}_{q_{\phi}(z|x)}\Bigg[\log \frac{q_{\phi}(z|x)}{p_{\theta}(z|x)}\Bigg]}_{=D_{KL}(q_{\phi}(z|x)||p_{\theta}(z|x))}
\end{align*}

To get more intuition about ELBO, we can express ELBO as follows:
\begin{align*}
	\mathcal{L}(\phi,\theta) & = \mathbb{E}_{q_{\phi}(z|x)}\Bigg[\log \frac{p_{\theta}(x,z)}{q_{\phi}(z|x) }\Bigg]\\
	& = \mathbb{E}_{q_{\phi}(z|x)}\Bigg[\log p_{\theta}(x,z)-\log q_{\phi}(z|x)\Bigg]\\
	& = \mathbb{E}_{q_{\phi}(z|x)}\Bigg[\log p_{\theta}(x)+\log p_{\theta}(z|x)-\log q_{\phi}(z|x)\Bigg]\\
	& = \log p_{\theta}(x) - D_{\textrm{KL}}(q_{\phi}(z|x)||p_{\theta}(z|x))\\
	& \leq \log p_{\theta}(x)
\end{align*}

ELBO can be also written as follows:
\begin{align*}
\mathcal{L}(\phi,\theta) & = \mathbb{E}_{q_{\phi}(z|x)}\Bigg[\log \frac{p_{\theta}(x,z)}{q_{\phi}(z|x) }\Bigg]\\
& = \mathbb{E}_{q_{\phi}(z|x)}\Bigg[\log p_{\theta}(x,z)-\log q_{\phi}(z|x)\Bigg]\\
& = \mathbb{E}_{q_{\phi}(z|x)}\Bigg[\log p_{\theta}(z)+\log p_{\theta}(x|z)-\log q_{\phi}(z|x)\Bigg]\\
& = \mathbb{E}_{q_{\phi}(z|x)}[\log p_{\theta}(x|z)] - D_{\textrm{KL}}(q_{\phi}(z|x)||p_{\theta}(z))\\
\end{align*}

We can get a conclusion that maximizing ELBO is equivalent to minimizing the KL divergence through the above equation. Fianlly, the log-likelihood can be rewritten as follows:
\begin{align*}
	\log p_{\theta}(x) = \mathcal{L}(\phi,\theta) + D_{\textrm{KL}}(q_{\phi}(z|x)||p_{\theta}(z|x))
\end{align*}


%\section{Variational Lower Bound}
%Function $g(\xi, x)$ is called variational lower bound for function $f(x)$ iff
%\begin{itemize}
%	\item For all $\xi$ for all $x$ if follows $f(x)\geq g(\xi, x)$
%	\item For any $x_0$ there exists $\xi(x_0)$ such that $f(x_0)=g(\xi(x_0), x_0)$
%\end{itemize} 

\subsection{Expectation Maximization}
We want to maximize ELBO, $\mathcal{L}(q,\theta)$ to minimize KL divergence between $q(Z)$ and $\log p(Z|X,\theta)$.
$$\max_{q,\theta}\mathcal{L}(q,\theta) = \max_{q,\theta}\int q(Z)\log \frac{p(X,Z|\theta)}{q(Z)}dZ.$$
We start from initial point $\theta_0$ and iteratively repeat \Ni E-step and \Nii M-step, iteratively:
\begin{itemize}
	\item E-Step: $\theta_0$ is fixed. 
		$$q(Z) = \argmax_{q}\mathcal{L}(q,\theta) = \argmin_{q}\textrm{KL}(q(Z)|p(Z|X,\theta)) = p(Z|X,\theta_0).$$ 
		\begin{itemize}
			\item This is because, maximizing ELBO is equal to minimizing KL divergence and the minimum $q$ can be achieved when $q$ is equal to $p(Z|X,\theta_0)$.
			\item Now, we just have to evaluate $p(Z|X,\theta_0)$.
		\end{itemize}
	\item M-Step: $q$ is fixed.
		$$\theta_* = \argmax_{\theta}\mathcal{L}(q,\theta) = \argmax_{\theta}\mathbb{E}_{q(Z)}[\log p(X,Z|\theta)]$$
		\begin{itemize}
			\item Can be accomplished by taking derivatives
			\item Set $\theta_0=\theta_*$ and go to the E-Step until convergence
		\end{itemize}
	
\end{itemize}

\subsection{Categorical Latent Variables}
$z_i \in \{1,...,K\}$
$$p(x_i|\theta) = \sum_{k=1}^{K}p(x_i|k,\theta)p(z_i=k|\theta)$$
is simply a finite mixture of distributions. 

E-Step:
$$q(z_i=k) = p(z_i=k|x_i,\theta) = \frac{p(x_k|z_i=k,\theta)p(z_i=k|\theta)}{\sum_{l=1}^{K}p(x_i|z_i=l,\theta)p(z_i=l|\theta)}$$
M-Step:
$$\argmax_{\theta}\mathbb{E}_{q(Z)}[\log p(X,Z|\theta)] = \sum_{i=1}^{n}\mathbb{E}_{q(z_i)}[\log p(x_i,z_i|\theta)] = \sum_{i=1}^{n}\sum_{k=1}^{K}q(z_i=k)\log p(x_i,k|\theta)$$

For GMM, we model $p(x|z)$ as Gaussian.

%\subsection{Continous Latent Variables}
%Continuous latent variables can be regarded as a mixture model of continous distributions. 
%$$p(x_i|\theta) = \int p(z_i|x_i,\theta) dz_i = \int p(x_i|z_i,\theta)p(z_i|\theta) dz_i$$
%E-step can be done in a closed from only in case of conjugate distributions, otherwise the true posterior is intractadble.
%$$q(z_i) = p(z_i|x_i,\theta) = \frac{p(x_k|z_i,\theta)p(z_i|\theta)}{\int p(x_i|z_i,\theta)p(z_i|\theta)dz_i}$$
%
%Typically, continuous latent variables are used for dimension reduction techniques also known as \textbf{representation learning.}

% \part{Deep Generative Models}
\chapter{Hidden Markov Models}
\section{Introduction}
The HMM is based on the Markov chain assumption. A Markov chain is a model
that tells us something about the probabilities of sequences of random variables,
states, each of which can take on values from some set. These sets can be words, or
tags, or symbols representing anything, like the weather.

There are two important assumptions:
\begin{itemize}
	\item Markov assumption
	\item Output independence: $p(x_i|z_1,\dots,z_i,\dots,z_T,x_1,\dots,x_i,\dots,x_T) = p(x_i|z_i)$
\end{itemize}

\subsection{Conditional Independence}
If two events $A$ and $B$ are \textbf{conditionally independent} given an event $C$ then,
\begin{itemize}
	\item $P(A\cap B|C) = P(A|C)P(B|C)$. 
	\item $P(A|B,C) = P(A|C)$
\end{itemize}

\subsection{Notation}

\begin{itemize}
	\item $X = (x_i, x_2,\dots, x_T)$
	% \item $x_i\in\{c_1,...,c_m\}$
	\item Initial state probabilities: $p(z_1) \sim \textrm{Multinomial}(\pi_1,...,\pi_k)$, need to learn $\pi$
	\item Transition probability:
	$$p(z_t|z_{t-1}=i)\sim \textrm{Multinomial}(a_{i,1},...,a_{i,k})$$
	, where $a_{i,j} = p(z_t=j|z_{t-1}=i)$ and $i$ and $j$ denote clusters or states, respectively.
	\item Emission probability:
	$$p(x_t|z_{t}=i)\sim \textrm{Multinomial}(b_{i,1},...,b_{i,m})$$
	, where $b_{i,j} = p(x_t=j|z_{t}=i)$
\end{itemize}


\section{Bayesian Network}
\subsection{Bayes Ball}

\begin{figure}[h!]
	\centering
	\includegraphics[scale=0.3]{./images/hmm/bayes.jpg}
	\caption{Bayes ball}
	\label{fig:bayes}
\end{figure}

\begin{itemize}
	\item Cascading: $P(Z|Y,X) = P(Z|Y)$. The information of $Y$ decouples $X$ and $Z$.
	\item Common parent: $P(X,Z|Y) = P(X|Y)P(Z|Y)$. The information of $Y$ decouples $X$ and $Z$.
	\item V-Structure (common child): Unlike the above two cases, the information of $Y$ couples $X$ and $Z$.
		$$P(X,Y,Z) = P(X)P(Y)P(Y|X,Z).$$
\end{itemize}

\subsection{Potential Function}
Potential function is a function which is not a probability function, but it can become a probability function by normalizing it. 
$$P(A,B,C,D) = P(A|B)P(B|C)P(C|D)P(D)$$

\begin{figure}[h]
	\centering
	\includegraphics[scale=0.5]{./images/hmm/cascade.pdf}
\end{figure}

\begin{itemize}
	\item Cliques: $\Psi(a,b), \Psi(b,c), \Psi(c,d)$
	\item Separators $\phi(b), \phi(c)$
\end{itemize}
Given a clique tree with cliques and separators, the joint probability distribution is defined as follows:
% By using potential functions, we can express the joint probability as
\begin{align*}
	P(A,B,C,D) &= P(U) = \frac{\prod_N \Psi(N)}{\prod_L\phi(L)}= \frac{ \Psi(a,b)\Psi(b,c)\Psi(c,d)}{\phi(b)\phi(c)}\\
\end{align*}
An effect of an observation propagates through the clique graph $\to$ \textbf{Belief propagation}. How to propagate the belief? \textbf{Absorption rule}!

Let's say we have some new observations about $A$, then it affects the clique $\Psi(a,b)$. The updated clique is now $\Psi^*(a,b)$. Similarly, $\phi^*(b) = \sum_A\Psi^*(a,b)$. Subsequently, $\Psi^*(b,c) = \Psi^(b,c)\frac{\phi^*(b)}{\phi(b)}$.

\newpage
\section{Hidden Markov Models}
\label{sec:hmm}

\begin{figure}[h]
	\begin{center}
		\includegraphics[scale=0.7]{./images/hmm/hmm_figure.pdf}
	\end{center}
	\caption{HMM Structure}
	\label{fig:HMM}
\end{figure}

The observation can be discrete or continuous. If the latent factors are continuous, then HMM is often referred as \textbf{Kalman filter}. 

\begin{itemize}
	\item Initial state probability: $P(z_1)\sim \textrm{Mult}(\pi_1, \dots, \pi_k)$
	\item Transition probability: $P(z_t|z^i_{t-1}=1)\sim \textrm{Mult}(a_{i,1}, \dots, a_{i,k})$, \\ where $P(z_t^j=1|z_{t-1}^i=1) = a_{i,j}$
	\item Emission probability: $P(x_t|z_t^i=1)\sim \textrm{Mult}(b_{i,1}, \dots, b_{i,m})\sim f(x_t|\theta_i)$,\\ where $P(x_t^j=1|z_{t}^i=1) = b_{i,j}$. The probability of observing $x_j$ at the  $i$-th cluster. 
\end{itemize}
Note that $i$ and $j$ are indices of clusters. 

There are three main problems in HMM:
\begin{enumerate}
	\item Evaluation Questions (likelihood): %Forward algorithm
	\begin{itemize}
		\item Given $\boldsymbol{\pi}\mathbf{, a, b}, X$
		\item Find $p(X|M, \boldsymbol{\pi}\mathbf{, a, b})$
		\item How much are $X$ likely to be observed by a model $M$?
	\end{itemize}
	
	\item Decoding Questions:
	\begin{itemize}
		\item Given $\boldsymbol{\pi}\mathbf{, a, b}, X$
		\item Find $\argmax_Z p(Z|X, M, \boldsymbol{\pi}\mathbf{, a, b})$
		\item What is the most probable sequence of $Z$ (latent states)? 
	\end{itemize}
	
	\item Learning Questions: Forward-Backward (Baum-Welch)
	\begin{itemize}
		\item Given $X$
		\item Find $\argmax_{\boldsymbol{\pi}\mathbf{, a, b}} p(X|M, \boldsymbol{\pi}\mathbf{, a, b})$
		\item What would be the optimal model parameters? 
	\end{itemize}
\end{enumerate}

% For a given hidden state, we can easily compute the output likelihood.

\section{Evaluation: Forward-Backward Probability}
% The forward–backward algorithm is an inference algorithm for hidden Markov models which computes the posterior marginals of all hidden state variables given a sequence of observations/emissions

% The term forward–backward algorithm is also used to refer to any algorithm belonging to the general class of algorithms that operate on sequence models in a forward–backward manner.

% In the first pass, the forward–backward algorithm computes a set of forward probabilities which provide, for all $t\in \{1,\dots ,T\}$, the probability of ending up in any particular state given the first $t$ observations in the sequence, i.e. $P(X_{t}\ |\ o_{1:t})$. In the second pass, the algorithm computes a set of backward probabilities which provide the probability of observing the remaining observations given any starting point $t$, i.e. $P(o_{t+1:T}\ |\ X_{t})$. These two sets of probability distributions can then be combined to obtain the distribution over states at any specific point in time given the entire observation sequence:

% These two sets of probability distributions can then be combined to obtain the distribution over states at any specific point in time given the entire observation sequence:
% $$P(X_{t}\ |\ o_{1:T})=P(X_{t}\ |\ o_{1:t},o_{t+1:T})\propto P(o_{t+1:T}\ |\ X_{t})P(X_{t}|o_{1:t})$$
% The forward–backward algorithm can be used to find the most likely state for any point in time. However, It cannot be used to find the most likely sequence of states.

\subsection{Joint Probability}
We can factorize the joint distribution of HMM in \Cref{fig:HMM} by using a Bayesian approach as follows:. 
\begin{align}
	p(X,Z) &= p(x_1,\dots,x_t, z_1,\dots,z_t)\\ 
		   &= p(z_1)p(x_1|z_1)p(z_2|z_1),\dots,p(x_{t}|z_{t})p(z_{t}|z_{t-1})
\end{align}
The key assumption involved in factorizing the Markov chain within a Hidden Markov Model (HMM) is \textit{conditional independence} among certain components of the state variables. Here's a detailed breakdown of what this assumption means:
\begin{itemize}
	\item Independence of State Components: The transition of each component $z_t^k$ only depends on its corresponding previous component $z_{t-1}^k$ and is independent of other components.
\end{itemize}
As the number of latent factor increases, it is getting harder to decode the latent factors. 

\subsection{Marginal Probability}
% \subsection{Forward Probability}
We want to compute the likelihood of sequence $X$ which is given by
$$p(X|\boldsymbol{\pi}\mathbf{, a, b}) = \sum_Z p(X, Z|\boldsymbol{\pi}\mathbf{, a, b})$$
The computation can be done as follows:
\begin{align*}
	p(X) &= \sum_Z p(X,Z)\\
	& = \sum_{z_1}\dots\sum_{z_t}p(x_1,\dots,x_t,z_1,\dots,z_t)\\
	& = \sum_{z_1}\dots\sum_{z_t}\pi_{z_{1}}\prod_{t=2}^{T}a_{z_{t-1},z_t}\prod_{t=1}^{T}b_{z_{t},x_t}
\end{align*}
The last step is done by using the factorization above. The computation of this equation requires lots of computations, so we will change it into a \textbf{recursive form} by using the factorization rule $p(a,b,c) = p(a)p(b|a)p(c|a,b)$. 
\begin{align}
	p(&x_1,\dots,x_t,z_t^k=1) = \sum_{z_{t-1}}p(x_1,\dots,x_{t-1}, x_t,z_{t-1},z_t^k=1)\\
	&= \sum_{z_{t-1}} p(\underbrace{x_1,\dots,x_{t-1}, z_{t-1}}_{a}, \underbrace{x_t}_{c}, \underbrace{z_t^k=1}_{b})\\
	& = \sum_{z_{t-1}} p(x_1,\dots,x_{t-1},z_{t-1}) p(z_t^k=1|x_1,\dots,x_{t-1},z_{t-1})p(x_t|z_t^k=1, x_1,\dots,x_{t-1},z_{t-1}) \\
	&\hspace{0.5cm} \because p(a,b,c) = p(a)p(b|a)p(c|a,b) \textrm{ or by the structure of HMM}\nonumber\\ 
	& = \sum_{z_{t-1}} p(x_1,\dots,x_{t-1},z_{t-1}) p(z_t^k=1|z_{t-1}) p(x_t|z_t^k=1)\\
	& = p(x_t|z_t^k=1) \sum_{z_{t-1}} p(x_1,\dots,x_{t-1},z_{t-1}) p(z_t^k=1|z_{t-1}) \\
	& = b_{z^k_t,x_t} \sum_{z_{t-1}} p(x_1,\dots,x_{t-1},z_{t-1}) a_{z_{t-1},z_t^k}
	\label{eq:hmm_eval_fact}
\end{align}
\begin{itemize}
	\item In the second line, the $x_{t-1}$ and $z_{t-1}$ are grouped together. 
	\item Then, we can find the HMM structure by factorizing the equation. 
	\item In the fourth line, $x$ terms are removed, since $z_t$ only relies on $z_{t-1}$ by the Markov assumption. Similarly, $x_t$ only depends on $z_t$. We can interpret this by using Bayes ball too. 
\end{itemize}
% In the fifth step, we assume that $z_t=k$ is given, thus by Markov assumption, we only need $z_{t-1}$. 
Now we can find a recursive structure of $p(x_1,\dots,x_{t},z_{t}^k=1)$ as follows:
$$\alpha_t^k = p(x_1,\dots,x_{t},z_{t}^k=1) = b_{k,x_t}\sum_i \alpha_{t-1}^ia_{i,k}$$
, where \textbf{$\alpha_t^k$ is the probabilities of being in state $k$ after observing the first $t$ observations.} Thus, 
\begin{align*}
	p(x_1,\dots,x_{t}) & = \sum_{\mathbf{z}} p(x_1,\dots,x_{t},z)\\
	& = \sum_{k} \alpha_t^k
\end{align*}
% \begin{itemize}
% 	\item $\alpha_t^k$: \textbf{Forward probability}. Probabilities of being in state $k$ after observing the first $t$ observations.
% 	% \item $a_{i,k}$: transition probability
% 	% \item $b_{k,x_t}$: observation (or emission) probability
% \end{itemize}
Note that $\alpha_t^k$ is also called \textbf{Forward probability}.

\subsection{Forward Algorithm}
Forward probability solves the evaluation problem. Essentially, this is a dynamic programming, so it calculates required values in a bottom-up manner. 
\begin{itemize}
	\item Forward probability: $\alpha_t^k$, $Time\times States$
\end{itemize}
%\LinesNumbered
\begin{algorithm}
	Create a probability matrix $forward[M,T] = \alpha_t^k$\\
	Initialization: \\
	\For {\textrm{each state} k=1,...,M}{
		$\alpha_1^k\leftarrow \pi_kb_{k,x_1}$
	}
	\For {\textrm{time step} t=2,...,T}{
		\For {\textrm{each step} k=1,...,M}{
			$\alpha_t^k = b_{k,x_t}\sum_i \alpha_{t-1}^ia_{i,k}$
			}
		
	}
	Return $p(X) = \sum_i^M \alpha_T^i$
	\caption{Forward Algorithm}
	\label{algo:forward_algorithm}
\end{algorithm}
%\begin{algorithm}
%	Init: $\alpha_1^k = b_{k,x_1}\pi_k$\\
%	\For{t=1,...,T}{
%		$\alpha_t^k = b_{k,x_t}\sum_i \alpha_{t-1}^ia_{i,k}$
%	}
%	Return $p(X) = \sum_i\alpha_T^i$
%	\caption{Forward Algorithm}
%	\label{algo:forward_algorithm}
%\end{algorithm}
Note again that 
$$p(X) = p(x_1,...,x_T) =\sum_i\alpha_T^i = \sum_i p(x_1,...,x_T, z_T^i=1)$$
Note also that the forward-algorithm returns $p(X)$ and forward probability is the probability of being in state $k$ after observing the first $t$ observations without $Z$. 

\subsection{Backward Probability}
The forward probability only considers an observation at $t$. To determine the $z_t$, we need to leverage the future observations. \textbf{The backward probability $\beta$ is the probability of seeing the observations from time $t+i$ to the end, given that we are in state $k$ at time $t$.} 
$$\beta_t^k = p(x_{t+1},\dots,x_T|z_t^k=1)$$
We want to compute $p(z_t^k=1|X)$ rather than $p(x_1,\dots,x_t, z_t^k=1)$. In other words, we will leverage the whole observations $X$. 
\begin{align*}
	p(z_t^k=1,X) &= p(x_1,\dots,x_t, z_t^k=1, x_{t+1},\dots,x_T)\\
	& = p(x_1,\dots,x_t, z_t^k=1)p(x_{t+1},\dots,x_T|x_1,\dots,x_t, z_t^k=1)\\
	& = p(x_1,\dots,x_t, z_t^k=1)p(x_{t+1},\dots,x_T|z_t^k=1)\\
	& = \alpha_{t}^k\beta_{t}^k
\end{align*}
We already know that $p(x_1,\dots,x_t, z_t^k=1) = \alpha_t^k$. We just need to compute backward probability as follows:
\begin{align*}
	\beta_t^k &= p(x_{t+1},\dots,x_T|z_t^k=1)\\
	& = \sum_{z_{t+1}}p(\underbrace{z_{t+1}}_{a}, \underbrace{x_{t+1}}_b,\underbrace{x_{t+2},\dots,x_T}_c|z_t^k=1)\\
	& = \sum_{i} p(z_{t+1}^i=1|z_t^k=1)p(x_{t+1}|z_{t+1}^i=1,z_t^k=1)p(x_{t+2},\dots,x_T|x_{t+1},z_{t+1}^i=1,z_t^k=1)\\
	& \because p(a,b,c) = p(a)p(b|a)p(c|a,b)\\
	& = \sum_{i} p(z_{t+1}^i=1|z_t^k=1)p(x_{t+1}|z_{t+1}^i=1)p(x_{t+2},\dots,x_T|z_{t+1}^i=1)\\
	& = \sum_{i}a_{k,i}b_{i,x_{t+1}} \beta_{t+1}^i
\end{align*}

Another recursive structure:
\begin{align*}
	p(z_t^k=1,X) &= \alpha_{t}^k\beta_{t}^k\\
	& = b_{k,x_t}\sum_i \alpha_{t-1}^ia_{i,k} \times \sum_{i}a_{k,i}b_{i,x_{t}} \beta_{t+1}^i
\end{align*}
This means at time $t$, the latent label is belong to some class $k$ and this can be computed by using the forward probability and the backward probability. Now we can compute
\begin{align*}
p(z_t^k=1|X) &= \frac{p(z_t^k=1,X)}{p(X)} = \frac{\alpha_{t}^k\beta_{t}^k}{p(X)}
\end{align*}
Then, 
$$k_t = \argmax_{k}p(z_t^k=1|X)$$
Note that this is for a single latent variable at a single time step given the whole observation $X$, but we want to decode a sequence of latent variables. Thus, we need some decoding algorithm.

\section{Decoding: Viterbi Algorithm}
For any model, such as an HMM, that contains hidden variables, \textbf{the task of determining which sequence of variables is the underlying source of some sequence of observations is called the decoding task}.

We might propose to find the best sequence as follows: 
\begin{enumerate}
	\item For each possible hidden state sequence (HHH, HHC, HCH, etc.), we could run the forward algorithm and compute the likelihood of the observation sequence given that hidden state sequence.
	\item Then, we could choose the hidden state sequence with the maximum observation likelihood.
\end{enumerate}  
However, this is not a feasible solution, because there are an exponentially large number of state sequences.

Instead, the most common decoding algorithms for HMMs is the \textbf{Viterbi algorithm}. Like the forward algorithm, \textbf{Viterbi} is a kind of \textbf{dynamic programming algorithm.}

Note that the Viterbi algorithm is identical to the forward algorithm except that it takes the \textbf{max} over the previous path probabilities whereas the forward algorithm takes the \textbf{sum}. This is because, we want to obtain \textbf{the most probable latent variable sequence}. Note also that the Viterbi algorithm has one component that the forward algorithm doesn't have: \textbf{backpointers}. The reason is that while the forward algorithm needs to produce an observation likelihood, the Viterbi algorithm must produce a probability and also the most likely state sequence. We compute this best state sequence by keeping track of the path of hidden states that led to each state and then at the end backtracing the best path to the beginning (the Viterbi backtrace).

We can leverage the forward-backward probabilities:
\begin{itemize}
	\item $k^* = \argmax_{k}p(z_t^k=1|X) = \argmax_{k}p(z^k_t=1,X) = \argmax_{k}\alpha_{t}^k\beta_{t}^k$
\end{itemize}
We will use a forward approach:
% \setcounter{equation}{0}
\begin{align}
	V_t^k &= \max_{z_1,\dots,z_{t-1}}p(x_1,\dots,x_{t-1},z_1,\dots,z_{t-1},x_t,z_t^k=1)\\ 
	& = \max_{z_1,\dots,z_{t-1}}p(x_t,z_t^k=1|x_1,\dots,x_{t-1},z_1,\dots,z_{t-1})p(x_1,\dots,x_{t-1},z_1,\dots,z_{t-1})\\
	& = \max_{z_1,\dots,z_{t-1}}p(x_t,z_t^k=1|z_{t-1})p(x_1,\dots,x_{t-2},z_1,\dots,z_{t-2}, x_{t-1}, z_{t-1})\\
	& = \max_{z_{t-1}}p(x_t,z_t^k=1|z_{t-1})\max_{z_1,\dots,z_{t-2}}p(x_1,\dots,x_{t-2},z_1,\dots,z_{t-2}, x_{t-1}, z_{t-1})\\
	& = \max_{i\in z_{t-1}}p(x_t,z_t^k=1|z_{t-1}^i=1)V_{t-1}^i\\
	& = \max_{i\in z_{t-1}}p(x_t|z_t^k=1)p(z_t^k=1|z_{t-1}^i=1)V_{t-1}^i\\
	& = p(x_t|z_t^k=1)\max_{i\in z_{t-1}}p(z_t^k=1|z_{t-1}^i=1)V_{t-1}^i\\
	& = b_{k,x_t}\max_{i\in z_{t-1}}a_{i,k}V_{t-1}^i
\end{align}
\begin{itemize}
	\item $V_{t}^k$ is Viterbi variable which denotes the probability that the HMM is in state $k$ at $t$ after observing the first $t$ observations and $t-1$ latent variables. In another words, this is the probability of most likely sequence of states ending at state $z_t=k$.
	\item The first line assumes that the observation at time $t$ and the latent variable are fixed and also the fourth line has the recursive structure.
	\item The third step, only $z_{t-1}$ can affect the $z_{t}$, so we can remove all other unnecessary variables.
	\item The step six can be derived by the HMM structure. 
	\item $i\in z_{t-1}$ simply denotes the index of potential cluster at $t-1$.
	\item We have already computed the backward and the forward probabilities. So we just need to apply the Viterbi algorithm. 
	% \item $\textrm{idx}(x_t)$
\end{itemize}

Note that  Also note that we present the most probable path by taking the maximum over all possible previous state sequences $\max_{z_1,\dots,z_{t-1}}$. Like other DP-algorithm, Viterbi fills each cell recursively. 

%\LinesNumberedHidden
\begin{algorithm}
	$V_t^k = viterbi[M,T]$, where $M$ is the number states\\
	% Initialization: $\pi$ is the initial probability of being state $k$\\
	\For{k=1,\dots,M}{
		$V_1^k \leftarrow \pi_{z_k}b_{k,x_1}$\\
		$backpointer[k,1]\leftarrow 0$
	}
	\For{t=2,\dots,T}{
		\For{k=1,\dots,M}{
			$V_t^k \leftarrow b_{k,x_t}\max_{k'} V_t^{k'}a_{k',k}$, where $k'$ is the previous state.\\
			$backpointer[k,t]\leftarrow b_{k,x_t}\argmax_{k'} V_t^{k'}a_{k',k}$
		}
	}
	$bestpathprob \leftarrow \max_{k}V_T^{k}$ \quad //termination step
	
	$bestpathpointer \leftarrow \argmax_{k}V_T^{k}$ \quad//termination step
	
	$bestpath \leftarrow $ the path starting at state $bestpathpointer$, that follows backpointer[] to states back in time
	
	Return $bestpathpointer$, $bestpathprob$

	\caption{Viterbi Algorithm}
	\label{algo:viterbi}
\end{algorithm}

Viterbi algorithm typically shows some technical issues:
\begin{itemize}
	\item Underflow problems $\to$ log $V$.
\end{itemize}

\section{Learning: Baum-Welch Algorithm}
We have to learn HMM parameters with only $X$. Baum-Welch algorithm or Forward-Backward Algorithm is a standard training algorithm for HMM. The algorithm let us train both the transition and the emission probabilities of the HMM. If we do not have the information about $Z$, then we can assign the most probable $Z$ given $X$.

\begin{itemize}
	\item Given $X$, estimate parameters $\pi, a, b$.
		% $$\theta^* = \argmax_\theta \ln \sum_Z P(X,Z|\theta).$$
	\item Then, find the most probable $Z$ given the parameters. 
	% \item We don't have $Z, \pi, a, b$, so we need to find out them.
\end{itemize}
We will use EM algorithm!

\subsection{EM Algorithm}
\begin{align*}
	P(X|\theta) = \sum_Z P(X,Z|\theta) \to \ln P(X|\theta) = \ln \sum_Z P(X,Z|\theta).
\end{align*}
We cannot directly estimate the log-likelihood function, so we will estimate the expectation of it. 
\begin{align*}
	Q(\theta, \theta^{old}) &= \mathbb{E}_{Z}\ln P(X,Z|\theta) \\
							&= \sum_Z p(Z|X,\theta^{old})\ln P(X,Z|\theta)\\
							&= \sum_Z p(Z|X,\pi^t, a^t, b^t)\ln P(X,Z|\pi, a, b).
\end{align*}
Note that $p(X,Z) = \pi_{z_{1}}\prod_{t=2}^{T}a_{z_{t-1},z_t}\prod_{t=2}^{T}b_{z_{t},x_t}$. Thus, $\ln p(X,Z) = \ln \pi_{z_{1}}+\sum_{t=2}^{T}\ln a_{z_{t-1},z_t}+\sum_{t=1}^{T}\ln b_{z_{t},x_t}$. Therefore
$$Q(\theta, \theta^{old}) = \sum_Z p(Z|X, \theta^{old}) \bigg(\ln \pi_{z_{1}}+\sum_{t=2}^{T}\ln a_{z_{t-1},z_t}+\sum_{t=1}^{T}\ln b_{z_{t},x_t}\bigg).$$
To optimize the above function we will use the Lagrange method as follows: 
$$\mathcal{L}(\pi, a, b) = Q(\theta, \theta^{old}) - \lambda_\pi \bigg(\sum_{i=1}^K\pi_i-1\bigg) - \sum_i^K\lambda_{a_i} \bigg(\sum_{j=1}^Ka_{i,j}-1\bigg) - \sum_i^K\lambda_{b_i} \bigg(\sum_{j=1}^Kb_{i,j}-1\bigg).$$
The constraints are for forcing the sum of each probability is equal to 1. 

Now, take a partial derivative for each parameter. Let's take a derivative with regard to $\pi_i$ first. Then, 
\begin{align*}
	\frac{\partial \mathcal{L}}{\partial \pi_i} &= \frac{\partial Q(\theta, \theta^{old})}{\partial \pi_i} - \lambda_\pi\\
												&= \frac{\partial }{\partial \pi_i}\sum_Z p(Z|X, \theta^{old}) \ln \pi_{z_{1}} - \lambda_\pi\\
												&= \frac{p(z_1^i=1|X, \theta^{old})}{\pi_i} - \lambda_\pi\\
	\frac{\partial \mathcal{L}}{\partial \lambda_{\pi_i}} &= \sum_{i=1}^K\pi_i - 1 = 0 \to \sum_{i=1}^K\pi_i = 1.
\end{align*}
By setting the derivative is equal to zero, 
\begin{align*}
 \pi_i = \frac{p(z_1^i=1|X, \theta^{old})}{\lambda_\pi}. 
\end{align*}
By using the constraint of $\pi$, the Lagrange multiplier $\lambda_\pi$ must be a normalizer. 
\begin{align*}
	\pi_i = \frac{p(z_1^i=1|X, \theta^{old})}{\sum_{j=1}^K p(z_1^j=1|X, \theta^{old})}. 
\end{align*}
Similarly, we can compute other parameters too. 
\begin{align*}
	a^{t+1}_{i,j} &= \frac{\sum_{t=2}^T p(z_{t-1}^i=1, z_t^j=1|X, \theta^{old})}{\sum_{t=2}^T p(z_{t-1}^i=1|X, \theta^{old})}.\\ 
	b^{t+1}_{i,j} &= \frac{\sum_{t=1}^T p(z_{t1}^i=1|X, \theta^{old})I(x_t=j)}{\sum_{t=1}^T p(z_{t}^i=1|X, \theta^{old})}, 
\end{align*}
where $I(x)$ is an indicator function which returns 1 if $x$ is true and 0, otherwise. 



\section{Python Implementation}
\label{sec:hmm_python}

\subsection{Viterbi Algorithm}
The Viterbi algorithm is a dynamic programming algorithm used to determine the most probable sequence of hidden states in a Hidden Markov Model (HMM) based on a sequence of observations. 

The algorithm works by recursively computing the probability of the most likely sequence of hidden states that ends in each state for each observation.

At each time step, the algorithm computes the probability of being in each state and emits the current observation based on the probabilities of being in the previous states and making a transition to the current state.

Assuming we have an HMM with N hidden states and T observations, the Viterbi algorithm can be summarized as follows:

    Initialization: At time t=1, we set the probability of the most likely path ending in state i for each state i to the product of the initial state probability pi and the emission probability of the first observation given state i. This is denoted by: delta[1,i] = pi * b[i,1].
    Recursion: For each time step t from 2 to T, and for each state i, we compute the probability of the most likely path ending in state i at time t by considering all possible paths that could have led to state i. This probability is given by:

$$delta[t,i] = max_j(delta[t-1,j] * a[j,i] * b[i,t])$$

Here, a[j,i] is the probability of transitioning from state j to state i, and b[i,t] is the probability of observing the t-th observation given state I.

We also keep track of the most likely previous state that led to the current state i, which is given by:

$$psi[t,i] = argmax_j(delta[t-1,j] * a[j,i])$$

\begin{itemize}
	\item Termination: The probability of the most likely path overall is given by the maximum of the probabilities of the most likely paths ending in each state at time $T$. That is, $P* = max_i(delta[T,i])$.
	\item Backtracking: Starting from the state $i*$ that gave the maximum probability at time $T$, we recursively follow the psi values back to time $t=1$ to obtain the most likely path of hidden states.
\end{itemize}

The Viterbi algorithm is an efficient and powerful tool that can handle long sequences of observations using dynamic programming.


% \begin{lstlisting}[language=Python]
% import torch.optim as optim
% epsilon = 2./255

% delta = torch.zeros_like(pig_tensor, requires_grad=True) # init delta
% opt = optim.SGD([delta], lr=1e-1) # Update delta

% for t in range(30):
%     pred = model(norm(pig_tensor + delta))
% 	# For gradient ascent -CELoss
%     loss = -nn.CrossEntropyLoss()(pred, torch.LongTensor([341])) 
%     if t % 5 == 0:
%         print(t, loss.item())

%     opt.zero_grad()
%     loss.backward()
%     opt.step()
%     delta.data.clamp_(-epsilon, epsilon) # infinity norm

% print("True class probability:", nn.Softmax(dim=1)(pred)[0,341].item())
% \end{lstlisting}

\section{Summary}
\begin{itemize}
	\item The loss function can be decomposed.
		$$L_\text{VLB} = L_T + L_{T-1} + \dots + L_0 .$$
	\item $L_T = D_\text{KL}(q(\mathbf{x}_T \vert \mathbf{x}_0) \parallel p_\theta(\mathbf{x}_T))$
		\begin{itemize}
			\item Constant $\approx 0$ since $x_T$ is a Gaussian noise.
		\end{itemize}
	\item $L_t = D_\text{KL}(q(\mathbf{x}_{t-1} \vert \mathbf{x}_{t}, \mathbf{x}_0) \parallel p_\theta(\mathbf{x}_{t-1} \vert\mathbf{x}_{t})) \text{ for } t>1 $
	%\item $L_t = D_\text{KL}(q(\mathbf{x}_t \vert \mathbf{x}_{t+1}, \mathbf{x}_0) \parallel p_\theta(\mathbf{x}_t \vert\mathbf{x}_{t+1})) \text{ for }1 \leq t \leq T-1 $
		\begin{itemize}
			\item This is the main part.
		\end{itemize}
	\item $L_0 = - \log p_\theta(\mathbf{x}_0 \vert \mathbf{x}_1)$
		\begin{itemize}
			\item Can be modeled by a separate decoder.
		\end{itemize}
	\item $q(\rvx_t|\rvx_0) = \mathcal{N}(\sqrt{\bar{\alpha}_t}\rvx_0, (1-\bar{\alpha}_t)I )$
	\item $q(\rvx_t \vert \rvx_{t-1}) &= \mathcal{N}(\mathbf{x}_t; \sqrt{1 - \beta_t}\rvx_{t-1}, \beta_t\mathbf{I}),$
		\begin{itemize}
			\item[] We can sample by $\rvx_t=\sqrt{1 - \beta_t}\rvx_{t-1}+ \sqrt{\beta_t}\epsilon$
		\end{itemize}
	\item $p_\theta(\mathbf{x}_{t-1} \vert \mathbf{x}_t) = \mathcal{N}(\mathbf{x}_{t-1}; \boldsymbol{\mu}_\theta(\mathbf{x}_t, t), \boldsymbol{\Sigma}_\theta(\mathbf{x}_t, t))$.
		\begin{itemize}
			\item We need to learn mean and variance.
			\item DDPM kept the variance fixed and let the neural network only learn the mean $\mu_\theta$.
			\item $\boldsymbol{\Sigma}_\theta(\mathbf{x}_t, t)) = \sigma_t^2\mathbf{I}$ and set $\sigma_t^2 = \beta_t$.
		\item Improved DDPM model trains $\sigma$ also.
		\end{itemize}
	\item One can reparameterize the mean to make the nerual network learn the added noise via a network $\epsilon_\theta$.
		$$\mu_\theta(\rvx_t, t) = \frac{1}{\sqrt{\alpha_t}}\bigg(\rvx_t - \frac{\beta_t}{\sqrt{1-\bar{\alpha}_t}}\underbrace{\epsilon_\theta(\rvx_t,t)}_{\text{Network}}\bigg)$$
	\item Final objective function $L_t$ is
		$$||\epsilon -\epsilon_\theta(\rvx_t,t)||^2 = ||\epsilon -\epsilon_\theta(\sqrt{\bar{\alpha}_t}\rvx_0+\sqrt{(1-\bar{\alpha}_t)}\epsilon,t)||^2 $$
		\begin{itemize}
			\item $t\sim\text{Unif}[\{1,..,T\}]$ 
			\item $\rvx_t=\sqrt{\bar{\alpha}_t}\rvx_0+\sqrt{(1-\bar{\alpha}_t)}\epsilon \sim q(\rvx_t|\rvx_0)$
			\item $\epsilon\sim \mathcal{N}(0,I)$
		\end{itemize}
	\item $\rvx_t$ is perturbed by $\epsilon$ and the noise prediction network $\epsilon_\theta$ predicts $\epsilon$.
\end{itemize}

\newpage
\begin{algorithm}[t]
	\caption{Training}
	\label{alg:diffusion_training}
	\Repeat{\textrm{converged}}{
			$\rvx_0\sim q(\rvx_0)$\\
			$t\sim \text{Unif}[\{1,\cdots,T\}]$\\
			$\epsilon\sim \mathcal{N}(0,I)$\\
			Take gradient descent step on
			$\nabla_\theta||\epsilon -\epsilon_\theta(\sqrt{\bar{\alpha}_t}\rvx_0+\sqrt{(1-\bar{\alpha}_t)}\epsilon,t)||^2$
		}
\end{algorithm}

The training process is given by
\begin{enumerate}
	\item $\rvx_0\sim q(\rvx_0)$ 
	\item Sample a noise level $t$ between $1$ and $T$ (\ie random time step).
	\item Sample a noise from a Gaussian distribution and perturb the input by the sampling equation.
	\item NN is trained to predict this noise $\epsilon$ used for generating $\rvx_t$.
	\item $\beta$ is often scheduled linearly.
	\item $\Sigma$ is set equal to $\beta$.
\end{enumerate}

The sampling process is given by
\begin{algorithm}[h]
	\caption{Sampling}
	\label{alg:diffusion_sampling}
		$\rvx_T\sim \mathcal{N}(0,I)$\\
		\For{$t=T,\cdots,1$}{
			$\rvz\sim \mathcal{N}(0,I)$\\
			\State $\rvx_{t-1}= \frac{1}{\sqrt{\alpha_t}}\bigg(\rvx_t-\frac{1-\alpha_t}{\sqrt{1-\bar{\alpha}_t}}\boldsymbol{\epsilon}_\theta(\rvx_t,t)\bigg)+\Sigma_t\rvz$
		}
		\textbf{return} $\rvx_0$
\end{algorithm*}
\begin{itemize}
	\item Ancestral sampling.
	\item $T$ is typically around 1,000
\end{itemize}
	


\chapter{Explicit Generative Models}
\section{Variational Autoencoder}

Our goal is to find the data distribution $p(X)$. \Cref{fig:dgm} represents a general structure of deep generative model. As you can see, we first sample $z\sim p(z)$ and feed it into a deep neural network $f(z)$ and output $x$.

\begin{figure}[h]
	\begin{center}
		\includegraphics[scale=0.5]{./images/generative/dgm.pdf}
	\end{center}
	\caption{General structure of deep generative models. This model does not infer $z$ from $x$.}
	\label{fig:dgm}
\end{figure}

VAE performs an inference by introducing a probabilistic encoder, called inference network. VAEs are generative model with a latent variable distributed according to some distribution $p(z_i)$. The observed variable is distributed according to a conditional distribution 
$$p_\theta(x_i|z_i)$$
This conditioning means the latent variable values are the one most likely given the observations. We also create a distribution $q_\phi(z_i|x_i)$. We would like to be able to encode our data into the latent variable space. Let's model the distribution.

\begin{itemize}
	\item $p_\theta(x_i|z_i)\sim \mathcal{N}(x_{i}|\mu(z_i), \sigma^2(z_i))$: A probabilistic decoder (or generative network, $\theta$)
	\item $q_\phi(z_i|x_i)$: A probabilistic encoder (or inference network $\phi$). We can choose a family of distributions for our conditional distribution $q$ (\eg standard Gaussian distribution). 
		$$q_\phi(z_i|x_i) = \mathcal{N}(z_i|\mu(x_i, W_1), \sigma^2(x_i, W_2)I),$$
	where $W_1$ and $W_2$ are network weights and collectively denoted as $\phi$. We create a neural network to model the distribution $q$ from our data in a non-linear manner. The outputs of the network are $\mu$ and $\sigma$. 
\end{itemize}

\begin{figure}[h]
	\begin{center}
		\includegraphics[scale=0.5]{./images/generative/encoder.pdf}
	\end{center}
	\caption{Overview of variational autoencoder.}
	\label{fig:vae}
\end{figure}



\begin{align*}
	p(X,Z|\theta) &=  \prod_{i=1}^{n}\underbrace{p(x_i|z_i,\theta)}_{\textrm{Likelihood, Generator}}\underbrace{p(z_i|\theta)}_{\textrm{Prior on latent variable}}\\
	&= \prod_{i=1}^{n} \mathcal{N}(x_{i}|\underbrace{\mu(z_i), \sigma^2(z_i)}_{\textrm{Non-linear}}) \mathcal{N}(z_i|0, I)
\end{align*}

Subsequently, marginal distributions can be expressed as follows under i.i.d. assumption:
\begin{align*}
	p(X|\theta) &= \prod_{i=1}^{n} p(x_i|\theta) \\
	&= \prod_{i=1}^{n} \int p(x_i, z_j|\theta) dz_j \\
	& = \prod_{i=1}^{n} \int p(x_i|z_i, \theta)p(z_i|\theta)dz_i \\
	& = \prod_{i=1}^{n} \int \mathcal{N}(x_{i}|\mu(z_i), \sigma^2(z_i)) \underbrace{\mathcal{N}(z_i|0, I)}_{\textrm{Mixture weight}} dz_i
\end{align*}

\begin{itemize}
	\item As you can see, the marginal distribution $p(X|\theta)$ becomes a mixture of Gaussian (infinite mixture of Gaussian). 
	\item Even though $p(x|z)$ and $p(z)$ are normal, $p(x)$ is not normal, because it is a mixture distribution.
	\item The non-linearity of Gaussian parameters (modeled by a neural network), conjugacy between the prior and the likelihood does not hold anymore.
%	\item Diagonal covariance matrix does not mean the independence between elements of $x$.
	\item Again, $\mu$ and $\sigma$ is non-linear function of $z$ modeled by some non-linear neural network. The neural network works as a powerful non-linear parameter approximator (based on universal approximation theorem). 
	\item Simple prior is used. Let's consider the data $x$ is an image of $100\times 100$ pixels. Then the covariance matrix has to be $10000\times 10000$. Thus, it is common to set a simple prior such as the standard Gaussian (covariance matrix is diagonal matrix). However, even if we set a simple distribution, with the infinite mixture of Gaussian, we can model any distribution.
	\item VAE uses a global parametric model to predict the local
		variational parameters for each data point (\textbf{amortized inference}). 
%	\item Under the simple standard Gaussian prior assumption, the generator, $p(x_i|z_i,\theta)$, returns factorized Gaussian whose mean and variance are non-linear functions of latent variable modelled by deep neural network parameterized by $\theta$.
%	$$p(x_i|z_i,\theta) = \mathcal{N}(x_{i}|\mu(z_i), \sigma^2(z_i))$$
%	\item VAEs uses a simple prior over latent variables and complicated and powerful generator (neural network).
	\item It allows to convert complicated large-dimensional data distributions into simple lower-dimensional latent variable representations.  
%	\item $Std = e^{\frac{1}{2}\log (Var)}$, thus output is the log var
\end{itemize}

\subsection{VAE Optimization}
We can train VAE using variational inference with the following objective function, ELBO:
$$\mathcal{L}(\phi,\theta) = \mathbbm{E}_{q_{\phi}(z|x)}[\log p_{\theta}(x|z)] - D_{\textrm{KL}}(q_{\phi}(z|x)||p_{\theta}(z))$$
Let's closely look at this objective function:
\begin{itemize}
	\item In $q_{\phi}(z|x)$, $x$ is a given data, so it is not stochatic. How to sample $z$?
	\item $q$ has to be deterministic and differentiable. 
	\item[] $\to$ \textbf{Reparameterization trick}!
		$$\tilde{z}\sim q_\phi(z|x) \to \tilde{z}\sim g_{\phi}(\epsilon, x)$$, where $\epsilon\sim p(\epsilon).$

	\item Estimated by using Monte-Carlo estimation 
		$$\mathbbm{E}_{q_{\phi}(z|x)}[\log p_{\theta}(x|z)]\approx \frac{1}{N}\sum_j \log p_{\theta}(x_i|z_j).$$
\end{itemize}

\subsection{Conditional VAE}
If we have label information about data, then it would provide a better optimization of VAE model. Recall that the following objective function is the objective of the original VAE:
$$\mathcal{L}(\phi,\theta) = \mathbbm{E}_{q_{\phi}(z|x)}[\log p_{\theta}(x|z)] - D_{\textrm{KL}}(q_{\phi}(z|x)||p_{\theta}(z))$$
In conditional VAE, 
$$\mathcal{L}(\phi,\theta) = \mathbbm{E}_{q_{\phi}(z|x, y)}[\log p_{\theta}(x|y, z)] - D_{\textrm{KL}}(q_{\phi}(z|x, y)\ ||\ p_{\theta}(z|y))$$

\begin{align}
	\log p(X|Y) & = \ln\int q(Z|X, Y) \frac{p(X, Z|Y)}{q(Z|X, Y)}dZ\\
					 & \geq \underbrace{\int q(Z|X, Y) \ln\frac{p(X, Z|Y)}{q(Z|X, Y)}dZ}_{\textrm{ELBO, } \mathcal{L}(q,\theta)} \quad\textrm{by Jensen's Inequality.}\\
					 &\dots\\
					 &\dots\\
					 &= \mathbb{E}_{q(Z|X,Y)} [\ln p(X|Z,Y)]  - KL(q(Z|X,Y)||p(Z|Y)) 
	% & = \int q(Z)\log \frac{p(X,Z|\theta)}{p(Z|X,\theta)}dZ\\
	% & = \int q(Z)\log \frac{q(Z)p(X,Z|\theta)}{q(Z)p(Z|X,\theta)}dZ\\
	% & = \int q(Z)\log \frac{p(X,Z|\theta)}{q(Z)}dZ+ \int q(Z)\log \frac{q(Z)}{p(Z|X,\theta)}dZ\\
	% & = \underbrace{\int q(Z)\log \frac{p(X,Z|\theta)}{q(Z)}dZ}_{\textrm{ELBO, } \mathcal{L}(q,\theta)}+ \textrm{KL}(q(Z)||\log p(Z|X,\theta))
\end{align}
Note that not we have a prior $p_{\theta}(z|y)$. However, we have no idea about latent variable $z$, so we simply assume that we cannot impact the $z$ by $y$. Thus, we typically set it as a standard normal distribution. Also, we can simply concatenate the input $X$ with $Y$. 

\subsection{Variational Deep Embedding (VaDE)}
The generative process of VADE $p(x, z, c) = p(x|z)p(z|c)p(c)$:
\begin{itemize}
	\item Choose a cluster $c\sim Cat(\pi)$
	\item Choose a latent vector $z\sim \mathcal{N}(\mu_c, \sigma_c^2I)$
	\item Choose a sample $x$:
		\begin{align*}
			x\sim
			\begin{cases}
				Ber(\mu_x)\quad &\textrm{If }$x$ \textrm{ is binary} \\
				\mathcal{N}(\mu_x, \sigma_x^2I) \quad &\textrm{else}
			\end{cases}
		\end{align*}
\end{itemize}
ELBO of VaDE:
\begin{align}
	\log p(X) & = \ln\int \sum_c p(X,Z,C) dz \\
					 & \geq \underbrace{\int q(Z,C|X) \ln\frac{p(X, Z, C)}{q(Z,C|X)}dZ}_{\textrm{ELBO}} 
\end{align}
The ELBO can be decomposed as follows:
\begin{align*}
\mathcal{L}_{ELBO} &= \mathbb{E}_q(z,c|x)\bigg[\ln\frac{p(x,z,c)}{q(z,c|x)}\bigg]\\
				   &= \mathbb{E}_q(z,c|x)[\ln p(x,z,c) - \ln q(z,c|x)]\\
				   &= \mathbb{E}_q(z,c|x)[\ln p(x|z) + \ln p(z|c)+ \ln p(c)-\ln q(z|x)-\ln q(c|x)]
\end{align*}
By using two factorizations:
\begin{itemize}
	\item $p(x, z, c) = p(x|z)p(z|c)p(c)$
	\item $q(z,c|x) \approx q(z|x)q(c|x)$ (Mean-field assumption)
		\begin{itemize}
			\item $q(z|x)\sim \mathcal{N}$: encoder, estimate mean and variance.
			\item $q(c|x)$: assignment probability of Gaussian mixture model
		\end{itemize}
\end{itemize}

\subsection{Importance Weighted VAE}


% \begin{itemize}
% 	\item 
% 		$$\mathbbm{E}_{q_{\phi}(z|x)}[\log p_{\theta}(x|z)]\approx \frac{1}{N}\sum_j \log p_{\theta}(x_i|z_j).$$
% 	\item The second term can be solve analytically for some distributions like Gaussian.
% 	\item For training, we need to use a reparameterization trick. 
% \end{itemize}




%\footnotetext[1]{Uncorrelated relationship does not imply the independence (independence makes covariance to be diagonal). If two variables are uncorrelated, $Cov(x_i,x_j)=0 $, there is no linear relationship between them.}
%\footnotetext[2]{In practice, simple prior could be a problem.}

\chapter{Implicit Generative Models}
\section{Generative Adversarial Networks}
\label{sec:gan}
\begin{itemize}
	\item Generator's distribution: $p_{g}$
	\item Prior on input noise: $p_{z}(z)$
	\item Mapping to data space: $z\rightarrow x$ through $G(z;\theta_{g})$
	\item[] a differentiable multilayer perceptron with parameter $\theta_{g}$
	\item $D(x;\theta_{d})$: a differentiable multilayer perceptron with parameter $\theta_{d}$. It outputs a single scalar 
	\item $D(x)$: probability that $x$ (real) came from the data rather than $p_g$ (fake)
\end{itemize}

\begin{figure}[h]
	\begin{center}
		\includegraphics[scale=0.5]{./images/generative/gan/gan_model.pdf}
	\end{center}
	\caption{GAN structure}
	\label{fig:gan}
\end{figure}

\subsection{Discriminator}
The discriminator's goal is to maximize the following equation given $G$
\begin{equation*}
\mathbbm{E}_{x \sim p_{data}(x)}\log(D(x))+\mathbbm{E}_{z \sim p_{z}(z)}\log(1-D(G(z)))
\end{equation*}
The optimal discriminator given $G$ can be denoted as $D^*_{G}$. To get the optimal discriminator, define a value function
\begin{equation*}
V(G,D):= \mathbbm{E}_{x \sim p_{data}(x)}\log(D(x))+\mathbbm{E}_{z \sim p_z(z)}\log(1-D(G(z))).
\end{equation*}
Then, $D_G^* = \text{argmax}_D V(G,D)$

However, the generator $G$ wants to minimize the value function given $D=D^*_G$. 
\begin{equation*}
G^* = \text{argmin}_G V(G,D_G^*).
\end{equation*}

\begin{equation*}
\min_{G}\max_{D}V(D,G) =  \mathbbm{E}_{x\sim p_{data}(x)}[\textrm{log}D(x)]+\mathbbm{E}_{z\sim p_{z}(z)}[\textrm{log}(1-D(G(z)))]
\end{equation*}
\begin{itemize}
	\item $\min_{G} \rightarrow$ try to generate fake data that is similar to real data
	\item $\max_{D} \rightarrow$ try to assign correct label \footnotemark
\end{itemize}

At this point, we must show that this optimization problem has a unique solution $G^*$ and that this solution satisfies $p_G=p_{data}$.

\footnotetext{The above equation is trained separately at the same time, don't get confused}

One big idea from the GAN paper–, which is different from other approaches is that $G$ \textbf{need not be invertible}. Many pieces of notes online miss this fact when they try to replicate the proof and incorrectly use the change of variables formula from calculus (which would depend on $G$ being invertible). Rather, the whole proof relies on this equality:
\begin{equation*}
	\mathbbm{E}_{z \sim p_{z}(z)}\log(1-D(G(z))) = \mathbbm{E}_{x \sim p_{G}(x)}\log(1-D(x)) .
\end{equation*}

With the above equality, 
\begin{align*}
	&\mathbbm{E}_{x \sim p_{data}(x)}\log(D(x))+\mathbbm{E}_{z \sim p_z(z)}\log(1-D(G(z)))\\
	&=\int_{x} p_{data}(x)\log D(x) \, \mathrm{d}x + \int_{z} p(z)\log ( 1- D(G(z))) \, \mathrm{d}z\\ 
	&= \int_{x} p_{data}(x)\log D(x) + p_G(x) \log ( 1- D(x)) \, \mathrm{d}x
\end{align*}

Additionally, we will use the following property:
\begin{align*}
	f(y)= a \log y + b \log(1-y).
\end{align*}
To find a critical point,
$$f^\prime(y) = 0 \Rightarrow \frac{a}{y} - \frac{b}{1-y} = 0 \Rightarrow y = \frac{a}{a+b}$$
If $a+b\neq0$, do the second derivative test:
$$f^{\prime\prime}\big ( \frac{a}{a+b} \big) = - \frac{a}{(\frac{a}{a+b})^2} - \frac{b}{(1-\frac{a}{a+b})^2} < 0
$$
If $a,b\in (0,1)$, $\frac{a}{a+b}$ is a maximum.

By rewriting the equation,
\begin{align*}
V(G,D) &= \int_{x} p_{data}(x)\log D(x) + p_G(x) \log ( 1- D(x)) \, \mathrm{d}x \\
& \leq \int_x \max_y {p_{data}(x)\log y + p_G(x) \log ( 1- y)}\, \mathrm{d}x 
\end{align*}
Thus, if $D(x) = \frac{p_{data}}{p_{data}+p_G}$, then we can achieve the maximum $V(G,D)$. 

\subsection{Generator}
If we achieve the optimal $G$ (i.e., $p_G = p_{data}$), then $D$ would be completely confused and $D^*_G(x) = \frac{p_{data}}{p_{data}+p_G}=\frac{1}{2}$ (it means that $D$ cannot make a clear decision.).

The global minimum of the virtual training criterion $C(G)=\max_DV(G,D)$ is acheived if and only if $p_{G}=p_{data}$. Let's plug $D^*_G(x)$ into the criterion then, 
\begin{equation*}
	C(G) = \int_{x} p_{data}(x)\log \big (\frac{p_{data}(x)}{p_{G}(x)+p_{data}(x)} \big )  + p_G(x) \log\big ( \frac{p_{G}(x)}{p_{G}(x)+p_{data}(x)}\big ) \, \mathrm{d}x. 
\end{equation*}
To get the minimum $C(G)$, we can use the Jansen-Shannon divergence:

\begin{align*}
	D_{JS}(p_{data}||p_{G}) & = \frac{1}{2}\Bigg[D_{KL}\Big(p_{data}\Big|\Big|\frac{p_{data}+p_{G}}{2}\Big)+D_{KL}\Big(p_{G}\Big|\Big|\frac{p_{data}+p_{G}}{2}\Big)\Bigg]\\
	& = \frac{1}{2}\Bigg[\Bigg(\int_x p_{data}(x)\textrm{log}\Bigg(\frac{2p_{data}(x)}{p_{data}(x)+p_{G}(x)}\Bigg)dx\Bigg)+\Bigg(\int_x p_{G}(x)\textrm{log}\Bigg(\frac{2p_{G}(x)}{p_{data}(x)+p_{G}(x)}\Bigg)dx\Bigg)\Bigg]\\
	& = \frac{1}{2}\Bigg[\Bigg(\int_x p_{data}(x)\textrm{log}2+p_{data}(x)\textrm{log}\Bigg(\frac{p_{data}(x)}{p_{data}(x)+p_{G}(x)}\Bigg)dx\Bigg)+\\
	&\hspace{1cm}\Bigg(\int_x p_{G}(x)\textrm{log}2+p_{G}(x)\textrm{log}\Bigg(\frac{p_{G}(x)}{p_{data}(x)+p_{G}(x)}\Bigg)dx\Bigg)\Bigg]\\
	& = \frac{1}{2}\Bigg[\Bigg(\textrm{log}2+\int_x p_{data}(x)\textrm{log}\Bigg(\frac{2p_{data}(x)}{p_{data}(x)+p_{G}(x)}\Bigg)dx\Bigg)+\\
	& \hspace{1cm}\Bigg(\textrm{log}2+\int_x p_{G}(x)\textrm{log}\Bigg(\frac{2p_{G}(x)}{p_{data}(x)+p_{G}(x)}\Bigg)dx\Bigg)\Bigg]\\
	& = \frac{1}{2}(\textrm{log}4+C(G))
\end{align*}

Thus,
$$C(G) = -\textrm{log}4 + 2D_{JS}(p_{data}||p_{G})$$

Since the Jensen-Shannon divergence between two distributions is always non-negative and zero only when they are equal, we have shown that $C^* = -\textrm{log}(4)$ is the global minimum of $C(G)$ and that the only solution is $p_G=p_{data}$, i.e., the generative model perfectly replicating the data generating process.

\begin{figure}[h]
	\centering
	\includegraphics[scale=0.3]{./images/generative/gan/gan_algorithm.png}
	\caption{Training GAN}
	\label{fig:algorithm}
\end{figure}

\section{Some notes}
What would be the optimal discriminator that separates the two different distributions $p(x)$ and $q(x)$? It turns out that it is
$$f(x) = \frac{q(x)}{p(x)+q(x)}$$
Actually, there are many choices for classifiers e.g., KL-divergence

\begin{enumerate}
	\item What do we need to learn a classifier?
	\begin{itemize}
		\item Only samples from $p(x)$ and $q(x)$
	\end{itemize}
	\item How do we parameterize $q(x)$?
	\begin{itemize}
		\item Parametric density function (Gaussian)
		\item Define implicitly (GANs approach): define mapping from one (noise) to another (data or image)
	\end{itemize}
\end{enumerate}

The orignal GAN does not learn the data distributions. 





\section{Wasserstein Generative Adversarial Networks}
\subsection{KL Divergence}
Definition:
$$D_\textrm{KL}(q(x)||p(x)) = \int q(x)\log \frac{q(x)}{p(x)}dx$$
\begin{itemize}
	\item Forward KL: 
	\begin{itemize}
		\item If $q(z)\rightarrow 0, \textrm{Forward KL}\rightarrow \infty$ 
		\item Zero avoiding for $q(z)$ 
	\end{itemize}
	\item Reverse KL:
	\begin{itemize}
		\item If $p(z)\rightarrow 0, \textrm{Reverse KL}\rightarrow \infty$ 
		\item Zero forcing: $q(z)\rightarrow 0$ 
	\end{itemize}
\end{itemize}

Typically, $p(x)$ and $q(x)$ are far apart at the initial state. 

\begin{figure}[h]
	\begin{center}
		\includegraphics[scale=0.5]{./images/generative/gan/twodist.pdf}
	\end{center}
	\caption{Two distributions: $p(x)$ and $q(x)$}
	\label{fig:}
\end{figure}

Thus, both the forward KL and the reverse KL suffers an unstability issue. Specifically, in each case, if the denominator goes to zero, then the divergence goes to infinity. 

\subsection{Jensen-Shannon Divergence}

Definition:
$$D_{JS}(p_{data}||p_{G}) = \frac{1}{2}\Bigg[D_{KL}\Big(p_{data}\Big|\Big|\frac{p_{data}+p_{G}}{2}\Big)+D_{KL}\Big(p_{G}\Big|\Big|\frac{p_{data}+p_{G}}{2}\Big)\Bigg]$$

The KL divergence's issue can be alleviated by JS-divergence. Consider a simple example in Fig. \ref{fig:wassersteinexample}6
\begin{align*}
	\forall (x, y) \in P, x = 0 \text{ and } y \sim U(0, 1)\\
	\forall (x, y) \in Q, x = \theta, 0 \leq \theta \leq 1 \text{ and } y \sim U(0, 1)
\end{align*}

\begin{figure}[h]
	\begin{center}
		\includegraphics[scale=0.25]{./images/generative/gan/wassersteinexample.png}
	\end{center}
	\caption{Two distributions: $p(x)$ and $q(x)$}
	\label{fig:wassersteinexample}
\end{figure}

\begin{align*}
	D_\textrm{KL}(q(x)||p(x)) &= \infty\\
	D_\textrm{KL}(p(x)||q(x)) &= \infty\\
	D_{JS}(p_{data}||p_{G}) &= \frac{1}{2}\Bigg[D_{KL}\Big(p_{data}\Big|\Big|\frac{p_{data}+p_{G}}{2}\Big)+D_{KL}\Big(p_{G}\Big|\Big|\frac{p_{data}+p_{G}}{2}\Big)\Bigg]\\
	& = \frac{1}{2}\Bigg[D_{KL}\Big(p_{data}\Big|\Big|\frac{p_{data}}{2}\Big)++D_{KL}\Big(p_{G}\Big|\Big|\frac{p_{G}}{2}\Big)\Bigg]\\
	& = \frac{1}{2}[\log 2 + \log 2] = \log 2\\
	W(p,q) & = |\theta|
\end{align*}
Therefore, Jensen-Shannon divergence is more stabler than KL divergece. This is one of the reasons why GAN, which uses JS divergence works better than VAE, which uses KL divergence. 

However, JS divergence also has some problem. If the value is close to $\frac{1}{2}\log 2$, then the gradient will be very small or close to zero, because the divergence is close to constant. It means that a training speed is very slow. Thus, we need a better metric. 

\subsection{Wasserstein Distance}
Wasserstein Distance is a measure of the distance between two probability distributions. It is also called Earth Mover’s distance, short for EM distance, because informally it can be interpreted as the minimum energy cost of moving and transforming a pile of dirt in the shape of one probability distribution to the shape of the other distribution.
\begin{equation*}
	W(p_r, p_g) = \inf_{\gamma \sim \Pi(p_r, p_g)} \mathbb{E}_{(x, y) \sim \gamma}[\| x-y \|]
\end{equation*}

\begin{itemize}
	\item $\Pi$: is the transportation plan and the set of all possible joint probability distributions between $p_r$ and $p_g$. One joint distribution $\gamma \sim \Pi(p_r, p_g)$ describes one transport plan.\
	\item $\mathbb{E}_{x, y \sim \gamma} \| x-y \| = \sum_{x, y} \gamma(x, y) \| x-y \|$
	\item Finally, we take the minimum one among the costs of all dirt moving solutions as the EM distance (by infimum). 
\end{itemize}

\section{WGAN}
However, consider all possible joint distribution is intractable, so dual solution can be used. 

$$W(p_r, p_g) = \frac{1}{K} \sup_{\| f \|_L \leq K} \mathbb{E}_{x \sim p_r}[f(x)] - \mathbb{E}_{x \sim p_g}[f(x)]$$

So to calculate the Wasserstein distance, we just need to find a 1-Lipschitz function. To enforce the constraint, WGAN applies a very simple clipping to restrict the maximum weight value in $f$, i.e. the weights of the discriminator

Suppose this function $f$ comes from a family of $K$-Lipschitz continuous functions, $\{f_w\}_{w\in W}$, parameterized by $w$. In the modified Wasserstein-GAN, the ``discriminator'' model is used to learn $w$ to find a good $f_w$ and the loss function is configured as measuring the Wasserstein distance between $p_r$ and $p_g$.

$$L(p_r, p_g) = W(p_r, p_g) = \max_{w \in W} \mathbb{E}_{x \sim p_r}[f_w(x)] - \mathbb{E}_{z \sim p_r(z)}[f_w(g_\theta(z))]$$

There are two ways to satisfy the Lipschitz continuity:
\begin{itemize}
	\item Weight clipping
	\item Gradient Penalty
\end{itemize}

\subsection{Lipschitz continuity}
The function $f$ in the new form of Wasserstein metric is demanded to satisfy $\| f \|_L \leq K$, meaning it should be $K$-Lipschitz continuous. 

A real-valued function $f: \mathbb{R} \rightarrow \mathbb{R}$ is called $K$-Lipschitz continuous if there exists a real constant $K\geq 0$ such that, for all $x_1, x_2 \in \mathbb{R}$
$$\lvert f(x_1) - f(x_2) \rvert \leq K \lvert x_1 - x_2 \rvert$$
Here $K$ is known as a Lipschitz constant for function $f(\cdot)$. Functions that are everywhere continuously differentiable is Lipschitz continuous, because the derivative, estimated as $\frac{\lvert f(x_1) - f(x_2) \rvert}{\lvert x_1 - x_2 \rvert}$, has bounds. However, a Lipschitz continuous function may not be everywhere differentiable, such as $f(x) = \lvert x \rvert$

\begin{figure}[h]
	\begin{center}
		\includegraphics[scale=0.25]{./images/generative/gan/wgan.jpeg}
	\end{center}
	\caption{WGAN}
	\label{fig:wgan}
\end{figure}

\begin{figure}[h]
	\begin{center}
		\includegraphics[scale=0.2]{./images/generative/gan/wgan_2.jpeg}
	\end{center}
	\caption{WGAN}
\end{figure}

\section{InfoGAN: Interpretable Representation Learning by Information Maximizing Generative Adversarial Nets}
\label{sec:q}

\subsection{Joint Entropy}
\begin{align*}
H(X,Y) = \mathbbm{E}_{X,Y}[-\log p(x,y)] = -\sum_{x,y}p(x,y)\log p(x,y)
\end{align*}
\subsection{Conditional Entropy}
\begin{align*}
H(X|Y) &= \mathbbm{E}_{Y}[H(X,Y)] \\
&= -\sum_{y\sim p_Y(y)}p(y)\sum_{x\sim p_X(x)}p(x|y)\log p(x|y)\\
& = -\sum_{y\sim p_Y(y)}\sum_{x\sim p_X(x)}p(y)p(x|y)\log p(x|y)\\
& = -\sum_{y\sim p_Y(y)}\sum_{x\sim p_X(x)}p(x,y)\log p(x|y) = -\mathbbm{E}_{x,y}[\log p(x|y)]\\
& = -\sum_{y\sim p_Y(y)}\sum_{x\sim p_X(x)}p(x,y)\log \frac{p(x,y)}{p(y)}\\
& = -\sum_{y\sim p_Y(y)}\sum_{x\sim p_X(x)}p(x,y)\log p(x,y) + \sum_{y\sim p_Y(y)}\sum_{x\sim p_X(x)}p(x,y)\log p(y)\\
& = H(X,Y) - H(Y)
\end{align*}
\subsection{Variational Mutual Information Maximization}
\begin{align*}
I(c;G(z,c)) &= H(c) - H(c|G(z,c))\\
& = H(c) + \int\int p(c=c',x=G(z,c))\log p(c=c'|x=G(z,c)) dc' dz\\
& = H(c) + \mathbbm{E}_{x\sim G(z,c),c'\sim p(c|x)}[\log p(c'|x)]\\
& = H(c) + \mathbbm{E}_{x\sim G(z,c)}\mathbbm{E}_{c'\sim p(c|x)}[\log p(c'|x)]\\
& = H(c) + \mathbbm{E}_{x\sim G(z,c)}\mathbbm{E}_{c'\sim p(c|x)}\Bigg[\log \frac{p(c'|x)Q(c'|x)}{Q(c'|x)}\Bigg]\\
& = H(c) + \mathbbm{E}_{x\sim G(z,c)}\mathbbm{E}_{c'\sim p(c|x)}\Bigg[\log \frac{p(c'|x)}{Q(c'|x)}\Bigg] + \mathbbm{E}_{x\sim G(z,c)}\mathbbm{E}_{c'\sim p(c|x)}\Big[\log Q(c'|x)\Big]\\
& = H(c) + \mathbbm{E}_{x\sim G(z,c)}\Bigg[D_{KL}(p(c'|x)||Q(c'|x))\Bigg] + \mathbbm{E}_{x\sim G(z,c)}\mathbbm{E}_{c'\sim p(c|x)}\Big[\log Q(c'|x)\Big]\\
& \geq H(c) + \mathbbm{E}_{x\sim G(z,c)}\mathbbm{E}_{c'\sim p(c|x)}\Big[\log Q(c'|x)\Big] \footnotemark
\end{align*}

Thus we get a lower bound for the mutual information as follows:

$$I(c;G(z,c)) \geq H(c) + \mathbbm{E}_{x\sim G(z,c)}\mathbbm{E}_{c'\sim p(c|x)}\Big[\log Q(c'|x)\Big]$$

However, we still have a problem. We need to sample $c$ from $p(c|x)$. Thus, we need to replace it with a known distribution. Firstly, with the reasoning that $x\sim G(z,c)$ means sample $c$ from $p(c)$ then sample $x$ from $G(z,c)$. So we can express $\mathbbm{E}_{x\sim G(z,c)}$ with $\mathbbm{E}_{c\sim p(c)}\mathbbm{E}_{x\sim G(z,c)}$. and by the Lemma \ref{lemma:1}, 
\begin{align*}
I(c;G(z,c)) &\geq H(c) + \mathbbm{E}_{x\sim G(z,c)}\mathbbm{E}_{c'\sim p(c|x)}\Big[\log Q(c'|x)\Big]\\
&= H(c) + \mathbbm{E}_{c\sim p(c)}\mathbbm{E}_{x\sim G(z,c)}\mathbbm{E}_{c'\sim p(c|x)}\Big[\log Q(c'|x)\Big]\\
& = H(c) + \mathbbm{E}_{c\sim p(c)}\mathbbm{E}_{x\sim G(z,c)}\Big[\log Q(c'|x)\Big] \footnotemark
\end{align*}

Thus, we can directly sample $c$ from the known distribution instead of $p(c|x)$.

\begin{lemma}
	For random variables $X, Y$ and function $f(x, y)$ under suitable regularity conditions:
	$$\mathbbm{E}_{x\sim X, y\sim Y|x}[f(x,y)] = \mathbbm{E}_{x\sim X, y\sim Y|x, x'\sim X|y}[f(x',y)]$$
	\begin{proof}
		\begin{align*}
		\mathbbm{E}_{x\sim X, y\sim Y|x}[f(x,y)] &=\int_x P(x)\int_y P(y|x)f(x,y)dydx\\
		& = \int_x\int_yP(x,y)f(x,y)dydx\\
		& = \int_{x'}\int_yP(x',y)f(x',y)dydx'\\
		& = \int_{x'}\int_y P(y)P(x'|y)f(x',y)dydx'\\
		& = \int_{x'}\int_y\int_{x} P(x,y)P(x'|y)f(x',y)dxdydx'\\
		& = \int_{x}P(x)\int_y P(x|y) \int_{x'} P(x'|y)f(x',y)dxdydx'\\
		& = \mathbbm{E}_{x\sim X, y\sim Y|x, x'\sim X|y}[f(x',y)]
		\end{align*}
	\end{proof}
	\label{lemma:1}
\end{lemma} 


% \chapter{Hidden Markov Models}
\section{Introduction}
The HMM is based on the Markov chain assumption. A Markov chain is a model
that tells us something about the probabilities of sequences of random variables,
states, each of which can take on values from some set. These sets can be words, or
tags, or symbols representing anything, like the weather.

There are two important assumptions:
\begin{itemize}
	\item Markov assumption
	\item Output independence: $p(x_i|z_1,\dots,z_i,\dots,z_T,x_1,\dots,x_i,\dots,x_T) = p(x_i|z_i)$
\end{itemize}

\subsection{Conditional Independence}
If two events $A$ and $B$ are \textbf{conditionally independent} given an event $C$ then,
\begin{itemize}
	\item $P(A\cap B|C) = P(A|C)P(B|C)$. 
	\item $P(A|B,C) = P(A|C)$
\end{itemize}

\subsection{Notation}

\begin{itemize}
	\item $X = (x_i, x_2,\dots, x_T)$
	% \item $x_i\in\{c_1,...,c_m\}$
	\item Initial state probabilities: $p(z_1) \sim \textrm{Multinomial}(\pi_1,...,\pi_k)$, need to learn $\pi$
	\item Transition probability:
	$$p(z_t|z_{t-1}=i)\sim \textrm{Multinomial}(a_{i,1},...,a_{i,k})$$
	, where $a_{i,j} = p(z_t=j|z_{t-1}=i)$ and $i$ and $j$ denote clusters or states, respectively.
	\item Emission probability:
	$$p(x_t|z_{t}=i)\sim \textrm{Multinomial}(b_{i,1},...,b_{i,m})$$
	, where $b_{i,j} = p(x_t=j|z_{t}=i)$
\end{itemize}


\section{Bayesian Network}
\subsection{Bayes Ball}

\begin{figure}[h!]
	\centering
	\includegraphics[scale=0.3]{./images/hmm/bayes.jpg}
	\caption{Bayes ball}
	\label{fig:bayes}
\end{figure}

\begin{itemize}
	\item Cascading: $P(Z|Y,X) = P(Z|Y)$. The information of $Y$ decouples $X$ and $Z$.
	\item Common parent: $P(X,Z|Y) = P(X|Y)P(Z|Y)$. The information of $Y$ decouples $X$ and $Z$.
	\item V-Structure (common child): Unlike the above two cases, the information of $Y$ couples $X$ and $Z$.
		$$P(X,Y,Z) = P(X)P(Y)P(Y|X,Z).$$
\end{itemize}

\subsection{Potential Function}
Potential function is a function which is not a probability function, but it can become a probability function by normalizing it. 
$$P(A,B,C,D) = P(A|B)P(B|C)P(C|D)P(D)$$

\begin{figure}[h]
	\centering
	\includegraphics[scale=0.5]{./images/hmm/cascade.pdf}
\end{figure}

\begin{itemize}
	\item Cliques: $\Psi(a,b), \Psi(b,c), \Psi(c,d)$
	\item Separators $\phi(b), \phi(c)$
\end{itemize}
Given a clique tree with cliques and separators, the joint probability distribution is defined as follows:
% By using potential functions, we can express the joint probability as
\begin{align*}
	P(A,B,C,D) &= P(U) = \frac{\prod_N \Psi(N)}{\prod_L\phi(L)}= \frac{ \Psi(a,b)\Psi(b,c)\Psi(c,d)}{\phi(b)\phi(c)}\\
\end{align*}
An effect of an observation propagates through the clique graph $\to$ \textbf{Belief propagation}. How to propagate the belief? \textbf{Absorption rule}!

Let's say we have some new observations about $A$, then it affects the clique $\Psi(a,b)$. The updated clique is now $\Psi^*(a,b)$. Similarly, $\phi^*(b) = \sum_A\Psi^*(a,b)$. Subsequently, $\Psi^*(b,c) = \Psi^(b,c)\frac{\phi^*(b)}{\phi(b)}$.

\newpage
\section{Hidden Markov Models}
\label{sec:hmm}

\begin{figure}[h]
	\begin{center}
		\includegraphics[scale=0.7]{./images/hmm/hmm_figure.pdf}
	\end{center}
	\caption{HMM Structure}
	\label{fig:HMM}
\end{figure}

The observation can be discrete or continuous. If the latent factors are continuous, then HMM is often referred as \textbf{Kalman filter}. 

\begin{itemize}
	\item Initial state probability: $P(z_1)\sim \textrm{Mult}(\pi_1, \dots, \pi_k)$
	\item Transition probability: $P(z_t|z^i_{t-1}=1)\sim \textrm{Mult}(a_{i,1}, \dots, a_{i,k})$, \\ where $P(z_t^j=1|z_{t-1}^i=1) = a_{i,j}$
	\item Emission probability: $P(x_t|z_t^i=1)\sim \textrm{Mult}(b_{i,1}, \dots, b_{i,m})\sim f(x_t|\theta_i)$,\\ where $P(x_t^j=1|z_{t}^i=1) = b_{i,j}$. The probability of observing $x_j$ at the  $i$-th cluster. 
\end{itemize}
Note that $i$ and $j$ are indices of clusters. 

There are three main problems in HMM:
\begin{enumerate}
	\item Evaluation Questions (likelihood): %Forward algorithm
	\begin{itemize}
		\item Given $\boldsymbol{\pi}\mathbf{, a, b}, X$
		\item Find $p(X|M, \boldsymbol{\pi}\mathbf{, a, b})$
		\item How much are $X$ likely to be observed by a model $M$?
	\end{itemize}
	
	\item Decoding Questions:
	\begin{itemize}
		\item Given $\boldsymbol{\pi}\mathbf{, a, b}, X$
		\item Find $\argmax_Z p(Z|X, M, \boldsymbol{\pi}\mathbf{, a, b})$
		\item What is the most probable sequence of $Z$ (latent states)? 
	\end{itemize}
	
	\item Learning Questions: Forward-Backward (Baum-Welch)
	\begin{itemize}
		\item Given $X$
		\item Find $\argmax_{\boldsymbol{\pi}\mathbf{, a, b}} p(X|M, \boldsymbol{\pi}\mathbf{, a, b})$
		\item What would be the optimal model parameters? 
	\end{itemize}
\end{enumerate}

% For a given hidden state, we can easily compute the output likelihood.

\section{Evaluation: Forward-Backward Probability}
% The forward–backward algorithm is an inference algorithm for hidden Markov models which computes the posterior marginals of all hidden state variables given a sequence of observations/emissions

% The term forward–backward algorithm is also used to refer to any algorithm belonging to the general class of algorithms that operate on sequence models in a forward–backward manner.

% In the first pass, the forward–backward algorithm computes a set of forward probabilities which provide, for all $t\in \{1,\dots ,T\}$, the probability of ending up in any particular state given the first $t$ observations in the sequence, i.e. $P(X_{t}\ |\ o_{1:t})$. In the second pass, the algorithm computes a set of backward probabilities which provide the probability of observing the remaining observations given any starting point $t$, i.e. $P(o_{t+1:T}\ |\ X_{t})$. These two sets of probability distributions can then be combined to obtain the distribution over states at any specific point in time given the entire observation sequence:

% These two sets of probability distributions can then be combined to obtain the distribution over states at any specific point in time given the entire observation sequence:
% $$P(X_{t}\ |\ o_{1:T})=P(X_{t}\ |\ o_{1:t},o_{t+1:T})\propto P(o_{t+1:T}\ |\ X_{t})P(X_{t}|o_{1:t})$$
% The forward–backward algorithm can be used to find the most likely state for any point in time. However, It cannot be used to find the most likely sequence of states.

\subsection{Joint Probability}
We can factorize the joint distribution of HMM in \Cref{fig:HMM} by using a Bayesian approach as follows:. 
\begin{align}
	p(X,Z) &= p(x_1,\dots,x_t, z_1,\dots,z_t)\\ 
		   &= p(z_1)p(x_1|z_1)p(z_2|z_1),\dots,p(x_{t}|z_{t})p(z_{t}|z_{t-1})
\end{align}
The key assumption involved in factorizing the Markov chain within a Hidden Markov Model (HMM) is \textit{conditional independence} among certain components of the state variables. Here's a detailed breakdown of what this assumption means:
\begin{itemize}
	\item Independence of State Components: The transition of each component $z_t^k$ only depends on its corresponding previous component $z_{t-1}^k$ and is independent of other components.
\end{itemize}
As the number of latent factor increases, it is getting harder to decode the latent factors. 

\subsection{Marginal Probability}
% \subsection{Forward Probability}
We want to compute the likelihood of sequence $X$ which is given by
$$p(X|\boldsymbol{\pi}\mathbf{, a, b}) = \sum_Z p(X, Z|\boldsymbol{\pi}\mathbf{, a, b})$$
The computation can be done as follows:
\begin{align*}
	p(X) &= \sum_Z p(X,Z)\\
	& = \sum_{z_1}\dots\sum_{z_t}p(x_1,\dots,x_t,z_1,\dots,z_t)\\
	& = \sum_{z_1}\dots\sum_{z_t}\pi_{z_{1}}\prod_{t=2}^{T}a_{z_{t-1},z_t}\prod_{t=1}^{T}b_{z_{t},x_t}
\end{align*}
The last step is done by using the factorization above. The computation of this equation requires lots of computations, so we will change it into a \textbf{recursive form} by using the factorization rule $p(a,b,c) = p(a)p(b|a)p(c|a,b)$. 
\begin{align}
	p(&x_1,\dots,x_t,z_t^k=1) = \sum_{z_{t-1}}p(x_1,\dots,x_{t-1}, x_t,z_{t-1},z_t^k=1)\\
	&= \sum_{z_{t-1}} p(\underbrace{x_1,\dots,x_{t-1}, z_{t-1}}_{a}, \underbrace{x_t}_{c}, \underbrace{z_t^k=1}_{b})\\
	& = \sum_{z_{t-1}} p(x_1,\dots,x_{t-1},z_{t-1}) p(z_t^k=1|x_1,\dots,x_{t-1},z_{t-1})p(x_t|z_t^k=1, x_1,\dots,x_{t-1},z_{t-1}) \\
	&\hspace{0.5cm} \because p(a,b,c) = p(a)p(b|a)p(c|a,b) \textrm{ or by the structure of HMM}\nonumber\\ 
	& = \sum_{z_{t-1}} p(x_1,\dots,x_{t-1},z_{t-1}) p(z_t^k=1|z_{t-1}) p(x_t|z_t^k=1)\\
	& = p(x_t|z_t^k=1) \sum_{z_{t-1}} p(x_1,\dots,x_{t-1},z_{t-1}) p(z_t^k=1|z_{t-1}) \\
	& = b_{z^k_t,x_t} \sum_{z_{t-1}} p(x_1,\dots,x_{t-1},z_{t-1}) a_{z_{t-1},z_t^k}
	\label{eq:hmm_eval_fact}
\end{align}
\begin{itemize}
	\item In the second line, the $x_{t-1}$ and $z_{t-1}$ are grouped together. 
	\item Then, we can find the HMM structure by factorizing the equation. 
	\item In the fourth line, $x$ terms are removed, since $z_t$ only relies on $z_{t-1}$ by the Markov assumption. Similarly, $x_t$ only depends on $z_t$. We can interpret this by using Bayes ball too. 
\end{itemize}
% In the fifth step, we assume that $z_t=k$ is given, thus by Markov assumption, we only need $z_{t-1}$. 
Now we can find a recursive structure of $p(x_1,\dots,x_{t},z_{t}^k=1)$ as follows:
$$\alpha_t^k = p(x_1,\dots,x_{t},z_{t}^k=1) = b_{k,x_t}\sum_i \alpha_{t-1}^ia_{i,k}$$
, where \textbf{$\alpha_t^k$ is the probabilities of being in state $k$ after observing the first $t$ observations.} Thus, 
\begin{align*}
	p(x_1,\dots,x_{t}) & = \sum_{\mathbf{z}} p(x_1,\dots,x_{t},z)\\
	& = \sum_{k} \alpha_t^k
\end{align*}
% \begin{itemize}
% 	\item $\alpha_t^k$: \textbf{Forward probability}. Probabilities of being in state $k$ after observing the first $t$ observations.
% 	% \item $a_{i,k}$: transition probability
% 	% \item $b_{k,x_t}$: observation (or emission) probability
% \end{itemize}
Note that $\alpha_t^k$ is also called \textbf{Forward probability}.

\subsection{Forward Algorithm}
Forward probability solves the evaluation problem. Essentially, this is a dynamic programming, so it calculates required values in a bottom-up manner. 
\begin{itemize}
	\item Forward probability: $\alpha_t^k$, $Time\times States$
\end{itemize}
%\LinesNumbered
\begin{algorithm}
	Create a probability matrix $forward[M,T] = \alpha_t^k$\\
	Initialization: \\
	\For {\textrm{each state} k=1,...,M}{
		$\alpha_1^k\leftarrow \pi_kb_{k,x_1}$
	}
	\For {\textrm{time step} t=2,...,T}{
		\For {\textrm{each step} k=1,...,M}{
			$\alpha_t^k = b_{k,x_t}\sum_i \alpha_{t-1}^ia_{i,k}$
			}
		
	}
	Return $p(X) = \sum_i^M \alpha_T^i$
	\caption{Forward Algorithm}
	\label{algo:forward_algorithm}
\end{algorithm}
%\begin{algorithm}
%	Init: $\alpha_1^k = b_{k,x_1}\pi_k$\\
%	\For{t=1,...,T}{
%		$\alpha_t^k = b_{k,x_t}\sum_i \alpha_{t-1}^ia_{i,k}$
%	}
%	Return $p(X) = \sum_i\alpha_T^i$
%	\caption{Forward Algorithm}
%	\label{algo:forward_algorithm}
%\end{algorithm}
Note again that 
$$p(X) = p(x_1,...,x_T) =\sum_i\alpha_T^i = \sum_i p(x_1,...,x_T, z_T^i=1)$$
Note also that the forward-algorithm returns $p(X)$ and forward probability is the probability of being in state $k$ after observing the first $t$ observations without $Z$. 

\subsection{Backward Probability}
The forward probability only considers an observation at $t$. To determine the $z_t$, we need to leverage the future observations. \textbf{The backward probability $\beta$ is the probability of seeing the observations from time $t+i$ to the end, given that we are in state $k$ at time $t$.} 
$$\beta_t^k = p(x_{t+1},\dots,x_T|z_t^k=1)$$
We want to compute $p(z_t^k=1|X)$ rather than $p(x_1,\dots,x_t, z_t^k=1)$. In other words, we will leverage the whole observations $X$. 
\begin{align*}
	p(z_t^k=1,X) &= p(x_1,\dots,x_t, z_t^k=1, x_{t+1},\dots,x_T)\\
	& = p(x_1,\dots,x_t, z_t^k=1)p(x_{t+1},\dots,x_T|x_1,\dots,x_t, z_t^k=1)\\
	& = p(x_1,\dots,x_t, z_t^k=1)p(x_{t+1},\dots,x_T|z_t^k=1)\\
	& = \alpha_{t}^k\beta_{t}^k
\end{align*}
We already know that $p(x_1,\dots,x_t, z_t^k=1) = \alpha_t^k$. We just need to compute backward probability as follows:
\begin{align*}
	\beta_t^k &= p(x_{t+1},\dots,x_T|z_t^k=1)\\
	& = \sum_{z_{t+1}}p(\underbrace{z_{t+1}}_{a}, \underbrace{x_{t+1}}_b,\underbrace{x_{t+2},\dots,x_T}_c|z_t^k=1)\\
	& = \sum_{i} p(z_{t+1}^i=1|z_t^k=1)p(x_{t+1}|z_{t+1}^i=1,z_t^k=1)p(x_{t+2},\dots,x_T|x_{t+1},z_{t+1}^i=1,z_t^k=1)\\
	& \because p(a,b,c) = p(a)p(b|a)p(c|a,b)\\
	& = \sum_{i} p(z_{t+1}^i=1|z_t^k=1)p(x_{t+1}|z_{t+1}^i=1)p(x_{t+2},\dots,x_T|z_{t+1}^i=1)\\
	& = \sum_{i}a_{k,i}b_{i,x_{t+1}} \beta_{t+1}^i
\end{align*}

Another recursive structure:
\begin{align*}
	p(z_t^k=1,X) &= \alpha_{t}^k\beta_{t}^k\\
	& = b_{k,x_t}\sum_i \alpha_{t-1}^ia_{i,k} \times \sum_{i}a_{k,i}b_{i,x_{t}} \beta_{t+1}^i
\end{align*}
This means at time $t$, the latent label is belong to some class $k$ and this can be computed by using the forward probability and the backward probability. Now we can compute
\begin{align*}
p(z_t^k=1|X) &= \frac{p(z_t^k=1,X)}{p(X)} = \frac{\alpha_{t}^k\beta_{t}^k}{p(X)}
\end{align*}
Then, 
$$k_t = \argmax_{k}p(z_t^k=1|X)$$
Note that this is for a single latent variable at a single time step given the whole observation $X$, but we want to decode a sequence of latent variables. Thus, we need some decoding algorithm.

\section{Decoding: Viterbi Algorithm}
For any model, such as an HMM, that contains hidden variables, \textbf{the task of determining which sequence of variables is the underlying source of some sequence of observations is called the decoding task}.

We might propose to find the best sequence as follows: 
\begin{enumerate}
	\item For each possible hidden state sequence (HHH, HHC, HCH, etc.), we could run the forward algorithm and compute the likelihood of the observation sequence given that hidden state sequence.
	\item Then, we could choose the hidden state sequence with the maximum observation likelihood.
\end{enumerate}  
However, this is not a feasible solution, because there are an exponentially large number of state sequences.

Instead, the most common decoding algorithms for HMMs is the \textbf{Viterbi algorithm}. Like the forward algorithm, \textbf{Viterbi} is a kind of \textbf{dynamic programming algorithm.}

Note that the Viterbi algorithm is identical to the forward algorithm except that it takes the \textbf{max} over the previous path probabilities whereas the forward algorithm takes the \textbf{sum}. This is because, we want to obtain \textbf{the most probable latent variable sequence}. Note also that the Viterbi algorithm has one component that the forward algorithm doesn't have: \textbf{backpointers}. The reason is that while the forward algorithm needs to produce an observation likelihood, the Viterbi algorithm must produce a probability and also the most likely state sequence. We compute this best state sequence by keeping track of the path of hidden states that led to each state and then at the end backtracing the best path to the beginning (the Viterbi backtrace).

We can leverage the forward-backward probabilities:
\begin{itemize}
	\item $k^* = \argmax_{k}p(z_t^k=1|X) = \argmax_{k}p(z^k_t=1,X) = \argmax_{k}\alpha_{t}^k\beta_{t}^k$
\end{itemize}
We will use a forward approach:
% \setcounter{equation}{0}
\begin{align}
	V_t^k &= \max_{z_1,\dots,z_{t-1}}p(x_1,\dots,x_{t-1},z_1,\dots,z_{t-1},x_t,z_t^k=1)\\ 
	& = \max_{z_1,\dots,z_{t-1}}p(x_t,z_t^k=1|x_1,\dots,x_{t-1},z_1,\dots,z_{t-1})p(x_1,\dots,x_{t-1},z_1,\dots,z_{t-1})\\
	& = \max_{z_1,\dots,z_{t-1}}p(x_t,z_t^k=1|z_{t-1})p(x_1,\dots,x_{t-2},z_1,\dots,z_{t-2}, x_{t-1}, z_{t-1})\\
	& = \max_{z_{t-1}}p(x_t,z_t^k=1|z_{t-1})\max_{z_1,\dots,z_{t-2}}p(x_1,\dots,x_{t-2},z_1,\dots,z_{t-2}, x_{t-1}, z_{t-1})\\
	& = \max_{i\in z_{t-1}}p(x_t,z_t^k=1|z_{t-1}^i=1)V_{t-1}^i\\
	& = \max_{i\in z_{t-1}}p(x_t|z_t^k=1)p(z_t^k=1|z_{t-1}^i=1)V_{t-1}^i\\
	& = p(x_t|z_t^k=1)\max_{i\in z_{t-1}}p(z_t^k=1|z_{t-1}^i=1)V_{t-1}^i\\
	& = b_{k,x_t}\max_{i\in z_{t-1}}a_{i,k}V_{t-1}^i
\end{align}
\begin{itemize}
	\item $V_{t}^k$ is Viterbi variable which denotes the probability that the HMM is in state $k$ at $t$ after observing the first $t$ observations and $t-1$ latent variables. In another words, this is the probability of most likely sequence of states ending at state $z_t=k$.
	\item The first line assumes that the observation at time $t$ and the latent variable are fixed and also the fourth line has the recursive structure.
	\item The third step, only $z_{t-1}$ can affect the $z_{t}$, so we can remove all other unnecessary variables.
	\item The step six can be derived by the HMM structure. 
	\item $i\in z_{t-1}$ simply denotes the index of potential cluster at $t-1$.
	\item We have already computed the backward and the forward probabilities. So we just need to apply the Viterbi algorithm. 
	% \item $\textrm{idx}(x_t)$
\end{itemize}

Note that  Also note that we present the most probable path by taking the maximum over all possible previous state sequences $\max_{z_1,\dots,z_{t-1}}$. Like other DP-algorithm, Viterbi fills each cell recursively. 

%\LinesNumberedHidden
\begin{algorithm}
	$V_t^k = viterbi[M,T]$, where $M$ is the number states\\
	% Initialization: $\pi$ is the initial probability of being state $k$\\
	\For{k=1,\dots,M}{
		$V_1^k \leftarrow \pi_{z_k}b_{k,x_1}$\\
		$backpointer[k,1]\leftarrow 0$
	}
	\For{t=2,\dots,T}{
		\For{k=1,\dots,M}{
			$V_t^k \leftarrow b_{k,x_t}\max_{k'} V_t^{k'}a_{k',k}$, where $k'$ is the previous state.\\
			$backpointer[k,t]\leftarrow b_{k,x_t}\argmax_{k'} V_t^{k'}a_{k',k}$
		}
	}
	$bestpathprob \leftarrow \max_{k}V_T^{k}$ \quad //termination step
	
	$bestpathpointer \leftarrow \argmax_{k}V_T^{k}$ \quad//termination step
	
	$bestpath \leftarrow $ the path starting at state $bestpathpointer$, that follows backpointer[] to states back in time
	
	Return $bestpathpointer$, $bestpathprob$

	\caption{Viterbi Algorithm}
	\label{algo:viterbi}
\end{algorithm}

Viterbi algorithm typically shows some technical issues:
\begin{itemize}
	\item Underflow problems $\to$ log $V$.
\end{itemize}

\section{Learning: Baum-Welch Algorithm}
We have to learn HMM parameters with only $X$. Baum-Welch algorithm or Forward-Backward Algorithm is a standard training algorithm for HMM. The algorithm let us train both the transition and the emission probabilities of the HMM. If we do not have the information about $Z$, then we can assign the most probable $Z$ given $X$.

\begin{itemize}
	\item Given $X$, estimate parameters $\pi, a, b$.
		% $$\theta^* = \argmax_\theta \ln \sum_Z P(X,Z|\theta).$$
	\item Then, find the most probable $Z$ given the parameters. 
	% \item We don't have $Z, \pi, a, b$, so we need to find out them.
\end{itemize}
We will use EM algorithm!

\subsection{EM Algorithm}
\begin{align*}
	P(X|\theta) = \sum_Z P(X,Z|\theta) \to \ln P(X|\theta) = \ln \sum_Z P(X,Z|\theta).
\end{align*}
We cannot directly estimate the log-likelihood function, so we will estimate the expectation of it. 
\begin{align*}
	Q(\theta, \theta^{old}) &= \mathbb{E}_{Z}\ln P(X,Z|\theta) \\
							&= \sum_Z p(Z|X,\theta^{old})\ln P(X,Z|\theta)\\
							&= \sum_Z p(Z|X,\pi^t, a^t, b^t)\ln P(X,Z|\pi, a, b).
\end{align*}
Note that $p(X,Z) = \pi_{z_{1}}\prod_{t=2}^{T}a_{z_{t-1},z_t}\prod_{t=2}^{T}b_{z_{t},x_t}$. Thus, $\ln p(X,Z) = \ln \pi_{z_{1}}+\sum_{t=2}^{T}\ln a_{z_{t-1},z_t}+\sum_{t=1}^{T}\ln b_{z_{t},x_t}$. Therefore
$$Q(\theta, \theta^{old}) = \sum_Z p(Z|X, \theta^{old}) \bigg(\ln \pi_{z_{1}}+\sum_{t=2}^{T}\ln a_{z_{t-1},z_t}+\sum_{t=1}^{T}\ln b_{z_{t},x_t}\bigg).$$
To optimize the above function we will use the Lagrange method as follows: 
$$\mathcal{L}(\pi, a, b) = Q(\theta, \theta^{old}) - \lambda_\pi \bigg(\sum_{i=1}^K\pi_i-1\bigg) - \sum_i^K\lambda_{a_i} \bigg(\sum_{j=1}^Ka_{i,j}-1\bigg) - \sum_i^K\lambda_{b_i} \bigg(\sum_{j=1}^Kb_{i,j}-1\bigg).$$
The constraints are for forcing the sum of each probability is equal to 1. 

Now, take a partial derivative for each parameter. Let's take a derivative with regard to $\pi_i$ first. Then, 
\begin{align*}
	\frac{\partial \mathcal{L}}{\partial \pi_i} &= \frac{\partial Q(\theta, \theta^{old})}{\partial \pi_i} - \lambda_\pi\\
												&= \frac{\partial }{\partial \pi_i}\sum_Z p(Z|X, \theta^{old}) \ln \pi_{z_{1}} - \lambda_\pi\\
												&= \frac{p(z_1^i=1|X, \theta^{old})}{\pi_i} - \lambda_\pi\\
	\frac{\partial \mathcal{L}}{\partial \lambda_{\pi_i}} &= \sum_{i=1}^K\pi_i - 1 = 0 \to \sum_{i=1}^K\pi_i = 1.
\end{align*}
By setting the derivative is equal to zero, 
\begin{align*}
 \pi_i = \frac{p(z_1^i=1|X, \theta^{old})}{\lambda_\pi}. 
\end{align*}
By using the constraint of $\pi$, the Lagrange multiplier $\lambda_\pi$ must be a normalizer. 
\begin{align*}
	\pi_i = \frac{p(z_1^i=1|X, \theta^{old})}{\sum_{j=1}^K p(z_1^j=1|X, \theta^{old})}. 
\end{align*}
Similarly, we can compute other parameters too. 
\begin{align*}
	a^{t+1}_{i,j} &= \frac{\sum_{t=2}^T p(z_{t-1}^i=1, z_t^j=1|X, \theta^{old})}{\sum_{t=2}^T p(z_{t-1}^i=1|X, \theta^{old})}.\\ 
	b^{t+1}_{i,j} &= \frac{\sum_{t=1}^T p(z_{t1}^i=1|X, \theta^{old})I(x_t=j)}{\sum_{t=1}^T p(z_{t}^i=1|X, \theta^{old})}, 
\end{align*}
where $I(x)$ is an indicator function which returns 1 if $x$ is true and 0, otherwise. 



\section{Python Implementation}
\label{sec:hmm_python}

\subsection{Viterbi Algorithm}
The Viterbi algorithm is a dynamic programming algorithm used to determine the most probable sequence of hidden states in a Hidden Markov Model (HMM) based on a sequence of observations. 

The algorithm works by recursively computing the probability of the most likely sequence of hidden states that ends in each state for each observation.

At each time step, the algorithm computes the probability of being in each state and emits the current observation based on the probabilities of being in the previous states and making a transition to the current state.

Assuming we have an HMM with N hidden states and T observations, the Viterbi algorithm can be summarized as follows:

    Initialization: At time t=1, we set the probability of the most likely path ending in state i for each state i to the product of the initial state probability pi and the emission probability of the first observation given state i. This is denoted by: delta[1,i] = pi * b[i,1].
    Recursion: For each time step t from 2 to T, and for each state i, we compute the probability of the most likely path ending in state i at time t by considering all possible paths that could have led to state i. This probability is given by:

$$delta[t,i] = max_j(delta[t-1,j] * a[j,i] * b[i,t])$$

Here, a[j,i] is the probability of transitioning from state j to state i, and b[i,t] is the probability of observing the t-th observation given state I.

We also keep track of the most likely previous state that led to the current state i, which is given by:

$$psi[t,i] = argmax_j(delta[t-1,j] * a[j,i])$$

\begin{itemize}
	\item Termination: The probability of the most likely path overall is given by the maximum of the probabilities of the most likely paths ending in each state at time $T$. That is, $P* = max_i(delta[T,i])$.
	\item Backtracking: Starting from the state $i*$ that gave the maximum probability at time $T$, we recursively follow the psi values back to time $t=1$ to obtain the most likely path of hidden states.
\end{itemize}

The Viterbi algorithm is an efficient and powerful tool that can handle long sequences of observations using dynamic programming.


% \begin{lstlisting}[language=Python]
% import torch.optim as optim
% epsilon = 2./255

% delta = torch.zeros_like(pig_tensor, requires_grad=True) # init delta
% opt = optim.SGD([delta], lr=1e-1) # Update delta

% for t in range(30):
%     pred = model(norm(pig_tensor + delta))
% 	# For gradient ascent -CELoss
%     loss = -nn.CrossEntropyLoss()(pred, torch.LongTensor([341])) 
%     if t % 5 == 0:
%         print(t, loss.item())

%     opt.zero_grad()
%     loss.backward()
%     opt.step()
%     delta.data.clamp_(-epsilon, epsilon) # infinity norm

% print("True class probability:", nn.Softmax(dim=1)(pred)[0,341].item())
% \end{lstlisting}

\section{Summary}
\begin{itemize}
	\item The loss function can be decomposed.
		$$L_\text{VLB} = L_T + L_{T-1} + \dots + L_0 .$$
	\item $L_T = D_\text{KL}(q(\mathbf{x}_T \vert \mathbf{x}_0) \parallel p_\theta(\mathbf{x}_T))$
		\begin{itemize}
			\item Constant $\approx 0$ since $x_T$ is a Gaussian noise.
		\end{itemize}
	\item $L_t = D_\text{KL}(q(\mathbf{x}_{t-1} \vert \mathbf{x}_{t}, \mathbf{x}_0) \parallel p_\theta(\mathbf{x}_{t-1} \vert\mathbf{x}_{t})) \text{ for } t>1 $
	%\item $L_t = D_\text{KL}(q(\mathbf{x}_t \vert \mathbf{x}_{t+1}, \mathbf{x}_0) \parallel p_\theta(\mathbf{x}_t \vert\mathbf{x}_{t+1})) \text{ for }1 \leq t \leq T-1 $
		\begin{itemize}
			\item This is the main part.
		\end{itemize}
	\item $L_0 = - \log p_\theta(\mathbf{x}_0 \vert \mathbf{x}_1)$
		\begin{itemize}
			\item Can be modeled by a separate decoder.
		\end{itemize}
	\item $q(\rvx_t|\rvx_0) = \mathcal{N}(\sqrt{\bar{\alpha}_t}\rvx_0, (1-\bar{\alpha}_t)I )$
	\item $q(\rvx_t \vert \rvx_{t-1}) &= \mathcal{N}(\mathbf{x}_t; \sqrt{1 - \beta_t}\rvx_{t-1}, \beta_t\mathbf{I}),$
		\begin{itemize}
			\item[] We can sample by $\rvx_t=\sqrt{1 - \beta_t}\rvx_{t-1}+ \sqrt{\beta_t}\epsilon$
		\end{itemize}
	\item $p_\theta(\mathbf{x}_{t-1} \vert \mathbf{x}_t) = \mathcal{N}(\mathbf{x}_{t-1}; \boldsymbol{\mu}_\theta(\mathbf{x}_t, t), \boldsymbol{\Sigma}_\theta(\mathbf{x}_t, t))$.
		\begin{itemize}
			\item We need to learn mean and variance.
			\item DDPM kept the variance fixed and let the neural network only learn the mean $\mu_\theta$.
			\item $\boldsymbol{\Sigma}_\theta(\mathbf{x}_t, t)) = \sigma_t^2\mathbf{I}$ and set $\sigma_t^2 = \beta_t$.
		\item Improved DDPM model trains $\sigma$ also.
		\end{itemize}
	\item One can reparameterize the mean to make the nerual network learn the added noise via a network $\epsilon_\theta$.
		$$\mu_\theta(\rvx_t, t) = \frac{1}{\sqrt{\alpha_t}}\bigg(\rvx_t - \frac{\beta_t}{\sqrt{1-\bar{\alpha}_t}}\underbrace{\epsilon_\theta(\rvx_t,t)}_{\text{Network}}\bigg)$$
	\item Final objective function $L_t$ is
		$$||\epsilon -\epsilon_\theta(\rvx_t,t)||^2 = ||\epsilon -\epsilon_\theta(\sqrt{\bar{\alpha}_t}\rvx_0+\sqrt{(1-\bar{\alpha}_t)}\epsilon,t)||^2 $$
		\begin{itemize}
			\item $t\sim\text{Unif}[\{1,..,T\}]$ 
			\item $\rvx_t=\sqrt{\bar{\alpha}_t}\rvx_0+\sqrt{(1-\bar{\alpha}_t)}\epsilon \sim q(\rvx_t|\rvx_0)$
			\item $\epsilon\sim \mathcal{N}(0,I)$
		\end{itemize}
	\item $\rvx_t$ is perturbed by $\epsilon$ and the noise prediction network $\epsilon_\theta$ predicts $\epsilon$.
\end{itemize}

\newpage
\begin{algorithm}[t]
	\caption{Training}
	\label{alg:diffusion_training}
	\Repeat{\textrm{converged}}{
			$\rvx_0\sim q(\rvx_0)$\\
			$t\sim \text{Unif}[\{1,\cdots,T\}]$\\
			$\epsilon\sim \mathcal{N}(0,I)$\\
			Take gradient descent step on
			$\nabla_\theta||\epsilon -\epsilon_\theta(\sqrt{\bar{\alpha}_t}\rvx_0+\sqrt{(1-\bar{\alpha}_t)}\epsilon,t)||^2$
		}
\end{algorithm}

The training process is given by
\begin{enumerate}
	\item $\rvx_0\sim q(\rvx_0)$ 
	\item Sample a noise level $t$ between $1$ and $T$ (\ie random time step).
	\item Sample a noise from a Gaussian distribution and perturb the input by the sampling equation.
	\item NN is trained to predict this noise $\epsilon$ used for generating $\rvx_t$.
	\item $\beta$ is often scheduled linearly.
	\item $\Sigma$ is set equal to $\beta$.
\end{enumerate}

The sampling process is given by
\begin{algorithm}[h]
	\caption{Sampling}
	\label{alg:diffusion_sampling}
		$\rvx_T\sim \mathcal{N}(0,I)$\\
		\For{$t=T,\cdots,1$}{
			$\rvz\sim \mathcal{N}(0,I)$\\
			\State $\rvx_{t-1}= \frac{1}{\sqrt{\alpha_t}}\bigg(\rvx_t-\frac{1-\alpha_t}{\sqrt{1-\bar{\alpha}_t}}\boldsymbol{\epsilon}_\theta(\rvx_t,t)\bigg)+\Sigma_t\rvz$
		}
		\textbf{return} $\rvx_0$
\end{algorithm*}
\begin{itemize}
	\item Ancestral sampling.
	\item $T$ is typically around 1,000
\end{itemize}
	



\begin{figure}[h]
	\centering
	\includegraphics[scale=0.53]{./images/generative/flows/generative_models.png}
	\caption{An overview of generative models.}
\end{figure}

\begin{itemize}
	\item Flow-models get an exact estimate of the likelihood of your sample, as well as in the reverse direction. 
	\item VAEs optimize a lower bound on the (log) likelihood 
	\item GANs minimize a discrepancy between your input and transformed noise distributions. 
\end{itemize}



\section{The Method of Transformations of Random Variables}

If we are interested in finding the PDF of $Y=g(X)$, where $g(\cdot)$ is some deterministic transformation of $X$, and the function $g$ satisfies following properties, we can utilize a method called the method of transformations.
\begin{itemize}
	\item $g(x)$ is differentiable;
	\item $g(x)$ is a strictly (or monotonically) increasing function, that is, if $x_1<x_2$, then $g(x_1)<g(x_2)$.
\end{itemize}

% Now, let $X$ be a continuous random variable and $Y=g(X)$. We will show that you can directly find the PDF of $Y$ using the following formula.
% \begin{equation*}
% 	f_Y(y) = 
% 	\begin{cases}
% 	\frac{f_X(x_1)}{g'(x_1)}=f_X(x_1). \frac{dx_1}{dy} & \quad \textrm{where } g(x_1)=y\\
% 	0 & \quad \textrm{if }g(x)=y \textrm{ does not have a solution}
% 	\end{cases} 
% \end{equation*}

% Note that the derivative $\frac{dx}{dy}$ or $\frac{d}{dy}(g^{-1}(y))$ \textbf{measures how $X$ changes with respect to $Y$}.
% % Note that start with the function $y=f^{-1}(x)$. Write this as $x=f(y)$ and differentiate both sides implicitly with respect to$x$ using the Chain Rule:
% % \begin{align*}
% % 	1 &= f'(y)\frac{dy}{dx}\\
% % 	\frac{dy}{dx} &= \frac{1}{f'(y)}\\
% % 	y &= f^{-1}(x)\\
% % 	[f^{-1}]'(x) &= \frac{1}{f'(f^{-1}(x))}
% % \end{align*}
% Since $g$ is strictly increasing, its inverse function $g^{-1}$ is well defined. You can imagine a simple function like a linear function, \eg $Y=3X+1$. Then, for each $y\in R_Y$, there exists a \textbf{unique} $x_1$ such that $g(x_1)=y$. We can write $x_1=g^{-1}(y)$. Then,
% \begin{align*}
% 	\{Y\leq y\} = \{g(X)\leq y\} = \{X \leq g^{-1}(x)\}.
% \end{align*}
% Thus, 
% \begin{align*}
% 	F_Y(y) &= P(Y\leq y)\\
% 		   &= P(g(X)\leq y)\\
% 		   &= P(X\leq g^{-1}(y))\quad \text{, since }g \text{ is strictly increasing.}\\
% 		   &= F_X(g^{-1}(y)).
% \end{align*}
% To find the PDF of $Y$, we differentiate $F_Y(y)$ as follows:
% \begin{align*}
% 	f_Y(y) &= \frac{d}{dy}F_X(x_1)\quad \text{by } g(x_1)=y\\
% 		   &=\frac{dx_1}{dy}\cdot \underbrace{\frac{d}{dx_1}F_X(x_1)}_{=F'_X(x_1)}\\
% 		   &=\frac{dx_1}{dy}f_X(x_1)\\
% 		   &=f_X(g^{-1}(y))\left|\frac{d}{dy}(g^{-1}(y))\right|
% 		   % &= \frac{f_X(x_1)}{g'(x_1)} \quad \text{, since } \frac{dx}{dy}=\frac{1}{\frac{dy}{dx}}.
% \end{align*}
% We can repeat the same argument for the case where $g$ is \textbf{strictly decreasing}. In that case, $g'(x_1)$ will be \textbf{negative}, so we need to use $|g'(x_1)|$ . Thus, we can state the following theorem for a \textit{strictly monotonic function}. (A function $g:R\to R$ is called strictly monotonic if it is strictly increasing or strictly decreasing.)

% Actually, we assumed that $g$ was one-to-one out of convenience: the condition that $g$ is one-to-one is not necessary for change of variables to work: Consider a continuous random variable $X$ with domain $R_X$, and let $Y=g(X)$. Suppose that we can partition $R_X$ into a finite number of intervals such that $g(x)$ is strictly monotone and differentiable on each partition. Then the PDF of $Y$ is given by 
% \begin{align*}
% 	f_Y(y)= \sum_{i=1}^{n} \frac{f_X(x_i)}{|g'(x_i)|}= \sum_{i=1}^{n} f_X(x_i).
% 	\left|\frac{dx_i}{dy}\right|,
% \end{align*}
% where $x_1,\dots,x_n$ are real solutions to $g(x)=y$.

% \subsection{Intuitive Explanation}
How to derive the PDF of the random variable $Y=g(X)$ when one knows the PDF of the random variable $X$? If $X$ is discrete, we can derive the pmf for $Y$ by simply summing up the probability mass for all the $x$'s such that $f(x)=y$. For a general function $g$, there is no direct formula to get the PDF of the random variable $Y=g(X)$ knowing $p(X)$. There is a formula in case when $h$ is a differentiable one-to-one mapping from the range (\ie the support) of $X$ to the range of $Y$.

Take for example a random variable $X\sim \mathcal{N}(\mu, \sigma)$ and set $Y=\exp(X)$. The figure below shows some simulations of $X$ and the corresponding values of $Y$. The density of $X$ is shown in blue and the one of $Y$ is shown in orange in the vertical direction.

\begin{figure}[t]
	\centering
	\includegraphics[scale=0.23]{./images/generative/flows/change_of_vars.pdf}
\end{figure}

\begin{figure}[t]
    \centering
    \includegraphics[scale=0.7]{./images/generative/flows/change_intuition.pdf}
    \caption{Area would be approximately $p(z)dz = q(z)dx$. Thus, $q(x) = p(z)\Big|\frac{dz}{dx}\Big|$}
    \label{fig:change_intuition}
\end{figure}


% \begin{figure}[ht]
%     \centering
%     \begin{minipage}[b]{0.45\textwidth}
%         \centering
%         \includegraphics[width=\textwidth]{./images/generative/flows/sample.png}
%     \end{minipage}
%     \hfill
%     \begin{minipage}[b]{0.45\textwidth}
%         \centering
%         \includegraphics[width=\textwidth]{./images/generative/flows/sample2.png}
%     \end{minipage}
% \end{figure}
Now the question is: knowing the density of $X$, what is the density of $Y$?
Taking a point $y$ in the range of $Y$, the PDF $f_Y$ provides the probability of $Y$, belong to a small area $dy$ around $y$ by the formula below
$$P(Y\in dy)\approx f_Y(y)|dy|,$$
where $P(Y\in dy)$ is the area below the curve. Similarly, we can define
$$P(X\in dx)\approx f_X(x)|dx|$$
The above two areas are approximately the same in case of very small region. Note that if $dy$ and $dx$ are very small, we can approximate the derivative of $g'(x)=\frac{|dy|}{|dx|}$. Compactly, this can be expressed as follows:
$$P(X\in dx) = f_X(x)\frac{|dy|}{g'(x)}$$
With $y=g(x)$ we can get 
\begin{align*}
	P(X\in dx)\approx P(Y\in dy) &= f_X(x)\frac{|dy|}{g'(x)}\\
	& = f_X(g^{-1}(y))\frac{|dy|}{g'(g^{-1}(y))}\\
	& = f_X(g^{-1}(y))|dy|(g^{-1})'(y)
\end{align*}
The last line is by the derivative of inverse function which is 
\begin{align*}
	\frac{d}{dx}f^{-1}(x) = \frac{1}{f'(f^{-1}(x))}
\end{align*}
Then, 
\begin{align*}
	P(Y\in dy)\approx f_Y(y)|dy| = f_X(g^{-1}(y))|(g^{-1})'(y)|\cdot |dy|
\end{align*}
Finally, we can get 
$$f_Y(y) = f_X(g^{-1}(y))|(g^{-1})'(y)|$$
Note that the absolute is determined by the function $h$. This is the so-called \textit{change of variables formula}.



\subsection{Vector to Vector}

$Z$ and $X$ be random variables which are related by a mapping $f:\mathbb{R}^n\to \mathbb{R}^n$ such that $X=f(Z)$ and $Z=f^{-1}(X)$. Then
\begin{align*}
	p_X(\mathbf{x}) = p_Z(f^{-1}(\mathbf{x})) \left\vert \text{det}\left(\frac{\partial f^{-1}(\mathbf{x})}{\partial \mathbf{x}}\right) \right\vert
\end{align*}

Note that for any invertible square matrix $A$ over a field (\eg real or complex numbers),
\begin{align*}
	\det\bigl(A^{-1}\bigr)=\frac{1}{\det(A)}.
\end{align*}

Let
$$
A=\begin{bmatrix}
2 & 1\\[2pt]
3 & 4
\end{bmatrix}.
$$

Then, determinant of $A$ is given by
$$
\det(A)=2\cdot4-3\cdot1 = 8-3 = 5.
$$

The inverse of $A$ is 
$$
A^{-1}= \frac{1}{5}\begin{bmatrix}
4 & -1\\[2pt] -3 & 2
\end{bmatrix}.
$$

Then, the determinant of $A^{-1}$ is
$$
\det(A^{-1}) = \frac{1}{5^2}\bigl(4\cdot2-(-3)(-1)\bigr)
=\frac{1}{25}(8-3)=\frac{5}{25}=\frac{1}{5}.
$$

You can confirm the property:

$$
\det(A^{-1})=\frac{1}{5}=\frac{1}{\det(A)}.
$$


For example, we can transform $(x_1,x_2)$ to $(r,\theta)$ via $x_1=r\cos\theta$ and $x_2=r\sin\theta$.

Then
\[
J_{y\to x}=
\begin{pmatrix}
\dfrac{\partial x_1}{\partial r} & \dfrac{\partial x_1}{\partial\theta}\\[6pt]
\dfrac{\partial x_2}{\partial r} & \dfrac{\partial x_2}{\partial\theta}
\end{pmatrix}
=
\begin{pmatrix}
\cos\theta & -r\sin\theta\\
\sin\theta & \phantom{-}r\cos\theta
\end{pmatrix},
\tag{3}
\]
so
\[
\bigl|\det J_{y\to x}\bigr|
 = \bigl|\,r\cos^{2}\theta + r\sin^{2}\theta\,\bigr|
 = |r|.
\tag{4}
\]

Hence
\[
p_{\mathbf{y}}(\mathbf{y}) = p_{\mathbf{x}}(\mathbf{x})\,|J_{y\to x}|
\quad\Longrightarrow\quad
p_{R,\Theta}(r,\theta) = p_{X_1,X_2}(x_1,x_2)\,r.
\tag{5--6}
\]

For a two dimensional random vector $(X,Y)$ with density $p_{X,Y}$, 
\begin{align*}
	Pr((X,Y)\in A) = \int\int_A p_{X,Y}(x,y)dxdy.
\end{align*}
For a infinitesimally small region, we can approximate as follows:
\begin{align*}
	\int_x^{x+dx} p_{X}(x)dx \approx  p_{X}(x)dx.
\end{align*}

Similarly, we can get the probability as follows:
\[
P\!\bigl(r\le R\le r+dr,\;
        \theta\le\Theta\le\theta+d\theta\bigr)
  = p_{R,\Theta}(r,\theta)\,dr\,d\theta.
\tag{7}
\]
Note that the length of arc is $r\times d\theta$. Thus, the area $r\,dr\,d\theta$ or probability is given by
\[
P\!\bigl(r\le R\le r+dr,\;
        \theta\le\Theta\le\theta+d\theta\bigr)
  = p_{X,Y}(r\cos\theta, r\sin\theta)\,r\,dr\,d\theta,
\tag{8--9}
\]
so that finally
\[
p_{R,\Theta}(r,\theta)
  = p_{X,Y}(r\cos\theta, r\sin\theta)\,r.
\tag{10}
\]

% \begin{figure}[h]
%   \centering
%   % Replace the filename with the actual graphic if you have it.
%   \includegraphics[width=.55\linewidth]{polar_patch}
%   \caption{Change of variables from polar to Cartesian.
%            The area of the shaded patch is $r\,dr\,d\theta$.}
%   \label{fig:polarPatch}
% \end{figure}

\section{Distribution Modeling}
	What we want to learn (or model) is $p_\theta(\rvx_0)\approx p(\rvx_0)$ (approximate data distribution).
	\begin{itemize}
%		\item $p_\theta(\rvx_0)$: a distribution of output (denoised) image.
%			\begin{itemize}
%				\item This is our target.
%			\end{itemize}
		\item $p_\theta(\rvx_0) = \int p_\theta(\rvx_{0:T})d\rvx_{1:T}$
		\item It is intractable to compute all trajectories.
			$$\argmax_\theta\mathbb{E}_{\rvx_0\sim p}[\log p_\theta(\rvx_0)] = \mathbb{E}_{\rvx_0\sim p}\bigg[\log\int p_\theta(\rvx_{0:T})d\rvx_{1:T}\bigg].$$
		\item Thus, we will use variational lower bound with KL-Div:
	\end{itemize}
\begin{align}
\log p_\theta(\rvx_0) 
	&=  \log\int p(\rvx_{0:T})d\rvx_{1:T}\\
	&= \log\int p(\rvx_{0:T})\frac{q(\rvx_{1:T}|\rvx_0)}{q(\rvx_{1:T}|\rvx_0)}d\rvx_{1:T}\\
	&= \log\int q(\rvx_{1:T}|\rvx_0)\frac{p(\rvx_{0:T})}{q(\rvx_{1:T}|\rvx_0)}d\rvx_{1:T}\\
	&\geq \int q(\rvx_{1:T}|\rvx_0)\log\frac{p(\rvx_{0:T})}{q(\rvx_{1:T}|\rvx_0)}d\rvx_{1:T}\\
	&= \mathbb{E}_{q(\rvx_{1:T}|\rvx_0)}\bigg[\log\frac{p(\rvx_{0:T})}{q(\rvx_{1:T}|\rvx_0)}\bigg] \quad \to \textrm{ELBO}\\
	&= \mathbb{E}_{q(\rvx_{1:T}|\rvx_0)}\bigg[\log p(\rvx_T)\prod_{t=1}^T\frac{p(\rvx_{t-1}|\rvx_t)}{q(\rvx_{t}|\rvx_{t-1})}\bigg]\\
	&= \mathbb{E}_{q(\rvx_{1:T}|\rvx_0)}\bigg[\log \frac{p(\rvx_T)p(\rvx_0|\rvx_{1})\prod_{t=1}^{T-1} p(\rvx_{t}|\rvx_{t+1})}{q(\rvx_T|\rvx_{T-1})\prod_{t=1}^{T-1}  q(\rvx_{t}|\rvx_{t-1})}\bigg]\\
	&= \mathbb{E}_{q(\rvx_{1:T}|\rvx_0)}\bigg[\log \frac{p(\rvx_T)p(\rvx_0|\rvx_{1})}{q(\rvx_T|\rvx_{T-1})}\bigg] + \mathbb{E}_{q(\rvx_{1:T}|\rvx_0)}\bigg[\log \prod_{t=1}^{T-1}\frac{ p(\rvx_{t}|\rvx_{t+1})}{q(\rvx_{t}|\rvx_{t-1})}\bigg]\\
	&= \mathbb{E}_{q(\rvx_{1:T}|\rvx_0)}[\log p(\rvx_0|\rvx_{1})]+\mathbb{E}_{q(\rvx_{1:T}|\rvx_0)}\bigg[\log \frac{p(\rvx_T)}{q(\rvx_T|\rvx_{T-1})}\bigg]\\ 
	&\quad + \mathbb{E}_{q(\rvx_{1:T}|\rvx_0)}\bigg[\sum_{t=1}^{T-1}\log \frac{ p(\rvx_{t}|\rvx_{t+1})}{q(\rvx_{t}|\rvx_{t-1})}\bigg]\\
	&= \dots + \blue{\mathbb{E}_{q(\rvx_{1:T}|\rvx_0)}\bigg[\log \frac{p(\rvx_T)}{q(\rvx_T|\rvx_{T-1})}\bigg]} + \sum_{t=1}^{T-1}\mathbb{E}_{q(\rvx_{1:T}|\rvx_0)}\bigg[\log \frac{ p(\rvx_{t}|\rvx_{t+1})}{q(\rvx_{t}|\rvx_{t-1})}\bigg]\\
	% % &= \log\int q(\rvx_{1:T}|\rvx_0)p(\rvx_T)\prod_{t=1}^T\frac{p(\rvx_{t-1}|\rvx_t)}{q(\rvx_{t}|\rvx_{t-1})}d\rvx_{1:T}
		&= \dots + \int_{x_1}\dots\int_{x_T}q(\rvx_{1},\dots,\rvx_{T})\bigg[\log \frac{p(\rvx_T)}{q(\rvx_T|\rvx_{T-1})}\bigg]dx_1\dots dx_T + \dots\\
		&= \dots + \int_{x_T}\int_{x_{T-1}}\bigg[\log \frac{p(\rvx_T)}{q(\rvx_T|\rvx_{T-1})}\bigg]\dots\int_{x_1} q(\rvx_{1},\dots,\rvx_{T})dx_1\dots dx_T + \dots\\
		&= \dots + \int_{x_T}\int_{x_{T-1}}\bigg[\log \frac{p(\rvx_T)}{q(\rvx_T|\rvx_{T-1})}\bigg] q(\rvx_{t},\rvx_{t-1})dx_{T-1} dx_T + \dots\\
		&= \dots + \mathbb{E}_{q(\rvx_{t}, \rvx_{t-1}|\rvx_0)}\bigg[\log \frac{p(\rvx_T)}{q(\rvx_T|\rvx_{T-1})}\bigg] + \sum_{t=1}^{T-1}\mathbb{E}_{q(\rvx_{t-1},\rvx_t,\rvx_{t+1}|\rvx_0)}\bigg[\log \frac{ p(\rvx_{t}|\rvx_{t+1})}{q(\rvx_{t}|\rvx_{t-1})}\bigg]\\
		&= \mathbb{E}_{q(\rvx_{1}|\rvx_0)}[\log p(\rvx_0|\rvx_{1})] + \mathbb{E}_{q(\rvx_{t}, \rvx_{t-1}|\rvx_0)}\bigg[\log \frac{p(\rvx_T)}{q(\rvx_T|\rvx_{T-1})}\bigg]\\ 
		&\quad+ \sum_{t=1}^{T-1}\mathbb{E}_{q(\rvx_{t-1},\rvx_t,\rvx_{t+1}|\rvx_0)}\bigg[\log \frac{ p(\rvx_{t}|\rvx_{t+1})}{q(\rvx_{t}|\rvx_{t-1})}\bigg]\\
		&= \mathbb{E}_{q(\rvx_{1}|\rvx_0)}[\log p(\rvx_0|\rvx_{1})] - \mathbb{E}_{q(\rvx_{t-1}|\rvx_0)}[D_{KL}(q(\rvx_T|\rvx_{T-1})\| p(\rvx_T))]\\
		&\,- \sum_{t=1}^{T-1}\mathbb{E}_{q(\rvx_{t-1},\rvx_{t+1}|\rvx_0)}D_{KL}[q(\rvx_{t}|\rvx_{t-1})\|p(\rvx_{t}|\rvx_{t+1})]\\
		& \because q(\rvx_{t}, \rvx_{t-1}|\rvx_0) = \frac{q(\rvx_{t}, \rvx_{t-1},\rvx_0)}{q(\rvx_0)} = \frac{q(\rvx_{t}|\rvx_{t-1}, \rvx_0)q(\rvx_{t-1}, \rvx_0)}{q(\rvx_0)} = q(\rvx_{t}|\rvx_{t-1})q(\rvx_{t-1}|\rvx_0)
	% % &= \log\int q(\rvx_{1:T}|\rvx_0)p(\rvx_T)\prod_{t=1}^T\frac{p(\rvx_{t-1}|\rvx_t)}{q(\rvx_{t}|\rvx_{t-1})}d\rvx_{1:T}
	\label{eq:diffusion_mle}
\end{align}
\begin{itemize}
	\item The sixth step is done by Markov property.
	\item The first term is a \textit{reconstruction term}. 
	\item The second term is a \textit{prior matching term}. This term requires no optimization, as it has no trainable parameters; furthermore, as we have assumed a large enough $T$ such that the final distribution is Gaussian, this term effectively becomes zero.
	\item The last term is a \textit{consistency term}. It endeavors to make the distribution at xt consistent, from both forward and backward processes. That is, a denoising step from a noisier image should match the corresponding noising step from a cleaner image, for every intermediate timestep.  This term is minimized when we train $p_\theta(\rvx_t|\rvx_{t+1})$ to match the Gaussian distribution $q(\rvx_t|\rvx_{t-1})$, which is defined in \Cref{eq:forward_diffusion}. % \item For a small $\beta$, the forward and the backward processes can be identical.
	\item Under this derivation, all terms of the ELBO are computed as expectations, and can therefore be approximated using Monte Carlo estimates. 
	\item However, actually optimizing the ELBO using the terms we just derived might be suboptimal, because the consistency term is computed as an expectation over two random variables $\{x_{t-1}, x_{t+1}\}$ for every time step, the variance of its Monte Carlo estimate could potentially be higher than a term that is estimated using only one random variable per time step. As it is computed by summing up $T-1$ consistency terms, the final estimated value of the ELBO may have high variance for large $T$ values.
\end{itemize}

Let us instead try to derive a form for our ELBO where each term is computed as an expectation over \textbf{only one random variable at a time}. The key insight is that we can rewrite encoder transitions as $q(x_t|x_{t-1}) = q(x_t|x_{t-1}, x_0)$, where the extra conditioning term is superfluous due to the Markov property. Then, according to Bayes rule, we can rewrite each transition as:
\begin{align*}
	q(\rvx_t|\rvx_{t-1}, \rvx_0)= \frac{q(\rvx_{t-1}|\rvx_t, \rvx_0)q(\rvx_t| \rvx_0)}{q(\rvx_{t-1}| \rvx_0)} 
\end{align*}
Armed with this equation, we can factorize the ELBO again as follows:
\begin{align}
		L &= \mathbb{E}_{q(\rvx_{1:T}|\rvx_0)}\bigg[\log\frac{p(\rvx_{0:T})}{q(\rvx_{1:T}|\rvx_0)}\bigg] \quad \to \textrm{ELBO}\\		
		&= \mathbb{E}_{q(\rvx_{1:T}|\rvx_0)}\bigg[\log p(\rvx_T)\prod_{t=1}^T\frac{p(\rvx_{t-1}|\rvx_t)}{q(\rvx_{t}|\rvx_{t-1})}\bigg]\\
		&= \mathbb{E}_{q(\rvx_{1:T}|\rvx_0)}\bigg[\log \frac{p(\rvx_T)p(\rvx_0|\rvx_{1})\prod_{t=2}^{T} p(\rvx_{t-1}|\rvx_{t})}{q(\rvx_1|\rvx_{0})\prod_{t=2}^{T}  q(\rvx_{t}|\rvx_{t-1})}\bigg]\\
		&= \mathbb{E}_{q(\rvx_{1:T}|\rvx_0)}\bigg[\log \frac{p(\rvx_T)p(\rvx_0|\rvx_{1})\prod_{t=2}^{T} p(\rvx_{t-1}|\rvx_{t})}{q(\rvx_1|\rvx_{0})\prod_{t=2}^{T}  q(\rvx_{t}|\rvx_{t-1}, \blue{\rvx_0})}\bigg] \quad \textrm{ by Markov Property}\\
		&= \mathbb{E}_{q(\rvx_{1:T}|\rvx_0)}\bigg[\log \frac{p(\rvx_T)p(\rvx_0|\rvx_{1})}{q(\rvx_1|\rvx_{0})}+\log \prod_{t=2}^{T}\frac{ p(\rvx_{t-1}|\rvx_{t})}{q(\rvx_{t}|\rvx_{t-1}, \rvx_0)}\bigg]\\
		&= \mathbb{E}_{q(\rvx_{1:T}|\rvx_0)}\bigg[\log \frac{p(\rvx_T)p(\rvx_0|\rvx_{1})}{q(\rvx_1|\rvx_{0})}+\log \prod_{t=2}^{T}\frac{ p(\rvx_{t-1}|\rvx_{t})}{\frac{q(\rvx_{t-1}|\rvx_t, \rvx_0)q(\rvx_t| \rvx_0)}{q(\rvx_{t-1}| \rvx_0)}}\bigg]\\
		&= \mathbb{E}_{q(\rvx_{1:T}|\rvx_0)}\bigg[\log \frac{p(\rvx_T)p(\rvx_0|\rvx_{1})}{q(\rvx_1|\rvx_{0})}+\log \frac{q(\rvx_1|\rvx_0)}{q(\rvx_T|\rvx_0)}+\log \prod_{t=2}^{T}\frac{ p(\rvx_{t-1}|\rvx_{t})}{q(\rvx_{t-1}|\rvx_t, \rvx_0)}\bigg]\\
		&= \mathbb{E}_{q(\rvx_{1:T}|\rvx_0)}\bigg[\log \frac{p(\rvx_T)p(\rvx_0|\rvx_{1})}{q(\rvx_T|\rvx_{0})}+\log \prod_{t=2}^{T}\frac{ p(\rvx_{t-1}|\rvx_{t})}{q(\rvx_{t-1}|\rvx_t, \rvx_0)}\bigg]\\
		&= \mathbb{E}_{q(\rvx_{1}|\rvx_0)}[\log p(\rvx_0|\rvx_{1})]+\mathbb{E}_{q(\rvx_{T}|\rvx_0)}\bigg[\log \frac{p(\rvx_T)}{q(\rvx_T|\rvx_{0})}\bigg]+\ \sum_{t=2}^{T}\mathbb{E}_{q(\rvx_{t},\rvx_{t-1}|\rvx_0)}\bigg[\log\frac{ p(\rvx_{t-1}|\rvx_{t})}{q(\rvx_{t-1}|\rvx_t, \rvx_0)}\bigg]\\
		&= \mathbb{E}_{q(\rvx_{1}|\rvx_0)}[\log p(\rvx_0|\rvx_{1})]-D_{KL}_(q(\rvx_{T}|\rvx_0)||p(\rvx_T))-\sum_{t=2}^{T}\mathbb{E}_{q(\rvx_{t}|\rvx_0)}[D_{KL}(q(\rvx_{t-1}|\rvx_t, \rvx_0)\|p(\rvx_{t-1}|\rvx_{t}))]
		% &= \mathbb{E}_{q(\rvx_{1:T}|\rvx_0)}\bigg[\log \frac{p(\rvx_T)p(\rvx_0|\rvx_{1})}{q(\rvx_T|\rvx_{T-1})}\bigg] + \mathbb{E}_{q(\rvx_{1:T}|\rvx_0)}\bigg[\log \prod_{t=1}^{T-1}\frac{ p(\rvx_{t}|\rvx_{t+1})}{q(\rvx_{t}|\rvx_{t-1})}\bigg]\\
% 		&= \mathbb{E}_{q}\Bigg[-\log p_\theta(x_T)-\log\frac{\prod_{t=1}^Tp_\theta(x_{t-1}|x_t)}{\prod_{t=1}^Tq(x_{t}|x_{t-1})}\Bigg]\\
% 		&= \mathbb{E}_{q}\Bigg[-\log p_\theta(x_T)-\sum_{t\geq 1}\log\frac{p_\theta(x_{t-1}|x_t)}{q(x_{t}|x_{t-1})}\Bigg]\\
% 		&= \mathbb{E}_{q}\Bigg[-\log p_\theta(x_T)-\sum_{t=2}^T\log\frac{p_\theta(x_{t-1}|x_t)}{q(x_{t}|x_{t-1})}-\log \frac{p_\theta(x_{0}|x_1)}{q(x_{1}|x_{0})}\Bigg]\\
% 		&= \mathbb{E}_{q}\Bigg[-\log p_\theta(x_T)-\sum_{t=2}^T\log\frac{p_\theta(x_{t-1}|x_t)}{q(x_{t-1}|x_{t},x_0)}\cdot \red{ \frac{q(x_{t-1}|x_0)}{q(x_{t}|x_0)}}\\
% 		&\quad -\log \frac{p_\theta(x_{0}|x_1)}{q(x_{1}|x_{0})}\Bigg]\\
	\label{eq:diffusion_objective}
\end{align}
Let's closely look at the last three terms:
\begin{itemize}
	\item $\mathbb{E}_{q(\rvx_{1}|\rvx_0)}[\log p(\rvx_0|\rvx_{1})]$: reconstruction term. 
	\item $D_{KL} (q(\rvx_{T}|\rvx_0)||p(\rvx_T))$: Prior matching term.
		\begin{itemize}
			\item No trainable parameters 
		\end{itemize}
	\item $\sum_{t=2}^{T}\mathbb{E}_{q(\rvx_{t}|\rvx_0)}[D_{KL}(q(\rvx_{t-1}|\rvx_t, \rvx_0)\|p(\rvx_{t-1}|\rvx_{t}))]$: Denoising matching term.
		\begin{itemize}
			\item $p_\theta(\rvx_{t−1}|\rvx_t)$ as an approximation to tractable, ground-truth denoising transition step $q(\rvx_{t−1}|\rvx_t, \rvx_0)$. The $q(\rvx_{t−1}|\rvx_t, \rvx_0)$ transition step can act as a ground-truth signal, since it defines how to denoise a noisy image with access to what the final, completely denoised image $\rvx_0$ should be. This term is therefore minimized when the two denoising steps match as closely as possible, as measured by their KL Divergence.
		\end{itemize}
\end{itemize}
In the last term, $q(\rvx_{t-1}|\rvx_t, \rvx_0)$ can be further factorized as follows:
\begin{align*}
	q(\rvx_{t-1}|\rvx_t, \rvx_0) = \frac{q(\rvx_{t}|\rvx_{t-1}, \rvx_0)q(\rvx_{t-1}|\rvx_0)}{q(\rvx_{t}|\rvx_0)}=\frac{q(\rvx_{t}|\rvx_{t-1})q(\rvx_{t-1}|\rvx_0)}{q(\rvx_{t}|\rvx_0)}.
\end{align*}
Then, $q(\rvx_{t}|\rvx_{t-1}, \rvx_0) = q(\rvx_{t}|\rvx_{t-1})$ by Markov property. Now, let's compute the KL-divergence in the denoising matching term. 

\begin{align}
	q(\rvx_{t-1}|\rvx_t, \rvx_0) &=\frac{q(\rvx_{t}|\rvx_{t-1})q(\rvx_{t-1}|\rvx_0)}{q(\rvx_{t}|\rvx_0)}\\
								 &= \frac{\mathcal{N}(\mathbf{x}_t; \sqrt{\bar{\alpha}_t} \rvx_{t-1}, (1-\bar{\alpha}_t)\mathbf{I})\mathcal{N}(\mathbf{x}_{t-1}; \sqrt{\bar{\alpha}_{t-1}} \rvx_{0}, (1-\bar{\alpha}_{t-1})\mathbf{I})}{\mathcal{N}(\mathbf{x}_t; \sqrt{\bar{\alpha}_t} \rvx_{0}, (1-\bar{\alpha}_t)\mathbf{I})}\\
								 & \vdots\\
								 &\propto \mathcal{N}\bigg(\rvx_{t-1};\underbrace{\frac{\sqrt{{\alpha}_{t}}(1-\bar{\alpha}_{t-1})\rvx_t+\sqrt{\bar{\alpha}_{t-1}}(1-{\alpha}_{t})\rvx_0}{1-\bar{\alpha}_{t}}}_{\mu_q(\rvx_t,\rvx_0)}, \underbrace{\frac{(1-{\alpha}_{t})(1-\bar{\alpha}_{t-1})}{1-\bar{\alpha}_{t}}\mathbf{I}}_{\Sigma_{q}(t)}\bigg)
	\label{eq:diffusion_kl_true_denoising}
\end{align}

We have therefore shown that at each step, $\rvx_{t-1} \sim q(\rvx_{t-1}|\rvx_t, \rvx_0)$ is normally distributed, with mean $\mu_q(\rvx_t,\rvx_0)$ that is a function of $\rvx_t$ and $\rvx_0$, and variance $\Sigma_q(t)$ as a function of $\alpha$ coefficients. 

% These $\alpha$ coefficients are known and fixed at each time step; they are either set permanently when modeled as hyperparameters, or treated as the current inference output of a network that seeks to model them. We can rewrite the variance equation as $\Sigma_q(t) = \sigma^2_q(t)I$, where:
% $$\sigma^2_q(t) = \frac{(1-{\alpha}_{t})(1-\bar{\alpha}_{t-1})}{1-\bar{\alpha}_{t}}.$$
% In order to match approximate denoising transition step $p_\theta(\rvx_{t-1}|\rvx_{t})$ to ground-truth denoising transition step $q(\rvx_{t-1}|\rvx_t, \rvx_0)$ as closely as possible, we can also model it as a Gaussian. Furthermore, as all $\alpha$ terms are known to be frozen at each timestep, we can immediately construct the variance of the approximate denoising transition step to also be $\Sigma_q(t) = \sigma^2_q(t)I$.

% The following equation is the KL divergence between two normal distributions $P$ and $Q$:
% $$D_{KL}(P\|Q) = \frac{1}{2}\left[\log\frac{|\Sigma_2|}{|\Sigma_1|} - n + \text{tr} \{ \Sigma_2^{-1}\Sigma_1 \} + (\mu_2 - \mu_1)^T \Sigma_2^{-1}(\mu_2 - \mu_1)\right]$$
% We can optimize the KL divergence between two normal distributions to minimize the difference between the two means: 
% \begin{align}
% 	\mathcal{L} &= \min_\theta D_{KL}(q(\rvx_{t-1}|\rvx_t, \rvx_0)\|p_\theta(\rvx_{t-1}|\rvx_{t}))\\
% 				& \vdots\\
% 				&= \min_\theta \frac{1}{2\sigma^2_q(t)}\big[\|\boldsymbol{\mu}_\theta- \boldsymbol{\mu}_q\|^2_2\big],
% 	\label{eq:diffusion_kl_mean}
% \end{align}
% where $\boldsymbol{\mu}_q$ is a shorthand for $\boldsymbol{\mu}_q(\rvx_t,\rvx_0)$.

We can use the \Cref{eq:diffusion_kl_true_denoising} to compute the optimal $\theta$ by solving $\min_\theta D_{KL}(q(\rvx_{t-1}|\rvx_t, \rvx_0)\|p_\theta(\rvx_{t-1}|\rvx_{t}))$, but we can simplify the optimization problem by using \Cref{eq:diffusion_forward_sampling}:
\begin{align*}
	\rvx_t &= \sqrt{\bar{\alpha}_t} \mathbf{x}_{0}+ \sqrt{1-\bar{\alpha}_t} \odot \epsilon\\
	\mathbf{x}_{0} &= \frac{\rvx_t - \sqrt{1-\bar{\alpha}_t}\odot \epsilon}{\sqrt{\bar{\alpha}_t} } 
\end{align*}
By plugging the above equation into $\boldsymbol{\mu}_q(\rvx_t,\rvx_0)$, we can exclude $\rvx_0$ term as follows:
\begin{align}
	\boldsymbol{\mu}_q(\rvx_t,\rvx_0) &= \frac{\sqrt{{\alpha}_{t}}(1-\bar{\alpha}_{t-1})\rvx_t+\sqrt{\bar{\alpha}_{t-1}}(1-{\alpha}_{t})\rvx_0}{1-\bar{\alpha}_{t}}\\
									  &= \frac{\sqrt{{\alpha}_{t}}(1-\bar{\alpha}_{t-1})\rvx_t+\sqrt{\bar{\alpha}_{t-1}}(1-{\alpha}_{t})\frac{\rvx_t - \sqrt{1-\bar{\alpha}_t}\odot \epsilon}{\sqrt{\bar{\alpha}_t} }}{1-\bar{\alpha}_{t}}\\
									  &\vdots\\
									  &= \frac{1}{\sqrt{\alpha_t}}\rvx_t - \frac{1-\alpha_t}{\sqrt{1-\bar{\alpha_t}}\sqrt{\alpha_t}}\boldsymbol{\epsilon}_0
	\label{eq:diffusion_mean_simple}
\end{align}
Thus, we can force $\boldsymbol{\mu}_\theta(\mathbf{x}_t, t)$, which has no dependency on $\rvx_0$ term, to match the $\boldsymbol{\mu}_q$. Since the $\rvx_t$ is given at training time, we just need to predict the $\boldsymbol{\epsilon}_t$. Then, we can express $\boldsymbol{\mu}_\theta(\mathbf{x}_t, t)$ as follows:
\begin{align*}
\boldsymbol{\mu}_\theta(\mathbf{x}_t, t) &= \frac{1}{\sqrt{\alpha_t}} \bigg( \mathbf{x}_t - \frac{1 - \alpha_t}{\sqrt{1 - \bar{\alpha}_t}} \boldsymbol{\epsilon}_\theta(\mathbf{x}_t, t) \bigg) \\
\end{align*}
Thus, $\rvx_{t-1}\sim p_\theta(\mathbf{x}_{t-1} \vert \mathbf{x}_t)$ can be expressed as follows:
\begin{align*}
	\mathcal{N}\Bigg(\mathbf{x}_{t-1}; \frac{1}{\sqrt{\alpha_t}} \bigg( \mathbf{x}_t - \frac{1 - \alpha_t}{\sqrt{1 - \bar{\alpha}_t}} \boldsymbol{\epsilon}_\theta(\mathbf{x}_t, t) \bigg), \boldsymbol{\Sigma}_\theta(\mathbf{x}_t, t)\Bigg)
\end{align*}
Note that we can use the variance derived in \Cref{eq:diffusion_kl_true_denoising} instead of estimating it from a network for simplicity. Finally, we can solve the KL-divergence term:

\begin{align}
	\mathcal{L} &= \min_\theta D_{KL}(q(\rvx_{t-1}|\rvx_t, \rvx_0)\|p_\theta(\rvx_{t-1}|\rvx_{t}))\\
				& \vdots\\
				&= \min_\theta \frac{1}{2\sigma^2_q(t)}\frac{(1-\alpha_t)^2}{(1-\bar{\alpha}_t)\alpha_t}\big[\|\boldsymbol{\epsilon}_0- \hat{\boldsymbol{\epsilon}}_\theta(\rvx_t,t)\|^2_2\big].
	\label{eq:diffusion_kl_mean_simple}
\end{align}
Here $\hat{\boldsymbol{\epsilon}}_\theta(\rvx_t,t)$ is a is a neural network that learns to predict the source noise $\boldsymbol{\epsilon}_0 \sim \mathcal{N}(\boldsymbol{\epsilon}}; \mathbf{0}, \mathbf{I})$ that determines $\rvx_t$ from $\rvx_0$. We have therefore shown that the overall learning objective is equivalent to learning to predict the noise.



\section{Summary}
\begin{itemize}
	\item The loss function can be decomposed.
		$$L_\text{VLB} = L_T + L_{T-1} + \dots + L_0 .$$
	\item $L_T = D_\text{KL}(q(\mathbf{x}_T \vert \mathbf{x}_0) \parallel p_\theta(\mathbf{x}_T))$
		\begin{itemize}
			\item Constant $\approx 0$ since $x_T$ is a Gaussian noise.
		\end{itemize}
	\item $L_t = D_\text{KL}(q(\mathbf{x}_{t-1} \vert \mathbf{x}_{t}, \mathbf{x}_0) \parallel p_\theta(\mathbf{x}_{t-1} \vert\mathbf{x}_{t})) \text{ for } t>1 $
	%\item $L_t = D_\text{KL}(q(\mathbf{x}_t \vert \mathbf{x}_{t+1}, \mathbf{x}_0) \parallel p_\theta(\mathbf{x}_t \vert\mathbf{x}_{t+1})) \text{ for }1 \leq t \leq T-1 $
		\begin{itemize}
			\item This is the main part.
		\end{itemize}
	\item $L_0 = - \log p_\theta(\mathbf{x}_0 \vert \mathbf{x}_1)$
		\begin{itemize}
			\item Can be modeled by a separate decoder.
		\end{itemize}
	\item $q(\rvx_t|\rvx_0) = \mathcal{N}(\sqrt{\bar{\alpha}_t}\rvx_0, (1-\bar{\alpha}_t)I )$
	\item $q(\rvx_t \vert \rvx_{t-1}) &= \mathcal{N}(\mathbf{x}_t; \sqrt{1 - \beta_t}\rvx_{t-1}, \beta_t\mathbf{I}),$
		\begin{itemize}
			\item[] We can sample by $\rvx_t=\sqrt{1 - \beta_t}\rvx_{t-1}+ \sqrt{\beta_t}\epsilon$
		\end{itemize}
	\item $p_\theta(\mathbf{x}_{t-1} \vert \mathbf{x}_t) = \mathcal{N}(\mathbf{x}_{t-1}; \boldsymbol{\mu}_\theta(\mathbf{x}_t, t), \boldsymbol{\Sigma}_\theta(\mathbf{x}_t, t))$.
		\begin{itemize}
			\item We need to learn mean and variance.
			\item DDPM kept the variance fixed and let the neural network only learn the mean $\mu_\theta$.
			\item $\boldsymbol{\Sigma}_\theta(\mathbf{x}_t, t)) = \sigma_t^2\mathbf{I}$ and set $\sigma_t^2 = \beta_t$.
		\item Improved DDPM model trains $\sigma$ also.
		\end{itemize}
	\item One can reparameterize the mean to make the nerual network learn the added noise via a network $\epsilon_\theta$.
		$$\mu_\theta(\rvx_t, t) = \frac{1}{\sqrt{\alpha_t}}\bigg(\rvx_t - \frac{\beta_t}{\sqrt{1-\bar{\alpha}_t}}\underbrace{\epsilon_\theta(\rvx_t,t)}_{\text{Network}}\bigg)$$
	\item Final objective function $L_t$ is
		$$||\epsilon -\epsilon_\theta(\rvx_t,t)||^2 = ||\epsilon -\epsilon_\theta(\sqrt{\bar{\alpha}_t}\rvx_0+\sqrt{(1-\bar{\alpha}_t)}\epsilon,t)||^2 $$
		\begin{itemize}
			\item $t\sim\text{Unif}[\{1,..,T\}]$ 
			\item $\rvx_t=\sqrt{\bar{\alpha}_t}\rvx_0+\sqrt{(1-\bar{\alpha}_t)}\epsilon \sim q(\rvx_t|\rvx_0)$
			\item $\epsilon\sim \mathcal{N}(0,I)$
		\end{itemize}
	\item $\rvx_t$ is perturbed by $\epsilon$ and the noise prediction network $\epsilon_\theta$ predicts $\epsilon$.
\end{itemize}

\newpage
\begin{algorithm}[t]
	\caption{Training}
	\label{alg:diffusion_training}
	\Repeat{\textrm{converged}}{
			$\rvx_0\sim q(\rvx_0)$\\
			$t\sim \text{Unif}[\{1,\cdots,T\}]$\\
			$\epsilon\sim \mathcal{N}(0,I)$\\
			Take gradient descent step on
			$\nabla_\theta||\epsilon -\epsilon_\theta(\sqrt{\bar{\alpha}_t}\rvx_0+\sqrt{(1-\bar{\alpha}_t)}\epsilon,t)||^2$
		}
\end{algorithm}

The training process is given by
\begin{enumerate}
	\item $\rvx_0\sim q(\rvx_0)$ 
	\item Sample a noise level $t$ between $1$ and $T$ (\ie random time step).
	\item Sample a noise from a Gaussian distribution and perturb the input by the sampling equation.
	\item NN is trained to predict this noise $\epsilon$ used for generating $\rvx_t$.
	\item $\beta$ is often scheduled linearly.
	\item $\Sigma$ is set equal to $\beta$.
\end{enumerate}

The sampling process is given by
\begin{algorithm}[h]
	\caption{Sampling}
	\label{alg:diffusion_sampling}
		$\rvx_T\sim \mathcal{N}(0,I)$\\
		\For{$t=T,\cdots,1$}{
			$\rvz\sim \mathcal{N}(0,I)$\\
			\State $\rvx_{t-1}= \frac{1}{\sqrt{\alpha_t}}\bigg(\rvx_t-\frac{1-\alpha_t}{\sqrt{1-\bar{\alpha}_t}}\boldsymbol{\epsilon}_\theta(\rvx_t,t)\bigg)+\Sigma_t\rvz$
		}
		\textbf{return} $\rvx_0$
\end{algorithm*}
\begin{itemize}
	\item Ancestral sampling.
	\item $T$ is typically around 1,000
\end{itemize}
	

\section{Score Matching}
\begin{itemize}
	\item Suppose $\{\rvx_0,\cdots,\rvx_N\}$, where each data point (\eg image, video, or text)) is sampled independently from a data distribution $p(\rvx)$. 
	\item Given the dataset, the goal of generative modeling is to fit a model to the data distribution such that we can synthesize new data points at will by sampling from the model. 
	\item One way is to directly model the distribution function as in likelihood-based models. Let $f_\theta(\rvx)\in \mathbb{R}^d$, then we can define a density function:
	$$p_\theta(\rvx) = \frac{\exp^{-f_{\theta}(\rvx)}}{Z_\theta}$$.
	\item $f_\theta(\rvx)\in \mathbb{R}^d$ is often called unnormalized probabilistic model or energy-based model.
	\item Energy-based model originates from the Gibbs distribution in statistical physics.
	\item $p_\theta(\rvx)$ can be trained by maximizing the log-likelihood of the data.
		\begin{align*}
			\max_\theta \sum_i^N \log p_\theta(\rvx_i).
		\end{align*}
	\item The gradient of the loglikelihood is given by
		\begin{align*}
			\nabla_\theta \log p_\theta(\rvx) &= \nabla_\theta f_\theta(\rvx)- \nabla_\theta Z_\theta\\
			\nabla_\theta Z_\theta	&= \frac{\nabla_\theta Z_\theta}{ Z_\theta}\\
									&= \frac{1}{ Z_\theta}\nabla_\theta \int\exp(f_\theta(\rvx))d\rvx \\
									&= \frac{1}{ Z_\theta} \int\exp(f_\theta(\rvx))\nabla_\theta f_\theta(\rvx)d\rvx \\
									&=  \int \frac{1}{ Z_\theta}\exp(f_\theta(\rvx))\nabla_\theta f_\theta(\rvx)d\rvx \\
									&=  \int p_\theta(\rvx)\nabla_\theta f_\theta(\rvx)d\rvx \\
									&=  \mathbb{E}_{p_\theta(\rvx)}[\nabla_\theta f_\theta(\rvx)] \\
			\nabla_\theta \log p_\theta(\rvx) &= \nabla_\theta f_\theta(\rvx) - \mathbb{E}_{p_\theta(\rvx)}[\nabla_\theta f_\theta(\rvx)]
		\end{align*}
	\item However, it is undesirable, since $Z_\theta$ is intractable.
		\begin{itemize}
			\item For instance, a gray scale image of $100\times 100$ has $256^{10,000}$ space. 
		\end{itemize}
	\item Thus, we have to sidestep the issue by using some solutions, for instance:
		\begin{itemize}
			% \item Architecture: NMF
			\item Approximate by using VAE or MCMC
		\end{itemize}
\end{itemize}

Instead, we can leverage \textbf{\textit{Stein Score}}: 
\begin{itemize}
	\item By modeling a score function, instead of the density function, we can sidestep the difficulty of computing the intractable normalizaing constants.
	\item \textit{Stein Score} function: $\nabla_\rvx\log p(\rvx)$.
		\begin{itemize}
			\item \textbf{Not a gradient w.r.t. model parameters.}
			\item Gradient of the log probability density function.
			\item Not same as the score in stat.
		\end{itemize}
	\item It is a direction that maximizes a log data density.
	\item A model for approximating the score function is called a \textit{score-based model} $s_\theta(\rva)$.
	\item Score-based models does not have to compute the intractable normalizing constant, $Z_\theta$.
		$$s_\theta(\rvx) = \nabla_\rvx\log p_\theta(\rvx) = -\nabla_\rvx f_{\theta}(\rvx)- \underbrace{\nabla_\rvx \log Z_\theta}_{\text{Constant}}.$$
	\end{itemize}

\subsection{Fisher Divergence}
We need to know about \textit{Fisher Divergence}:
\begin{itemize}
	\item Given \textit{i.i.d.} samples $\{\rvx_1, \cdots, \rvx_N\}\sim p_{data}(\rvx) = p(\rvx).$
	\item Estimating the score function $\nabla_\rvx \log p(\rvx)$.
	\item Score model $s_\theta(\rvx):\mathbb{R}^D\to \mathbb{R}^D$.
	\item Use score estimator $s_\theta(x)$:
		$$\mathcal{L}_{\theta} = \frac{1}{2}\mathbb{E}_{p(\rvx)}\big[||\nabla_\rvx \log p(\rvx)-s_\theta(\rvx)||^2_2\big]\,.$$
	\item It is called \textit{Fisher divergence}.
	\item Intuitively, the Fisher divergence compares the squared distance between the ground-truth data score and the score-based model. 
	\begin{itemize}
		\item It changes the problem into a \textit{regression} problem.
	\end{itemize}
	\item Direct computation of the divergence is \textbf{infeasible} due to the unknown data score $\nabla_\rvx \log p(\rvx)$.
	\begin{itemize}
		\item Since we have no access to the true data distribution $p(\rvx)$.
	\end{itemize}
\end{itemize}

Fortunately, there exists a family of methods called \textbf{\textit{score matching}} that minimize the Fisher divergence without knowledge of the ground-truth data score.
\begin{itemize}
	\item Score matching objectives can directly be estimated on a dataset and optimized with stochastic gradient descent, analogous to the log-likelihood objective for training likelihood-based models (with known normalizing constants).
	\item We can train the score-based model by minimizing a score matching objective, without requiring adversarial optimization.
\end{itemize}
\begin{align*}
	\mathcal{L}_{\theta} &=  \mathbb{E}_{p(\rvx)}\bigg[\frac{1}{2}||s_\theta(x)||^2_2+tr(\nabla_{\rvx}s_\theta(x))\bigg]\\
	&\approx \frac{1}{N}\sum_{i=1}^N\bigg[\frac{1}{2}||s_\theta(x)||^2_2+tr(\nabla_{\rvx}s_\theta(x))\bigg]
	\label{eq:score_integration}
\end{align*}
\begin{itemize}
	\item $\{\rvx_1, \cdots, \rvx_N\}\sim  p(\rvx)$
	\item $\nabla_{\rvx}s_\theta(x)$: Jacobian
	\item Remove the dependency of $p(\rvx)$
\end{itemize}
\begin{align}
	\mathcal{L}_{\theta} &= \frac{1}{2}\mathbb{E}_{p(x)}\big[||\nabla_x \log p(x)-s_\theta(x)||^2_2\big]\\
&= \frac{1}{2}\mathbb{E}_{p(x)}\big[\big(\nabla_x \log p(x)-s_\theta(x)\big)^2\big]\\
&= \frac{1}{2}\int p(x)(\nabla_x \log p(x)-s_\theta(x))^2dx\\
&= \underbrace{\frac{1}{2}\int p(x)(\nabla_x \log p(x))^2dx}_{\textrm{independent from theta}}+\frac{1}{2}\int p(x)s_\theta(x)^2dx-\int p(x)s_\theta(x)\nabla_x \log p(x)dx\\
	&= \dots-\int p(x)s_\theta(x)\nabla_x \log p(x)dx\\
	&= \dots-\int p(x)s_\theta(x)\frac{\nabla_x p(\rvx)}{p(x)}dx\\
	&= \dots-\int \nabla_\rvx p(x)s_\theta(x)dx\\
	&= \dots-\bigg[p(x)s_\theta(x)\bigg]^{\infty}_{x=-\infty}+\int  p(x) \nabla_x s_\theta(x)dx\\
	&= \frac{1}{2}\int p(x)s_\theta(x)^2dx+\int  p(x) \nabla_x s_\theta(x)dx+\textrm{const}\\
	&= \frac{1}{2}\mathbb{E}_{p(x)}[s_\theta(x)^2]+\mathbb{E}_{p(x)}[\nabla_xs_\theta(x)]+ \textrm{const}.
\end{align}
\begin{itemize}
	\item The second last term used the integration by parts.
	\item The last step is done by a boundary condition assumption which makes score function to be zero (\cf Sliced score matching paper).
		\begin{itemize}
			\item $p_{data}(x)\to 0$ as $|x|\to \infty$.
			\item In other words, gradient vanishes on the boundary. 
		\end{itemize}
\end{itemize}
% \begin{align*}
% 	&\quad \quad \frac{1}{2}\mathbb{E}_{p(x)}\big[(s_\theta(x)-\nabla_x \log p(x))^2\big]\\
% 	&= \frac{1}{2}\int p(x)(\nabla_x \log p(x))^2dx+\frac{1}{2}\int p(x)s_\theta(x)^2dx\\
% 	&\quad+\int  p(x) \nabla_x s_\theta(x)dx\\
% 	&= 
% \end{align*}

\subsection{Langevin Dynamics}
\begin{itemize}
	\item Once we have trained a score-based model $s_\theta(\rvx)\approx \nabla_\rvx\log p(\mathbf{x})$, we can use an iterative procedure called \textit{Langevin Dynamics} (LD) \cite{sgld2011} to draw samples from it.
	\item LD provides an MCMC procedure to sample from a distribution $p(\rvx)$ using only its score function.
\end{itemize}

	$$ \mathbf{x}_t \leftarrow \mathbf{x}_{t-1} + \frac{\epsilon}{2} \nabla_\rvx\log p(\mathbf{x}_{t-1}) + \sqrt{\epsilon} \mathbf{z}_t$$
\begin{itemize}
	\item Specifically, it initializes the chain from an arbitrary prior distribution $\rvx_0\sim \pi(\rvx)$, and then iterates the following
	\item Sample from $p(x)$ using only the score $\nabla_x\log p(x)$.
	\item $\mathbf{z}_t \sim \mathcal{N}(\mathbf{0},\mathbf{I})$.
	\item $\epsilon$ is the step size.
\item As $T\to \infty$ and $\epsilon\to 0$, $\rvx_T$ will become the true probability density $p(\rvx)$.
\end{itemize}


\part{Natural Language Processing}
\chapter{Introduction}
\section{Evaluation Metrics}
\label{sec:nlp_eval_metrics}

In the context of a bigram model, likelihood refers to the probability of observing a sequence of words in a corpus based on the bigram model's parameters. The bigram model assumes that the probability of a word in a sequence depends only on the immediately preceding word, making it a **Markov model of order 1**.

A bigram model represents the probability of a word sequence \( W = (w_1, w_2, \ldots, w_n) \) as a product of conditional probabilities:
\[
P(W) = P(w_1) \prod_{i=2}^n P(w_i | w_{i-1})
\]
Here, \( P(w_i | w_{i-1}) \) is the conditional probability of word \( w_i \) given the preceding word \( w_{i-1} \).

The **likelihood** of a sequence of words (data) \( W = (w_1, w_2, \ldots, w_n) \) under the bigram model is:
\[
L(\theta | W) = P(W | \theta) = P(w_1 | \theta) \prod_{i=2}^n P(w_i | w_{i-1}, \theta)
\]
Here, \( \theta \) represents the model parameters, specifically the probabilities \( P(w_i | w_{i-1}) \).

\paragraph{Maximum Likelihood Estimation (MLE) in the Bigram Model}

To find the parameters \( P(w_i | w_{i-1}) \) that maximize the likelihood of the observed data, the **maximum likelihood estimation (MLE)** approach is commonly used.

The bigram probabilities \( P(w_i | w_{i-1}) \) are estimated from the frequency counts in the training corpus:
\[
P(w_i | w_{i-1}) = \frac{\text{Count}(w_{i-1}, w_i)}{\text{Count}(w_{i-1})}
\]
where:
\begin{itemize}
	\item \( \text{Count}(w_{i-1}, w_i) \) is the number of times the bigram \( (w_{i-1}, w_i) \) appears in the corpus.
	\item \( \text{Count}(w_{i-1}) \) is the number of times the word \( w_{i-1} \) appears.
\end{itemize}

To simplify computations, the log-likelihood of the sequence is used:
\[
\log L(\theta | W) = \log P(w_1 | \theta) + \sum_{i=2}^n \log P(w_i | w_{i-1}, \theta)
\]

\begin{itemize}
	\item Recall: TP/(TP+FN). Find all relevant cases whithin a dataset.
	\item Precision: TP/(TP+FP): While recall expresses the ability to find all relevant instances in a dataset, precision expresses the proportion of the data points our model says was relevant actually were relevant.
	\item The F1 score is the harmonic mean of precision and recall taking both metrics into account in the following equation:
\end{itemize}

\subsection{Perplexity}

Intuitively, perplexity can be understood as a \textit{measure of uncertainty}. The perplexity of a language model can be seen as the level of perplexity. Consider a language model with an entropy of three bits, in which each bit encodes two possible outcomes of equal probability. This means that when predicting a symbol, that language model has to choose among $2^3=8$ possible options. Thus, we can argue that this language model has a perplexity of 8.

It can be modeled as $2^H(P,Q)$:
\begin{align*}
	PPL(W) &= P(w_1,\cdots, w_N)^{-\frac{1}{N}}\\
	&\approx \Bigg(\prod_{i=1}^N P(w_i|w_{<i})\Bigg)^{-\frac{1}{N}}\\
	&= \sqrt[n]{\frac{1}{\prod_{i=1}^{N} P(w_{i}|w_{<i})}}
\end{align*}

Let's derive it from a cross-entropy. We want to optimize $P_\theta$ instead of the true distribution $P$:
\begin{align}
	% H & \approx -\sum_{i=1}^{N} \log P(w_{i}|w_{<i})\\
	\mathcal{L}_{CE} &= -\mathbb{E}_{w\sim P} [P_\theta(w_{i}|w_{<i})]\\
	&\approx -\frac{1}{N} \sum_{i=1}^{N} \log P_\theta(w_{i}|w_{<i})\\
	&= -\frac{1}{N} \log \prod_{i=1}^{N} P_\theta(w_{i}|w_{<i})\\
	&=  \log \Bigg(\prod_{i=1}^{N} P_\theta(w_{i}|w_{<i}) \Bigg)^{-\frac{1}{N}}\\
	&=  \log \sqrt[N]{\frac{1}{\prod_{i=1}^{N} P_\theta(w_{i}|w_{<i})}}\\
	\label{eq:ppl_entropy}
\end{align}
Thus, $PPL(W) = \exp\Big(\mathcal{L}_{CE}\Big).$

\subsection{Cross-Entropy and Perplexity}
\begin{align*}
	H(P,Q) &= -\sum_{x}P(x)\log Q(x) \\
	&= -\sum_{x}P(x) [\log P(x) + \log Q(x) - \log P(x)] \\
	&= -\sum_{x}P(x)\Bigg[\log P(x) + \log\frac{Q(x)}{P(x)}\Bigg] \\
	&= H(P) + D_{KL}(P||Q)
\end{align*}
It should be noted that since the empirical entropy $H(P)$ is unoptimizable, when we train a language model with the objective of minimizing the cross entropy loss, the true objective is to minimize the $KL$-divergence of the distribution, which was learned by our language model from the empirical distribution of the language.

\chapter{Transformer}
\section{Attention Mechanism}
\label{sec:nlp_attention}
The attention mechanism mimics the retrieval of a value $v_i$ for a query $q$ based on a key $k_i$ in database.
$$attn(q, k, v) = \sum_i sim(q,k_i)\times v_i$$
\begin{figure}[h]
	\centering
	\includegraphics[scale=2.0]{./images/transformer/attention_database.pdf}
	\caption{The most similar key will be selected by measuring a similarity between a query and a key.}
\end{figure}

\begin{figure}[h]
	\centering
	\includegraphics[scale=0.8]{./images/transformer/attention.pdf}
	\caption{The similarity $s_t$ is computed by a query and keys}
\end{figure}
There are several choices for a similarity function.
\begin{itemize}
	\item $q^Tk_i$: dot product.
	\item $\frac{q^Tk_i}{\sqrt{d}}$: scaled dot product.
	\item $q^TWk_i$: general dot product.
	\item $w_q^Tq+ w_k^Tk_i$: additive similarity.
\end{itemize}
Finally, the attention score can be computed by using a softmax:
$$a_i = \frac{\exp(s_i)}{\sum_j \exp(s_j)}$$

\section{Transformer}
\label{sec:nlp_transformer}

Attention:
$$attn(Q,K,V) = softmax(\frac{Q^TK}{\sqrt{d_k}})V$$
Masked attention:
$$\textrm{MA}(Q,K,V) = softmax\bigg(\frac{Q^TK+M}{\sqrt{d_k}}\bigg)V,$$
where $M$ is a matrix of 0 and $-\infty$. Note that $-\infty$ will make $exp$ term to be zero.




\part{Advanced Topics}
\chapter{Neural Ordinary Differential Equations}
\section{Preliminary}
\label{sec:node_preliminary}

\subsection{Euler Method}
The Euler method is a simple numerical technique used to solve ordinary differential equations (ODEs) of the form \( \frac{dy}{dt} = f(t, y) \). It is an initial value problem where we seek to find the function \( y(t) \) given an initial condition \( y(t_0) = y_0 \). Here's a step-by-step explanation of the Euler method:


\paragraph{Problem Setup:} Given,
\begin{itemize}
	\item A differential equation \( \frac{dy}{dt} = f(t, y) \)
	\item An initial condition \( y(t_0) = y_0 \)
\end{itemize}

\paragraph{Discretization:} The idea is to approximate the solution at discrete points. Let's denote:
\begin{itemize}
	\item \( t_n \) as the \( n \)-th time step
	\item \( y_n \) as the approximation of \( y(t_n) \)
\end{itemize}
We define a step size \( h \) such that \( t_{n+1} = t_n + h \).

\paragraph{Euler's Approximation:} Using the first-order Taylor series expansion, we can approximate \( y(t) \) at \( t_{n+1} \) as:
\[ y_{n+1} \approx y_n + h \cdot f(t_n, y_n) \]

\paragraph{Iterative Process:} Starting from the initial condition \( (t_0, y_0) \):
\begin{enumerate}
	\item Calculate the next value using the formula:
	\[ y_{n+1} = y_n + h \cdot f(t_n, y_n) \]
	\item Repeat the process for \( n = 0, 1, 2, \ldots \) until the desired value of \( t \) is reached.
\end{enumerate}

\paragraph{Example:} Let's solve the differential equation \( \frac{dy}{dt} = y \) with the initial condition \( y(0) = 1 \) using the Euler method.

\begin{itemize}
	\item Set the step size \( h \) (e.g., \( h = 0.1 \)).
	\item Start with \( t_0 = 0 \) and \( y_0 = 1 \).
\end{itemize}

Using the Euler formula:
\[ y_{1} = y_0 + h \cdot f(t_0, y_0) = 1 + 0.1 \cdot 1 = 1.1 \]
\[ y_{2} = y_1 + h \cdot f(t_1, y_1) = 1.1 + 0.1 \cdot 1.1 = 1.21 \]
\[ y_{3} = y_2 + h \cdot f(t_2, y_2) = 1.21 + 0.1 \cdot 1.21 = 1.331 \]

Euler's method can be visualized as taking small steps along the curve defined by the differential equation, using the slope at the current point to determine the direction of the next step.

\begin{figure}[h]
	\centering
	\includegraphics[scale=0.3]{./images/node/euler_method.png}
\end{figure}

\paragraph{Advantages:}
\begin{itemize}
	\item Simple to understand and implement.
	\item Requires only basic arithmetic operations.
\end{itemize}
\paragraph{Disadvantages:}
\begin{itemize}
	\item Low accuracy for large step sizes.
	\item Can become unstable if the step size is not chosen appropriately.
	\item Errors accumulate over time, leading to less accurate solutions.
\end{itemize}

Euler's method is often used as a basic introduction to numerical methods for solving ODEs, and more sophisticated methods like the \textit{Runge-Kutta} methods are used for more accurate solutions.



\section{Neural ODE}


Models such as residual networks, recurrent neural network decoders, and normalizing flows build complicated transformations by composing a sequence of transformations to a hidden state:
$$\rvh_{t+1} = \rvh_{t}+f(\rvh_{t}, \theta_t).$$

The $\rvh$ is iteratively updated as follows:
\begin{align*}
	\rvh_{2} &= \rvh_{1}+f(\rvh_{1}, \theta)\\
	\rvh_{3} &= \rvh_{2}+f(\rvh_{2}, \theta) = \rvh_{1}+f(\rvh_{1}, \theta)+f(\rvh_{2}, \theta)\\
	\vdots
\end{align*}
These iterative update can be seen as \textit{Euler discretization} of the \textit{continuous} transformation. Think of traditional neural networks as a sequence of steps. You give it some input, it goes through several steps (layers), and you get an output. Instead of thinking in steps, Neural ODEs think in \textbf{continuous change over time}. Note that this is the key contribution of this approach.   

Euler method can be expressed as follows: 
$$y_n = y_{n-1}+h\frac{\partial y_{n-1}}{\partial x_{n-1}},$$
where $h$ is the step size. In NODE, they view the $f$ as an ordinary differential equation, which depends on the state at time $t$ and parameter $\theta$. The following equation is the shape of Euler method:
$$y_n = y_{1}+h\frac{\partial y_{1}}{\partial x_{1}}+h\frac{\partial y_{2}}{\partial x_{2}}+\cdots+h\frac{\partial y_{n-1}}{\partial x_{n-1}}.$$
In NODE, 




\chapter{State Space Model}
\chapter{Introduction to Regression Methods}
\label{chapter:regression_intro}
\input{./sections/regression/regression}
\chapter{Recursive Least Squares}
\input{./sections/regression/recursive}
\chapter{Logistic Regression}
\input{./sections/regression/logistic_reg}
\chapter{Bayesian Regression}
\input{./sections/regression/bayes}



\section{Kalman Filter}
\label{sec:advanced_kalman}

\href{https://www.kalmanfilter.net/default.aspx}{Ref: Kalman Filter Tutorial}


% \section{Legendre Memory Units}
% \label{sec:nlp_lmu}

\section{Efficiently Modeling Long Sequences with Structured State-Spaces}
\label{sec:nlp_ssm}
The Linear State-Space Layer (LSSL) is a simple sequence model that maps a one-dimensional function  or sequence $u(t)\to y(t)$ through an implicit state $x(t)$ by simulating a linear continuous-time state-space representation in discrete-time
\begin{align*}
	x'(t) &= \mathbf{A}x(t)+\mathbf{B}u(t),\\
	y(t) &= \mathbf{C}x(t)+\mathbf{D}u(t).
\end{align*}

The first equation maps a single dimensional input signal $u(t)$ (or sequence) to an $N$-dim latent state (or hidden state) $x'(t)$ with the current state $x(t)$. The $\mathbf{A}$ and $\mathbf{B}$ can be considered as an non-linear mapping or transition matrices (\ie learnable parameters) to reflect the impact of the current state and the input, respectively. Finally, we project the input and the updated state to a one-dim output signal $y(t)$ (\ie sequence).  

Our goal is to simply use the SSM as a black-box representation in a deep sequence model, where $\rmA, \rmB, \rmC$, and $\rmD$ are parameters learned by gradient descent. We will omit the parameter $\rmD$ for exposition (or equivalently, assume $\rmD=0$, because the term $\rmD u$ can be viewed as a \textit{skip connection} that doesn't depend on the hidden state ($x$)).

An SSM maps a input $u(t)$ to a state representation vector $x(t)$ and an output $y(t)$. For simplicity, we assume the input and output are one-dimensional, and the state representation is $N$-dimensional. The first equation defines the change in $x(t)$ over time.

\begin{figure}[t]
	\centering
	\includegraphics[scale=0.6]{./images/nlp/ssm.pdf}
	\caption{S4 Model}
	\label{fig:nlp_s4_model}
\end{figure}

\paragraph{Discretization} We want to find a discrete-time state-space model. We can represent it by approximating a continuous model (\ie $h(t_k) \approx h(k\Delta)$) as follows:
\begin{align*}
	x'(t) &= \overline{\mathbf{A}}x(t)+\overline{\mathbf{B}}u(t),\\
	y(t) &= \mathbf{C}x(t)+\underbrace{\mathbf{D}u(t)}_{=0},
\end{align*}
where $\overline{\mathbf{A}} = \mathbf{I}+\Delta \mathbf{A}$ and $\overline{\mathbf{B}} = \Delta \mathbf{B}$, respectively. They are derived by
\begin{align*}
	x'(t) &= \overline{\mathbf{A}}x(t)+\overline{\mathbf{B}}u(t)\\
		  &= \lim_{\Delta\to 0} \frac{x(t+\Delta)-x(t)}{\Delta},
\end{align*}
where $\Delta$ is a step size. Note that $\Delta$ is a learnable parameter to be determined during a training phase. Subsequently, we get 
\begin{align*}
	x(t+\Delta)-x(t) = \Delta x'(t).
\end{align*}
Equivalently, 
\begin{align*}
	x(t+\Delta) = \Delta x'(t)+x(t). 
\end{align*}
Plugging $x'(t)$ into the above equation, we get:
\begin{align*}
	x(t+\Delta) &= \Delta (\mathbf{A}x(t)+\mathbf{B}u(t)) +x(t) \\
				&= \underbrace{(\mathbf{I}+\Delta \mathbf{A})}_{\overline{\mathbf{A}}}x(t)+\underbrace{\Delta \mathbf{B}}_{\overline{\mathbf{B}}}u(t).
\end{align*}
We can say $x(t+\Delta)=x(t+1)$, since it is the next state we care. Thus, 
\begin{align*}
	x(t+1)= (\mathbf{I}+\Delta \mathbf{A})x(t)+\Delta \mathbf{B}u(t).
\end{align*}
Note that the original paper utilizes a special rule called \textit{Zero-Order Hold} to approximate the $\overline{A}$ and $\overline{B}$. 

Alternatively, we can use the trapezoidal method. The trapezoidal rule works by approximating the region under the graph of the function $f(x)$ as a trapezoid and calculating its area. It follows that
$$\int _{a}^{b}f(x)\,dx\approx (b-a)\cdot {\tfrac {1}{2}}(f(a)+f(b)).$$
Thus, the integral of $x'(t)$ from $t_n$ to $t_{n+1}$ can be approximated using the trapezoidal rule. The exact integral is:
\begin{align*}
	x(t_{n+1})-x(t_n) &= \int_{t_n}^{t_{n+1}}x'(t)dt\\
					  &\approx \frac{1}{2}\Delta (\mathbf{A}x(t_{n+1})+\mathbf{B}u(t_{n+1})+\mathbf{A}x(t_n) + \mathbf{B}u(t_n)),
\end{align*}
where $\Delta = t_{n+1}-t_n$. Then, we have 
\[ x(t_{n+1}) - \frac{\Delta}{2} \mathbf{A}x(t_{n+1}) = x(t_n) + \frac{\Delta}{2} \mathbf{A}x(t_n) + \frac{\Delta}{2} \mathbf{B}u(t_n) + \frac{\Delta}{2} \mathbf{B}u(t_{n+1}) \]

\[ \bigg(\mathbf{I} - \frac{\Delta}{2} \mathbf{A}\bigg) x(t_{n+1}) = \bigg(\mathbf{I} - \frac{\Delta}{2} \mathbf{A}\bigg) x(t_n) + \Delta \mathbf{B} \frac{(u(t_n) + u(t_{n+1})}{2} \]

\[ x(t_{n+1}) = \left( \mathbf{I} - \frac{\Delta}{2} \mathbf{A} \right)^{-1} \left( \mathbf{I} + \frac{\Delta}{2} \mathbf{A} \right) x(t_n) + \left( \mathbf{I} - \frac{\Delta}{2} \mathbf{A} \right)^{-1} \frac{\Delta}{2} \mathbf{B} (u(t_{n+1})) \]
Finally, we get
\begin{align*}
	\overline{\mathbf{A}} &= \left( \mathbf{I} - \frac{\Delta}{2} \mathbf{A} \right)^{-1} \left( \mathbf{I} + \frac{\Delta}{2} \mathbf{A} \right)\\
	\overline{\mathbf{B}} &= \left( \mathbf{I} - \frac{\Delta}{2} \mathbf{A} \right)^{-1} \Delta \mathbf{B}
\end{align*}
Note that we assume that $u(t_{n+1})\approx u(t_{n}).$ We can represent the update process as follows:
At $t=0$
\begin{align*}
	x(0) &= \overline{\mathbf{B}}u(0),\\
	y(0) &= \mathbf{C}x(0)
\end{align*}

At $t=1$
\begin{align*}
	x(1) &= \overline{\mathbf{A}}x(0)+\overline{\mathbf{B}}u(1),\\
	y(1) &= \mathbf{C}x(1)
\end{align*}

At $t=2$
\begin{align*}
	x(2) &= \overline{\mathbf{A}}x(1)+\overline{\mathbf{B}}u(2),\\
	y(2) &= \mathbf{C}x(2).
\end{align*}
Note that this update process is equivalent to the RNN's update process. 

\paragraph{As a Convolution} 
The above process can be viewed as an one-dimensional convolution. 

\begin{align*}
	x(0) &= \overline{\mathbf{B}}u(0),\\
	y(0) &= \mathbf{C}x(0) = \mathbf{C}\overline{\mathbf{B}}u(0)\\
		 &\\
	x(1) &= \overline{\mathbf{A}}x(0)+\overline{\mathbf{B}}u(1)=\overline{\mathbf{A}}\overline{\mathbf{B}}u(0)+\overline{\mathbf{B}}u(1)\\
	y(1) &= \mathbf{C}x(1) = \mathbf{C}(\overline{\mathbf{A}}\overline{\mathbf{B}}u(0)+\overline{\mathbf{B}}u(1)) = \mathbf{C}\overline{\mathbf{A}}\overline{\mathbf{B}}u(0)+\mathbf{C}\overline{\mathbf{B}}u(1)\\
		 &\\
	x(2) &= \overline{\mathbf{A}}x(1)+\overline{\mathbf{B}}u(2)=\overline{\mathbf{A}}(\overline{\mathbf{A}}\overline{\mathbf{B}}u(0)+\overline{\mathbf{B}}u(1))+\overline{\mathbf{B}}u(2)=\overline{\mathbf{A}}^2\overline{\mathbf{B}}u(0)+\overline{\mathbf{A}}\overline{\mathbf{B}}u(1)+\overline{\mathbf{B}}u(2)\\
	y(2) &= \mathbf{C}x(2)=\mathbf{C}(\overline{\mathbf{A}}^2\overline{\mathbf{B}}u(0)+\overline{\mathbf{A}}\overline{\mathbf{B}}u(1)+\overline{\mathbf{B}}u(2))=\mathbf{C}\overline{\mathbf{A}}^2\overline{\mathbf{B}}u(0)+\mathbf{C}\overline{\mathbf{A}}\overline{\mathbf{B}}u(1)+\mathbf{C}\overline{\mathbf{B}}u(2)\\
\end{align*}
We get a general formula:
\begin{align*}
	y(t) &= \mathbf{C}\overline{\mathbf{A}}^t\overline{\mathbf{B}}u(0)+\mathbf{C}\overline{\mathbf{A}}^{t-1}\overline{\mathbf{B}}u(1)+\dots+\mathbf{C}\overline{\mathbf{B}}u(t)\\
		 &= \sum_{t=0}^T \mathbf{C}\overline{\mathbf{A}}^{T-t}\overline{\mathbf{B}}u(t)
\end{align*}
It turns out that the above equation is a one-dimensional convolution by a kernel $\overline{\mathbf{K}}$:
\begin{align*}
	y = x*\overline{\mathbf{K}}.
\end{align*}
It is generally referred to as the SSM convolution kernel in the literature, and its size is equivalent to the entire input sequence. This convolution kernel is calculated by Fast Fourier Transform (FFT)

\begin{figure}[t]
	\centering
	\includegraphics[scale=0.95]{./images/state_space/mamba_conv.pdf}
	\caption{}
	\label{fig:mamba_conv}
\end{figure}

Let's say the kernel size is 4 with zero-padding,  (See \Cref{fig:mamba_conv}). During training, we can train the model as a convolutional neural network so that we can leverage the parallel training. During inference (\ie decoding stage), we can switch to the recurrent mode for near-constant time inference. Please note here, that if you look at the kernels you can see that they are fixed. 


In the convolution kernel developed above, $\mathbf{\bar{C}}$ and $\mathbf{\bar{B}}$, are learnable scalars.
Concerning $\mathbf{\bar{A}}$, we've seen that in our convolution kernel, it's expressed as a power of $k$ at time $k$. This can be very time-consuming to calculate, so we're looking for a fixed $\mathbf{\bar{A}}$. For this, the best option is to have it diagonal:


Note that we can scale the single-dimensional SSM to multi-dimensional vector by putting a SSM for each dimension. For instance, there is going to be 256 SSMs if we want to learn a 256 embeddings.  


\section{Mamba: Linear-Time Sequence Modeling with Selective State Spaces}
\label{sec:ssm_mamba}
https://blog.premai.io/s4-and-mamba/

We can immediately notice the importance of the matrix $\mathbf{A}$. The Mamba leverages a special matrix called HiPPO. The HiPPO is a $N x N$ matrix specifically designed to approximate all the input signals by using Legendre polynomials (like Fourier series). 
\begin{align*}
	\mathbf{A}_{nk} = 
\begin{cases}
	(2n+1)^{1/2}(2k+1)^{1/2}\, &\textrm{ if } n>k\\
	n+1\, &\textrm{ if } n=k\\
	0\, &\textrm{ else }
\end{cases}
\end{align*}
This matrix helps to compress the history of the information by masking the next tokens, which is similar to the masked self-attention. Note that the matrix just needs to be computed once. 





\part{Appendix}
\renewcommand{\thesection}{\Alph{section}.\arabic{section}}
\setcounter{section}{0}

\begin{appendices}
\chapter{Appendix}

\section{Vector Notations}

\begin{itemize}
	\item $\rvx = (x_1, x_2,..., x_n)$ or
	\item 
		\begin{align*}
			\rvx = \begin{bmatrix}x_1\\ \vdots\\ x_n\end{bmatrix}
		\end{align*}
\end{itemize}

\subsection{Differentiate}

Differentiate a vector $y$ by a scalar $\rvx$:
\begin{align*}
	\frac{\partial \rvy}{\partial x} = \begin{bmatrix}\frac{\partial y_1}{\partial x}\\ \vdots\\ \frac{\partial y_n}{\partial x}\end{bmatrix}
\end{align*}

Differentiate a scalar $y$ by a vector $\rvx$:
\begin{align*}
	\frac{\partial y}{\partial \rvx} = \bigg[\frac{\partial y}{\partial x_1}, \cdots, \frac{\partial y}{\partial x_n}\bigg]
\end{align*}

Differentiate a vector $y$ by a vector $\rvx$:
\begin{align*}
	\frac{\partial \rvy}{\partial \rvx} = \begin{bmatrix}
		\frac{\partial y_1}{\partial x_1} & \cdots & \frac{\partial y_1}{\partial x_n}\\ 
		\vdots & \ddots & \vdots\\ 
		\frac{\partial y_n}{\partial x_1} & \cdots & \frac{\partial y_n}{\partial x_n}
\end{bmatrix}
\end{align*}

$\rva^T\rvx$ is a scalar value, so 

\begin{align*}
	\frac{\partial \rva^T\rvx}{\partial \rvx} &= \bigg[\frac{\partial (\rva^T\rvx)}{\partial x_1} \cdots \frac{\partial (\rva^T\rvx)}{\partial x_n}\bigg] = \bigg[\frac{\partial (a_1x_1+\dots+a_nx_n)}{\partial x_1}, \cdots, \frac{\partial (a_1x_1+\dots+a_nx_n)}{\partial x_n}\bigg]\\ 
	&= [a_1, \dots, a_n] = \rva^T
\end{align*}

\begin{align*}
	A\rvx = \begin{bmatrix}
		a_{11} & \dots & a_{1n}\\
		\vdots & \ddots & \vdots\\
		a_{m1} & \dots & a_{mn}\\
	\end{bmatrix}\times
	\begin{bmatrix}
		x_1\\
		\vdots\\
		x_n
		\end{bmatrix} = \begin{bmatrix}
		\sum_{i=1}^n a_{1i}x_i\\
		\vdots\\
		\sum_{i=1}^n a_{mi}x_i\\
	\end{bmatrix}
\end{align*}
Thus, 
\begin{align*}
	\frac{\partial A\rvx}{\partial \rvx} =\begin{bmatrix}
		\frac{\partial \sum_{i=1}^n a_{1i}x_i}{\partial x_1}& \dots & \frac{\partial \sum_{i=1}^n a_{1i}x_i}{\partial x_n}\\
		\vdots & \ddots & \vdots\\
		\frac{\partial \sum_{i=1}^n a_{mi}x_i}{\partial x_1} & \dots & \frac{\partial \sum_{i=1}^n a_{mi}x_i}{\partial x_n}
	\end{bmatrix} = A
\end{align*}

\section{KL Divergence between Two Normal Distribution}
\begin{align*}
D_\text{KL}(P||Q) & = \mathbb{E}_P\Big[\textrm{log}\frac{P}{Q}\Big]\\
\end{align*}
Consider two multivariate Gaussians in $\mathbb{R}^n$, $P_1$ and $P_2$

\begin{align*}
D_\text{KL}(P||Q) &= \int \left[ \frac{1}{2} \log\frac{|\Sigma_2|}{|\Sigma_1|} - \frac{1}{2} (x-\mu_1)^T\Sigma_1^{-1}(x-\mu_1) + \frac{1}{2} (x-\mu_2)^T\Sigma_2^{-1}(x-\mu_2) \right] \times p(x) dx \\
&= \frac{1}{2} \log\frac{|\Sigma_2|}{|\Sigma_1|} - \frac{1}{2} \text{tr} \left\{E[(x - \mu_1)(x - \mu_1)^T] \ \Sigma_1^{-1} \right\} + \frac{1}{2} E[(x - \mu_2)^T \Sigma_2^{-1} (x - \mu_2)] \\
&= \frac{1}{2} \log\frac{|\Sigma_2|}{|\Sigma_1|} - \frac{1}{2} \text{tr}\ \{I_n \} + \frac{1}{2} (\mu_1 - \mu_2)^T \Sigma_2^{-1} (\mu_1 - \mu_2) + \frac{1}{2} \text{tr} \{ \Sigma_2^{-1} \Sigma_1 \} \\
&= \frac{1}{2}\left[\log\frac{|\Sigma_2|}{|\Sigma_1|} - n + \text{tr} \{ \Sigma_2^{-1}\Sigma_1 \} + (\mu_2 - \mu_1)^T \Sigma_2^{-1}(\mu_2 - \mu_1)\right]
\end{align*}

Trace tricks:
$$x^TAx = tr[x^TAx] = tr[xx^TA]$$
$$tr[A+B] = tr[A]+tr[B]$$
$$E[(x-\mu)^T \Sigma^{-1} (x-\mu)]= tr(E[(x-\mu)(x-\mu)^T] \Sigma^{-1})$$
$$\Sigma = E[(X-\mu)(X-\mu)^T]=E[XX^T]-\mu\mu^T$$
$$E[XX^T] = \Sigma + \mu\mu^T$$

Note that the determinant of a diagonal matrix could be computed as product of its diagonal.



\section{Various Tricks}

\subsection{Spectral Normalization}

A persisting challenge in the training of GANs is the performance control of the discriminator. The derivative of discriminator could be unbounded and even incomputable, so they introduced a regularization on the derivative of discriminator called, Lipchitz continuity, which bound the gradient.

Neural network is actually a composite function. So if we make each function to satisfy the Lipchitz continuity, then we can make whole network satisfy it. Lipchitz continuity of a linear operator can be seen as 

\begin{align*}
	||f(x_1)-f(x_2)||_2 &\leq L||x_1-x_2||_2\\
	||Ax_1-Ax_2||_2 &\leq L||x_1-x_2||_2\\
	\frac{||Ax||_2}{||x||_2}&\leq L\\
	\sigma_{max}\underbrace{\sup_x \frac{||Ax||_2}{||x||_2}}_{Spectral Norm}&\leq L, \quad \textrm{Since inequaility holds for all }x
\end{align*}
, where $\sigma_{max}$ is the maximum singular value. Note that the spectral norm is from the linear algebra. We can make the matrix $A$ Lipchitz continuous by 
\begin{align*}
1 = \underbrace{\sup_x \frac{||\frac{A}{\sigma_{max}}x||_2}{||x||_2}}_{Spectral Norm}\leq L
\end{align*}


\subsection{Moving Averaging}

\subsection{Weight Averaging}

\subsection{Quality Measurements}

\section{f-Divergence}
In probability theory, an $f$-divergence is a function $D_{f}(p||q)$ that measures the difference between two probability distributions $p$ and $q$. It helps the intuition to think of the divergence as an average, weighted by the function $f$, of the odds ratio given by $p$ and $q$.

For distributions $p$ and $q$, f-divergence is defined as:
$$D_{f}(p||q) = \int_{\mathcal{X}}f\Bigg(\frac{p(x)}{q(x)}\Bigg)q(x)dx$$

\begin{itemize}
	\item KL-divergence: $f(t) = t\log t$
	\item Reversed KL-divergence: $f(t) = -\log t$
	\item Total variation: $f(t) = \frac{1}{2}|t-1|$
	$$D_{f}(p||q) = \int_{\mathcal{X}}|p(x)-q(x)|dx$$
\end{itemize}

\section{Lipchitz Continuous}
The function $f$ in the new form of Wasserstein metric is demanded to satisfy $\| f \|_L \leq K$, meaning it should be $K$-Lipschitz continuous.

A real-valued function $f: \mathbb{R} \rightarrow \mathbb{R}$ is called $K$-Lipschitz continuous if there exists a real constant $K\geq 0$ such that, for all $x_1, x_2 \in \mathbb{R}$
$$\lvert f(x_1) - f(x_2) \rvert \leq K \lvert x_1 - x_2 \rvert$$

\section{Singular Value}
All singular values can be calculated via the singular value decomposition (SVD)
$$A = U\Sigma V^T$$
, where $U$ is the left singular vectors and $V$ is the right singular vectors. However, if we just want the maximum singular value then we just need to find corresponding vectors
$$\sigma = uAv^T$$
Actually, there is a simpler way to find the maximum singular value e.g., power iteration. 



\end{appendices}


\backmatter
% bibliography, glossary and index would go here.

\nocite{*}
\bibliographystyle{unsrt}
\bibliography{references}
\end{document}
