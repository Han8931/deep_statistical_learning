\section{Logistic Regression}
\label{sec:logistic_regression}

Logistic regression corresponds to the following binary classification model:
$$p(y|\mathbf{x},\mathbf{w})=\textrm{Ber}(y|\sigma(\mathbf{w}^T\mathbf{x}))$$

Logistic regression modles a logit (log odds) through a linear model. For binary data, the goal is to model the probability $p$ that one of two outcomes occurs. The logit function is $\textrm{log}\frac{p}{1-p}$, which varies between $-\infty$ and $+\infty$ as $p$ varies between $0$ and $1$.
$$\textrm{log}\frac{p}{1-p} = w_0x_0 +  w_1x_1 + ... + w_nx_n$$
Note that the logistic regression model assumes that the log-odds (\textit{logit}) of an observation $y$ can be expressed as a linear function. In this context, the logit function is called the \textbf{\textit{link function}} because it ``links'' the probability to the linear function of the predictor variables.

% Simplest solution to model a dependant variable $y$ is a linear regression. However, $y$ should be in a range of $[0,1]$. So we need to introduce the logit function. 

% The linear regression can be generalized to the classification setting with two changes:
% \begin{itemize}
% 	\item Replacing the Gaussian distribution for $y$ with a Bernoulli distribution: $p(y|\mathbf{x},\mathbf{w})=Ber(y|\mu(\mathbf{x}))$
% 	\item Squashing input data into sigmoid function $\sigma(\eta)$ that range from 0 to 1: $\sigma(\eta)\triangleq \frac{1}{1+exp(-\eta)}$.
% \end{itemize}
% $$p(y|\mathbf{x},\mathbf{w})=Ber(y|\sigma(\mathbf{w}^T\mathbf{x})),\$$

The negative log-likelihood for logistic regression is given by
\begin{align*}
	\textrm{NLL}(\mathbf{w}) &= -\sum_{i=1}^{N}\textrm{log}[\mu_i^{\mathds{I}(y_i=1)}\times (1-\mu_i)^{\mathds{I}(y_i=0)}]\\
	&=-\sum_{i=1}^{N}[y_i\textrm{log}\mu_i + (1-y_i) \textrm{log}(1-\mu_i)], \textrm{ }\footnotemark
\end{align*}
where $\mu=\sigma(\mathbf{w}^T\mathbf{x})$. This is also called \textbf{cross-entropy} error function. 
\footnotetext{$\mathds{I}(y_i=1) = y_i$, because $y_i\in \{0, 1\}$ is a binary variable}

Another way to express \textrm{NLL} is as follows. Suppose $\hat{y}_i\in\{-1,+1\}$ instead of $y_i\in\{0,1\}$. We have $p(y=1)=\frac{1}{1+\mathrm{exp}(-\mathbf{w}^T\mathbf{x})}$ and $p(y=-1)=\frac{1}{1+\mathrm{exp}(+\mathbf{w}^T\mathbf{x})}$. Hence
\begin{align*}
	\textrm{NLL}(\mathbf{w}) &= -\frac{1}{N}\sum_{n=1}^N [\mathbb{I}(\hat{y}_n=1)\log(\sigma(a_n))+\mathbb{I}(\hat{y}_n=-1)\log(\sigma(-a_n))]\\
							 &= -\frac{1}{N}\sum_{n=1}^N \log(\sigma(\hat{y}_na_n))\\
							 &=  \frac{1}{N}\sum_{i=1}^{N}\textrm{log}(1+\mathrm{exp}(-\hat{y}_i\mathbf{w}^T\mathbf{x}_i).
\end{align*}
Note that the sigmoid is used for compressing the output into $[0,1]$ and $\sigma(-a_n) = 1-\sigma(a_n)$. Unlike the linear regression, there is no closed from solution for logistic regression, thus we need optimization algorithms for it. Typically, optimization process involves the gradient and Hessian. 
\begin{align*}
	\mathbf{g}&=\frac{d}{d\mathbf{w}}\mathrm{NLL}(\mathbf{w})=\frac{d}{d\mu_i}\mathrm{NLL}(\mathbf{w})\frac{d\mu_i}{d\mathbf{h}}\frac{d\mathbf{h}}{d\mathbf{w}}\\
	& = \sum_{i=1}\Bigg[-\frac{y_i}{\mu_i} + \frac{(1-y_i) }{(1-\mu_i)}\Bigg]\frac{d\mu_i}{d\mathbf{h}}\frac{d\mathbf{h}}{d\mathbf{w}}=\sum_{i=1}\Bigg[\frac{\mu_i-y_i }{\mu_i(1-\mu_i)}\Bigg]\frac{d\mu_i}{d\mathbf{h}}\frac{d\mathbf{h}}{d\mathbf{w}}\\
	&=\sum_{i}(\mu_i-y_i)\mathbf{x}_i=\mathbf{X}^T(\boldsymbol{\mu}-\mathbf{y})\\
	\frac{d\mu_i}{d\mathbf{h}}& = \mu_i(1-\mu_i)\\
	\frac{d\mathbf{h}}{d\mathbf{w}}& = \mathbf{x}_i
\end{align*}
where $\mathbf{h}=\mathbf{w}^T\mathbf{x}$. 

We can also use the second-order method. 
\begin{align*}
\mathbf{H}&=\frac{d}{d\mathbf{w}}g(\mathbf{w})^T=\sum_{i}(\nabla_{\mathbf{w}}\mu_i)\mathbf{x}_i^T=\sum_{i}\mu_i(1-\mu_i)\mathbf{x}_i\mathbf{x}_i^T\\
&=\mathbf{X}^T\mathbf{S}\mathbf{X},
\end{align*}
where $\mathbf{S}\triangleq \mathrm{diag}(\mu_i(1-\mu_i))$. Note that $\mathbf{H}$ is positive definite, because the \textrm{NLL} is convex and has a global minimum. 
