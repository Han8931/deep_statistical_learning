\section{Introduction}
\label{state:sec:intro}


\subsection{State-Space Model}
\begin{align*}
	\frac{dX}{dt} &= {A}X+{B}U,\\
	Y &= {C}X+DU.
\end{align*}
The first and the second equations are known as \textit{state equation} and \textit{output equation}, respectively. The state equation tells us that how the state vector changes with the state vector and the input. 

\begin{itemize}
	\item $X$ and ${dX}/{dt}$ are the state vector and the differential state vector respectively. 
	\item $U$ and $Y$ are scalar input vector and scalar output vector respectively. 
	\item $A$ is the system matrix.
	\item $B$ and $C$ are the input and the output matrices.
	\item $D$ is the feed-forward matrix.
\end{itemize}


The state space model (SSM) can be defined as follows:
\begin{align*}
	x'(t) &= \mathbf{A}x(t)+\mathbf{B}u(t),\\
	y(t) &= \mathbf{C}x(t)+\mathbf{D}u(t).
\end{align*}

It maps a single dimensional input signal $u(t)$ to an $N$-dim latent state $x(t)$ before projecting to a one-dim output signal $y(t)$.  

Our goal is to simply use the SSM as a black-box representation in a deep sequence model, where $\rmA, \rmB, \rmC$, and $\rmD$ are parameters learned by gradient descent. We will omit the parameter $\rmD$ for exposition (or equivalently, assume $\rmD=0$, because the term $\rmD u$ can be viewed as a skip connection and is easy to compute).

An SSM maps a input $u(t)$ to a state representation vector $x(t)$ and an output $y(t)$. For simplicity, we assume the input and output are one-dimensional, and the state representation is $N$-dimensional. The first equation defines the change in $x(t)$ over time.

% \begin{lstlisting}[language=Python]
% 	def random_SSM(rng, N):
% 		a_r, b_r, c_r = jax.random.split(rng, 3)
% 		A = jax.random.uniform(a_r, (N, N))
% 		B = jax.random.uniform(b_r, (N, 1))
% 		C = jax.random.uniform(c_r, (1, N))
%     return A, B, C
% \end{lstlisting}
