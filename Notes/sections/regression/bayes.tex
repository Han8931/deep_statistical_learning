\section{Bayesian Regression}
\label{sec:regression_bayes}

Integrate over all $\theta$

\begin{align}
P(heads \mid D) =& \int_{\theta} P(heads, \theta \mid D) d\theta\\
 =& \int_{\theta} P(heads \mid \theta, D) P(\theta \mid D) d\theta \quad \because  \textrm{Chain rule, } P(A,B|C)=P(A|B,C)P(B|C)\\ 
  =& \int_{\theta} \theta P(\theta \mid D) d\theta\\ 
  =&E\left[\theta|D\right]\\
 =&\frac{n_H + \alpha}{n_H + \alpha + n_T + \beta}
\end{align}


% \chapter{Introduction}
\section{Curve Fitting}
We can assume that a target variable has a Gaussian distribution with a mean equal to the value $y(x,\mathbf{w})$ of the polynomial curven given by
\begin{equation}
	p(t|x, \mathbf{w}, \beta) = \mathcal{N}(t|y(x,\mathbf{w}), \beta^{-1}),
	\label{eq:curve}
\end{equation}
\begin{itemize}
	\item $t$: target variable
		$$t = y(\rvx, \rvw) +\epsilon,$$
	where $\epsilon$ is a zero mean Gaussian noise with precision (inverse variance) $\beta$. 
	% \item $x$: input
	% \item $\beta$: an inverse variance of the distribution.
\end{itemize}

We not use the training data $\{\mathbf{x,y}\}$ to determine the values of the unknown parameters $\mathbf{w}$ and $\beta$ by maximum likelihood. If the data are assumed to be drawn independently from the distribution, then the likelihood function is given by 
\begin{equation}
	p(\mathbf{t}|\mathbf{x,w},\beta) = \prod_{n=1}^{N}\mathcal{N}(t_n|y(x_n,\mathbf{w}), \beta^{-1}).
	\label{eq:curve_ml}
\end{equation}

We can take a step towards a more Bayesian approach and introduce a prior distribution over the polynomial coefficients $\mathbf{w}$. For simplicity, we can use a Gaussian distribution from
\begin{equation}
	p(\mathbf{w}|\alpha) = \mathcal{N}(\mathbf{w|0},\alpha^{-1}\mathbf{I}),
	\label{eq:prior_hyper}
\end{equation}
where $\alpha$ is the precision of the distribution. Using Bayes' theorem, the posterior distribution for $\mathbf{w}$ is proportional to the product of the prior distribution and the likelihood function
\begin{equation}
	p(\mathbf{w|x,t},\alpha,\beta)\propto p(\mathbf{w|x,t},\beta)p(\mathbf{w},\alpha).
	\label{eq:bayes_reg}
\end{equation}
We can now determined $\mathbf{w}$ by finding the most probable value of $\mathbf{w}$ given the data, in other words by maximizing the posterior distribution, MAP (maximum posterior). Taking a negative logarithm, then we can find that the maximum of the posterior is given by the minimum of 
\begin{equation}
	\frac{\beta}{2}\sum_{n=1}^N \{y(x_n,\mathbf{w})-t_n\}^2+\frac{\alpha}{2}\mathbf{w}^T\mathbf{w}.
	\label{eq:bayes_}
\end{equation}
Thus we see that maximizing the posteiror distribution si equivalent to minimizing the regularized sum of squares error function encountered earler with a regularization parameter given by $\lambda = \alpha/\beta$.



\section{Bayesian Curve Fitting}


% \section{Adding Noise to Regression Predictors is Ridge Regression}
% In linear regression, we seek a vector $\hat{\beta}$ which solves the following optimization problem:
% \begin{equation}
% 	\hat{\beta} = \arg\min_{\beta}|y-X\beta|^2
% 	\label{eq:linear_regression}
% \end{equation}

% Ridge regression is

% \begin{equation}
% 	\hat{\beta} = \arg\min_{\beta}|y-X\beta|^2+\lambda|\beta|^2
% 	\label{eq:ridge_linear_regression}
% \end{equation}

% $$\varepsilon_1,\varepsilon_2,\cdots\sim \mathcal{N}(1,\sigma),$$
% We can add a multiplicitive random noise to $\mathbf{X}$.
% $$x_{ij}\to \varepsilon_{ij}x_{ij}$$

% $$\hat{\beta} \sim \arg\min_{\beta} E_{G}[|y-G\cdot X\beta|^2]$$
% where $G$ is a matrix of random Gaussian noise. 

% \begin{align*}
% 	\mathbb{E} \left[ \left| y - (G * X) \beta  \right|^2 \right] &= E \left[ y^t y - 2 y^t (G * X) \beta + \beta^t (G * X)^t (G * X) \right] \\
% &= y^t y - 2 y^t (E[G] * X) \beta + \beta^t E \left[ M \right] \beta \\
% &= y^t y - 2 y^t X \beta + \beta^t X^t X \beta + \beta^t diag(\sigma^2) X^t X \beta \\
% &= \left| y - X \beta \right|^2 + \beta^t diag(\sigma^2) X^t X \beta \\
% &= \left| y - X \beta \right|^2 + \sigma^2 \left| \Gamma \beta \right|^2
% \end{align*}
% where $M = (G\cdot X)^T(G\cdot X)$
% $$m_{ij} = \sum_k e_{ki}e_{kj}x_{ki}x_{kj}$$

	
