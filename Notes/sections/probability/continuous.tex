\section{Continuous and Mixed Random Variables}
Remember that discrete random variables can take only a countable number of possible values. On the other hand, a continuous random variable $X$ has a range in the form of an interval or a union of non-overlapping intervals on the real line (possibly the whole real line). Also, for any $x\in \mathbb{R}$, $P(X=x)=0$. Thus, we need to develop new tools to deal with continuous random variables. The good news is that the theory of continuous random variables is completely analogous to the theory of discrete random variables. Indeed, if we want to oversimplify things, we might say the following: take any formula about discrete random variables, and then replace sums with integrals, and replace PMFs with probability density functions (PDFs), and you will get the corresponding formula for continuous random variables. 

\section{Probability Density Function (PDF)}


