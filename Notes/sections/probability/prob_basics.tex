\section{Probability}
\label{sec:intro_prob}

The probability of event $A$ as
\begin{align*}
	P(A) = \frac{\text{Number of times A occurs}}{\text{Total number of outcomes}}
\end{align*}
This commonsense understanding of probability is called the \textit{relative frequency definition}.

\begin{definition}{Independence}
	\begin{align*}
		X\perp Y \leftrightarrow p(X,Y)=p(X)p(Y)
	\end{align*}
\end{definition}
\begin{definition}{Conditional independence}
	\begin{align*}
		X\perp Y|Z \leftrightarrow p(X,Y|Z)=p(X|Z)p(Y|Z)
	\end{align*}
\end{definition}
All the dependencies between $X$ and $Y$ are mediated via $Z$. If $X$ and $Y$ are conditionally independent, then 
\begin{align*}
	p(X|Y,Z)&=\frac{p(X,Y|Z)}{p(Y|Z)}\\
	&=\frac{p(X|Z)p(Y|Z)}{p(Y|Z)}\\
	&=p(X|Z).
\end{align*}


\subsection{The law of total expectation}
$$\mathbb{E}[\mathbb{E}[Y|X]] = \mathbb{E}[Y].$$

It is worth trying to parse this formula more carefully and adding some subscripts:
$$\mathbb{E}_X[\mathbb{E}_{Y|X}[Y|X]] = \mathbb{E}[Y].$$
Intuitively, this expression has a divide and conquer flavour, \ie what it says is that to compute the average of a random variable $Y$, you can first compute its average over a bunch of partitions of the sample space (where some other random variable $X$ is fixed to different values), and then average the resulting averages. 

Example: Suppose I had a population of people, 47\% of whom weer men and the remaining 53\% were women. Suppose that the average height of the men was 70 inches, and the women was 71 inches. What is the average height of the entire population?

By the law of total expectation:

\begin{align*}
	\mathbb{E}[H] &= \mathbb{E}[\mathbb{E}[H|S]]\\
				  &= \mathbb{E}[H|S]\mathbb{P}(S=m) + \mathbb{E}[H|S]\mathbb{P}(S=f)\\
			   &= 70\times 0.47 + 71\times 0.53 = 70.53
\end{align*}

\section{Transformations of Random Variable}
\subsection{Formal Definition}
Suppose $X$ is a continuous random variable with pdf $f(x)$. If we define $Y=g(X)$, where $g(\cdot)$ is a monotonically increasing function, then the pdf of $Y$ can be obtained as follows:
\begin{align*}
	p(Y\leq y) &= p(g(X)\leq y)\\
	& = p(X\leq g^{-1}(y))
\end{align*}
This can be re-written as by definition
\begin{align*}
F_Y(y) = F_X(g^{-1}(y))
\end{align*}
By differentiating the CDFs on both sides w.r.t. $y$, we can get the pdf of $Y$. If the function $g(\cdot)$ is monotonically increasing, then the pdf of $Y$ is given by
$$f_Y(y) = f_X(g^{-1}(y))\frac{d}{dy}g^{-1}(y)$$
On the other hand, if it is monotonically decreasing, then the pdf of $Y$ is given by
$$f_Y(y) = - f_X(g^{-1}(y))\frac{d}{dy}g^{-1}(y)$$
Compactly, the above two equations can be combined into a following equation:
$$f_Y(y) = f_X(g^{-1}(y))\Bigg|\frac{d}{dy}g^{-1}(y)\Bigg|$$

Let's look at what happens to the pdf of an RV when we pass the RV through some function. Suppose that we have two RVs, $X$ and $Y$, related to one another by the monotonic functions $g(\cdot)$ and $h(\cdot)$:
\begin{align*}
	Y &= g(X)\\
	X &= g^{-1}(X) = h(Y)
\end{align*}
If we know the pdf (probability density function) of $X$ 

\subsection{Intuition}
Given a random variable $z$ and its known probability density function $z\sim \pi(z)$, we would like to construct a new random variable using a one-to-one mapping function $x=f(z)$. The function $f$ is invertible, so $z = f^{-1}(x)$. Now the question is how to infer the unknown probability density function of the new variable, $p(x)$?

$$
\begin{aligned}
& \int p(x)dx = \int \pi(z)dz = 1 \scriptstyle{\text{   ; Definition of probability distribution.}}\\
& p(x) = \pi(z) \left\vert\frac{dz}{dx}\right\vert = \pi(f^{-1}(x)) \left\vert\frac{d f^{-1}(x)}{dx}\right\vert = \pi(f^{-1}(x)) \vert (f^{-1})'(x) \vert
\end{aligned}$$

In multivariate case, 

\begin{align}
\mathbf{z} &\sim \pi(\mathbf{z}), \mathbf{x} = f(\mathbf{z}), \mathbf{z} = f^{-1}(\mathbf{x}) \\
p(\mathbf{x}) 
&= \pi(\mathbf{z}) \left\vert \det \dfrac{d \mathbf{z}}{d \mathbf{x}} \right\vert  
= \pi(f^{-1}(\mathbf{x})) \left\vert \det \dfrac{d f^{-1}}{d \mathbf{x}} \right\vert
\end{align}

\section{Bernoulli distribution}
\label{sec:bernoulli}

The probability of $x=1$ will be denoted by the parameter $\mu$ so that 
$$p(x=1|\mu) = \mu,$$
where $0 \leq \mu \leq 1$, from which it follows that $p(x=0|\mu) = 1-\mu$. The probability distribution over $x$ can be written in the form 
\begin{align}
	Bern(x|\mu) = \mu^x(1-\mu)^{1-x},
	\label{eq:bernoulli}
\end{align}
which is known as the \textit{Bernoulli distribution}.  

Now suppose we have a dataset $\mathcal{D} = \{x_1, \dots, x_{N}\}$. We can construct the likelihood function, which is a function of $\mu$, on the assumption that the observations are drawn independently from $p(x|\mu)$, so that

$$p(\mathcal{D}|\mu) = \prod_{n=1}^Np(x_n|\mu)= \prod_{n=1}^N\mu^{x_n}(1-\mu)^{1-x_n}$$

In a frequentist, setting, we can estimate a value for $\mu$ by maximizing the likelihood function, or equivalently maximizing the logarithm of the likelihood. 
$$L(w) = p(\rvt|\rvw, \sigma^2) = $$

\section{Gaussian Distribution}
For a $D$-dimensional vector $\rvx$, the multivariate Gaussian distribution takes the form
\begin{align}
	\mathcal{N}(\mathbf{x}|\boldsymbol{\mu},\boldsymbol{\Sigma}) &= \frac{1}{(2\pi)^{D/2}|\boldsymbol{\Sigma}|^{1/2}}\exp\bigg(-\frac{1}{2}(\mathbf{x}-\boldsymbol{\mu})^T\boldsymbol{\Sigma}^{-1}(\mathbf{x}-\boldsymbol{\mu})\bigg)\\
	\label{eq:normal_distribution}
\end{align}

\subsection{Conditional Gaussian Distribution}
Consider first the case of conditional distributions. Suppose $\rvx$ is a $D$-dimensional vector with Gaussian distribution $\mathcal{N}(\mathbf{x}|\boldsymbol{\mu},\boldsymbol{\Sigma})$ and that we partition $\rvx$ into two disjoint subsets $\rvx_a$ and $\rvx_b$. Thus, $\rvx_a$ has $M$ components and $\rvx_b$ has $D-M$ components.
\begin{align*}
	\rvx = \begin{bmatrix}
		\rvx_a\\
		\rvx_b
	\end{bmatrix}.
\end{align*}
Similarly, 
\begin{align*}
	\boldsymbol{\mu} = \begin{bmatrix}
		\boldsymbol{\mu}_a\\
		\boldsymbol{\mu}_b
	\end{bmatrix} 
\end{align*}
and the covariance matrix is given by
\begin{align*}
	\boldsymbol{\Sigma} = \begin{bmatrix}
		\boldsymbol{\Sigma}_{aa} & \boldsymbol{\Sigma}_{ab}\\
		\boldsymbol{\Sigma}_{ba} & \boldsymbol{\Sigma}_{bb} 
	\end{bmatrix} 
\end{align*}
Note that the symmetry $\mSigma^T = \mSigma$ implies that $\mSigma_{ab}^T=\mSigma_{ba}$. We can also define a \textit{precision matrix} as follows: 
$$\mLambda \equiv \mSigma^{-1}$$
We also introduce a partitioned form of the precision matrix:
\begin{align}
	\boldsymbol{\Lambda} = \begin{bmatrix}
		\mLambda_{aa} & \mLambda_{ab}\\
		\mLambda_{ba} & \mLambda_{bb} 
	\end{bmatrix} 
	\label{eq:precision_matrix}
\end{align}
Because the inverse of a symmetric matrix is also symmetric, we see that $\mLambda_{aa}$ and $\mLambda_{bb}$ are symmetric and $\mLambda_{ab}^T =\mLambda_{ba}$. Note that, for instance, $\mLambda_{aa}$ is not simply given by the inverse of $\boldsymbol{\Sigma}_{aa}$. 

Now let's compute the conditional probability:
\begin{align*}
	-\frac{1}{2}(\mathbf{x}-\boldsymbol{\mu})^T\boldsymbol{\Sigma}^{-1}(\mathbf{x}-\boldsymbol{\mu}) &= \frac{1}{2}\Bigg(\bigg(\begin{bmatrix}
		\rvx_a\\
		\rvx_b
	\end{bmatrix}-
	\begin{bmatrix}
		\boldsymbol{\mu}_a\\
		\boldsymbol{\mu}_b
	\end{bmatrix} 
\bigg)^T\begin{bmatrix}
		\mLambda_{aa} & \mLambda_{ab}\\
		\mLambda_{ba} & \mLambda_{bb} 
	\end{bmatrix}\bigg(\begin{bmatrix}
		\rvx_a\\
		\rvx_b
	\end{bmatrix}-
	\begin{bmatrix}
		\boldsymbol{\mu}_a\\
		\boldsymbol{\mu}_b
	\end{bmatrix}\bigg)\Bigg)\\
	&= -\frac{1}{2}\big((\rvx_a-\vmu_a)^T\mLambda_{aa}(\rvx_a-\vmu_a)+(\rvx_a-\vmu_a)^T\mLambda_{ab}(\rvx_a-\vmu_a)\\
	&\quad+(\rvx_a-\vmu_a)^T\mLambda_{ba}(\rvx_a-\vmu_a)+(\rvx_a-\vmu_a)^T\mLambda_{bb}(\rvx_a-\vmu_a)\big)\\
\end{align*}



