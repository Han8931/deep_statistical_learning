\chapter{Introduction to Probability}
\section{Introduction}
\label{sec:intro_prob}

\href{https://www.probabilitycourse.com/chapter5/5_2_3_conditioning_independence.php}{Reference: Introduction to Probability, Statistics, and Random Processes.}

The probability of event $A$ as
\begin{align*}
	P(A) = \frac{\text{Number of times A occurs}}{\text{Total number of outcomes}}
\end{align*}
This commonsense understanding of probability is called the \textit{relative frequency definition}.

Bayes'Rule is often written by rearranging the above equation to obtain
\begin{align*}
	P(A|B) = \frac{P(B|A)P(A)}{P(B)}
\end{align*}

We define a random variable (RV) as a functional mapping from a set of experimental outcomes (the domain) to a set of real numbers (the range).

The most fundamental property of an RV $X$ is its cumulative distribution function (CDF) $F_X(x)$, defined as 
$$F_{X}(x) = P(X\leq x).$$
Some properties of the CDF are given by
\begin{align*}
	F_X(x)&= [0,1]\\
	F_X(-\infty)&= 0\\
	F_X(\infty)&= 1\\
	F_X(a)&\leq F_X(b), \quad \text{if }a\leq b \\
	P(a<x\leq b)&=F_X(b)-F_X(a)
\end{align*}
The probability density function (PDF) $f_X(x)$ is defined as the derivative of the CDF:
\begin{align*}
	f_X(x) = \frac{dF_X(x)}{dx}.
\end{align*}
Some properties of the PDF that can be obtained from this definition are
\begin{align*}
	F_X(x) &= \int_{-\infty}^xf_X(z)dz\\
	f_X(x)&\geq 0\\
	\int_{-\infty}^{\infty}f_X(x)dx&= 1\\
	P(a<x\leq b)&=\int_{a}^bf_X(x)dx
\end{align*}
The $Q$-function of an RV is defined as one minus the CDF. This is equal to the probability that the RV is greater than the argument of the function: 
\begin{align*}
	Q(x)&= 1-F_X(x)\\
		&= P(X>x).
\end{align*}
Suppose that we know that the event $X\in I=[a,b]$ has occurred. Call this event $A$. The conditional distribution of the RV $X$ given $A$ can be defined as
\begin{align*}
	F_{X|A}(x)&= P(X\leq x|A)\\
			&= P(X\leq x|a\leq X \leq b)\\
			&= \frac{P(X\leq x,a\leq X \leq b)}{P(A)}\\
\end{align*}
Now if $x\leq 0$, then $F_{X|A}(x)=0$. One the other hand, if $a\leq x \leq b$, we have
\begin{align*}
	F_{X|A}(x)&= P(X\leq x|a\leq X \leq b)\\
			&= \frac{P(a\leq X \leq b)}{P(A)}\\
			&= \frac{F_X(x)-F_X(a)}{F_X(b)-F_X(a)}
\end{align*}
Finally, if $x>b$, then $F_{X|A}(x)=1$. Thus, we obtain
\begin{align*}
F_{X|A}(x)=
	\begin{cases}
	1\quad x>b\\	
	\frac{F_X(x)-F_X(a)}{F_X(b)-F_X(a)}\quad a\leq x <b\\	
	0\quad \text{Otherwise}\
	\end{cases}
\end{align*}
We can obtain the conditional PDF of $X$ given $A$ by differentiating $F_{X|A}$ as follows:

Now the conditional PDF of $X$ given $A$, denoted by $f_{X|A}(x)$ is 
\begin{align*}
	f_{X|A}(x) = 
	\begin{cases}
	\frac{f_{X}(x)}{P(A)}\quad a\leq x <b\\	
	0\quad \text{Otherwise}\
	\end{cases}
\end{align*}

It is insightful if we derive the above formula for fX|A(x) directly from the definition of the PDF for continuous random variables. Recall that the PDF of X can be defined as
\begin{align*}
	f_X(x)=\lim_{\Delta \rightarrow 0^+} \frac{P(x<X \leq x+\Delta)}{\Delta}.
\end{align*}
Now, the conditional PDF of $X$ given $A$, denoted by $f_{X|A}(x)$, is

\begin{align*}
	 f_{X|A}(x)&=\lim_{\Delta \rightarrow 0^+} \frac{P(x<X \leq x+\Delta|A)}{\Delta}\\
	 &=\lim_{\Delta \rightarrow 0^+} \frac{P(x<X \leq x+\Delta,A)}{\Delta P(A)}\\
	 &=\lim_{\Delta \rightarrow 0^+} \frac{P(x<X \leq x+\Delta,a \leq X \leq b)}{\Delta P(A)}.
\end{align*}
Now consider two cases. If $a\leq x < b$, then
\begin{align*}
	 f_{X|A}(x)&=\lim_{\Delta \rightarrow 0^+} \frac{P(x<X \leq x+\Delta,a \leq X \leq b)}{\Delta P(A)}\\
	 &=\frac{1}{P(A)}\lim_{\Delta \rightarrow 0^+} \frac{P(x<X \leq x+\Delta)}{\Delta}\\
	 &=\frac{f_X(x)}{P(A)}.
\end{align*}

The conditional expectation and variance are defined by replacing the PDF by conditional PDF in the definitions of expectation and variance. 
\begin{align*}
	\mathbb{E}[X|A] &= \int_{-\infty}^{\infty}xf_{X|A}(x)dx\\
	\mathbb{E}[g(X)|A]	&= \int_{-\infty}^{\infty}g(x)f_{X|A}(x)dx\\
	\text{Var}[X|A]	&=\mathbb{E}[X^2|A]-\left(\mathbb{E}[X|A]\right)^2
\end{align*}


% \begin{definition}{Independence}
% 	\begin{align*}
% 		X\perp Y \leftrightarrow p(X,Y)=p(X)p(Y)
% 	\end{align*}
% \end{definition}
% \begin{definition}{Conditional independence}
% 	\begin{align*}
% 		X\perp Y|Z \leftrightarrow p(X,Y|Z)=p(X|Z)p(Y|Z)
% 	\end{align*}
% \end{definition}
% All the dependencies between $X$ and $Y$ are mediated via $Z$. If $X$ and $Y$ are conditionally independent, then 
% \begin{align*}
% 	p(X|Y,Z)&=\frac{p(X,Y|Z)}{p(Y|Z)}\\
% 	&=\frac{p(X|Z)p(Y|Z)}{p(Y|Z)}\\
% 	&=p(X|Z).
% \end{align*}
%
%
% \subsection{The law of total expectation}
% $$\mathbb{E}[\mathbb{E}[Y|X]] = \mathbb{E}[Y].$$
%
% It is worth trying to parse this formula more carefully and adding some subscripts:
% $$\mathbb{E}_X[\mathbb{E}_{Y|X}[Y|X]] = \mathbb{E}[Y].$$
% Intuitively, this expression has a divide and conquer flavour, \ie what it says is that to compute the average of a random variable $Y$, you can first compute its average over a bunch of partitions of the sample space (where some other random variable $X$ is fixed to different values), and then average the resulting averages. 
%
% Example: Suppose I had a population of people, 47\% of whom weer men and the remaining 53\% were women. Suppose that the average height of the men was 70 inches, and the women was 71 inches. What is the average height of the entire population?
%
% By the law of total expectation:
%
% \begin{align*}
% 	\mathbb{E}[H] &= \mathbb{E}[\mathbb{E}[H|S]]\\
% 				  &= \mathbb{E}[H|S]\mathbb{P}(S=m) + \mathbb{E}[H|S]\mathbb{P}(S=f)\\
% 			   &= 70\times 0.47 + 71\times 0.53 = 70.53
% \end{align*}
%
% \section{Transformations of Random Variable}
% \subsection{Formal Definition}
% Suppose $X$ is a continuous random variable with pdf $f(x)$. If we define $Y=g(X)$, where $g(\cdot)$ is a monotonically increasing function, then the pdf of $Y$ can be obtained as follows:
% \begin{align*}
% 	p(Y\leq y) &= p(g(X)\leq y)\\
% 	& = p(X\leq g^{-1}(y))
% \end{align*}
% This can be re-written as by definition
% \begin{align*}
% F_Y(y) = F_X(g^{-1}(y))
% \end{align*}
% By differentiating the CDFs on both sides w.r.t. $y$, we can get the pdf of $Y$. If the function $g(\cdot)$ is monotonically increasing, then the pdf of $Y$ is given by
% $$f_Y(y) = f_X(g^{-1}(y))\frac{d}{dy}g^{-1}(y)$$
% On the other hand, if it is monotonically decreasing, then the pdf of $Y$ is given by
% $$f_Y(y) = - f_X(g^{-1}(y))\frac{d}{dy}g^{-1}(y)$$
% Compactly, the above two equations can be combined into a following equation:
% $$f_Y(y) = f_X(g^{-1}(y))\Bigg|\frac{d}{dy}g^{-1}(y)\Bigg|$$
%
% Let's look at what happens to the pdf of an RV when we pass the RV through some function. Suppose that we have two RVs, $X$ and $Y$, related to one another by the monotonic functions $g(\cdot)$ and $h(\cdot)$:
% \begin{align*}
% 	Y &= g(X)\\
% 	X &= g^{-1}(X) = h(Y)
% \end{align*}
% If we know the pdf (probability density function) of $X$ 
%
% \subsection{Intuition}
% Given a random variable $z$ and its known probability density function $z\sim \pi(z)$, we would like to construct a new random variable using a one-to-one mapping function $x=f(z)$. The function $f$ is invertible, so $z = f^{-1}(x)$. Now the question is how to infer the unknown probability density function of the new variable, $p(x)$?
%
% $$
% \begin{aligned}
% & \int p(x)dx = \int \pi(z)dz = 1 \scriptstyle{\text{   ; Definition of probability distribution.}}\\
% & p(x) = \pi(z) \left\vert\frac{dz}{dx}\right\vert = \pi(f^{-1}(x)) \left\vert\frac{d f^{-1}(x)}{dx}\right\vert = \pi(f^{-1}(x)) \vert (f^{-1})'(x) \vert
% \end{aligned}$$
%
% In multivariate case, 
%
% \begin{align}
% \mathbf{z} &\sim \pi(\mathbf{z}), \mathbf{x} = f(\mathbf{z}), \mathbf{z} = f^{-1}(\mathbf{x}) \\
% p(\mathbf{x}) 
% &= \pi(\mathbf{z}) \left\vert \det \dfrac{d \mathbf{z}}{d \mathbf{x}} \right\vert  
% = \pi(f^{-1}(\mathbf{x})) \left\vert \det \dfrac{d f^{-1}}{d \mathbf{x}} \right\vert
% \end{align}
%
% \section{Bernoulli distribution}
% \label{sec:bernoulli}
%
% The probability of $x=1$ will be denoted by the parameter $\mu$ so that 
% $$p(x=1|\mu) = \mu,$$
% where $0 \leq \mu \leq 1$, from which it follows that $p(x=0|\mu) = 1-\mu$. The probability distribution over $x$ can be written in the form 
% \begin{align}
% 	Bern(x|\mu) = \mu^x(1-\mu)^{1-x},
% 	\label{eq:bernoulli}
% \end{align}
% which is known as the \textit{Bernoulli distribution}.  
%
% Now suppose we have a dataset $\mathcal{D} = \{x_1, \dots, x_{N}\}$. We can construct the likelihood function, which is a function of $\mu$, on the assumption that the observations are drawn independently from $p(x|\mu)$, so that
%
% $$p(\mathcal{D}|\mu) = \prod_{n=1}^Np(x_n|\mu)= \prod_{n=1}^N\mu^{x_n}(1-\mu)^{1-x_n}$$
%
% In a frequentist, setting, we can estimate a value for $\mu$ by maximizing the likelihood function, or equivalently maximizing the logarithm of the likelihood. 
% $$L(w) = p(\rvt|\rvw, \sigma^2) = $$
%
% \section{Gaussian Distribution}
For a $D$-dimensional vector $\rvx$, the multivariate Gaussian distribution takes the form
\begin{align}
	\mathcal{N}(\mathbf{x}|\boldsymbol{\mu},\boldsymbol{\Sigma}) &= \frac{1}{(2\pi)^{D/2}|\boldsymbol{\Sigma}|^{1/2}}\exp\bigg(-\frac{1}{2}(\mathbf{x}-\boldsymbol{\mu})^T\boldsymbol{\Sigma}^{-1}(\mathbf{x}-\boldsymbol{\mu})\bigg)\\
	\label{eq:normal_distribution}
\end{align}

\subsection{Conditional Gaussian Distribution}
Consider first the case of conditional distributions. Suppose $\rvx$ is a $D$-dimensional vector with Gaussian distribution $\mathcal{N}(\mathbf{x}|\boldsymbol{\mu},\boldsymbol{\Sigma})$ and that we partition $\rvx$ into two disjoint subsets $\rvx_a$ and $\rvx_b$. Thus, $\rvx_a$ has $M$ components and $\rvx_b$ has $D-M$ components.
\begin{align*}
	\rvx = \begin{bmatrix}
		\rvx_a\\
		\rvx_b
	\end{bmatrix}.
\end{align*}
Similarly, 
\begin{align*}
	\boldsymbol{\mu} = \begin{bmatrix}
		\boldsymbol{\mu}_a\\
		\boldsymbol{\mu}_b
	\end{bmatrix} 
\end{align*}
and the covariance matrix is given by
\begin{align*}
	\boldsymbol{\Sigma} = \begin{bmatrix}
		\boldsymbol{\Sigma}_{aa} & \boldsymbol{\Sigma}_{ab}\\
		\boldsymbol{\Sigma}_{ba} & \boldsymbol{\Sigma}_{bb} 
	\end{bmatrix} 
\end{align*}
Note that the symmetry $\mSigma^T = \mSigma$ implies that $\mSigma_{ab}^T=\mSigma_{ba}$. We can also define a \textit{precision matrix} as follows: 
$$\mLambda \equiv \mSigma^{-1}$$
We also introduce a partitioned form of the precision matrix:
\begin{align}
	\boldsymbol{\Lambda} = \begin{bmatrix}
		\mLambda_{aa} & \mLambda_{ab}\\
		\mLambda_{ba} & \mLambda_{bb} 
	\end{bmatrix} 
	\label{eq:precision_matrix}
\end{align}
Because the inverse of a symmetric matrix is also symmetric, we see that $\mLambda_{aa}$ and $\mLambda_{bb}$ are symmetric and $\mLambda_{ab}^T =\mLambda_{ba}$. Note that, for instance, $\mLambda_{aa}$ is not simply given by the inverse of $\boldsymbol{\Sigma}_{aa}$. 

Now let's compute the conditional probability:
\begin{align*}
	-\frac{1}{2}(\mathbf{x}-\boldsymbol{\mu})^T\boldsymbol{\Sigma}^{-1}(\mathbf{x}-\boldsymbol{\mu}) &= \frac{1}{2}\Bigg(\bigg(\begin{bmatrix}
		\rvx_a\\
		\rvx_b
	\end{bmatrix}-
	\begin{bmatrix}
		\boldsymbol{\mu}_a\\
		\boldsymbol{\mu}_b
	\end{bmatrix} 
\bigg)^T\begin{bmatrix}
		\mLambda_{aa} & \mLambda_{ab}\\
		\mLambda_{ba} & \mLambda_{bb} 
	\end{bmatrix}\bigg(\begin{bmatrix}
		\rvx_a\\
		\rvx_b
	\end{bmatrix}-
	\begin{bmatrix}
		\boldsymbol{\mu}_a\\
		\boldsymbol{\mu}_b
	\end{bmatrix}\bigg)\Bigg)\\
	&= -\frac{1}{2}\big((\rvx_a-\vmu_a)^T\mLambda_{aa}(\rvx_a-\vmu_a)+(\rvx_a-\vmu_a)^T\mLambda_{ab}(\rvx_a-\vmu_a)\\
	&\quad+(\rvx_a-\vmu_a)^T\mLambda_{ba}(\rvx_a-\vmu_a)+(\rvx_a-\vmu_a)^T\mLambda_{bb}(\rvx_a-\vmu_a)\big)\\
\end{align*}



