\chapter{Gaussian Process}
\href{https://github.com/jwangjie/Gaussian-Process-Regression-Tutorial}{Reference}


\section{Introduction}
\label{sec:gaussian_process}
A Gaussian process (GP) is a probability distribution over possible functions that fit a set of points. More formally, GPs are distributions over functions $f(x)$ of which the distribution is defined by a mean function $m(x)$ and positive definite covariance function $k(x,x')$ (\ie \textit{kernel}). Thus, it is a distribution over functions whose shape is defined by the kernel:
$$f(x) \sim \mathcal{GP}(m(x),k(x,x')).$$

Since we have the probability distribution over all possible functions, we can calculate the mean and the variance of the function to determine how confident in our predictions.

Let's say we have observations, and we have estimated functions $\rvf$ with these observations. Now say we have some new points $\mathbf{X}_*$ where we want to predict $f(\mathbf{X}_*)$.

The joint distribution of $\rvf$ and $\rvf_*$ can be modeled as:

\begin{align*}
	\begin{pmatrix}
		\rvf\\
		\rvf_*
		\end{pmatrix}\sim \mathcal{N} 
		\begin{pmatrix}
		\begin{pmatrix}
			m(\mathbf{X})\\
			m(\mathbf{X}_*)
		\end{pmatrix},
		\begin{pmatrix}
			\mathbf{K} & \mathbf{K}_*\\
			\mathbf{K}_*^T & \mathbf{K}_{**}
		\end{pmatrix}
		\end{pmatrix},
\end{align*}
% where 
% However, what we want is $\rvf_*$, predictions for new observations, which is a conditional distribution over $\rvf_*$. 
% Let $f(x) = \phi(x)^Tw$, with $w\sim \mathcal{N}(0, \alpha^{-1}\mathbf{I})$. Then, $m(x) = \mathbb{E}[f(x)] = \mathbb{E}[w]^T\phi(x) = 0$
% $k(x, x') = \mathbb{E}[(f(x)-m(x))(f(x')-m(x'))] =\mathbb{E}[f(x)(f(x')] $
where 
\begin{align*}
	\mathbf{K }&= \kappa(\mathbf{X,X})\\
	\mathbf{K}_* &= \kappa(\mathbf{X},\mathbf{X}_*)\\
	\mathbf{K}_{**} &= \kappa(\mathbf{X}_*,\mathbf{X}_*).
\end{align*}
The mean is assumed to be $(m(\mathbf{X}), m(\mathbf{X}_*))=0$. 

While this equation describes the joint probability distribution $P(\rvf, \rvf_*|\mathbf{X}, \mathbf{X}_*)$ over $\rvf$ and $\rvf_*$ in regressions, we need the conditional distribution $P(\rvf_*|\rvf, \mathbf{X}, \mathbf{X}_*)$ from the joint distribution. The conditional distribution can be achieved by the Marginal and conditional distributions of MVN theorem:

\begin{theorem} Marginals and conditionals of an MVN:
A random vector \(\mathbf{X}\) follows a multivariate normal distribution:
\[
\mathbf{X} \sim \mathcal{N}(\mathbf{\mu}, \mathbf{\Sigma})
\]
where \(\mathbf{X}\) is partitioned as:
\[
\mathbf{X} = \begin{bmatrix}
\mathbf{X}_1 \\
\mathbf{X}_2
\end{bmatrix}
\]
with corresponding partitions of the mean vector \(\mathbf{\mu}\) and the covariance matrix \(\mathbf{\Sigma}\):
\[
\mathbf{\mu} = \begin{bmatrix}
\mathbf{\mu}_1 \\
\mathbf{\mu}_2
\end{bmatrix}, \quad
\mathbf{\Sigma} = \begin{bmatrix}
\mathbf{\Sigma}_{11} & \mathbf{\Sigma}_{12} \\
\mathbf{\Sigma}_{21} & \mathbf{\Sigma}_{22}
\end{bmatrix}
\]

Marginal Distributions:
The marginal distribution of \(\mathbf{X}_1\) is:
\[
\mathbf{X}_1 \sim \mathcal{N}(\mathbf{\mu}_1, \mathbf{\Sigma}_{11})
\]
The marginal distribution of \(\mathbf{X}_2\) is:
\[
\mathbf{X}_2 \sim \mathcal{N}(\mathbf{\mu}_2, \mathbf{\Sigma}_{22})
\]

Conditional Distributions:
The conditional distribution of \(\mathbf{X}_1\) given \(\mathbf{X}_2 = \mathbf{x}_2\) is:
\[
\mathbf{X}_1 | \mathbf{X}_2 = \mathbf{x}_2 \sim \mathcal{N}(\mathbf{\mu}_{1|2}, \mathbf{\Sigma}_{1|2})
\]
where the conditional mean \(\mathbf{\mu}_{1|2}\) and the conditional covariance \(\mathbf{\Sigma}_{1|2}\) are given by:
\[
\mathbf{\mu}_{1|2} = \mathbf{\mu}_1 + \mathbf{\Sigma}_{12}\mathbf{\Sigma}_{22}^{-1} (\mathbf{x}_2 - \mathbf{\mu}_2)
\]
\[
\mathbf{\Sigma}_{1|2} = \mathbf{\Sigma}_{11} - \mathbf{\Sigma}_{12}\mathbf{\Sigma}_{22}^{-1}\mathbf{\Sigma}_{21}
\]

\end{theorem}

By using the theorem, we get
$$\rvf_*|\rvf, \mathbf{X}, \mathbf{X}_*\sim \mathcal{N}(\mathbf{K}_*^T\mathbf{K}\rvf, \mathbf{K}_{**}-\mathbf{K}_*^T\mathbf{K}^{-1}\mathbf{K}_{*})$$

In practice, we typically have noisy estimations of target functions, $y = f(x)+\epsilon$. By assuming the i.i.d., Gaussian noise with variance $\sigma_n^2$, the prior on these noisy observations then becomes $cov(y) = \mathbf{K}+\sigma_n^2\mathbf{I}$. The joint distribution of the observed target values and the function values at the test locations under the prior as 
\begin{align*}
	\begin{bmatrix}
		\rvy\\
		\rvf_*
	\end{bmatrix} \sim \mathcal{N}
	\begin{bmatrix}
		\mathbf{0}, \begin{bmatrix}
			\mathbf{K}+\sigma_n^2\mathbf{I} & \mathbf{K}_*\\
			\mathbf{K}_*^T & \mathbf{K}_{**}
		\end{bmatrix}
	\end{bmatrix}
\end{align*}
By deriving the condional distribution, we get the predictive equations for Gaussian process regression:






