\renewcommand{\thesection}{\Alph{section}.\arabic{section}}
\setcounter{section}{0}

\begin{appendices}
\chapter{Appendix}

\section{Vector Notations}

\begin{itemize}
	\item $\rvx = (x_1, x_2,..., x_n)$ or
	\item 
		\begin{align*}
			\rvx = \begin{bmatrix}x_1\\ \vdots\\ x_n\end{bmatrix}
		\end{align*}
\end{itemize}

\subsection{Differentiate}

Differentiate a vector $y$ by a scalar $\rvx$:
\begin{align*}
	\frac{\partial \rvy}{\partial x} = \begin{bmatrix}\frac{\partial y_1}{\partial x}\\ \vdots\\ \frac{\partial y_n}{\partial x}\end{bmatrix}
\end{align*}

Differentiate a scalar $y$ by a vector $\rvx$:
\begin{align*}
	\frac{\partial y}{\partial \rvx} = \bigg[\frac{\partial y}{\partial x_1}, \cdots, \frac{\partial y}{\partial x_n}\bigg]
\end{align*}

Differentiate a vector $y$ by a vector $\rvx$:
\begin{align*}
	\frac{\partial \rvy}{\partial \rvx} = \begin{bmatrix}
		\frac{\partial y_1}{\partial x_1} & \cdots & \frac{\partial y_1}{\partial x_n}\\ 
		\vdots & \ddots & \vdots\\ 
		\frac{\partial y_n}{\partial x_1} & \cdots & \frac{\partial y_n}{\partial x_n}
\end{bmatrix}
\end{align*}

$\rva^T\rvx$ is a scalar value, so 

\begin{align*}
	\frac{\partial \rva^T\rvx}{\partial \rvx} &= \bigg[\frac{\partial (\rva^T\rvx)}{\partial x_1} \cdots \frac{\partial (\rva^T\rvx)}{\partial x_n}\bigg] = \bigg[\frac{\partial (a_1x_1+\dots+a_nx_n)}{\partial x_1}, \cdots, \frac{\partial (a_1x_1+\dots+a_nx_n)}{\partial x_n}\bigg]\\ 
	&= [a_1, \dots, a_n] = \rva^T
\end{align*}

\begin{align*}
	A\rvx = \begin{bmatrix}
		a_{11} & \dots & a_{1n}\\
		\vdots & \ddots & \vdots\\
		a_{m1} & \dots & a_{mn}\\
	\end{bmatrix}\times
	\begin{bmatrix}
		x_1\\
		\vdots\\
		x_n
		\end{bmatrix} = \begin{bmatrix}
		\sum_{i=1}^n a_{1i}x_i\\
		\vdots\\
		\sum_{i=1}^n a_{mi}x_i\\
	\end{bmatrix}
\end{align*}
Thus, 
\begin{align*}
	\frac{\partial A\rvx}{\partial \rvx} =\begin{bmatrix}
		\frac{\partial \sum_{i=1}^n a_{1i}x_i}{\partial x_1}& \dots & \frac{\partial \sum_{i=1}^n a_{1i}x_i}{\partial x_n}\\
		\vdots & \ddots & \vdots\\
		\frac{\partial \sum_{i=1}^n a_{mi}x_i}{\partial x_1} & \dots & \frac{\partial \sum_{i=1}^n a_{mi}x_i}{\partial x_n}
	\end{bmatrix} = A
\end{align*}

\section{KL Divergence between Two Normal Distribution}
\begin{align*}
D_\text{KL}(P||Q) & = \mathbb{E}_P\Big[\textrm{log}\frac{P}{Q}\Big]\\
\end{align*}
Consider two multivariate Gaussians in $\mathbb{R}^n$, $P_1$ and $P_2$

\begin{align*}
D_\text{KL}(P||Q) &= \int \left[ \frac{1}{2} \log\frac{|\Sigma_2|}{|\Sigma_1|} - \frac{1}{2} (x-\mu_1)^T\Sigma_1^{-1}(x-\mu_1) + \frac{1}{2} (x-\mu_2)^T\Sigma_2^{-1}(x-\mu_2) \right] \times p(x) dx \\
&= \frac{1}{2} \log\frac{|\Sigma_2|}{|\Sigma_1|} - \frac{1}{2} \text{tr} \left\{E[(x - \mu_1)(x - \mu_1)^T] \ \Sigma_1^{-1} \right\} + \frac{1}{2} E[(x - \mu_2)^T \Sigma_2^{-1} (x - \mu_2)] \\
&= \frac{1}{2} \log\frac{|\Sigma_2|}{|\Sigma_1|} - \frac{1}{2} \text{tr}\ \{I_n \} + \frac{1}{2} (\mu_1 - \mu_2)^T \Sigma_2^{-1} (\mu_1 - \mu_2) + \frac{1}{2} \text{tr} \{ \Sigma_2^{-1} \Sigma_1 \} \\
&= \frac{1}{2}\left[\log\frac{|\Sigma_2|}{|\Sigma_1|} - n + \text{tr} \{ \Sigma_2^{-1}\Sigma_1 \} + (\mu_2 - \mu_1)^T \Sigma_2^{-1}(\mu_2 - \mu_1)\right]
\end{align*}

Trace tricks:
$$x^TAx = tr[x^TAx] = tr[xx^TA]$$
$$tr[A+B] = tr[A]+tr[B]$$
$$E[(x-\mu)^T \Sigma^{-1} (x-\mu)]= tr(E[(x-\mu)(x-\mu)^T] \Sigma^{-1})$$
$$\Sigma = E[(X-\mu)(X-\mu)^T]=E[XX^T]-\mu\mu^T$$
$$E[XX^T] = \Sigma + \mu\mu^T$$

Note that the determinant of a diagonal matrix could be computed as product of its diagonal.



\section{Various Tricks}

\subsection{Spectral Normalization}

A persisting challenge in the training of GANs is the performance control of the discriminator. The derivative of discriminator could be unbounded and even incomputable, so they introduced a regularization on the derivative of discriminator called, Lipchitz continuity, which bound the gradient.

Neural network is actually a composite function. So if we make each function to satisfy the Lipchitz continuity, then we can make whole network satisfy it. Lipchitz continuity of a linear operator can be seen as 

\begin{align*}
	||f(x_1)-f(x_2)||_2 &\leq L||x_1-x_2||_2\\
	||Ax_1-Ax_2||_2 &\leq L||x_1-x_2||_2\\
	\frac{||Ax||_2}{||x||_2}&\leq L\\
	\sigma_{max}\underbrace{\sup_x \frac{||Ax||_2}{||x||_2}}_{Spectral Norm}&\leq L, \quad \textrm{Since inequaility holds for all }x
\end{align*}
, where $\sigma_{max}$ is the maximum singular value. Note that the spectral norm is from the linear algebra. We can make the matrix $A$ Lipchitz continuous by 
\begin{align*}
1 = \underbrace{\sup_x \frac{||\frac{A}{\sigma_{max}}x||_2}{||x||_2}}_{Spectral Norm}\leq L
\end{align*}


\subsection{Moving Averaging}

\subsection{Weight Averaging}

\subsection{Quality Measurements}

\section{f-Divergence}
In probability theory, an $f$-divergence is a function $D_{f}(p||q)$ that measures the difference between two probability distributions $p$ and $q$. It helps the intuition to think of the divergence as an average, weighted by the function $f$, of the odds ratio given by $p$ and $q$.

For distributions $p$ and $q$, f-divergence is defined as:
$$D_{f}(p||q) = \int_{\mathcal{X}}f\Bigg(\frac{p(x)}{q(x)}\Bigg)q(x)dx$$

\begin{itemize}
	\item KL-divergence: $f(t) = t\log t$
	\item Reversed KL-divergence: $f(t) = -\log t$
	\item Total variation: $f(t) = \frac{1}{2}|t-1|$
	$$D_{f}(p||q) = \int_{\mathcal{X}}|p(x)-q(x)|dx$$
\end{itemize}

\section{Lipchitz Continuous}
The function $f$ in the new form of Wasserstein metric is demanded to satisfy $\| f \|_L \leq K$, meaning it should be $K$-Lipschitz continuous. \citep{Lil2017}

A real-valued function $f: \mathbb{R} \rightarrow \mathbb{R}$ is called $K$-Lipschitz continuous if there exists a real constant $K\geq 0$ such that, for all $x_1, x_2 \in \mathbb{R}$
$$\lvert f(x_1) - f(x_2) \rvert \leq K \lvert x_1 - x_2 \rvert$$

\section{Singular Value}
All singular values can be calculated via the singular value decomposition (SVD)
$$A = U\Sigma V^T$$
, where $U$ is the left singular vectors and $V$ is the right singular vectors. However, if we just want the maximum singular value then we just need to find corresponding vectors
$$\sigma = uAv^T$$
Actually, there is a simpler way to find the maximum singular value e.g., power iteration. 



\end{appendices}
