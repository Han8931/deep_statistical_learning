\section{Evaluation: Forward-Backward Probability}
% The forward–backward algorithm is an inference algorithm for hidden Markov models which computes the posterior marginals of all hidden state variables given a sequence of observations/emissions

% The term forward–backward algorithm is also used to refer to any algorithm belonging to the general class of algorithms that operate on sequence models in a forward–backward manner.

% In the first pass, the forward–backward algorithm computes a set of forward probabilities which provide, for all $t\in \{1,\dots ,T\}$, the probability of ending up in any particular state given the first $t$ observations in the sequence, i.e. $P(X_{t}\ |\ o_{1:t})$. In the second pass, the algorithm computes a set of backward probabilities which provide the probability of observing the remaining observations given any starting point $t$, i.e. $P(o_{t+1:T}\ |\ X_{t})$. These two sets of probability distributions can then be combined to obtain the distribution over states at any specific point in time given the entire observation sequence:

% These two sets of probability distributions can then be combined to obtain the distribution over states at any specific point in time given the entire observation sequence:
% $$P(X_{t}\ |\ o_{1:T})=P(X_{t}\ |\ o_{1:t},o_{t+1:T})\propto P(o_{t+1:T}\ |\ X_{t})P(X_{t}|o_{1:t})$$
% The forward–backward algorithm can be used to find the most likely state for any point in time. However, It cannot be used to find the most likely sequence of states.

\subsection{Joint Probability}
We can factorize the joint distribution of HMM in \Cref{fig:HMM} by using a Bayesian approach as follows:. 

\begin{align}
	p(X,Z) = p(x_1,\dots,x_t, z_1,\dots,z_t) = p(z_1)p(x_1|z_1),p(z_2|z_1),\dots,p(x_{t}|z_{t}),p(z_{t}|z_{t-1})
	\label{eq:hmm_joint}
\end{align}

As the number of latent factor increases, it is getting harder to decode the latent factors. 

\subsection{Marginal Probability}
% \subsection{Forward Probability}
We want to compute the likelihood of sequence $X$ which is given by
$$p(X|\boldsymbol{\pi}\mathbf{, a, b}) = \sum_Z p(X, Z|\boldsymbol{\pi}\mathbf{, a, b})$$
The computation can be done as follows:
\begin{align*}
	p(X) &= \sum_Z p(X,Z)\\
	& = \sum_{z_1}\dots\sum_{z_t}p(x_1,\dots,x_t,z_1,\dots,z_t)\\
	& = \sum_{z_1}\dots\sum_{z_t}\pi_{z_{1}}\prod_{t=2}^{T}a_{z_{t-1},z_t}\prod_{t=1}^{T}b_{z_{t},x_t}
\end{align*}
The last step is done by using \Cref{eq:hmm_joint}). The computation of this equation requires lots of computations, so we will change it into a \textbf{recursive form} by using the factorization rule $p(a,b,c) = p(a)p(b|a)p(c|a,b)$. 

\begin{align}
	p(&x_1,\dots,x_t,z_t^k=1) = \sum_{z_{t-1}}p(x_1,\dots,x_{t-1}, x_t,z_{t-1},z_t^k=1)\\
	&= \sum_{z_{t-1}} p(\underbrace{x_1,\dots,x_{t-1}, z_{t-1}}_{a}, \underbrace{x_t}_{c}, \underbrace{z_t^k=1}_{b})\\
	& = \sum_{z_{t-1}} p(x_1,\dots,x_{t-1},z_{t-1}) p(z_t^k=1|x_1,\dots,x_{t-1},z_{t-1})p(x_t|z_t^k=1, x_1,\dots,x_{t-1},z_{t-1}) \\
	&\hspace{0.5cm} \because p(a,b,c) = p(a)p(b|a)p(c|a,b) \textrm{ or by the structure of HMM}\nonumber\\ 
	& = \sum_{z_{t-1}} p(x_1,\dots,x_{t-1},z_{t-1}) p(z_t^k=1|z_{t-1}) p(x_t|z_t^k=1)\\
	& = p(x_t|z_t^k=1) \sum_{z_{t-1}} p(x_1,\dots,x_{t-1},z_{t-1}) p(z_t^k=1|z_{t-1}) \\
	& = b_{z^k_t,x_t} \sum_{z_{t-1}} p(x_1,\dots,x_{t-1},z_{t-1}) a_{z_{t-1},z_t^k}
	\label{eq:hmm_eval_fact}
\end{align}

\begin{itemize}
	\item In the second line, the $x_{t-1}$ and $z_{t-1}$ are grouped together. 
	\item Then, we can find the HMM structure by factorizing the equation. 
	\item In the fourth line, $x$ terms are removed, since $z_t$ only relies on $z_{t-1}$ by the Markov assumption. Similarly, $x_t$ only depends on $z_t$. We can interpret this by using Bayes ball too. 
\end{itemize}
% In the fifth step, we assume that $z_t=k$ is given, thus by Markov assumption, we only need $z_{t-1}$. 
Now we can find a recursive structure of $p(x_1,\dots,x_{t},z_{t}^k=1)$ as follows:
$$\alpha_t^k = p(x_1,\dots,x_{t},z_{t}^k=1) = b_{k,x_t}\sum_i \alpha_{t-1}^ia_{i,k}$$
, where \textbf{$\alpha_t^k$ is the probabilities of being in state $k$ after observing the first $t$ observations.} Thus, 
\begin{align*}
	p(x_1,\dots,x_{t}) & = \sum_{\mathbf{z}} p(x_1,\dots,x_{t},z)\\
	& = \sum_{k} \alpha_t^k
\end{align*}
% \begin{itemize}
% 	\item $\alpha_t^k$: \textbf{Forward probability}. Probabilities of being in state $k$ after observing the first $t$ observations.
% 	% \item $a_{i,k}$: transition probability
% 	% \item $b_{k,x_t}$: observation (or emission) probability
% \end{itemize}
Note that $\alpha_t^k$ is also called \textbf{Forward probability}.

\subsection{Forward Algorithm}
Forward probability solves the evaluation problem. Essentially, this is a dynamic programming, so it calculates required values in a bottom-up manner. 
\begin{itemize}
	\item Forward probability: $\alpha_t^k$, $Time\times States$
\end{itemize}
%\LinesNumbered
\begin{algorithm}
	Create a probability matrix $forward[M,T] = \alpha_t^k$\\
	Initialization: \\
	\For {\textrm{each state} k=1,...,M}{
		$\alpha_1^k\leftarrow \pi_kb_{k,x_1}$
	}
	\For {\textrm{time step} t=2,...,T}{
		\For {\textrm{each step} k=1,...,M}{
			$\alpha_t^k = b_{k,x_t}\sum_i \alpha_{t-1}^ia_{i,k}$
			}
		
	}
	Return $p(X) = \sum_i^M \alpha_T^i$
	\caption{Forward Algorithm}
	\label{algo:forward_algorithm}
\end{algorithm}
%\begin{algorithm}
%	Init: $\alpha_1^k = b_{k,x_1}\pi_k$\\
%	\For{t=1,...,T}{
%		$\alpha_t^k = b_{k,x_t}\sum_i \alpha_{t-1}^ia_{i,k}$
%	}
%	Return $p(X) = \sum_i\alpha_T^i$
%	\caption{Forward Algorithm}
%	\label{algo:forward_algorithm}
%\end{algorithm}
Note again that 
$$p(X) = p(x_1,...,x_T) =\sum_i\alpha_T^i = \sum_i p(x_1,...,x_T, z_T^i=1)$$
Note also that the forward-algorithm returns $p(X)$ and forward probability is the probability of being in state $k$ after observing the first $t$ observations without $Z$. 

\subsection{Backward Probability}
The forward probability only considers an observation at $t$. To determine the $z_t$, we need to leverage the future observations. \textbf{The backward probability $\beta$ is the probability of seeing the observations from time $t+i$ to the end, given that we are in state $k$ at time $t$.} 
$$\beta_t^k = p(x_{t+1},\dots,x_T|z_t^k=1)$$
We want to compute $p(z_t^k=1|X)$ rather than $p(x_1,\dots,x_t, z_t^k=1)$. In other words, we will leverage the whole observations $X$. 
\begin{align*}
	p(z_t^k=1,X) &= p(x_1,\dots,x_t, z_t^k=1, x_{t+1},\dots,x_T)\\
	& = p(x_1,\dots,x_t, z_t^k=1)p(x_{t+1},\dots,x_T|x_1,\dots,x_t, z_t^k=1)\\
	& = p(x_1,\dots,x_t, z_t^k=1)p(x_{t+1},\dots,x_T|z_t^k=1)\\
	& = \alpha_{t}^k\beta_{t}^k
\end{align*}
We already know that $p(x_1,\dots,x_t, z_t^k=1) = \alpha_t^k$. We just need to compute backward probability as follows:
\begin{align*}
	\beta_t^k &= p(x_{t+1},\dots,x_T|z_t^k=1)\\
	& = \sum_{z_{t+1}}p(\underbrace{z_{t+1}}_{a}, \underbrace{x_{t+1}}_b,\underbrace{x_{t+2},\dots,x_T}_c|z_t^k=1)\\
	& = \sum_{i} p(z_{t+1}^i=1|z_t^k=1)p(x_{t+1}|z_{t+1}^i=1,z_t^k=1)p(x_{t+2},\dots,x_T|x_{t+1},z_{t+1}^i=1,z_t^k=1)\\
	& \because p(a,b,c) = p(a)p(b|a)p(c|a,b)\\
	& = \sum_{i} p(z_{t+1}^i=1|z_t^k=1)p(x_{t+1}|z_{t+1}^i=1)p(x_{t+2},\dots,x_T|z_{t+1}^i=1)\\
	& = \sum_{i}a_{k,i}b_{i,x_{t+1}} \beta_{t+1}^i
\end{align*}

Another recursive structure:
\begin{align*}
	p(z_t^k=1,X) &= \alpha_{t}^k\beta_{t}^k\\
	& = b_{k,x_t}\sum_i \alpha_{t-1}^ia_{i,k} \times \sum_{i}a_{k,i}b_{i,x_{t}} \beta_{t+1}^i
\end{align*}
This means at time $t$, the latent label is belong to some class $k$ and this can be computed by using the forward probability and the backward probability. Now we can compute
\begin{align*}
p(z_t^k=1|X) &= \frac{p(z_t^k=1,X)}{p(X)} = \frac{\alpha_{t}^k\beta_{t}^k}{p(X)}
\end{align*}
Then, 
$$k_t = \argmax_{k}p(z_t^k=1|X)$$
Note that this is for a single latent variable at a single time step given the whole observation $X$, but we want to decode a sequence of latent variables. Thus, we need some decoding algorithm.
