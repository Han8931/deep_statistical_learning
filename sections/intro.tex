\chapter{Introduction}
\section{Probability}



\begin{definition}{Independence}
	\begin{align*}
		X\perp Y \leftrightarrow p(X,Y)=p(X)p(Y)
	\end{align*}
\end{definition}
\begin{definition}{Conditional independence}
	\begin{align*}
		X\perp Y|Z \leftrightarrow p(X,Y|Z)=p(X|Z)p(Y|Z)
	\end{align*}
\end{definition}
All the dependencies between $X$ and $Y$ are mediated via $Z$. If $X$ and $Y$ are conditionally independent, then 
\begin{align*}
	p(X|Y,Z)&=\frac{p(X,Y|Z)}{p(Y|Z)}\\
	&=\frac{p(X|Z)p(Y|Z)}{p(Y|Z)}\\
	&=p(X|Z).
\end{align*}

\section{Transformations of Random Variable}
\subsection{Formal Definition}
Suppose $X$ is a continuous random variable with pdf $f(x)$. If we define $Y=g(X)$, where $g(\cdot)$ is a monotonically increasing function, then the pdf of $Y$ can be obtained as follows:
\begin{align*}
	p(Y\leq y) &= p(g(X)\leq y)\\
	& = p(X\leq g^{-1}(y))
\end{align*}
This can be re-written as by definition
\begin{align*}
F_Y(y) = F_X(g^{-1}(y))
\end{align*}
By differentiating the CDFs on both sides w.r.t. $y$, we can get the pdf of $Y$. If the function $g(\cdot)$ is monotonically increasing, then the pdf of $Y$ is given by
$$f_Y(y) = f_X(g^{-1}(y))\frac{d}{dy}g^{-1}(y)$$
On the other hand, if it is monotonically decreasing, then the pdf of $Y$ is given by
$$f_Y(y) = - f_X(g^{-1}(y))\frac{d}{dy}g^{-1}(y)$$
Compactly, the above two equations can be combined into a following equation:
$$f_Y(y) = f_X(g^{-1}(y))\Bigg|\frac{d}{dy}g^{-1}(y)\Bigg|$$

\subsection{Intuition}
Given a random variable $z$ and its known probability density function $z\sim \pi(z)$, we would like to construct a new random variable using a one-to-one mapping function $x=f(z)$. The function $f$ is invertible, so $z = f^{-1}(x)$. Now the question is how to infer the unknown probability density function of the new variable, $p(x)$?

$$
\begin{aligned}
& \int p(x)dx = \int \pi(z)dz = 1 \scriptstyle{\text{   ; Definition of probability distribution.}}\\
& p(x) = \pi(z) \left\vert\frac{dz}{dx}\right\vert = \pi(f^{-1}(x)) \left\vert\frac{d f^{-1}(x)}{dx}\right\vert = \pi(f^{-1}(x)) \vert (f^{-1})'(x) \vert
\end{aligned}$$

In multivariate case, 

\begin{align}
\mathbf{z} &\sim \pi(\mathbf{z}), \mathbf{x} = f(\mathbf{z}), \mathbf{z} = f^{-1}(\mathbf{x}) \\
p(\mathbf{x}) 
&= \pi(\mathbf{z}) \left\vert \det \dfrac{d \mathbf{z}}{d \mathbf{x}} \right\vert  
= \pi(f^{-1}(\mathbf{x})) \left\vert \det \dfrac{d f^{-1}}{d \mathbf{x}} \right\vert
\end{align}



\section{Gaussian Distribution}
For a $D$-dimensional vector $\rvx$, the multivariate Gaussian distribution takes the form
\begin{align}
	\mathcal{N}(\mathbf{x}|\boldsymbol{\mu},\boldsymbol{\Sigma}) &= \frac{1}{(2\pi)^{D/2}|\boldsymbol{\Sigma}|^{1/2}}\exp\bigg(-\frac{1}{2}(\mathbf{x}-\boldsymbol{\mu})^T\boldsymbol{\Sigma}^{-1}(\mathbf{x}-\boldsymbol{\mu})\bigg)\\
	\label{eq:normal_distribution}
\end{align}

\subsection{Conditional Gaussian Distribution}
Consider first the case of conditional distributions. Suppose $\rvx$ is a $D$-dimensional vector with Gaussian distribution $\mathcal{N}(\mathbf{x}|\boldsymbol{\mu},\boldsymbol{\Sigma})$ and that we partition $\rvx$ into two disjoint subsets $\rvx_a$ and $\rvx_b$. Thus, $\rvx_a$ has $M$ components and $\rvx_b$ has $D-M$ components.
\begin{align*}
	\rvx = \begin{bmatrix}
		\rvx_a\\
		\rvx_b
	\end{bmatrix}.
\end{align*}
Similarly, 
\begin{align*}
	\boldsymbol{\mu} = \begin{bmatrix}
		\boldsymbol{\mu}_a\\
		\boldsymbol{\mu}_b
	\end{bmatrix} 
\end{align*}
and the covariance matrix is given by
\begin{align*}
	\boldsymbol{\Sigma} = \begin{bmatrix}
		\boldsymbol{\Sigma}_{aa} & \boldsymbol{\Sigma}_{ab}\\
		\boldsymbol{\Sigma}_{ba} & \boldsymbol{\Sigma}_{bb} 
	\end{bmatrix} 
\end{align*}
Note that the symmetry $\mSigma^T = \mSigma$ implies that $\mSigma_{ab}^T=\mSigma_{ba}$. We can also define a \textit{precision matrix} as follows: 
$$\mLambda \equiv \mSigma^{-1}$$
We also introduce a partitioned form of the precision matrix:
\begin{align}
	\boldsymbol{\Lambda} = \begin{bmatrix}
		\mLambda_{aa} & \mLambda_{ab}\\
		\mLambda_{ba} & \mLambda_{bb} 
	\end{bmatrix} 
	\label{eq:precision_matrix}
\end{align}
Because the inverse of a symmetric matrix is also symmetric, we see that $\mLambda_{aa}$ and $\mLambda_{bb}$ are symmetric and $\mLambda_{ab}^T =\mLambda_{ba}$. Note that, for instance, $\mLambda_{aa}$ is not simply given by the inverse of $\boldsymbol{\Sigma}_{aa}$. 

Now let's compute the conditional probability:
\begin{align*}
	-\frac{1}{2}(\mathbf{x}-\boldsymbol{\mu})^T\boldsymbol{\Sigma}^{-1}(\mathbf{x}-\boldsymbol{\mu}) &= \frac{1}{2}\Bigg(\bigg(\begin{bmatrix}
		\rvx_a\\
		\rvx_b
	\end{bmatrix}-
	\begin{bmatrix}
		\boldsymbol{\mu}_a\\
		\boldsymbol{\mu}_b
	\end{bmatrix} 
\bigg)^T\begin{bmatrix}
		\mLambda_{aa} & \mLambda_{ab}\\
		\mLambda_{ba} & \mLambda_{bb} 
	\end{bmatrix}\bigg(\begin{bmatrix}
		\rvx_a\\
		\rvx_b
	\end{bmatrix}-
	\begin{bmatrix}
		\boldsymbol{\mu}_a\\
		\boldsymbol{\mu}_b
	\end{bmatrix}\bigg)\Bigg)\\
	&= -\frac{1}{2}\big((\rvx_a-\vmu_a)^T\mLambda_{aa}(\rvx_a-\vmu_a)+(\rvx_a-\vmu_a)^T\mLambda_{ab}(\rvx_a-\vmu_a)\\
	&\quad+(\rvx_a-\vmu_a)^T\mLambda_{ba}(\rvx_a-\vmu_a)+(\rvx_a-\vmu_a)^T\mLambda_{bb}(\rvx_a-\vmu_a)\big)\\
\end{align*}



% \chapter{Bayesian}
\section{Naive Bayes}
\label{sec:naive_bayes}

Let's say we have a prediction model and it gives us a prediction. We want to measure the precision of the prediction. For example, the prediction has 90\% chance of being a correct prediction. This can be modeled probabilistically. Given data $\rvx$, we want to know the probability of it being $y$. Similarly, if we have $N$ data, then with an i.i.d., assumption,
$$\prod_{i=1}^N p(y_i|\mathbf{x}_i).$$

What we want to is to build such a probabilistic model parameterized by some parameters. 

In this section, we discuss how to classify vectors of discrete-valued features $\mathbf{x}$. Recall that we discussed how to classify a feature vector $\mathbf{x}$ by applying Bayes rule to a generative classifier of the form 
  $$p(y=c|\mathbf{x},\boldsymbol{\theta})\propto p(\mathbf{x}|y=c, \boldsymbol{\theta})p(y=c|\boldsymbol{\theta})$$
  The key to using such models is specifying a suitable form for the class-conditional density $p(\mathbf{x}|y=c, \boldsymbol{\theta})$, which defines what kind of data we expect to see in each class. 
  \begin{itemize}
    \item  $\textbf{x} \in \{1,...,K\}^D$,
    \begin{itemize}
      \item $K$: the number of values for each feature.
      \item $D$: the number of features.
    \end{itemize}
    \item We will use a generative approach.
    \item Need to specify the class conditional distribution, $p(\mathbf{x}|y=c)$.
    \item A simple approach is to assume the features are \textbf{conditionally independence} given the class label.
		\begin{figure}[h]
			\centering
			\includegraphics[scale=0.5]{./images/conditional_independence.pdf}
		\end{figure}
    \item This allows us to write the class conditional density as a product of one dimensional densities:
    $$p(\mathbf{x}|y=c, \boldsymbol{\theta}) = \prod_{j=1}^{D}p(x_j|y=c,\boldsymbol{\theta}_{jc})$$
  \end{itemize}
  The resulting model is called a \textbf{naive Bayes classifier (NBC)}. The model is called ``naive'' since we assume the independence between the features, which is not true in practice. However, if often results in classifiers that work well.

  The form of the class-conditional density depends on the type of each feature. We give some possibilities below:
  \begin{itemize}
    \item In the case of real-valued features, we can use the Gaussian distribution: $p(\mathbf{x}|y=c, \boldsymbol{\theta}) = \prod_{j=1}^{D}\mathcal{N}(x_j|\mu_{jc}^2)$, where $\mu_{jc}$ is the mean of feature $j$ in objects of class $c$, and $\sigma_{jc}^2$ is its variance.
    \item In the case of binary features, we can use the Bernoulli distribution: $p(\mathbf{x}|y=c, \boldsymbol{\theta}) = \prod_{j=1}^{D}\textrm{Ber}(x_j|\mu_{jc})$, where $\mu_{jc}$ is the probability that feature $j$ occurs in class $c$. This is sometimes called the \textbf{multivariate Bernoulli naive Bayes} model.
    \item In the case of categorical features, $x_j\in \{1,...,K\}$, we can model the multinomial distribution: $p(\mathbf{x}|y=c, \boldsymbol{\theta}) = \prod_{j=1}^{D}\textrm{Cat}(x_j|\mu_{jc})$, where $\boldsymbol{\mu}_{jc}$ is a histogram over the $K$ possible values for $x_j$ in class $c$.
  \end{itemize}

The probability for a single data case is given by
$$p(\mathbf{x}_i,y_i|\boldsymbol{\theta}) = p(y_i|\boldsymbol{\pi})\prod_{j}p(x_{ij}|\boldsymbol{\theta}_j)=\prod_{c}\pi_{c}^{\mathds{I}(y_i=c)}\prod_{j}\prod_{c}p(x_{ij}|\boldsymbol{\theta}_{jc})^{\mathds{I}(y_i=c)},$$
where $\boldsymbol{\pi}$ is a vector of class probability. Hence the log-likelihood is given by
	$$\textrm{log}p(\mathcal{D}|\boldsymbol{\theta}) = \sum_{c=1}^{C}N_c\textrm{log}\pi_c+\sum_{j=1}^{D}\sum_{c=1}^{C}\sum_{i:y_i=c}\textrm{log}p(x_{ij}|\boldsymbol{\theta}_{jc})$$

	\begin{algorithm}[H]
		\SetAlgoLined
%		\KwResult{Write here the result }
		Initialize $N_c=0,N_{jc}=0$ \;
		\For{$i=1:N$}{
      $c=y_i$ //Class label of $i$-th example;

      $N_c:=N_c+1$;

  		\For{$j=1:D$}{
      \uIf{$x_{ij}=1$}{
        $N_{jc}:=N_{jc}+1$
        }
  		}
		}
  $\hat{\pi}=\frac{N_c}{N},\hat{\theta}_{jc}=\frac{N_{jc}}{N_c}$
	\caption{Fitting a naive Bayes classifier to binary features}
\end{algorithm}

\section{Logistic Regression}
\label{sec:logistic_regression}

Logistic regression corresponds to the following binary classification model parameterized by $\rvw$:
$$p(y|\mathbf{x},\mathbf{w})=\textrm{Ber}(y|\sigma(\mathbf{w}^T\mathbf{x}))$$

Logistic regression models \textit{logit}s (log odds) through a linear model. For binary data, the goal is to model the probability $p$ that one of two outcomes occurs. Recall that an ordinary linear regression model is not bounded. Thus, we will pass a linear model through a sigmoid function, which is also known as logistic function. 

$$\sigma(z) = \frac{1}{1+\exp^{z}},$$
where $z=wx+b.$
The sigmoid function has the property
$$\sigma(-x) = 1-\sigma(x).$$
The $z$ is often called the logit. Note that the inverse of the sigmoid is the log of the odds ratio $\frac{p}{1-p}.$

The logit function is $\textrm{log}\frac{p}{1-p}$, which varies between $-\infty$ and $+\infty$ as $p$ varies between $0$ and $1$.
$$\textrm{log}\frac{p}{1-p} = w_0x_0 +  w_1x_1 + ... + w_nx_n$$
Note that \textbf{the logistic regression model assumes that the log-odds (\textit{logit}) of an observation $y$ can be expressed as a linear function}. In this context, the logit function is called the \textbf{\textit{link function}} because it ``links'' the probability to the linear function of the predictor variables.

% Simplest solution to model a dependant variable $y$ is a linear regression. However, $y$ should be in a range of $[0,1]$. So we need to introduce the logit function. 

% The linear regression can be generalized to the classification setting with two changes:
% \begin{itemize}
% 	\item Replacing the Gaussian distribution for $y$ with a Bernoulli distribution: $p(y|\mathbf{x},\mathbf{w})=Ber(y|\mu(\mathbf{x}))$
% 	\item Squashing input data into sigmoid function $\sigma(\eta)$ that range from 0 to 1: $\sigma(\eta)\triangleq \frac{1}{1+exp(-\eta)}$.
% \end{itemize}
% $$p(y|\mathbf{x},\mathbf{w})=Ber(y|\sigma(\mathbf{w}^T\mathbf{x})),\$$

The negative log-likelihood for logistic regression is given by
\begin{align*}
	\textrm{NLL}(\mathbf{w}) &= -\ln \prod_{i=1}^N p(\rvx)^{\mathds{I}(y_i=1)}(1-p(\rvx))^{\mathds{I}(y_i=0)}\,\footnotemark\\
							 &= -\ln \prod_{i=1}^N \sigma(\mathbf{w}^T\rvx)^{\mathds{I}(y_i=1)}(1-\sigma(\mathbf{w}^T\rvx))^{\mathds{I}(y_i=0)}\\
							 &= -\sum_{i=1}^N y_i\ln\sigma(\mathbf{w}^T\rvx)+\ln(1-y_i)(1-\sigma(\mathbf{w}^T\rvx)).
							 % &= -\sum_{i=1}^{N}\textrm{log}[\mu_i^{\mathds{I}(y_i=1)}\times (1-\mu_i)^{\mathds{I}(y_i=0)}]\\
	% &=-\sum_{i=1}^{N}[y_i\textrm{log}\mu_i + (1-y_i) \textrm{log}(1-\mu_i)], \textrm{ }
\end{align*}
This is also called \textbf{cross-entropy} error function. 
\footnotetext{$\mathds{I}(y_i=1) = y_i$, because $y_i\in \{0, 1\}$ is a binary variable} 

To compute the derivative of NLL, we first need to know the following tricks:
\begin{itemize}
	\item The derivative of $\ln (x)$:
$$\frac{\partial }{\partial x}\ln (x) = \frac{1}{x}.$$
\item The derivative of the sigmoid is given by:
$$\frac{\partial \sigma(z)}{\partial x} = \sigma(x)(1-\sigma(x)).$$
\item Finally, the chain rule of derivative. Suppose we are computing the derivative of a composite function $f(x) = u(v(x))$. The derivative of $f(x)$ is the derivative of $u(x)$ with respect to $v(x)$ times the derivative of $v(x)$ with respect to $x$.
$$\frac{\partial f}{\partial x} = \frac{\partial u}{\partial v} \frac{\partial v}{\partial x}$$
\end{itemize}
The derivative of the loss function w.r.t., a single weight $w_j$ is given by
\begin{align*}
	\frac{\partial \mathcal{L}}{\partial w_j} &= \frac{\partial }{\partial w_j} -[y\ln \sigma(wx + b)+(1-y) \ln (1-\sigma(wx+b))]\\
											  &=  -[\frac{\partial }{\partial w_j}y\ln \sigma(wx + b)+\frac{\partial }{\partial w_j}(1-y) \ln (1-\sigma(wx+b))]\\
											  &= -\frac{y}{\sigma(wx + b)}\frac{\partial }{\partial w_j}\sigma(wx + b) - \frac{1-y}{1-\sigma(wx+b)} \frac{\partial }{\partial w_j}1-\sigma(wx+b)\\
											  &= -\bigg[\frac{y}{\sigma(wx + b)}-\frac{1-y}{1-\sigma(wx + b)}\bigg]\frac{\partial }{\partial w_j}\sigma(wx + b)\\
											  &= -\bigg[\frac{y-\sigma(wx + b)}{\sigma(wx + b)[1-\sigma(wx + b)]}\bigg]\sigma(wx + b)[1-\sigma(wx + b)]\frac{\partial \sigma(wx + b)}{\partial w_j}\\
											  &= -\bigg[\frac{y-\sigma(wx + b)}{\sigma(wx + b)[1-\sigma(wx + b)]}\bigg]\sigma(wx + b)[1-\sigma(wx + b)]x_j\\ 
											  &= -( y-\sigma(wx + b) )x_j\\
											  &= ( \sigma(wx + b)-y )x_j.
\end{align*}





Another way to express \textrm{NLL} is as follows. Suppose $\hat{y}_i\in\{-1,+1\}$ instead of $y_i\in\{0,1\}$. We have $p(y=1)=\frac{1}{1+\mathrm{exp}(-\mathbf{w}^T\mathbf{x})}$ and $p(y=-1)=\frac{1}{1+\mathrm{exp}(+\mathbf{w}^T\mathbf{x})}$. Hence
\begin{align*}
	\textrm{NLL}(\mathbf{w}) &= -\frac{1}{N}\sum_{n=1}^N [\mathbb{I}(\hat{y}_n=1)\log(\sigma(a_n))+\mathbb{I}(\hat{y}_n=-1)\log(\sigma(-a_n))]\\
							 &= -\frac{1}{N}\sum_{n=1}^N \log(\sigma(\hat{y}_na_n))\\
							 &=  \frac{1}{N}\sum_{i=1}^{N}\textrm{log}(1+\mathrm{exp}(-\hat{y}_i\mathbf{w}^T\mathbf{x}_i).
\end{align*}
Note that the sigmoid is used for compressing the output into $[0,1]$ and $\sigma(-a_n) = 1-\sigma(a_n)$. Unlike the linear regression, there is no closed from solution for logistic regression, thus we need optimization algorithms for it. Typically, optimization process involves the gradient and Hessian. 
\begin{align*}
	\mathbf{g}&=\frac{d}{d\mathbf{w}}\mathrm{NLL}(\mathbf{w})=\frac{d}{d\mu_i}\mathrm{NLL}(\mathbf{w})\frac{d\mu_i}{d\mathbf{h}}\frac{d\mathbf{h}}{d\mathbf{w}}\\
	& = \sum_{i=1}\Bigg[-\frac{y_i}{\mu_i} + \frac{(1-y_i) }{(1-\mu_i)}\Bigg]\frac{d\mu_i}{d\mathbf{h}}\frac{d\mathbf{h}}{d\mathbf{w}}=\sum_{i=1}\Bigg[\frac{\mu_i-y_i }{\mu_i(1-\mu_i)}\Bigg]\frac{d\mu_i}{d\mathbf{h}}\frac{d\mathbf{h}}{d\mathbf{w}}\\
	&=\sum_{i}(\mu_i-y_i)\mathbf{x}_i=\mathbf{X}^T(\boldsymbol{\mu}-\mathbf{y})\\
	\frac{d\mu_i}{d\mathbf{h}}& = \mu_i(1-\mu_i)\\
	\frac{d\mathbf{h}}{d\mathbf{w}}& = \mathbf{x}_i
\end{align*}
where $\mathbf{h}=\mathbf{w}^T\mathbf{x}$. 

We can also use the second-order method. 
\begin{align*}
\mathbf{H}&=\frac{d}{d\mathbf{w}}g(\mathbf{w})^T=\sum_{i}(\nabla_{\mathbf{w}}\mu_i)\mathbf{x}_i^T=\sum_{i}\mu_i(1-\mu_i)\mathbf{x}_i\mathbf{x}_i^T\\
&=\mathbf{X}^T\mathbf{S}\mathbf{X},
\end{align*}
where $\mathbf{S}\triangleq \mathrm{diag}(\mu_i(1-\mu_i))$. Note that $\mathbf{H}$ is positive definite, because the \textrm{NLL} is convex and has a global minimum. 



