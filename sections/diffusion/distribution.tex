\section{Distribution Modeling}
	What we want to learn (or model) is $p_\theta(\rvx_0)\approx p(\rvx_0)$ (approximate data distribution).
	\begin{itemize}
%		\item $p_\theta(\rvx_0)$: a distribution of output (denoised) image.
%			\begin{itemize}
%				\item This is our target.
%			\end{itemize}
		\item $p_\theta(\rvx_0) = \int p_\theta(\rvx_{0:T})d\rvx_{1:T}$
		\item It is intractable to compute all trajectories.
			$$\argmax_\theta\mathbb{E}_{\rvx_0\sim p}[\log p_\theta(\rvx_0)] = \mathbb{E}_{\rvx_0\sim p}\bigg[\log\int p_\theta(\rvx_{0:T})d\rvx_{1:T}\bigg].$$
		\item Thus, we will use variational lower bound with KL-Div:
	\end{itemize}
\begin{align}
\log p_\theta(\rvx_0) 
	&=  \log\int p(\rvx_{0:T})d\rvx_{1:T}\\
	&= \log\int p(\rvx_{0:T})\frac{q(\rvx_{1:T}|\rvx_0)}{q(\rvx_{1:T}|\rvx_0)}d\rvx_{1:T}\\
	&= \log\int q(\rvx_{1:T}|\rvx_0)\frac{p(\rvx_{0:T})}{q(\rvx_{1:T}|\rvx_0)}d\rvx_{1:T}\\
	&\geq \int q(\rvx_{1:T}|\rvx_0)\log\frac{p(\rvx_{0:T})}{q(\rvx_{1:T}|\rvx_0)}d\rvx_{1:T}\\
	&= \mathbb{E}_{q(\rvx_{1:T}|\rvx_0)}\bigg[\log\frac{p(\rvx_{0:T})}{q(\rvx_{1:T}|\rvx_0)}\bigg] \quad \to \textrm{ELBO}\\
	&= \mathbb{E}_{q(\rvx_{1:T}|\rvx_0)}\bigg[\log p(\rvx_T)\prod_{t=1}^T\frac{p(\rvx_{t-1}|\rvx_t)}{q(\rvx_{t}|\rvx_{t-1})}\bigg]\\
	&= \mathbb{E}_{q(\rvx_{1:T}|\rvx_0)}\bigg[\log \frac{p(\rvx_T)p(\rvx_0|\rvx_{1})\prod_{t=1}^{T-1} p(\rvx_{t}|\rvx_{t+1})}{q(\rvx_T|\rvx_{T-1})\prod_{t=1}^{T-1}  q(\rvx_{t}|\rvx_{t-1})}\bigg]\\
	&= \mathbb{E}_{q(\rvx_{1:T}|\rvx_0)}\bigg[\log \frac{p(\rvx_T)p(\rvx_0|\rvx_{1})}{q(\rvx_T|\rvx_{T-1})}\bigg] + \mathbb{E}_{q(\rvx_{1:T}|\rvx_0)}\bigg[\log \prod_{t=1}^{T-1}\frac{ p(\rvx_{t}|\rvx_{t+1})}{q(\rvx_{t}|\rvx_{t-1})}\bigg]\\
	&= \mathbb{E}_{q(\rvx_{1:T}|\rvx_0)}[\log p(\rvx_0|\rvx_{1})]+\mathbb{E}_{q(\rvx_{1:T}|\rvx_0)}\bigg[\log \frac{p(\rvx_T)}{q(\rvx_T|\rvx_{T-1})}\bigg]\\ 
	&\quad + \mathbb{E}_{q(\rvx_{1:T}|\rvx_0)}\bigg[\sum_{t=1}^{T-1}\log \frac{ p(\rvx_{t}|\rvx_{t+1})}{q(\rvx_{t}|\rvx_{t-1})}\bigg]\\
	&= \dots + \blue{\mathbb{E}_{q(\rvx_{1:T}|\rvx_0)}\bigg[\log \frac{p(\rvx_T)}{q(\rvx_T|\rvx_{T-1})}\bigg]} + \sum_{t=1}^{T-1}\mathbb{E}_{q(\rvx_{1:T}|\rvx_0)}\bigg[\log \frac{ p(\rvx_{t}|\rvx_{t+1})}{q(\rvx_{t}|\rvx_{t-1})}\bigg]\\
	% % &= \log\int q(\rvx_{1:T}|\rvx_0)p(\rvx_T)\prod_{t=1}^T\frac{p(\rvx_{t-1}|\rvx_t)}{q(\rvx_{t}|\rvx_{t-1})}d\rvx_{1:T}
		&= \dots + \int_{x_1}\dots\int_{x_T}q(\rvx_{1},\dots,\rvx_{T})\bigg[\log \frac{p(\rvx_T)}{q(\rvx_T|\rvx_{T-1})}\bigg]dx_1\dots dx_T + \dots\\
		&= \dots + \int_{x_T}\int_{x_{T-1}}\bigg[\log \frac{p(\rvx_T)}{q(\rvx_T|\rvx_{T-1})}\bigg]\dots\int_{x_1} q(\rvx_{1},\dots,\rvx_{T})dx_1\dots dx_T + \dots\\
		&= \dots + \int_{x_T}\int_{x_{T-1}}\bigg[\log \frac{p(\rvx_T)}{q(\rvx_T|\rvx_{T-1})}\bigg] q(\rvx_{t},\rvx_{t-1})dx_{T-1} dx_T + \dots\\
		&= \dots + \mathbb{E}_{q(\rvx_{t}, \rvx_{t-1}|\rvx_0)}\bigg[\log \frac{p(\rvx_T)}{q(\rvx_T|\rvx_{T-1})}\bigg] + \sum_{t=1}^{T-1}\mathbb{E}_{q(\rvx_{t-1},\rvx_t,\rvx_{t+1}|\rvx_0)}\bigg[\log \frac{ p(\rvx_{t}|\rvx_{t+1})}{q(\rvx_{t}|\rvx_{t-1})}\bigg]\\
		&= \mathbb{E}_{q(\rvx_{1}|\rvx_0)}[\log p(\rvx_0|\rvx_{1})] + \mathbb{E}_{q(\rvx_{t}, \rvx_{t-1}|\rvx_0)}\bigg[\log \frac{p(\rvx_T)}{q(\rvx_T|\rvx_{T-1})}\bigg]\\ 
		&\quad+ \sum_{t=1}^{T-1}\mathbb{E}_{q(\rvx_{t-1},\rvx_t,\rvx_{t+1}|\rvx_0)}\bigg[\log \frac{ p(\rvx_{t}|\rvx_{t+1})}{q(\rvx_{t}|\rvx_{t-1})}\bigg]\\
		&= \mathbb{E}_{q(\rvx_{1}|\rvx_0)}[\log p(\rvx_0|\rvx_{1})] - \mathbb{E}_{q(\rvx_{t-1}|\rvx_0)}[D_{KL}(q(\rvx_T|\rvx_{T-1})\| p(\rvx_T))]\\
		&\,- \sum_{t=1}^{T-1}\mathbb{E}_{q(\rvx_{t-1},\rvx_{t+1}|\rvx_0)}D_{KL}[q(\rvx_{t}|\rvx_{t-1})\|p(\rvx_{t}|\rvx_{t+1})]\\
		& \because q(\rvx_{t}, \rvx_{t-1}|\rvx_0) = \frac{q(\rvx_{t}, \rvx_{t-1},\rvx_0)}{q(\rvx_0)} = \frac{q(\rvx_{t}|\rvx_{t-1}, \rvx_0)q(\rvx_{t-1}, \rvx_0)}{q(\rvx_0)} = q(\rvx_{t}|\rvx_{t-1})q(\rvx_{t-1}|\rvx_0)
	% % &= \log\int q(\rvx_{1:T}|\rvx_0)p(\rvx_T)\prod_{t=1}^T\frac{p(\rvx_{t-1}|\rvx_t)}{q(\rvx_{t}|\rvx_{t-1})}d\rvx_{1:T}
	\label{eq:diffusion_mle}
\end{align}
\begin{itemize}
	\item The sixth step is done by Markov property.
	\item The first term is a \textit{reconstruction term}. 
	\item The second term is a \textit{prior matching term}. This term requires no optimization, as it has no trainable parameters; furthermore, as we have assumed a large enough $T$ such that the final distribution is Gaussian, this term effectively becomes zero.
	\item The last term is a \textit{consistency term}. It endeavors to make the distribution at xt consistent, from both forward and backward processes. That is, a denoising step from a noisier image should match the corresponding noising step from a cleaner image, for every intermediate timestep.  This term is minimized when we train $p_\theta(\rvx_t|\rvx_{t+1})$ to match the Gaussian distribution $q(\rvx_t|\rvx_{t-1})$, which is defined in \Cref{eq:forward_diffusion}. % \item For a small $\beta$, the forward and the backward processes can be identical.
	\item Under this derivation, all terms of the ELBO are computed as expectations, and can therefore be approximated using Monte Carlo estimates. 
	\item However, actually optimizing the ELBO using the terms we just derived might be suboptimal, because the consistency term is computed as an expectation over two random variables $\{x_{t-1}, x_{t+1}\}$ for every time step, the variance of its Monte Carlo estimate could potentially be higher than a term that is estimated using only one random variable per time step. As it is computed by summing up $T-1$ consistency terms, the final estimated value of the ELBO may have high variance for large $T$ values.
\end{itemize}

Let us instead try to derive a form for our ELBO where each term is computed as an expectation over \textbf{only one random variable at a time}. The key insight is that we can rewrite encoder transitions as $q(x_t|x_{t-1}) = q(x_t|x_{t-1}, x_0)$, where the extra conditioning term is superfluous due to the Markov property. Then, according to Bayes rule, we can rewrite each transition as:
\begin{align*}
	q(\rvx_t|\rvx_{t-1}, \rvx_0)= \frac{q(\rvx_{t-1}|\rvx_t, \rvx_0)q(\rvx_t| \rvx_0)}{q(\rvx_{t-1}| \rvx_0)} 
\end{align*}
Armed with this equation, we can factorize the ELBO again as follows:
\begin{align}
		L &= \mathbb{E}_{q(\rvx_{1:T}|\rvx_0)}\bigg[\log\frac{p(\rvx_{0:T})}{q(\rvx_{1:T}|\rvx_0)}\bigg] \quad \to \textrm{ELBO}\\
		&= \mathbb{E}_{q(\rvx_{1:T}|\rvx_0)}\bigg[\log p(\rvx_T)\prod_{t=1}^T\frac{p(\rvx_{t-1}|\rvx_t)}{q(\rvx_{t}|\rvx_{t-1})}\bigg]\\
		&= \mathbb{E}_{q(\rvx_{1:T}|\rvx_0)}\bigg[\log \frac{p(\rvx_T)p(\rvx_0|\rvx_{1})\prod_{t=2}^{T} p(\rvx_{t-1}|\rvx_{t})}{q(\rvx_1|\rvx_{0})\prod_{t=2}^{T}  q(\rvx_{t}|\rvx_{t-1})}\bigg]\\
		&= \mathbb{E}_{q(\rvx_{1:T}|\rvx_0)}\bigg[\log \frac{p(\rvx_T)p(\rvx_0|\rvx_{1})\prod_{t=2}^{T} p(\rvx_{t-1}|\rvx_{t})}{q(\rvx_1|\rvx_{0})\prod_{t=2}^{T}  q(\rvx_{t}|\rvx_{t-1}, \blue{\rvx_0})}\bigg] \quad \textrm{ by Markov Property}\\
		&= \mathbb{E}_{q(\rvx_{1:T}|\rvx_0)}\bigg[\log \frac{p(\rvx_T)p(\rvx_0|\rvx_{1})}{q(\rvx_1|\rvx_{0})}+\log \prod_{t=2}^{T}\frac{ p(\rvx_{t-1}|\rvx_{t})}{q(\rvx_{t}|\rvx_{t-1}, \rvx_0)}\bigg]\\
		&= \mathbb{E}_{q(\rvx_{1:T}|\rvx_0)}\bigg[\log \frac{p(\rvx_T)p(\rvx_0|\rvx_{1})}{q(\rvx_1|\rvx_{0})}+\log \prod_{t=2}^{T}\frac{ p(\rvx_{t-1}|\rvx_{t})}{\frac{q(\rvx_{t-1}|\rvx_t, \rvx_0)q(\rvx_t| \rvx_0)}{q(\rvx_{t-1}| \rvx_0)}}\bigg]\\
		&= \mathbb{E}_{q(\rvx_{1:T}|\rvx_0)}\bigg[\log \frac{p(\rvx_T)p(\rvx_0|\rvx_{1})}{q(\rvx_1|\rvx_{0})}+\log \frac{q(\rvx_1|\rvx_0)}{q(\rvx_T|\rvx_0)}+\log \prod_{t=2}^{T}\frac{ p(\rvx_{t-1}|\rvx_{t})}{q(\rvx_{t-1}|\rvx_t, \rvx_0)}\bigg]\\
		&= \mathbb{E}_{q(\rvx_{1:T}|\rvx_0)}\bigg[\log \frac{p(\rvx_T)p(\rvx_0|\rvx_{1})}{q(\rvx_T|\rvx_{0})}+\log \prod_{t=2}^{T}\frac{ p(\rvx_{t-1}|\rvx_{t})}{q(\rvx_{t-1}|\rvx_t, \rvx_0)}\bigg]\\
		&= \mathbb{E}_{q(\rvx_{1}|\rvx_0)}[\log p(\rvx_0|\rvx_{1})]+\mathbb{E}_{q(\rvx_{T}|\rvx_0)}\bigg[\log \frac{p(\rvx_T)}{q(\rvx_T|\rvx_{0})}\bigg]+\ \sum_{t=2}^{T}\mathbb{E}_{q(\rvx_{t},\rvx_{t-1}|\rvx_0)}\bigg[\log\frac{ p(\rvx_{t-1}|\rvx_{t})}{q(\rvx_{t-1}|\rvx_t, \rvx_0)}\bigg]\\
		&= \mathbb{E}_{q(\rvx_{1}|\rvx_0)}[\log p(\rvx_0|\rvx_{1})]-D_{KL}_(q(\rvx_{T}|\rvx_0)||p(\rvx_T))-\sum_{t=2}^{T}\mathbb{E}_{q(\rvx_{t}|\rvx_0)}[D_{KL}(q(\rvx_{t-1}|\rvx_t, \rvx_0)\|p(\rvx_{t-1}|\rvx_{t}))]
		% &= \mathbb{E}_{q(\rvx_{1:T}|\rvx_0)}\bigg[\log \frac{p(\rvx_T)p(\rvx_0|\rvx_{1})}{q(\rvx_T|\rvx_{T-1})}\bigg] + \mathbb{E}_{q(\rvx_{1:T}|\rvx_0)}\bigg[\log \prod_{t=1}^{T-1}\frac{ p(\rvx_{t}|\rvx_{t+1})}{q(\rvx_{t}|\rvx_{t-1})}\bigg]\\
% 		&= \mathbb{E}_{q}\Bigg[-\log p_\theta(x_T)-\log\frac{\prod_{t=1}^Tp_\theta(x_{t-1}|x_t)}{\prod_{t=1}^Tq(x_{t}|x_{t-1})}\Bigg]\\
% 		&= \mathbb{E}_{q}\Bigg[-\log p_\theta(x_T)-\sum_{t\geq 1}\log\frac{p_\theta(x_{t-1}|x_t)}{q(x_{t}|x_{t-1})}\Bigg]\\
% 		&= \mathbb{E}_{q}\Bigg[-\log p_\theta(x_T)-\sum_{t=2}^T\log\frac{p_\theta(x_{t-1}|x_t)}{q(x_{t}|x_{t-1})}-\log \frac{p_\theta(x_{0}|x_1)}{q(x_{1}|x_{0})}\Bigg]\\
% 		&= \mathbb{E}_{q}\Bigg[-\log p_\theta(x_T)-\sum_{t=2}^T\log\frac{p_\theta(x_{t-1}|x_t)}{q(x_{t-1}|x_{t},x_0)}\cdot \red{ \frac{q(x_{t-1}|x_0)}{q(x_{t}|x_0)}}\\
% 		&\quad -\log \frac{p_\theta(x_{0}|x_1)}{q(x_{1}|x_{0})}\Bigg]\\
	\label{eq:diffusion_objective}
\end{align}
Let's closely look at the last three terms:
\begin{itemize}
	\item $\mathbb{E}_{q(\rvx_{1}|\rvx_0)}[\log p(\rvx_0|\rvx_{1})]$: reconstruction term. 
	\item $D_{KL}_(q(\rvx_{T}|\rvx_0)||p(\rvx_T))$: Prior matching term.
		\begin{itemize}
			\item No trainable parameters 
		\end{itemize}
	\item $\sum_{t=2}^{T}\mathbb{E}_{q(\rvx_{t}|\rvx_0)}[D_{KL}(q(\rvx_{t-1}|\rvx_t, \rvx_0)\|p(\rvx_{t-1}|\rvx_{t}))]$: Denoising matching term.
		\begin{itemize}
			\item $p_\theta(\rvx_{t−1}|\rvx_t)$ as an approximation to tractable, ground-truth denoising transition step $q(\rvx_{t−1}|\rvx_t, \rvx_0)$. The $q(\rvx_{t−1}|\rvx_t, \rvx_0)$ transition step can act as a ground-truth signal, since it defines how to denoise a noisy image with access to what the final, completely denoised image $\rvx_0$ should be. This term is therefore minimized when the two denoising steps match as closely as possible, as measured by their KL Divergence.
		\end{itemize}
\end{itemize}
In the last term, $q(\rvx_{t-1}|\rvx_t, \rvx_0)$ can be further factorized as follows:
\begin{align*}
	q(\rvx_{t-1}|\rvx_t, \rvx_0) = \frac{q(\rvx_{t}|\rvx_{t-1}, \rvx_0)q(\rvx_{t-1}|\rvx_0)}{q(\rvx_{t}|\rvx_0)}=\frac{q(\rvx_{t}|\rvx_{t-1})q(\rvx_{t-1}|\rvx_0)}{q(\rvx_{t}|\rvx_0)}.
\end{align*}
Then, $q(\rvx_{t}|\rvx_{t-1}, \rvx_0) = q(\rvx_{t}|\rvx_{t-1})$ by Markov property. Now, let's compute the KL-divergence in the denoising matching term. 

\begin{align}
	q(\rvx_{t-1}|\rvx_t, \rvx_0) &=\frac{q(\rvx_{t}|\rvx_{t-1})q(\rvx_{t-1}|\rvx_0)}{q(\rvx_{t}|\rvx_0)}\\
								 &= \frac{\mathcal{N}(\mathbf{x}_t; \sqrt{\bar{\alpha}_t} \rvx_{t-1}, (1-\bar{\alpha}_t)\mathbf{I})\mathcal{N}(\mathbf{x}_{t-1}; \sqrt{\bar{\alpha}_{t-1}} \rvx_{0}, (1-\bar{\alpha}_{t-1})\mathbf{I})}{\mathcal{N}(\mathbf{x}_t; \sqrt{\bar{\alpha}_t} \rvx_{0}, (1-\bar{\alpha}_t)\mathbf{I})}\\
								 & \vdots\\
								 &\propto \mathcal{N}\bigg(\rvx_{t-1};\underbrace{\frac{\sqrt{{\alpha}_{t}}(1-\bar{\alpha}_{t-1})\rvx_t+\sqrt{\bar{\alpha}_{t-1}}(1-{\alpha}_{t})\rvx_0}{1-\bar{\alpha}_{t}}}_{\mu_q(\rvx_t,\rvx_0)}, \underbrace{\frac{(1-{\alpha}_{t})(1-\bar{\alpha}_{t-1})}{1-\bar{\alpha}_{t}}\mathbf{I}}_{\Sigma_{q}(t)}\bigg)
	\label{eq:diffusion_kl_true_denoising}
\end{align}

We have therefore shown that at each step, $\rvx_{t-1} \sim q(\rvx_{t-1}|\rvx_t, \rvx_0)$ is normally distributed, with mean $\mu_q(\rvx_t,\rvx_0)$ that is a function of $\rvx_t$ and $\rvx_0$, and variance $\Sigma_q(t)$ as a function of $\alpha$ coefficients. 

% These $\alpha$ coefficients are known and fixed at each time step; they are either set permanently when modeled as hyperparameters, or treated as the current inference output of a network that seeks to model them. We can rewrite the variance equation as $\Sigma_q(t) = \sigma^2_q(t)I$, where:
% $$\sigma^2_q(t) = \frac{(1-{\alpha}_{t})(1-\bar{\alpha}_{t-1})}{1-\bar{\alpha}_{t}}.$$
% In order to match approximate denoising transition step $p_\theta(\rvx_{t-1}|\rvx_{t})$ to ground-truth denoising transition step $q(\rvx_{t-1}|\rvx_t, \rvx_0)$ as closely as possible, we can also model it as a Gaussian. Furthermore, as all $\alpha$ terms are known to be frozen at each timestep, we can immediately construct the variance of the approximate denoising transition step to also be $\Sigma_q(t) = \sigma^2_q(t)I$.

% The following equation is the KL divergence between two normal distributions $P$ and $Q$:
% $$D_{KL}(P\|Q) = \frac{1}{2}\left[\log\frac{|\Sigma_2|}{|\Sigma_1|} - n + \text{tr} \{ \Sigma_2^{-1}\Sigma_1 \} + (\mu_2 - \mu_1)^T \Sigma_2^{-1}(\mu_2 - \mu_1)\right]$$
% We can optimize the KL divergence between two normal distributions to minimize the difference between the two means: 
% \begin{align}
% 	\mathcal{L} &= \min_\theta D_{KL}(q(\rvx_{t-1}|\rvx_t, \rvx_0)\|p_\theta(\rvx_{t-1}|\rvx_{t}))\\
% 				& \vdots\\
% 				&= \min_\theta \frac{1}{2\sigma^2_q(t)}\big[\|\boldsymbol{\mu}_\theta- \boldsymbol{\mu}_q\|^2_2\big],
% 	\label{eq:diffusion_kl_mean}
% \end{align}
% where $\boldsymbol{\mu}_q$ is a shorthand for $\boldsymbol{\mu}_q(\rvx_t,\rvx_0)$.

We can use the \Cref{eq:diffusion_kl_true_denoising} to compute the optimal $\theta$ by solving $\min_\theta D_{KL}(q(\rvx_{t-1}|\rvx_t, \rvx_0)\|p_\theta(\rvx_{t-1}|\rvx_{t}))$, but we can simplify the optimization problem by using \Cref{eq:diffusion_forward_sampling}:
\begin{align*}
	\rvx_t &= \sqrt{\bar{\alpha}_t} \mathbf{x}_{0}+ \sqrt{1-\bar{\alpha}_t} \odot \epsilon\\
	\mathbf{x}_{0} &= \frac{\rvx_t - \sqrt{1-\bar{\alpha}_t}\odot \epsilon}{\sqrt{\bar{\alpha}_t} } 
\end{align*}
By plugging the above equation into $\boldsymbol{\mu}_q(\rvx_t,\rvx_0)$, we can exclude $\rvx_0$ term as follows:
\begin{align}
	\boldsymbol{\mu}_q(\rvx_t,\rvx_0) &= \frac{\sqrt{{\alpha}_{t}}(1-\bar{\alpha}_{t-1})\rvx_t+\sqrt{\bar{\alpha}_{t-1}}(1-{\alpha}_{t})\rvx_0}{1-\bar{\alpha}_{t}}\\
									  &= \frac{\sqrt{{\alpha}_{t}}(1-\bar{\alpha}_{t-1})\rvx_t+\sqrt{\bar{\alpha}_{t-1}}(1-{\alpha}_{t})\frac{\rvx_t - \sqrt{1-\bar{\alpha}_t}\odot \epsilon}{\sqrt{\bar{\alpha}_t} }}{1-\bar{\alpha}_{t}}\\
									  &\vdots\\
									  &= \frac{1}{\sqrt{\alpha_t}}\rvx_t - \frac{1-\alpha_t}{\sqrt{1-\bar{\alpha_t}}\sqrt{\alpha_t}}\boldsymbol{\epsilon}_0
	\label{eq:diffusion_mean_simple}
\end{align}
Thus, we can force $\boldsymbol{\mu}_\theta(\mathbf{x}_t, t)$, which has no dependency on $\rvx_0$ term, to match the $\boldsymbol{\mu}_q$. Since the $\rvx_t$ is given at training time, we just need to predict the $\boldsymbol{\epsilon}_t$. Then, we can express $\boldsymbol{\mu}_\theta(\mathbf{x}_t, t)$ as follows:
\begin{align*}
\boldsymbol{\mu}_\theta(\mathbf{x}_t, t) &= \frac{1}{\sqrt{\alpha_t}} \bigg( \mathbf{x}_t - \frac{1 - \alpha_t}{\sqrt{1 - \bar{\alpha}_t}} \boldsymbol{\epsilon}_\theta(\mathbf{x}_t, t) \bigg) \\
\end{align*}
Thus, $\rvx_{t-1}\sim p_\theta(\mathbf{x}_{t-1} \vert \mathbf{x}_t)$ can be expressed as follows:
\begin{align*}
	\mathcal{N}\Bigg(\mathbf{x}_{t-1}; \frac{1}{\sqrt{\alpha_t}} \bigg( \mathbf{x}_t - \frac{1 - \alpha_t}{\sqrt{1 - \bar{\alpha}_t}} \boldsymbol{\epsilon}_\theta(\mathbf{x}_t, t) \bigg), \boldsymbol{\Sigma}_\theta(\mathbf{x}_t, t)\Bigg)
\end{align*}
Note that we can use the variance derived in \Cref{eq:diffusion_kl_true_denoising} instead of estimating it from a network for simplicity. Finally, we can solve the KL-divergence term:

\begin{align}
	\mathcal{L} &= \min_\theta D_{KL}(q(\rvx_{t-1}|\rvx_t, \rvx_0)\|p_\theta(\rvx_{t-1}|\rvx_{t}))\\
				& \vdots\\
				&= \min_\theta \frac{1}{2\sigma^2_q(t)}\frac{(1-\alpha_t)^2}{(1-\bar{\alpha}_t)\alpha_t}\big[\|\boldsymbol{\epsilon}_0- \hat{\boldsymbol{\epsilon}}_\theta(\rvx_t,t)\|^2_2\big].
	\label{eq:diffusion_kl_mean_simple}
\end{align}
Here $\hat{\boldsymbol{\epsilon}}_\theta(\rvx_t,t)$ is a is a neural network that learns to predict the source noise $\boldsymbol{\epsilon}_0 \sim \mathcal{N}(\boldsymbol{\epsilon}}; \mathbf{0}, \mathbf{I})$ that determines $\rvx_t$ from $\rvx_0$. We have therefore shown that the overall learning objective is equivalent to learning to predict the noise.


