\chapter{Introduction}
\section{KL Divergence}
% \href{https://dibyaghosh.com/blog/probability/kldivergence.html}{Reference}
The KL divergence can be defined as follows:
\begin{align*}
	D_{KL}(P||Q) = \mathbbm{E}_{x\sim P}\Bigg[\log \frac{P(X)}{Q(X)}\Bigg]
\end{align*}

\subsection{Properties}
\begin{itemize}
	\item Non symmetric
	\item $D_{KL} \in [0,\infty]$ 
	\item In order for the KL divergence to be finite, the support of $P$ needs to be in the support of $Q$. 
\end{itemize}

\subsection{Rewriting the Objective}
\begin{align*}
	D_{KL}(P||Q) &= \mathbbm{E}_{x\sim P}\Bigg[\log \frac{P(X)}{Q(X)}\Bigg]\\
	&= \mathbbm{E}_{x\sim P} [-\log Q(X)] - \mathcal{H}(P(X))
\end{align*}

\begin{itemize}
	\item $\mathbbm{E}_{x\sim P} [-\log Q(X)]$: Cross entropy
	\item $\mathcal{H}(P(X))$: Entropy of $P$
\end{itemize}

\subsection{Forward and Reverse KL}

Let's say there is a true distribution $P(X)$ with two modes and our approximation $Q(X)$ has one mode. Then,

\begin{itemize}
	\item Forward KL: $D_{KL}(P||Q)$
	\item Reverse KL: $D_{KL}(Q||P)$
\end{itemize}

\paragraph{Forward KL: Mean-Seeking Behavior}
\begin{align*}
\arg\min_{\theta}D_{KL}(P \| Q) &= \arg\min_{\theta} \mathbbm{E}_{x \sim P}[-\log Q_\theta(X)] - \mathcal{H}(P(X))\\
&= \arg\min_{\theta} \mathbbm{E}_{x \sim P}[-\log Q_\theta(X)]\\
&= \arg\max_{\theta} \mathbbm{E}_{x \sim P}[\log Q_\theta(X)]
\end{align*}

Intuition: $x$ will be sampled from the distribution $P$, and its value will be estimated from $Q$. Thus, there will be higher chance that $x$ will be sampled from a space with higher probablity in $P$. Therefore, $Q_\theta$ has to consider all modes, which have high probabilities. 

To use the forward KL, we have to have an access to the true model $P(X)$ for sampling. 

\paragraph{Reverse KL: Mode-Seeking Behavior}
\begin{align*}
\arg\min_{\theta}D_{KL}(Q \| P) &= \arg\min_{\theta} \mathbbm{E}_{x \sim Q_\theta}[-\log P(X)] - \mathcal{H}(Q_\theta(X))\\
&= \arg\max_{\theta} \mathbbm{E}_{x \sim Q_\theta}[\log P(X)] + \mathcal{H}(Q_{\theta}(X))
\end{align*}

Intuition: $x$ will be sampled from the distribution $Q$, and its value will be estimated from $P$. Thus, there will be higher chance that $x$ will be sampled from a space with higher probablity in $Q$. Therefore, to maximize the value, we need to focus on a single mode.  

To use the reverse KL, we have to be able to evaluate the true model $P(X)$. 


