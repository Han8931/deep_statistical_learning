\chapter{Fourier Analysis}
Reference: Y.D. Chong 2021, NTU Lecture Note

The Fourier transform is one of the most important mathematical tools used for analyzing functions. Given an arbitrary function $f(x)$, with a real domain $(x \in \mathbb{R})$, we can express it as a linear combination of complex waves

\section{Fourier Series}
\label{sec:fourier_series}

We begin by discussing the Fourier series, which is used to analyze functions that are periodic in their inputs. A periodic function $f(x)$ is a function of a real variable $x$ that repeats itself every time $x$ changes by $a$. The constant $a$ is called the period. We can write the periodicity condition as
$$f(x+a) = f(x), \forall x\in \mathbb{R}.$$
The value of $f(x)$ can be real or complex, but $x$ should be real.

Let's consider what it means to specify a periodic function $f(x)$. One way to specify the function is to give an explicit mathematical formula for it. Another approach might be to specify the function values in $−a/2 \leq x < a/2$. Since there's an uncountably infinite number of points in this domain, we can generally only achieve an approximate specification of $f$ this way, by giving the values of $f$ at a large but finite set $x$ points. 

There is another interesting approach to specifying f. We can express it as a linear combination of simpler periodic functions, consisting of sines and cosines:
$$f(x) = \sum_{n=1}^{\infty} \alpha_n \sin\bigg(\frac{2\pi n x}{a}\bigg)+\sum_{m=0}^{\infty}\beta_m \cos\bigg(\frac{2\pi m x}{a}\bigg).$$
Note that the index $n$ does not include 0; since the sine term with $n = 0$ vanishes for all $x$, it's redundant. The above formula is called a \textit{Fourier series}. Given the numbers $\{\alpha_n, \beta_m\}$, which are called the \textit{Fourier coefficients}, $f(x)$ can be calculated for any $x$. The Fourier coefficients are real if $f(x)$ is a real function, or complex if $f(x)$ is complex.

\subsection{Square-integrable functions}
Can arbitrary periodic functions always be expressed as a Fourier series? It turns out that a certain class of periodic functions, commonly encountered in physical contexts, are guaranteed to always be expressible as Fourier series. These are called square-integrable functions such that the integral of the square of the absolute value is finite:
\begin{align*}
\int_{-a/2}^{a/2}|f(x)|^2 dx < \infty.
\end{align*}
Unless otherwise stated, we will always assume that the functions we're dealing with are square-integrable.

\subsection{Complex Fourier series and inverse relations}

We have written the Fourier series as a sum of sine and cosine functions. However, sines and cosines can be expressed by exponential functions by using \textit{Euler's formula}. 
\begin{align}
	e^{ix}=\cos x+i\sin x
	\label{eq:euler_formula}
\end{align}
\begin{itemize}
	\item $\cos x=\frac{e^{ix}+e^{-ix}}{2}$
	\item $\sin x=\frac{e^{ix}-e^{-ix}}{2i}$
\end{itemize}
Thus, Fourier series can be expressed as follows:
$$f(x) = \sum_{n=-\infty}^{\infty}e^{2\pi i n x/a}f_n$$
\begin{itemize}
	\item $i$: complex number
	\item $n$: integer
	\item $a$: period
	\item $f_n$: Fourier coefficient
\end{itemize}


